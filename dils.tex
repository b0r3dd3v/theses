\documentclass[b]{subfiles}
\begin{document}
\chapter{Dilations}

\begin{parsec}[dils-intro]%
\begin{point}%
In this chapter we will study various dilation theorems.
The common theme is that a complicated map
    is actually the composition of a simpler map
    after a representation into a larger algebra.
We already saw a dilation theorem in disguise:
    using the Gel'fand--Naimark--Segal construction (see \sref{gns})
    one shows every state is a vector state on a larger algebra:
\end{point}
\begin{point}[dils-gns]{Exercise (GNS')}%
    Show that
    for each pu-map~$\omega\colon \scrA \to \C$
        from a~C$^*$-algebra~$\scrA$
    there is a Hilbert space~$\scrH$,
    a miu-map~$\varrho\colon \scrA \to \scrB(\scrH)$
    and vector~$x \in \scrH$
    such that~$\omega = h \after \varrho$
    where~$h \colon \scrB(\scrH) \to \C$
    is given by~$h(T) = \left<Tx,x\right>$.
\end{point}

\begin{point}%
Probably the most famous dilation theorem is that
of Stinespring\cite[Thm.~1]{stinespring}.
\end{point}

\begin{point}[stinespring-theorem]{Theorem (Stinespring)}
    For every cp-map~$\varphi\colon \scrA \to \scrB (\scrH)$
    there is a Hilbert space~$\scrK$,
        miu-map~$\varrho\colon \scrA \to \scrB(\scrK)$
        and bounded operator~$V\colon \scrH \to \scrK$
        such that~$\varphi = \ad_V \after \varrho$,
        where~$\Define{\ad_V} \colon \scrB(\scrK) \to \scrB(\scrH)$
        is given by~$\ad_V(T)\equiv V^*TV$.

Furthermore:
\begin{inparaenum}
\item
    if~$\varphi$ is normal, then~$\varrho$ is also normal \emph{and}
\item
    if~$\varphi$ is unital, then~$V$ is an isometry
        (or equivalently~$\ad_V$ is unital).
\end{inparaenum}
\end{point}

\begin{point}%
(We will see a detailed proof later in \TODO{}.)
Stinespring's theorem
is fundamental in the study
of quantum information and quantum computing:
it is used to prove entropy inequalities (e.g.~\cite{lindblad}),
bounds on optimal cloners (e.g.~\cite{werner}),
full completeness of quantum programming languages (e.g.~\cite{staton}),
security of quantum key distribution (e.g.~\cite{werner2}),
analyze quantum alternation (e.g.~\cite{prakash}),
to categorify quantum processes (e.g.~\cite{selinger}) \emph{and}
as an axiom to single out
quantum theory among information processing theories.\cite{chiribella}
A fair overview of all uses of Stinespring's theorem and its consequences
is a formidable task out of scope of this text.

Stinespring's theorem only applies
to maps of the form~$\scrA \to \scrB(\scrH)$
    and so do most of its usefull consequences.
One wonders:
    is there an extension of Stinespring's theorem
    to arbitrary np-maps~$\scrA \to \scrB$?
A different and less common question might be:
    is Stinespring's dilation categorical in some way.
That is: does it have a defining universal property?
Both questions turn out to be true:
Paschke's generalization of GNS for Hilbert C$^*$-modules\cite{paschke}
    turns out to have the same universal property
        as Stinespring's dilation and so it extends Stinespring
        to arbitrary np-maps.

We start of this chapter with a detailed proof of Stinespring's theorem.
We continue to show it obeys a universal property.
Before we move on to Paschke's GNS
    we need to develop Paschke's theory of self-dual Hilbert C$^*$-modules.
Paschke's work builds on Sakai's characterization of von Neumann algebras,
    which would take considerable effort to develop in detail.
Thus to be as complete as possible
we avoid Sakai's characterization (in contrast to our
    article \cite{wwpaschke})
    and give new proofs
    of Paschke's results where required.
One major difference is that we will use
    a restricted completion (see \TODO{}) of a uniform space
    instead of considering the dual space of a Hilbert C$^*$-module.
We finish the first part of this chapter
    by constructing Paschke's dilation
    and establishing it indeed extends Stinespring's dilation.
In the second part of this chaper
    we prove several results about Paschke's dilations.
\TODO{more intro}
\end{point}
\end{parsec}

\section*{Stinespring's theorem}
\begin{parsec}[dils-completion-to-hilb]%
\begin{point}%
For the GNS-construction
it was necessary to ``complete''\footnote{Note that
        we do not require an inner product to be definite.
    The inner product on a Hilbert space \emph{is} definite.
    Thus the completion will quotient out those vectors with
        zero norm with respect to the inner product.}
    a complex vector space with inner product into a Hilbert space.
This completion was only sketched in \sref{completion-inner-product-space}.
Here we will work through the details
    as the corresponding completion required
    in Paschke's dilation
    is more complex and its exposition will benefit
    from this familiar analogon.
\end{point}

\begin{point}[prop-complete-into-hilbert-space]{Proposition}%
    Let~$V$ be a complex vector space with inner
        product~$[\,\cdot\,,\,\cdot\,]$.
    There is a Hilbert space~$\scrH$
        together with bounded linear map~$\eta\colon V \to \scrH$
            such that
        \begin{inparaenum}
        \item
        $[v,w] = \left<\eta(v), \eta(w)\right>$
            for all~$v,w \in V$ and
        \item
        the image of~$\eta$ is dense in~$\scrH$.
        \end{inparaenum}
\begin{point}{Proof}%
We will form~$\scrH$ from the set of Cauchy sequences in~$V$
    with a little twist.
Recall two Cauchy
    sequences~$(v_n)_n$ and~$(w_n)_n$ in~$V$
    are said to be equivalent
    if for every~$\varepsilon > 0$
    there is a~$n_0$
    such that~$\| v_n - w_n \| \leq \varepsilon$
    for all~$n \geq n_0$,
    where $\|v\| \equiv \sqrt{[v,v]}$.
Call a Cauchy sequence~$(v_n)_n$ \Define{fast}
    if for each~$n_0$
    we have~$\| v_n - v_m\| \leq \frac{1}{2^{n_0}}$
    for all~$n,m \geq n_0$.
Clearly every Cauchy sequence has a fast subsequence
    which is (as are all subsequences) equivalent with it.
\begin{point}%
    Define~$\scrH$ to be the set of fast Cauchy sequences modulo
        equivalence.
For brevity we will denote an element of~$\scrH$,
    which is an equivalence class of Cauchy sequences, simply by
    a single representative.
Also, we often tersely write~$v$ for the Cauchy sequence~$(v_n)_n$.
The set~$\scrH$ is a metric space with the standard
    distance~$d(v, w) \equiv \lim_{n\to\infty} \| v_n - w_n\|$.
To show it's complete, assume
$v^1, v^2, \ldots$
is a fast Cauchy sequence of fast Cauchy sequences.
Assume~$n,m,k \geq N$.  We have
\begin{equation*}
    \| v^n_n - v^m_m \|
        \leq
    \| v^n_n - v^n_k \| +
    \| v^n_k - v^m_k \| +
    \| v^m_k - v^m_m \| \leq
    \| v^n_k - v^m_k \|+ \frac{2}{2^N}.
\end{equation*}
Because~$\lim_{k\to\infty} \|v_k^n-v_k^m \| =d(v^n,v^m) \leq 2^{-N}$
    we can find a~$k \geq N$
    such that~$\| v^n_k - v^m_k \| \leq \frac{2}{2^N}$
    and so~$\|v^n_n - v^m_m\| \leq \frac{4}{2^N}$.
    Thus~$(v^n_n)_n$ is a Cauchy sequence.
It's easily checked~$v^1, v^2, \ldots$
converges to~$(v^n_n)_n$ with respect to~$d$.
The sequence~$(v^n_n)_n$ might not be fast
    and so, might not be in~$\scrH$,
    but like any other Cauchy sequence,
    it has a fast subsequence.
The equivalence class
of this subsequence is the limit of~$v^1, v^2, \ldots$ in~$\scrH$.
We have shown~$\scrH$ is complete.
\end{point}
\begin{point}[prop-hilbert-space-completion-extension]%
Define~$\eta\colon V \to \scrH$
    to be the map which sends~$v$ to the equivalence class of the
    constant sequence~$(v)_n$.
By construction of~$\eta$ and~$\scrH$,
    the image of~$\eta$ is dense in~$\scrH$.
Let~$f\colon V \to X$
    be any uniformly continuous map to a complete metric space~$X$.
We will show there is a unique continuous~$g\colon \scrH \to X$
    such that~$g \after \eta = f$.
From the uniform continuity it easily follows
    that~$f$ maps Cauchy sequences to Cauchy sequences
    and preserves equivalence between them.
Together with the completeness of ~$X$
    there is a unique~$g\colon \scrH \to X$
    fixed by~$g((v_n)_n) = \lim_{n\to\infty}f(v_n)$.
Clearly~$g \after \eta = f$.
Finally, $g$ is unique as it's fixed on the image of~$\eta$, which is dense.
\end{point}
\begin{point}[completion-to-hilb-vect]%
Thus we directly get a scalar multiplication
    on~$\scrH$ by extending~$\eta \after r_z \colon V \to \scrH$
    where~$r_z(v) = zv$ is scalar multiplication by~$z \in C$,
    which is uniformly continuous.
In fact~$z (v_n)_n \equiv (z v_n)_n$.
Extending addition is less direct, but straight-forward:
    given representatives~$v,w \in \scrH$
    we know~$(v_n+w_n)_n$ is Cauchy in~$V$
    by uniform continuity of addition and so
    picking a fast subsequence~$v+w$ of $(v_n+w_n)_n$
        fixes an addition on~$\scrH$.
With this expression for addition and the similar one for
    scalar multiplication it is easy to see
    they turn~$\scrH$ into a complex vector space with zero~$\eta(0)$
    for which~$\eta$ is linear.
Extending the inner product is bit trickier.
\end{point}
\begin{point}%
To define the inner product on~$\scrH$,
first note that for Cauchy sequences~$v$ and~$w$ in~$V$
we have (using Cauchy--Schwarz, \sref{chilb-cs}, on the second line):
\begin{align*}
    \bigl|[v_n,w_n] - [v_m,w_m]\bigr|
    & \ =\  \bigl|[v_n,w_n-w_m] + [v_n - v_m,w_m]\bigr| \\
    & \ \leq\  \|v_n\| \|w_n - w_m\| + \|v_n-v_m\|\|w_m\|.
\end{align*}
Thus as~$(\|v_n\|)_n$ and~$(\|w_n\|)_n$ are bounded
we see~$([v_n,w_n])_n$ is Cauchy.
In a similar fashion we see~$\bigl|[v_n,w_n] - [v'_n,w'_n]\bigr|
    \leq \|v_n\| \|w_n - w_n'\| + \|v_n-v_n'\|\|w'_n\|\to 0$ as~$n\to \infty$
    for~$v'$ and~$w'$ Cauchy sequences equivalent
to~$v$ respectively~$w$.
Thus
$([v_n,w_n])_n$ is equivalent to
$([v'_n,w'_n])_n$
    and so
    we define~$\left<v,w\right>
        \equiv \lim_{n\to \infty} [v_n,w_n]$ on~$\scrH$.
With vector space structure from~\sref{completion-to-hilb-vect}
we easily see this is an inner product~$\scrH$.
The metric induced by the inner product coincides with~$d$:
\begin{equation*}
    \left<v-w,v-w\right>
   =\lim_{n}[v_n-w_n,v_n-w_n]
    =\lim_{n}\|v_n-w_n\|^2
    = d(v,w)^2.
\end{equation*}
And so~$\scrH$ is a Hilbert space.
Finally, $\left<\eta(v),\eta(w)\right>=[v,w]$ is direct.\qed
\end{point}
\end{point}
\end{point}
\end{parsec}

\begin{parsec}[dils-stinespring]%
\begin{point}%
We are ready to prove Stinespring's dilation theorem.
\begin{point}{Proof of Stinespring's theorem \sref{stinespring-theorem}}%
Let~$\varphi\colon \scrA \to \scrB(\scrH)$
    be a cp-map.
Write~$\scrA \odot \scrH$ for the tensor product of~$\scrA$ and~$\scrH$
    as vector spaces.
By linear extension,
the following fixes a sesquilinear form on~$\scrA \odot \scrH$:
\begin{equation*}
    [a\otimes x, b \otimes y] = \left<x, \varphi(a^*b)y\right>_{\scrH}.
\end{equation*}
As~$\varphi$ preserves involution as a positive map,
see \sref{cstar-p-implies-i}, this is also a symmetric form, i.e.: $[t,s]=\overline{[s,t]}$.
    Assume~$\sum^n_{i=1} a_i\otimes x_i$
is an arbitrary element of~$A \odot \scrH$.
To see $[\,\cdot\,,\,\cdot\,]$ is positive,
we need to show
\begin{equation*}
    0 \leq \bigl[\sum_i a_i\otimes x_i, \sum_j a_j\otimes x_j\bigr]
        = \sum_{i,j} [a_i\otimes x_i, a_j\otimes x_j]
        = \sum_{i,j} \left< x_i, \varphi(a_i^*a_j) x_j \right>.
\end{equation*}
The matrix algebra~$M_n\scrB(\scrH)$
acts on~$\scrH^{\oplus n}$
as~$(A (x_1,\ldots,x_n)^\T)_i = \sum_j A_{ij} x_j$.
(In fact, this gives an
    isomorphism~$M_n\scrB(\scrH) \cong \scrB(\scrH^{\oplus n})$.)
Writing~$x$ for the vector~$(x_1,\ldots,x_n)^\T$ in~$\scrH^{\oplus n}$,
    we get
\begin{equation}\label{eq-stinespring-norm-tensor}
    \sum_{i,j} \left< x_i, \varphi(a_i^*a_j) x_j \right>
    = \bigl<
    x,
    (M_n\varphi) \bigl(\, (a_i^*a_j)_{ij} \, \bigr)\,
    x \bigr>_{\scrH^{\oplus n}}.
\end{equation}
The matrix~$(a_i^*a_j)_{ij}$
is positive as~$(C^*C)_{ij} = a_i^*a_j$
    with~$C_{ij} \equiv \frac{1}{\sqrt{n}} a_j$.
    By complete positivity~$M_n\varphi(\,(a_i^*a_j)_{ij}\,)$ is positive
    and so is~\eqref{eq-stinespring-norm-tensor},
    hence~$[\,\cdot\,,\,\cdot\,]$ is an inner product.
Write~$\eta\colon A\odot \scrH \to \scrK$ for the Hilbert space completion
    described in~\sref{prop-complete-into-hilbert-space}.
\begin{point}[stinespring-extend-operator]%
Let~$T\colon A \odot \scrH \to A \odot \scrH$
be a bouned linear map.
We show there is an extension~$\hat{T} \colon \scrK \to \scrK$.
The operator $T$
uniformly continuous and so is~$\eta\after T \colon A\odot\scrH \to\scrK$.
Thus by \sref{prop-hilbert-space-completion-extension}
there is a unique continuous extension~$\hat{T} \colon \scrK \to \scrK$
with~$\hat{T}(\eta(t)) = \eta(T(t))$
    for all~$t \in \scrA\odot\scrH$.
Clearly~$\hat{T}$ is linear on the image of~$\eta$,
    which is dense and so~$\hat{T}$ is linear everwhere.
Hence~$\hat{T}$ is bounded, so~$\hat{T} \in \scrB(\scrK)$.
It is easy to see~$\widehat{T+S}=\hat{T}+\hat{S}$ and
    $\widehat{\lambda T} = \lambda \hat{T}$
    for operators~$S,T$ on~$\scrA \odot \scrH$ and~$\lambda \in \C$.
Also~$\widehat{TS} = \hat{T}\hat{S}$:
indeed~$\hat{T}\hat{S} \eta(t)
            = \hat{T} \eta(St)
            = \eta(TSt)$
    and so by uniqueness~$\hat{T}\hat{S} = \widehat{TS}$.
\end{point}
\begin{point}%
    Assume~$b\in \scrA$.
    Let~$\varrho_0(b)$ be the operator on~$\scrA \otimes \scrH$
    fixed by~$\varrho_0(b) a\otimes x = (b a) \otimes x$.
Clearly $\varrho_0$ is linear, unital and multiplicative.
We want to show~$\varrho_0(b)$ is bounded for fixed~$b \in \scrA$.
To show this, we claim that in~$M_n\scrB(\scrH)$
we have~$(a_i^*b^*ba_j)_{ij} \leq \|b\|^2 (a_i^*a_j)_{ij}$.
Indeed, as~$b^*b \leq \|b\|^2$,
    there is some~$c$ with~$c^*c = \|b\|^2 - b^*b$.
Define~$C_{ij} \equiv \frac{1}{\sqrt{n}} ca_j$.
We compute:
$ (C^*C)_{ij} = a_i^*c^*ca_j = a_i^* (\|b\|^2 - b^*b) a_j$
and get the claimed inequality with which
\begin{equation*}
    \begin{split}
    \bigl\| \sum_i (ba_i) \otimes x_i \bigr\|^2
    & = \bigl< x, (M_n\varphi)(\, (a_i^* b^*b a_j)_{ij}\,)x\bigr> \\
    &\leq \|b\|^2 \bigl< x, (M_n\varphi)(\, (a_i^* a_j)_{ij}\,)x\bigr>
    = \|b\|^2 \bigl\| \sum_i a_i \otimes x_i \bigr\|^2
    \end{split}
\end{equation*}
and so~$\varrho_0(b)$ is bounded.
Now define~$\varrho(b) \colon \scrA \to \scrB(\scrH)$
    by~$\varrho(b) \equiv \widehat{\varrho_0(b)}$.
\end{point}
\begin{point}%
We already know~$\varrho$ is a~mu-map.
We want to show it preserves involution: $\varrho(c^*) = \varrho(c)^*$
for all~$c \in \scrA$.
Indeed, for every~$a,b \in \scrA$ and~$x,y \in \scrH$
    we have
    \begin{equation*}
        [\varrho_0(c^*) \, a\otimes x,b \otimes y]
        = [(c^* a)\otimes x,b \otimes y]
        = \left<x, \varphi(a^*cb) y \right>
        = [a\otimes x,\varrho_0(c)\, b \otimes y].
    \end{equation*}
Hence~$
    \left<\varrho(c^*) \eta(a\otimes x), \eta(b\otimes y)\right>=
    \left< \eta(a\otimes x), \varrho(c)\eta(b\otimes y)\right>$.
    As the linear span of~$\eta(a\otimes x)$ is dense in~$\scrK$,
        we find~$\varrho(c^*) = \varrho(c)^*$, as desired.
\end{point}
\begin{point}%
    Let~$V_0 \colon \scrH \to \scrA \odot \scrH$
        be given by~$V_0 x = 1 \otimes x$.
        It's bounded: $\| V_0 x\| = \left<x, \varphi(1^*1) x\right>^{\frac{1}{2}}
        = \|\sqrt{\varphi(1)} x\| \leq \|\sqrt{\varphi(1)}\| \|x\|$.
    Define~$V \equiv \eta \after V_0$.
    For all~$a \in \scrA$ and~$x,y \in \scrH$ we have
            $[a \otimes x, V_0 y]
            = \left<x, \varphi(a^*)y\right>
            = \left<x, \varphi(a)^*y\right>
            = \left<\varphi(a) x, y\right>$.
    Thus~$V^*$ satisfies~$V^* \eta(a \otimes x) = \varphi(a)x$.
    Hence for all~$a \in \scrA$ and~$x \in \scrH$:
    $V^* \varrho(a) V x = V^* \varrho(a) \eta(1 \otimes x)
        = V^* \eta(a\otimes x)
        = \varphi(a)x$.
        So~$\ad_V \after \varrho = \varphi$.

    If~$\varphi$ is unital,
    then~$V^*Vx = V^* (1\otimes x) = \varphi(1)x=x$
        for all~$x \in \scrH$ and so~$V$ is an isometry.
    (Or equivalently~$\ad_V$ is unital.)
\end{point}
\begin{point}%
    It remains to be shown that~$\varrho$
        is normal when~$\varphi$ is.
    Assume~$D \subseteq \scrA$ is a bounded directed
    set of self adjoint elements.
    We need to show~$\sup_{d \in D} \varrho (d) = \varrho(\sup D)$.
    Let~$t$ be in the image of~$\eta$.
    It is sufficient to show~$\left<t, \varrho(\sup D) t\right>
        = \left<t, \sup_{d \in D} \varrho(d)t\right>$.
    We have~$t = \sum_{i=1}^n a_i \otimes x_i$
        for some~$n \in \N$, $a_1, \ldots, a_n \in \scrA$
            and~$x_1,\ldots, x_n \in \scrH$.
    Write~$x$ for the vector~$(x_1, \ldots, x_n)^\T$
        in~$\scrH^{\oplus n}$.
    Then
    \begin{align*}
        \left<t, \varrho(\sup D) t\right>
        &= \bigl< x, (M_n\varphi)\bigl(\, ( a_i^* (\sup D) a_j)_{ij} \,\bigr) x \bigr>_{\scrH^{\oplus n}} \\
        &= \sup_{d\in D} \bigl< x, (M_n\varphi)\bigl(\, ( a_i^* d a_j)_{ij} \,\bigr) x \bigr>_{\scrH^{\oplus n}}\\
        & = \sup_{d\in D} \left<t, \varrho(d)t\right>
            = \bigl<t, \sup_{d \in D} \varrho(d) t\bigr>.
    \end{align*}
    Here we used that~$\left<x, (\,\cdot\,)x\right>$,
        $M_n\varphi$ and 
    $d \mapsto (a_i^*da_j)_{ij}$ are normal.
    The first is normal by~\sref{hilb-suprema}.
    \TODO{}
    At the moment it will be a bit tricky to show~$(a_i^*da_j)_{ij}$
    is normal.  Later (\TODO{}) we will have a simpler argument.
    Define~$C_{ij} = \frac{1}{\sqrt{n}} a_j$
        and~$\Delta\colon \scrA \to M_n \scrA$
        by~$\Delta(d)_{ij} = \delta_{ij}$.\TODO{kronecker}.
    Then~$(C^*\Delta(d)C)_{ij} = a_i^* d a_j$.
    As~$C^*(\,\cdot\,)C$ is normal by~\sref{ad-normal},
        it is sufficient to show~$\Delta$ is normal.
    Without loss of generality, we may assume~$\scrA$
        is an algebra of operators on a Hilbert spaces~$\scrL$.
        By~\sref{vna-supremum-uslimit}
        $(d)_{d \in D}$ converges ultrastrongly to~$\sup D$.
    In particular it converges in the WOT.
    Similarly
    $(\Delta(d))_{d \in D}$
        converges in the WOT
        to~$\sup_{d \in D} \Delta(d)$.
        Let~$x \equiv (x_1, \ldots, x_n)^\T$ be a vector in~$\scrL^{\oplus n}$.
        Then~$
        (\left<x,  \Delta(d)x\right>)_{d \in D}
        = (\sum_i \left<x_i,  dx_i \right>)_{d \in D}$
        converges to
        $ \sum_i \left<x_i,  (\sup D )x_i \right>
            = \left<x,  \Delta(\sup D )x \right>$
    and so~$(\Delta(d))_{d \in D}$
    also converges in the WOT to $\Delta(\sup D)$.
    Hence~$\Delta(\sup D) = \sup_{d \in D} \Delta(D)$
        and so~$\Delta$ is normal.\qed
\end{point}
\end{point}
\begin{point}%
Stinespring's theorem generalizes GNS.
It seems tempting to directly prove Stinespring instead of GNS ---
however, we used GNS in the proof of Stinespring theorem to show
that~$\varrho$ is normal, when~$\varphi$ is.
\end{point}
\end{point}
\end{parsec}

\begin{parsec}%
Let~$\varphi\colon \scrA \to \scrB(\scrH)$
    be any ncp-map.
A \Define{normal Stinespring dilation}
    is a triplet~$(\scrK,\varrho, V)$
    with~$\scrK$ a Hilbert space,
    $\varrho\colon \scrA \to \scrB(\scrK)$ a nmiu-map
        (i.e.~normal representation)
        and~$V\colon \scrH \to \scrK$ a bounded linear map
        with~$\varphi = \ad_V \after \varrho$.
We just saw every~$\varphi$ has a normal Stinespring dilation.
If the linear span of~$\{ \varrho(a) Vx; \ a \in \scrA,\ x\in \scrH\}$
is dense in~$\scrK$
then the Stinespring dilation $(\scrK,\varrho,V)$ is said to be \Define{minimal}.
Every Stinespring dilation of~$\varphi$ can be restricted to a minimal one
    for~$\varphi$.
\begin{point}%
It is a well-known fact that all minimal normal Stinespring dilations
    for a fixed~$\varphi$ are unitarily equivalent,
    see for instance~\cite[Proposition 4.2]{paulsen}.
We will adapt its proof to show that a minimal normal Stinespring
    dilation admits a universal property \TODO{},
    which we will need later on.
This adapation is mostly straight-forward, except for the following lemma.
\end{point}
\begin{point}{Lemma}%
% \TODO{}\parpic[r]{
% $\xymatrix{\mathscr{A} \ar[r]^\varrho \ar[rd]_{\varrho'}&
% \mathscr{B} \\
% & \mathscr{C} \ar[u]_\sigma}$}
Let~$\varrho\colon \scrA \to \scrB$ and~$\varrho'\colon \scrA \to \scrC$
    be nmiu-maps between von Neumann algebras,
        and let~$\sigma \colon \scrC \to \scrB$ be a ncp-map
        such that~$\sigma \after \varrho' = \varrho$.\\
Then~$\sigma(\varrho'(a_1)c \varrho'(a_2)) =  \varrho(a_1) \sigma(c) \varrho(a_2)$
    for all~$a_1,a_2 \in \scrA$ and~$c \in \scrC$.
\begin{point}{Proof}%
By Theorem 3.1 of \cite{choi} (see \sref{choi}),
we know that for all~$c,d\in\scrC$:
\begin{equation}
    \sigma(d^*d) = \sigma(d)^*\sigma(d) \quad \implies \quad
    \sigma(cd) = \sigma(c)\sigma(d).\label{dils-eq-choi}
\end{equation}
Let~$a \in \scrA$.
We have~$\sigma(\varrho'(a)^* \varrho'(a))
    = \sigma(\varrho'(a^*a))
    = \varrho(a^*a)
    = \varrho(a)^*\varrho(a)
    = \sigma(\varrho'(a))^*\sigma(\varrho'(a))$.
By \eqref{dils-eq-choi},
    we have~$\sigma(c\varrho'(a)))
        = \sigma(c)\sigma(\varrho'(a)) = \sigma(c)\varrho(a)$
        for all~$c \in \scrC$.
Thus also (taking adjoints):
$\sigma(\varrho'(a)c) = \varrho(a)\sigma(c)$ for all~$c \in \scrC$.
Hence~$\sigma(\varrho'(a_1)c\varrho'(a_2))
            = \varrho(a_1)\sigma(c \varrho'(a_2))
            = \varrho(a_1) \sigma(c) \varrho(a_2)$
    for all~$a_1,a_2 \in \scrA$ and~$c\in \scrC$ as desired.\qed
\end{point}
\end{point}
\end{parsec}

\TODO{universal property}
\TODO{kraus}

\section*{Hilbert C$^*$-modules}

\begin{parsec}%
\begin{point}%
In \sref{prop-complete-into-hilbert-space}
    we saw how to complete a complex vector space with inner product
    to a Hilbert space.
Now we will see how to complete a right-$\scrB$-module
    with $\scrB$-valued inner product
    to a self-dual Hilbert~$\scrB$-module
    under the assumption~$\scrB$ is a von Neumann algebra.
We will use a different construction than Paschke.
He shows that for a pre-Hilbert $\scrB$-module~$X$
    the set of functionals~$X'$
        (i.e.~$\scrB$-module homomorphisms into~$\scrB$)
    turns out to  be a self-dual Hilbert~$\scrB$-module,
    which he uses as the completion of~$X$.
A considerable part of his paper \cite{paschke}
    is devoted to this construction.
It requires Sakai's characterization of von Neumann algebras,
    which we have not covered.
To avoid developing Sakai's theory,
    we give a different construction.
Instead of embedding~$X$ into a dual space,
    we will stay closer to the similar fact for Hilbert spaces
    and use an topological completion.
A simple metric completion will not do,
    we will need to complete $X$ as a \emph{uniform space}.
Uniformities are structures that sit between topological spaces
and metric spaces.
\end{point}
\begin{point}[dils-dfn-uniformity]{Definition}%
    A \Define{uniform space} is a set~$X$
    together with a family of relations
    $\Phi \subseteq \wp (X\times X)$ called \Define{entourages}
        satisfying the following conditions.
    \begin{enumerate}
    \item
        The set of entourages~$\Phi$ is a filter.
        That is:
        \begin{inparaenum}
        \item
        if~$\varepsilon,\delta \in \Phi$,
            then~$\varepsilon \cap \delta \in \Phi$
        \emph{and}
        \item
        if~$\varepsilon \subseteq \delta$ and~$\varepsilon \in \Phi$,
            then~$\varepsilon \in \Phi$.
        \end{inparaenum}
    \item
        For every~$\varepsilon \in \Phi$
            and~$x \in X$ we have~$x \mathrel\varepsilon x$.
    \item
        For each~$\varepsilon \in \Phi$,
            there is a~$\delta \in \Phi$
            such that~$\delta^2 \subseteq \varepsilon$.
        Thus, if~$x \mathrel\delta y$ and~$y \mathrel\delta z$
            then~$x \mathrel\varepsilon z$ for any $x,y,z \in X$.
    \item
        For every~$\varepsilon \in \Phi$,
            there is a~$\delta \in \Phi$
                with~$\delta^{-1}\subseteq \varepsilon$.
        Thus, if~$x \mathrel\delta y$ then~$y \mathrel\varepsilon x$
            for any~$x,y \in X$.
    \end{enumerate}
    A uniform space is \Define{Hausdorff}
        whenever~$x \mathrel\varepsilon y$
            for all~$\varepsilon \in \Phi$
            imlies~$x=y$.
\begin{point}%
The elements of~$X$ are the points of the uniform space.
The entourages~$\varepsilon \in \Phi$
    are generalized distances:
one reads~$x \mathrel\varepsilon y$ as
    `the point $x$ is~$\varepsilon$-close to~$y$'.
With this in mind, the second axiom states every point is arbitrarily close
    to itself.
The third axiom requires that for every entourage~$\varepsilon$
    there is an entourage which acts like~$\nicefrac{\varepsilon}{2}$.
There might be several~$\nicefrac{\varepsilon}{2}$
    which fit the bill.
Whenever we write~$\nicefrac{\varepsilon}{2}$
    we implicitly pick some entourage that
    satisfies~$(\nicefrac{\varepsilon}{2})^2 \subseteq \varepsilon$.
Also we will use the obvious shorthand~$\nicefrac{\varepsilon}{2}
=   \nicefrac{(\nicefrac{\varepsilon}{2})}{2}$.
\end{point}
\end{point}
\begin{point}{Exercise}%
Let~$X$ be a set together with
    a family of relations~$B \subseteq \wp (X \times X)$
    such that it satisfies axioms 2, 3 and 4
    of \sref{dils-dfn-uniformity}.
Write~$\Phi$ for the filter generated by~$B$
    (that is: $\delta \in \Phi$
        iff~$\varepsilon_1 \cap \ldots \cap \varepsilon_n \subseteq \delta$
            for some~$\varepsilon_1, \ldots, \varepsilon_n \in B$).
Show~$(X,\Phi)$ is a uniform space.
We call~$B$ a \Define{subbase} for~$X$.
\end{point}
\begin{point}[dils-uniformity-examples]{Examples}%
Using the previous exercise, it's easy
to describe the entourages of some common uniformities.
    \begin{enumerate}
        \item
    Let~$(X,d)$ be a metric space.
Define~$\hat\varepsilon \equiv \{(x,y);\ d(x,y) \leq \varepsilon\}$
for any $\varepsilon > 0$.
The set~$B \equiv \{ \hat\varepsilon; \ \varepsilon > 0\}$
is a subbase and so fixes a uniformity~$\Phi$ for~$X$.
        \item
    Let~$X$ be a set together with a (infinite)
            family~$(d_{\alpha})_{\alpha\in I}$
        of pseudometrics.
    Define~$E_{\alpha,\varepsilon} = \{ (x,y); \ d_\alpha(x,y)
            \leq \varepsilon\}$.
            Then~$B \equiv \{ E_{\alpha,\varepsilon}; \ \varepsilon > 0, \ 
                    \alpha \in I\}$
                    is a subbase and fixes
                    a uniformity~$\Phi$ on~$X$.
        Similarly a family of seminorms
            fixes a uniformity.
    \item
An important special case of the previous is the following.
Let~$\scrB$ be a von Neumann algebra
    together with the family of pseudometrics
    given by the np-functionals:
        that is~$d_f(x,y) \equiv |f(x-y)|$
            for a np-map ~$f\colon \scrB \to \C$.
The corresponding uniformity is called the \Define{ultraweak uniformity}.
    \end{enumerate}
\end{point}
\end{parsec}

\begin{parsec}%
\begin{point}{Definition}%
Let~$X$ be a uniform space with entourages~$\Phi$.
It is easy to translate the familiar notions for
    metric spaces to uniform spaces.
\begin{enumerate}
    \item A net~$(x_\alpha)_\alpha$ is said to
            \Define{converge} to~$x$
            if for each~$\varepsilon \in \Phi$
            there is an~$\alpha_0$
            such that for all~$\alpha > \alpha_0$
                we have~$x \mathrel\varepsilon x_\alpha$.
Note that a net can converge to two different points.
Indeed, if we start of with a non-trivial pseudometric,
this will be the case.
\item A net~$(x_\alpha)_\alpha$ is called \Define{Cauchy}
        if for each~$\varepsilon \in \Phi$
            there is an~$\alpha_0$
            such that for all~$\alpha,\beta > \alpha_0$
            we have~$x_\alpha \mathrel\varepsilon x_\beta$.
The uniform space $X$ is \Define{complete}
    when every Cauchy net converges.
\item
Let~$(Y,\Psi)$ be another uniform space.
A map~$f\colon X \to X'$
is said to be \Define{uniformly continuous}
if for each~$\varepsilon \in \Psi$
there is a~$\delta \in \Phi$
such that for all~$x \mathrel\delta y$
we have~$f(x) \mathrel\varepsilon f(y)$.
The map~$f$ is merely \Define{continuous}
if for each~$x$ and~$\varepsilon\in\Psi$,
there is a~$\delta \in \Phi$
such that for all~$x \mathrel{\delta} y$
we know~$f(x) \mathrel\varepsilon f(y)$.
\item
We say two Cauchy nets
$(x_\alpha)_{\alpha\in I}$
and$(y_\beta)_{\beta\in J}$ are \Define{equivalent},
in symbols: $(x_\alpha)_\alpha \sim (y_\beta)_\beta$,
when for every~$\varepsilon \in \Phi$
there are~$\alpha_0 \in I$ and~$\beta_0 \in J$
such that for all~$\alpha \leq \alpha_0$ and~$\beta \leq \beta_0$
we have~$x_\alpha \mathrel\varepsilon x_\beta$.

\item
A subset~$D \subseteq X$ is said to be \Define{dense}
    if for each~$\varepsilon \in \Phi$ and~$x \in X$,
    there is a~$y \in D$
    with~$x \mathrel\varepsilon y$.
\end{enumerate}
\end{point}
\begin{point}[dils-uniform-spaces-basics]{Exercise}%
    In the same setting as the previous definition.
    \begin{enumerate}
\item
    Show that equivalence of Cauchy nets is an equivalence relation.
    Show that if~$(x_\alpha)_\alpha$ is a subnet of~$(y_\alpha)_\alpha$,
        that~$(x_\alpha)_\alpha$ is equivalent to~$(y_\alpha)_\alpha$.
\item
    Prove that if~$(x_\alpha)_\alpha$ and~$(y_\alpha)_\alpha$
        are equivalent Cauchy nets and~$x_\alpha \to x$,
        then also~$y_\alpha \to x$.
\item
    Assume~$(x_\alpha)_\alpha$ is a Cauchy net with~$x_\alpha \to x$
        and~$x_\alpha \to y$.  Prove~$x = y$ whenever~$X$ is Hausdorff.
\item\label{ex-continuous-preserves-lims}
    Show that if~$f\colon X \to Y$ is a continuous
    between uniform spaces
        and we have~$x_\alpha \to x$ in~$X$,
        then~$f(x_\alpha) \to f(x)$ in~$Y$.
\item
    Assume~$f$ is a uniformly continuous map between uniform spaces.
    Show that $f$ maps Cauchy nets to Cauchy nets
    and furthermore that~$f$ maps equivalent Cauchy nets to equivalent
        Cauchy nets.
\item\label{ex-cauchy-from-dense-subset}
    Suppose~$D \subseteq X$ is a dense subset.
    Show that for each~$x \in X$
    there is a Cauchy net~$(d_\alpha)_{\alpha \in \Phi}$
    in~$D$ with~$d_\alpha \to x$ in~$X$.
\item
    Assume~$f,g\colon X \to Y$ are continuous maps between uniform spaces
        where~$Y$ is Hausdorff.
Conclude from \ref{ex-continuous-preserves-lims} and
    \ref{ex-cauchy-from-dense-subset}
    that~$f=g$ whenever they agree on a dense subset of~$X$.
    \end{enumerate}
\end{point}
\begin{point}[dils-product-uniformity]{Exercise}%
    Let~$(X_i)_{i \in I}$ be a family of sets with
        uniformities~$(\Phi_i)_{i \in I}$.
    For each~$i_0 \in I$
    and~$\varepsilon \in \Phi_{i_0}$,
    define a relation on
    $\Pi_{i \in I} X_i$ by
    $(x_i)_{i \in I} \mathrel{\hat\varepsilon} (y_i)_{i \in I}
    \iff x_{i_0} \mathrel\varepsilon y_{i_0}$.
    Show~$\{ \hat\varepsilon;\ \varepsilon \in \Phi_i, \ i\in I \}$
    is a subbase for $\Pi_{i \in I} X_i$;
    that the projections~$\pi_i \colon \Pi_{i \in I} X_i \to X_i$
    are uniformly continuous with respect to them
    \emph{and} that they make~$\Pi_{i \in I} X_i$
    into the product of~$(X_i)_{i \in I}$
    in the category of uniform spaces with uniformly continuous maps.
\end{point}
\end{parsec}
\begin{parsec}%
\begin{point}%
It is well known that a Hausdorff uniform space~$X$ can
    be embedded in a complete uniform space~$C$,
    see e.g.~\cite[Thm.~39.12]{willard}.
Call this embedding~$\eta\colon X \to C$.
It comes with a extension theorem:
    every uniformly continuous~$f\colon X \to Y$
    to some complete Hausdorff uniformity~$Y$
    can be uniquely lifted to a continuous
    map~$g\colon C \to X$ such that~$g\after \eta = f$.
The similar extension theorem
    for metric spaces allowed us to extend
    the inner product on a complex vector space
    to its metric completion.
Unfortunately, we could not find a suitable uniformity on~$\scrB$
    which is complete.
Luckily, we do not need full completeness.
\end{point}
\begin{point}{Theorem}%
    Let~$\scrB$ be a von Neumann algebra.
    Assume~$V$ is right-$\scrB$-module
        with $\scrB$-valued inner product~$[\,\cdot\,,\,\,\cdot\,]$.
    The family of seminorms on~$V$ given
        by~$\|x\|_f = f([x,x])^{\frac{1}{2}}$
        for np-maps~$f \colon \scrB \to \C$
        turns~$V$ into a uniform space.
    There is a self-dual Hilbert~$\scrB$-module~$X$
        together with a uniformly continuous
        $\scrB$-linear map~$\eta\colon V \to X$
        such that
        \begin{inparaenum}
        \item
        $[x,y] = \left<\eta(x),\eta(y)\right>$ and
        \item
            the image of~$V$ is dense in~$X$ with respect
            to the uniformity.
        \end{inparaenum}
\begin{point}{Proof}%
Let~$\Phi$ denote the set of entourages
    generated by the seminorms~$\|\cdot\|_f$ on~$V$.
Concretely:~$E \in \Phi$
    if and only if
    there is an~$\varepsilon > 0$, $n \in \N$
    and np-maps~$f_1, \ldots, f_n\colon\scrB \to \C$
    such that~$\| x-y\|_{f_i} \leq \varepsilon $
    for~$1 \leq i \leq n$ implies~$x \mathrel{E} y$.
\begin{point}{Fast and norm-bounded nets}%
    The set of entourages~$\Phi$ is a filter
        and thus can be used as index set for a net
        using reverse inclusion as order.
        Thus~$\varepsilon \geq \delta
        \ \Leftrightarrow\ \varepsilon \subseteq \delta$.
We say a net~$(x_\alpha)_{\alpha \in \Phi}$  indexed by entourages
    is \Define{fast} if for every~$\varepsilon \in \Phi$
        and~$\alpha,\beta \geq \varepsilon$
        we have~$x_\alpha \mathrel{\varepsilon^2} x_\beta$.
A net~$(x_\alpha)_{\alpha \in I}$
    is called \Define{norm-bounded}
    whenever~$(\|x_\alpha\|)_{\alpha}$ is bounded
    with~$\|x\| \equiv \| [x,x]\|^{\frac{1}{2}}$.
We will construct~$X$ using equivalence classes of
norm-bounded fast Cauchy nets in~$V$.
\end{point}
\begin{point}{Equivalent fast nets}%
Every Cauchy net is equivalent to a fast one, but this is not as
        easy as in the metric case.
Let~$(x_\alpha)_{\alpha \in I}$
    be an arbitrary Cauchy net in~$V$.
By definition we can find~$\alpha_\varepsilon \in I$
    such that for each~$\varepsilon \in \Phi$
    and~$\alpha,\beta \geq \alpha_\varepsilon$
    we have~$x_\alpha \mathrel{\varepsilon} x_\beta$.
Unfortunately~$\alpha_{(\,\cdot\,)}$ need not be order preserving
    and so~$(x_{\alpha_{\varepsilon}})_{\varepsilon\in\Phi}$
    need not be a subnet of~$(x_\alpha)_{\alpha \in I}$.
However, we claim the net~$(x_{\alpha_{\varepsilon}})_{\varepsilon\in\Phi}$
    is Cauchy, fast and equivalent to~$(x_{\alpha})_{\alpha\in I}$.
To show it's Cauchy and fast, assume~$\varepsilon \in \Phi$
    and~$\zeta,\xi \geq \varepsilon$
    (that is: $\zeta,\xi \subseteq \varepsilon$).
As~$I$ is a net, we can find~$\beta\in I$
    with~$\beta \geq \alpha_\zeta, \alpha_\xi$.
By definition of~$\alpha_{(\,\cdot\,)}$
    we have~$x_{\alpha_\zeta} \mathrel\zeta x_{\beta}
        \mathrel\xi x_{\alpha_\xi}$
        and so~$x_{\alpha_\zeta} \mathrel{\varepsilon^2}
                x_{\alpha_\xi}$, as desired.
To show equivalence, assume~$\varepsilon \in \Phi$ is given.
Assume~$\delta \geq \nicefrac{\varepsilon}{2}$
and~$\beta \geq \alpha_{\nicefrac{\varepsilon}{2}}$.
There is a~$\gamma \in I$ with~$\gamma \geq \alpha_\delta,\beta$.
Then~$x_\beta \mathrel{\nicefrac{\varepsilon}{2}}
x_{\gamma} \mathrel{\delta} x_{\alpha_{\delta}}$
and so as~$\delta\subseteq \nicefrac{\varepsilon}{2}$
we get~$x_\beta \mathrel\varepsilon x_{\alpha_\delta}$.

Clearly if~$(x_{\alpha})_{\alpha \in I}$
    is norm bounded,
    then so is~$(x_{\alpha_{\varepsilon}})_{\varepsilon \in \Phi}$.
If fast Cauchy nets~$(x_\alpha)_{\alpha \in \Phi}$
    and~$(y_\alpha)_{\alpha \in \Phi}$ are equivalent,
    we can find for every~$\varepsilon \in \Phi$
    some~$\beta \in \Phi$
    such that for all~$\gamma \geq \beta$
    we have~$x_\gamma \mathrel{\varepsilon} y_\gamma$.
\end{point}
\begin{point}{The uniform space~$N$}%
Write~$N$ for the set of norm-bounded fast Cauchy nets over~$V$.
Later we will define~$X$ as~$N$ modulo equivalence.
Because of a subtlety with the definition of the uniformity on~$X$ later,
    it is helpful to consider~$N$ separately.
Let~$\varepsilon \in \Phi$.
For nets~$(x_\alpha)_{\alpha}$
    and~$(y_\alpha)_{\alpha}$ in~$N$,
    define
    \begin{equation*}
        (x_\alpha)_\alpha \mathrel{\Define{\hat\varepsilon}}
            (y_\alpha)_\alpha
            \quad\Leftrightarrow\quad
        \exists \beta \in \Phi \, \forall \gamma \geq \beta. \ 
        x_\gamma \mathrel{\varepsilon} y_\gamma.
    \end{equation*}
If~$\varepsilon \subseteq \delta$,
then $\hat\varepsilon \subseteq \hat\delta$
and
$\hat\varepsilon_1 \after \hat\varepsilon_2 
\subseteq\widehat{\varepsilon_1 \after \varepsilon_2}$.
So~$
\widehat{\nicefrac{\varepsilon}{2}} \after
\widehat{\nicefrac{\varepsilon}{2}} \subseteq
\widehat{\nicefrac{\varepsilon}{2} \after
\nicefrac{\varepsilon}{2}} \subseteq \widehat{\varepsilon}$,
which is one requirement for~$\{ \hat\varepsilon; \ \varepsilon \in \Phi\}$
to be a subbase for~$N$.
The others are easy as well. 
Also~$\widehat{\varepsilon_1 \cap \varepsilon_2} = \hat{\varepsilon}_1
    \cap \hat{\varepsilon}_2$
    and so each entourage of~$N$ has some~$\hat\varepsilon$ as subset.
Note~$(x_\alpha)_\alpha$ and~$(y_\alpha)_\alpha$
are equivalent if and only if~$(x_\alpha)_\alpha \mathrel{\hat\varepsilon}
    (y_\alpha)_\alpha$ for all~$\varepsilon \in \Phi$.
\end{point}
\begin{point}{$N$ is uniformly-bounded complete}%
The uniform space~$N$ might not be complete:
    it is possible there are
    Cauchy nets of norm-bounded Cauchy nets
    which only tend to a Cauchy net which is not norm bounded.
Also for Cauchy nets~$((x^\gamma_\alpha)_\alpha)_\gamma$ in~$N$ where
$(\limsup_\alpha \|x^\gamma_\alpha\|)_\gamma$ is bounded
        it is not clear whether they tend
        to a norm-bounded Cauchy.
For this proof,
call a Cauchy net~$((x^\gamma_\alpha)_\alpha)_\gamma$ in~$N$
    \Define{uniformly bounded}
    if~$(x^\gamma_\alpha)_{\gamma,\alpha}$ is norm bounded in~$V$.
We will show uniformly-bounded Cauchy nets in~$N$ converge.
As every uniformly-bounded Cauchy net is equivalent to a fast one,
it is sufficient to show convergence of uniformly-bounded fast Cauchy nets.
Thus let~$((x^\gamma_\alpha)_\alpha)_\gamma$
    be a uniformly-bounded fast Cauchy net in~$N$.
First we will show~$(x^{\hat\alpha}_\alpha)_\alpha$
    is a uniformly-bounded Cauchy net.
It might not be fast, so
    formally~$((x^\gamma_\alpha)_\alpha)_\gamma$
    cannot converge to it,
    but it will converge to an equivalent fast Cauchy net.
As~$((x^\gamma_\alpha)_\alpha)_\gamma$
    is fast Cauchy,
    we know that for every~$\gamma_1,\gamma_2 \geq \hat\varepsilon$
we have~$
(x^{\gamma_1}_\alpha)_\alpha \mathrel{\hat\varepsilon^2}
        (x^{\gamma_2}_\alpha)_\alpha$.
So there is a~$\zeta_{\gamma_1,\gamma_2} \in \Phi$
such that for~$\alpha \geq \zeta_{\gamma_1,\gamma_2}$
    we have~$x^{\gamma_1}_\alpha
        \mathrel{\varepsilon^2}
        x^{\gamma_2}_\alpha$.
Assume~$\alpha,\beta \geq \varepsilon$.
Pick~$\xi \geq \alpha,\beta,\zeta_{\hat\alpha,\hat\beta}$.
Then we have $ x^{\hat{\alpha}}_\alpha
            \mathrel{\varepsilon^2}
        x^{\hat{\alpha}}_\xi
            \mathrel{\varepsilon^2}
        x^{\hat{\beta}}_\xi
            \mathrel{\varepsilon^2}
            x^{\hat{\beta}}_\beta$ and so~$(x_\alpha^{\hat\alpha})_\alpha$
            is Cauchy.
Clearly it's norm bounded.
Let~$(y_\alpha)_\alpha$ be a norm-bounded fast Cauchy net equivalent
to~$(x^{\hat\alpha}_\alpha)_\alpha$.
We want to prove~$(x^\gamma_\alpha)_\alpha \to (y_\alpha)_\alpha$.
As~$(y_\alpha)_\alpha$ is equivalent to~$(x^{\hat\alpha}_\alpha)_\alpha$
    we can find~$\alpha_0$
    such that~$y_\alpha \mathrel{\nicefrac{\varepsilon}{4}} x^{\hat\alpha}_\alpha$
    whenever~$\alpha \geq \alpha_0$.
As~$((x^\gamma_\alpha)_\alpha)_\gamma$
is fast Cauchy,
we know that~$(x^{\hat\alpha}_\beta)_\beta
\mathrel{\widehat{\nicefrac{\varepsilon}{4}}}
(x^\gamma_\beta)_\beta $
if~$\hat{\alpha}, \gamma \geq \widehat{\nicefrac{\varepsilon}{8}}$.
Thus there is some~$\beta_0$ such
that~$x^{\hat\alpha}_\beta
\mathrel{\nicefrac{\varepsilon}{4}}
x^\gamma_\beta $
for~$\beta \geq \beta_0$
and~$\hat{\alpha}, \gamma \geq \widehat{\nicefrac{\varepsilon}{8}}$.
Thus for~$\beta \geq \beta_0 \cap \nicefrac{\varepsilon}{8}$,
$\alpha \geq \alpha_0 \cap \nicefrac{\varepsilon}{8}$
and~$\gamma \geq \widehat{\nicefrac{\varepsilon}{8}}$
we get~$y_\alpha
\mathrel{\nicefrac{\varepsilon}{4}}
x^{\hat{\alpha}}_\alpha
\mathrel{\nicefrac{\varepsilon}{4}}
x^{\hat{\alpha}}_\beta
\mathrel{\nicefrac{\varepsilon}{4}}
x^\gamma_\beta
\mathrel{\nicefrac{\varepsilon}{4}}
x^\gamma_\alpha$.
Thus~$(y_\alpha)_\alpha \mathrel{\hat\varepsilon} (x^\gamma_\alpha)_\alpha$
whenever~$\gamma \geq \widehat{\nicefrac{\varepsilon}{8}}$,
so~$(x^\gamma_\alpha)_\alpha \to (y_\alpha)_\alpha$.
\end{point}
\begin{point}{The uniform space~$X$}%
Let~$X$ be the set of norm-bounded fast Cauchy nets modulo equivalence.
Unfortunately the relations~$\hat\varepsilon$ do not necessarily preserve
    equivalence of Cauchy nets:
    there might be~$(x_\alpha)_\alpha \sim (x'_\alpha)_\alpha$
    and~$(y_\alpha)_\alpha \sim (y'_\alpha)_\alpha$ in~$N$
such that~$(x_\alpha)_\alpha \mathrel{\hat\varepsilon} (y_\alpha)_\alpha$,
but not~$(x'_\alpha)_\alpha \mathrel{\hat\varepsilon} (y'_\alpha)_\alpha$
for some~$\varepsilon \in \Phi$.
Instead define for~$\varepsilon \in \Phi$:
\begin{equation*}
    (x_\alpha)_\alpha \mathrel{\Define{\tilde\varepsilon}}
    (y_\alpha)_\alpha \quad\Leftrightarrow\quad
    (x'_\alpha)_\alpha \mathrel{\hat\varepsilon}
    (y'_\alpha)_\alpha \quad
    \text{for all
        $(x'_\alpha)_\alpha \sim (x_\alpha)_\alpha$ 
    and $(y'_\alpha)_\alpha \sim (y_\alpha)_\alpha$.}
\end{equation*}
By definition~$\tilde\varepsilon$
    respects equivalence and so is also a relation on~$X$.
It is not hard to verify~$\{ \tilde\varepsilon ;\ \varepsilon \in \Phi\}$
is a subbase for~$X$
and~$\widetilde{\varepsilon_1 \cap \varepsilon_2} = \tilde{\varepsilon}_1
\cap \tilde{\varepsilon}_2$.
By construction~$X$ is a Hausdorff uniform space.
Write~$\tilde\Phi$ for the generated set of entourages on~$X$.
The entourages of~$X$ are exactly those relations
    which have some~$\tilde\varepsilon$
    as a subset.
Furthermore, if~$(x_\alpha)_\alpha \mathrel{\hat\varepsilon}
(y_\alpha)_\alpha $,
then~$(x_\alpha)_\alpha \mathrel{\tilde\varepsilon^3} (y_\alpha)_\alpha$.
\end{point}
\begin{point}{$\eta$ has dense range}%
Let~$\eta\colon V \to X$
be the map that sends~$x \in V$
to the equivalence class of the constant Cauchy net~$(x)_{\alpha \in \Phi}$.
To show the image of~$\eta$ is dense,
    assume~$(x_\alpha)_\alpha$
    is a representative for an element in~$X$
    and~$\varepsilon \in \tilde\Phi$.
There is some~$\delta \in \Phi$ with~$\tilde\delta \subseteq \varepsilon$.
As~$(x_\alpha)_\alpha$ is fast,
we have~$x_{\nicefrac{\delta}{8}} \mathrel{\nicefrac{\delta}{4}} x_\alpha$
for all~$\alpha \geq \nicefrac{\delta}{8}$.
Thus~$(x_{\nicefrac{\delta}{8}})_\alpha \mathrel{\widehat{\nicefrac{\delta}{4}}}
    (x_\alpha)_\alpha$
    and so~$(x_{\nicefrac{\delta}{8}})_\alpha \mathrel{\widetilde{\nicefrac{\delta}{4}}^3}
    (x_\alpha)_\alpha$ hence~$\eta(x_{\nicefrac{\delta}{8}}) \mathrel{\varepsilon}
    (x_\alpha)_\alpha$, as desired.
\end{point}
\begin{point}{$X$ is a right-$\scrB$-module}%
    First we will define an addition on~$X$.
    The addition on~$V$ is uniformly continuous
        as the uniformity on~$V$ is given by seminorms.
    (See \sref{dils-product-uniformity} for the uniformity on~$V^2$.)
    Thus for any~$(x_\alpha)_\alpha$
    and~$(y_\alpha)_\alpha$ in~$N$,
    the net~$(x_\alpha+y_\alpha)_\alpha$ is again Cauchy,
        see \sref{dils-uniform-spaces-basics}.
    Also because of the uniform continuity of addition:
    if~$(x_\alpha)_\alpha \sim (x'_\alpha)_\alpha$
    and~$(y_\alpha)_\alpha \sim (y'_\alpha)_\alpha$,
    then~$(x_\alpha+y_\alpha)_\alpha \sim
            (x'_\alpha+y'_\alpha)_\alpha$.
    Clearly~$(x_\alpha+y_\alpha)_\alpha$
        is uniformly bounded and thus has an element in~$N$ equivalent to it.
    This fixes an addition on~$X$,
        which turns it into an abelian group.
        By construction~$\eta(x+y) = \eta(x)+\eta(y)$.

Assume~$b \in \scrB$. We show the map~$r_b\colon V \to V$
given by~$r_b(x) = xb$ is also uniformly continuous,
which requires us to unfold the definitions further
than with addition.
Assume we are given an entourage in~$V$,
that is: np-maps~$f_1,\ldots,f_n\colon \scrB \to \C$
and~$\varepsilon > 0$.
For any np-map~$f\colon \scrB \to \C$,
    the map~$b*f$ given by~$(b * f)(x) \equiv f(b^* x b)$ is also an np-map.
    \TODO{}. 
    Clearly~$\|xb\|_f = f([xb,xb]) = f(b^* [x,x] b) = \|x\|_{b*f}$
    for any~$x \in V$.
Thus if~$\|x-y\|_{b*f_i} \leq \varepsilon$ for~$1 \leq i \leq n$,
then~$\|xb-yb\|_{f_i} = \|x-y\|_{b*f_i} \leq \varepsilon$
as well. Hence~$r_b$ is uniformly continuous.
As before, this uniform continuity together
with the fact~$r_b$ sends uniformly-bounded Cauchy nets
to uniformly-bounded Cauchy nets allows us
to define a right~$\scrB$-action on~$X$
by sending the equivalence class of~$(x_\alpha)_\alpha$
to~$(x_\alpha b)_\alpha$. By definition~$\eta(x)b = \eta(xb)$.
It is straight-forward to check this turns~$X$ into a right-$\scrB$-module.
\begin{point}{$X$ is a Hilbert~$\scrB$-module}%
For an np-map~$f\colon \scrB \to \C$
write~$[x,y]_f \equiv f([x,y])$.
As~$f$ is positive, this defines an inner product on~$V$.
Assume~$(x_\alpha)_\alpha$ and~$(y_\alpha)_\alpha$
    are~$\Phi$-indexed norm-bounded Cauchy nets in~$V$.
For any np-map $f\colon \scrB \to \C$ we have
    by Cauchy--Schwarz
\begin{align*}\label{eq-dils-V-inprod-cauchy}
    \bigl|f([x_\alpha, y_\alpha] - [x_\beta,y_\beta])\bigr|
    & \ =\  \bigl| [x_\alpha, y_\alpha - y_\beta]_f
        + [x_\alpha - x_\beta, y_\beta]_f\bigr| \\
    &\  \leq\  \|x_\alpha\|_f \|y_\alpha - y_\beta\|_f \numberthis
        + \|x_\alpha - x_\beta\|_f \| y_\beta \|_f.
\end{align*}
As~$(x_\alpha)_\alpha$ is norm bounded and
$\|x_\alpha\|_f^2 = f([x_\alpha,x_\alpha]) \leq \|f\| \|x_\alpha\|^2$,
we see~$(\|x_\alpha\|_f)_\alpha$ is bounded.
Similarly~$(\|y_\beta\|_f)_\beta$ is bounded.
Thus from \eqref{eq-dils-V-inprod-cauchy}
it follows~$([x_\alpha,y_\alpha])_\alpha$
    is a Cauchy net in the ultraweak uniformity of~$\scrB$,
    see \sref{dils-uniformity-examples}.
As~$\|[x_\alpha,y_\alpha]\| \leq \|x_\alpha\| \|y_\alpha\|$,
    see \TODO{}, it is also a norm-bounded net.
By \TODO{} norm-bounded Cauchy net in the ultraweak uniformity
    converge, so we may define
\begin{equation*}
    \bigl<(x_\alpha)_\alpha, (y_\beta)_\beta\bigr>
    \ \equiv\  \uwlim_{\alpha} \,[x_\alpha,y_\alpha].
\end{equation*}
Assume we are given~$\Phi$-indexed
norm-bounded Cauchy-nets~$(x')_\alpha$ and~$(y')_\alpha$
with~$ (x'_\alpha)_\alpha \sim
(x_\alpha)_\alpha$ and~$ (y'_\alpha)_\alpha \sim
(y_\alpha)_\alpha$.
Then, like \eqref{eq-dils-V-inprod-cauchy}:
\begin{align*}
    \bigl|f([x_\alpha, y_\alpha] - [x'_\alpha,y'_\alpha])\bigr|
    & \ =\  \bigl| [x_\alpha, y_\alpha - y'_\alpha]_f
        + [x_\alpha - x'_\alpha, y'_\alpha]_f\bigr| \\
    &\  \leq\  \|x_\alpha\|_f \|y_\alpha - y'_\alpha\|_f
        + \|x_\alpha - x'_\alpha\|_f \| y'_\alpha \|_f.
\end{align*}
From this it follows
    $\left<(x_\alpha)_\alpha, (y_\beta)_\beta\right> =
    \langle(x'_\alpha)_\alpha, (y'_\beta)_\beta\rangle$.
Thus~$\left<\,\cdot\,,\,\cdot\,\right>$
    is defined on~$X$.

Assume~$(x_\alpha)_\alpha$,
    $(y_\alpha)_\alpha$ and
    $(z_\alpha)_\alpha$ are representatives of elements of~$X$.
    Let~$(t_\alpha)_\alpha$
    be an element of~$N$ equivalent to
    $(y_\alpha + z_\alpha)_\alpha$,
    that is: a representative of the sum of
    $(y_\alpha)_\alpha$ and $(z_\alpha)_\alpha$ in~$X$.
    Clearly
    $[x_\alpha,y_\alpha+ z_\alpha] =
    [x_\alpha, y_\alpha] + [x_\alpha, z_\alpha]$
    and so by the ultraweak continuity of addition, see \TODO{},
    we have
    \begin{equation*}
    \uwlim_\alpha [x_\alpha,t_\alpha]
    =\uwlim_\alpha [x_\alpha, y_\alpha] + [x_\alpha, z_\alpha]
    =\uwlim_\alpha [x_\alpha, y_\alpha] + \uwlim_\alpha [x_\alpha, z_\alpha].
    \end{equation*}
    Thus~$\left<\,\cdot\,,\,\cdot\,\right>$
        is linear in the right argument.
    In a similar fashion one proves the other axioms
        for a~$\scrB$-valued inner product:
        $\left<\tau,\sigma\right>=\left<\sigma,\tau\right>^*$
        follows from the ultraweak continuity of~$(\,\cdot\,)^*$, see \TODO{};
       $\scrB$-linearity in the right argument
       follows from the ultraweak continuity
       of~$c \mapsto cb$ in~$\scrB$,  see \TODO{} \emph{and}
       $\left<\tau,\tau\right>\geq 0$
       follows from ultraweak-closedness of the positive cone of~$\scrB$.

    Assume~$\left<\tau,\tau\right> = 0$ for some~$\tau \in X$.
    Let~$(x_\alpha)_\alpha$ in~$N$ be a representative of~$\tau$.
    Then~$\uwlim_\alpha [x_\alpha,x_\alpha] = 0$.
    Thus for each~np-map~$f\colon \scrB \to \C$ and~$\varepsilon$
        we have~$f([x_\alpha,x_\alpha]) \leq \varepsilon$
        for sufficiently large~$\alpha$.
\end{point}
\end{point}
\begin{point}{Lifting maps along~$\eta$}%
    First we note the map~$\eta$ is uniformly continuous.
    Indeed: if~$\varepsilon \in \Phi$,
    then~$x \mathrel{\nicefrac{\varepsilon}{4}} y$
    implies~$\eta(x) \mathrel{\tilde{\varepsilon}} \eta(y)$,
    which is sufficient to show uniform continuity.
\end{point}
\end{point}
\end{point}
\end{parsec}

\end{document}

% vim: ft=tex.latex
