\documentclass[b]{subfiles}
\begin{document}
\chapter{Dilations}

\begin{parsec}[dils-intro]%
\begin{point}%
In this chapter we will study various dilation theorems.
The common theme is that a complicated map
    is actually the composition of a simpler map
    after a representation into a larger algebra.
We already saw a dilation theorem in disguise:
    using the Gel'fand--Naimark--Segal construction (see \sref{gns})
    one shows every state is a vector state on a larger algebra:
\end{point}
\begin{point}{Exercise (GNS')}\label{dils-gns}%
    Show that
    for each pu-map~$\omega\colon \scrA \to \C$
        from a~C$^*$-algebra~$\scrA$
    there is a Hilbert space~$\scrH$,
    a miu-map~$\varrho\colon \scrA \to \scrB(\scrH)$
    and vector~$x \in \scrH$
    such that~$\omega = h \after \varrho$
    where~$h \colon \scrB(\scrH) \to \C$
    is given by~$h(T) = \left<Tx,x\right>$.
\end{point}

\begin{point}%
Probably the most famous dilation theorem is that
of Stinespring\cite[Thm.~1]{stinespring}.
\end{point}

\begin{point}{Theorem (Stinespring)}\label{stinespring-theorem}
    For every cp-map~$\varphi\colon \scrA \to \scrB (\scrH)$
    there is a Hilbert space~$\scrK$,
        miu-map~$\varrho\colon \scrA \to \scrB(\scrK)$
        and bounded operator~$V\colon \scrH \to \scrK$
        such that~$\varphi = \ad_V \after \varrho$,
        where~$\Define{\ad_V} \colon \scrB(\scrK) \to \scrB(\scrH)$
        is given by~$\ad_V(T):= V^*TV$.

Furthermore:
\begin{inparaenum}
\item
    if~$\varphi$ is normal, then~$\varrho$ is also normal \emph{and}
\item
    if~$\varphi$ is unital, then~$V$ is an isometry
        (or equivalently~$\ad_V$ is unital).
\end{inparaenum}
\end{point}

\begin{point}%
(We will see a detailed proof later in \TODO{}.)
Stinespring's theorem
is fundamental in the study
of quantum information and quantum computing:
it is used to prove entropy inequalities (e.g.~\cite{lindblad}),
bounds on optimal cloners (e.g.~\cite{werner}),
full completeness of quantum programming languages (e.g.~\cite{staton}),
security of quantum key distribution (e.g.~\cite{werner2}),
analyze quantum alternation (e.g.~\cite{prakash}),
to categorify quantum processes (e.g.~\cite{selinger}) \emph{and}
as an axiom to single out
quantum theory among information processing theories.\cite{chiribella}
A fair overview of all uses of Stinespring's theorem and its consequences
is a formidable task out of scope of this text.

Stinespring's theorem only applies
to maps of the form~$\scrA \to \scrB(\scrH)$
    and so do most of its usefull consequences.
One wonders:
    is there an extension of Stinespring's theorem
    to arbitrary np-maps~$\scrA \to \scrB$?
A different and less common question might be:
    is Stinespring's dilation categorical in some way.
That is: does it have a defining universal property?
Both questions turn out to be true:
Paschke's generalization of GNS for Hilbert C$^*$-modules\cite{paschke}
    turns out to have the same universal property
        as Stinespring's dilation and so it extends Stinespring
        to arbitrary np-maps.

We start of this chapter with a detailed proof of Stinespring's theorem.
We continue to show it obeys a universal property.
Before we move on to Paschke's GNS
    we need to develop Paschke's theory of self-dual Hilbert C$^*$-modules.
Paschke's work builds on Sakai's characterization of von Neumann algebras,
    which would take considerable effort to develop in detail.
Thus to be as complete as possible
we avoid Sakai's characterization (in contrast to our
    article \cite{wwpaschke})
    and give new proofs
    of Paschke's results where required.
One major difference is that we will use
    a restricted completion (see \TODO{}) of a uniformity
    instead of considering the dual space of a Hilbert C$^*$-module.
We finish the first part of this chapter
    by constructing Paschke's dilation
    and establishing it indeed extends Stinespring's dilation.
In the second part of this chaper
    we prove several results about Paschke's dilations.
\TODO{more intro}
\end{point}
\end{parsec}

\begin{parsec}[dils-completion-to-hilb]%
\begin{point}%
For the GNS-construction
it was necessary to ``complete''\footnote{Note that
        we do not require an inner product to be definite.
    The inner product on a Hilbert space \emph{is} definite.
    Thus the completion will quotient out those vectors with
        zero norm with respect to the inner product.}
    a complex vector space with inner product into a Hilbert space.
This completion was only sketched in \sref{completion-inner-product-space}.
Here we will work through the details
    as the corresponding completion required
    in Paschke's dilation
    is more complex and its exposition will benefit
    from this familiar analogon.
\end{point}

\begin{point}{Proposition}\label{prop-complete-into-hilbert-space}%
    Let~$V$ be a complex vector space with inner
        product~$[\,\cdot\,,\,\cdot\,]$.
    There is a Hilbert space~$\scrH$
        together with bounded linear map~$\eta\colon V \to \scrH$
            such that
        \begin{inparaenum}
        \item
        $[v,w] = \left<\eta(v), \eta(w)\right>$
            for all~$v,w \in V$ and
        \item
        the image of~$\eta$ is dense in~$\scrH$.
        \end{inparaenum}
\begin{point}{Proof}%
We will form~$\scrH$ from the set of Cauchy sequences in~$V$
    with a little twist.
Recall two Cauchy
    sequences~$(v_n)_n$ and~$(w_n)_n$ in~$V$
    are said to be equivalent
    if for every~$\varepsilon > 0$
    there is a~$n_0$
    such that~$\| v_n - w_n \| \leq \varepsilon$
    for all~$n \geq n_0$,
    where $\|v\| \equiv \sqrt{[v,v]}$.
Call a Cauchy sequence~$(v_n)_n$ \Define{fast}
    if for each~$n_0$
    we have~$\| v_n - v_m\| \leq \frac{1}{2^{n_0}}$
    for all~$n,m \geq n_0$.
Clearly every Cauchy sequence has a fast subsequence
    which is (as are all subsequences) equivalent with it.
\begin{point}%
    Define~$\scrH$ to be the set of fast Cauchy sequences modulo
        equivalence.
For brevity we will denote an element of~$\scrH$,
    which is an equivalence class of Cauchy sequences, simply by
    a single representative.
Also, we often tersely write~$v$ for the Cauchy sequence~$(v_n)_n$.
The set~$\scrH$ is a metric space with the standard
    distance~$d(v, w) \equiv \lim_{n\to\infty} \| v_n - w_n\|$.
To show it's complete, assume
$v^1, v^2, \ldots$
is a fast Cauchy sequence of fast Cauchy sequences.
Assume~$n,m,k \geq N$.  We have
\begin{equation*}
    \| v^n_n - v^m_m \|
        \leq
    \| v^n_n - v^n_k \| +
    \| v^n_k - v^m_k \| +
    \| v^m_k - v^m_m \| \leq
    \| v^n_k - v^m_k \|+ \frac{2}{2^N}.
\end{equation*}
Because~$\lim_{k\to\infty} \|v_k^n-v_k^m \| =d(v^n,v^m) \leq 2^{-N}$
    we can find a~$k \geq N$
    such that~$\| v^n_k - v^m_k \| \leq \frac{2}{2^N}$
    and so~$\|v^n_n - v^m_m\| \leq \frac{4}{2^N}$.
    Thus~$(v^n_n)_n$ is a Cauchy sequence.
It's easily checked~$v^1, v^2, \ldots$
converges to~$(v^n_n)_n$ with respect to~$d$.
The sequence~$(v^n_n)_n$ might not be fast
    and so, might not be in~$\scrH$,
    but like any other Cauchy sequence,
    it has a fast subsequence.
The equivalence class
of this subsequence is the limit of~$v^1, v^2, \ldots$ in~$\scrH$.
We have shown~$\scrH$ is complete.
\end{point}
\begin{point}\label{prop-hilbert-space-completion-extension}%
Define~$\eta\colon V \to \scrH$
    to be the map which sends~$v$ to the equivalence class of the
    constant sequence~$(v)_n$.
By construction of~$\eta$ and~$\scrH$,
    the image of~$\eta$ is dense in~$\scrH$.
Let~$f\colon V \to X$
    be any uniformly continuous map to a complete metric space~$X$.
We will show there is a unique continuous~$g\colon \scrH \to X$
    such that~$g \after \eta = f$.
From the uniform continuity it easily follows
    that~$f$ maps Cauchy sequences to Cauchy sequences
    and preserves equivalence between them.
Together with the completeness of ~$X$
    there is a unique~$g\colon \scrH \to X$
    fixed by~$g((v_n)_n) = \lim_{n\to\infty}f(v_n)$.
Clearly~$g \after \eta = f$.
Finally, $g$ is unique as it's fixed on the image of~$\eta$, which is dense.
\end{point}
\begin{point}\label{completion-to-hilb-vect}%
Thus we directly get a scalar multiplication
    on~$\scrH$ by extending~$\eta \after r_z \colon V \to \scrH$
    where~$r_z(v) = zv$ is scalar multiplication by~$z \in C$,
    which is uniformly continuous.
In fact~$z (v_n)_n \equiv (z v_n)_n$.
Extending addition is less direct, but straight-forward:
    given representatives~$v,w \in \scrH$
    we know~$(v_n+w_n)_n$ is Cauchy in~$V$
    by uniform continuity of addition and so~$v+w \equiv (v_n+w_n)_n$
        fixes an addition on~$\scrH$.
With this expression for addition and the similar one for
    scalar multiplication it is easy to see
    they turn~$\scrH$ into a complex vector space with zero~$\eta(0)$
    for which~$\eta$ is linear.
Extending the inner product is bit trickier.
\end{point}
\begin{point}%
To define the inner product on~$\scrH$,
first note that for Cauchy sequences~$v$ and~$w$ in~$V$
we have (using Cauchy--Schwarz, \sref{chilb-cs}, on the second line):
\begin{align*}
    \bigl|[v_n,w_n] - [v_m,w_m]\bigr|
    & \ =\  \bigl|[v_n,w_n-w_m] + [v_n - v_m,w_m]\bigr| \\
    & \ \leq\  \|v_n\| \|w_n - w_m\| + \|v_n-v_m\|\|w_m\|.
\end{align*}
Thus as~$(\|v_n\|)_n$ and~$(\|w_n\|)_n$ are bounded
we see~$([v_n,w_n])_n$ is Cauchy.
In a similar fashion we see~$\bigl|[v_n,w_n] - [v'_n,w'_n]\bigr|
    \leq \|v_n\| \|w_n - w_n'\| + \|v_n-v_n'\|\|w'_n\|\to 0$ as~$n\to \infty$
    for~$v'$ and~$w'$ Cauchy sequences equivalent
to~$v$ respectively~$w$.
Thus
$([v_n,w_n])_n$ is equivalent to
$([v'_n,w'_n])_n$
    and so
    we define~$\left<v,w\right>
        \equiv \lim_{n\to \infty} [v_n,w_n]$ on~$\scrH$.
With vector space structure from~\sref{completion-to-hilb-vect}\TODO{fix ref}
we easily see this is an inner product~$\scrH$.
The metric induced by the inner product coincides with~$d$:
\begin{equation*}
    \left<v-w,v-w\right>
   =\lim_{n}[v_n-w_n,v_n-w_n]
    =\lim_{n}\|v_n-w_n\|^2
    = d(v,w)^2.
\end{equation*}
And so~$\scrH$ is a Hilbert space.
Finally, $\left<\eta(v),\eta(w)\right>=[v,w]$ is direct.\qed
\end{point}
\end{point}
\end{point}
\end{parsec}

\begin{parsec}[dils-stinespring]%
\begin{point}%
We are ready to prove Stinespring's dilation theorem.
\begin{point}{Proof of Stinespring's theorem, see \sref{stinespring-theorem}}%
Let~$\varphi\colon \scrA \to \scrB(\scrH)$
    be a cp-map.
Write~$\scrA \odot \scrH$ for the tensor product of~$\scrA$ and~$\scrH$
    as vector spaces.
By linear extension,
the following fixes a sesquilinear form on~$\scrA \odot \scrH$:
\begin{equation*}
    [a\otimes x, b \otimes y] = \left<x, \varphi(a^*b)y\right>_{\scrH}.
\end{equation*}
As~$\varphi$ preserves involution as a positive map,\TODO{ref}
this is also a symmetric form, i.e.: $[t,s]=\overline{[s,t]}$.
    Assume~$\sum^n_{i=1} a_i\otimes x_i$
is an arbitrary element of~$A \odot \scrH$.
To see $[\,\cdot\,,\,\cdot\,]$ is positive,
we need to show
\begin{equation*}
    0 \leq \bigl[\sum_i a_i\otimes x_i, \sum_j a_j\otimes x_j\bigr]
        = \sum_{i,j} [a_i\otimes x_i, a_j\otimes x_j]
        = \sum_{i,j} \left< x_i, \varphi(a_i^*a_j) x_j \right>.
\end{equation*}
The matrix algebra~$M_n\scrB(\scrH)$
acts on~$\scrH^{\oplus n}$
as~$(A (x_1,\ldots,x_n)^\T)_i = \sum_j A_{ij} x_j$.
(In fact, this gives an
    isomorphism~$M_n\scrB(\scrH) \cong \scrB(\scrH^{\oplus n})$.)
Writing~$x$ for the vector~$(x_1,\ldots,x_n)^\T$ in~$\scrH^{\oplus n}$,
    we get
\begin{equation}\label{eq-stinespring-norm-tensor}
    \sum_{i,j} \left< x_i, \varphi(a_i^*a_j) x_j \right>
    = \bigl<
    x,
    (M_n\varphi) \bigl(\, (a_i^*a_j)_{ij} \, \bigr)\,
    x \bigr>_{\scrH^{\oplus n}}.
\end{equation}
The matrix~$(a_i^*a_j)_{ij}$
is positive as~$(C^*C)_{ij} = a_i^*a_j$
    with~$C_{ij} \equiv \frac{1}{\sqrt{n}} a_j$.
    By complete positivity~$M_n\varphi(\,(a_i^*a_j)_{ij}\,)$ is positive
    and so is~\eqref{eq-stinespring-norm-tensor},
    hence~$[\,\cdot\,,\,\cdot\,]$ is an inner product.
Write~$\eta\colon A\odot \scrH \to \scrK$ for the Hilbert space completion
    described in~\sref{prop-complete-into-hilbert-space}.
\begin{point}\label{stinespring-extend-operator}%
Let~$T\colon A \odot \scrH \to A \odot \scrH$
be a bouned linear map.
We show there is an extension~$\hat{T} \colon \scrK \to \scrK$.
The operator $T$
uniformly continuous and so is~$\eta\after T \colon A\odot\scrH \to\scrK$.
Thus by \sref{prop-hilbert-space-completion-extension}
there is a unique continuous extension~$\hat{T} \colon \scrK \to \scrK$
with~$\hat{T}(\eta(t)) = \eta(T(t))$
    for all~$t \in \scrA\odot\scrH$.
Clearly~$\hat{T}$ is linear on the image of~$\eta$,
    which is dense and so~$\hat{T}$ is linear everwhere.
Hence~$\hat{T}$ is bounded, so~$\hat{T} \in \scrB(\scrK)$.
It is easy to see~$\widehat{T+S}=\hat{T}+\hat{S}$ and
    $\widehat{\lambda T} = \lambda \hat{T}$
    for operators~$S,T$ on~$\scrA \odot \scrH$ and~$\lambda \in \C$.
Also~$\widehat{TS} = \hat{T}\hat{S}$:
indeed~$\hat{T}\hat{S} \eta(t)
            = \hat{T} \eta(St)
            = \eta(TSt)$
    and so by uniqueness~$\hat{T}\hat{S} = \widehat{TS}$.
\end{point}
\begin{point}%
    Assume~$b\in \scrA$.
    Let~$\varrho_0(b)$ be the operator on~$\scrA \otimes \scrH$
    fixed by~$\varrho_0(b) a\otimes x = (b a) \otimes x$.
Clearly $\varrho_0$ is linear, unital and multiplicative.
We want to show~$\varrho_0(b)$ is bounded for fixed~$b \in \scrA$.
To show this, we claim that in~$M_n\scrB(\scrH)$
we have~$(a_i^*b^*ba_j)_{ij} \leq \|b\|^2 (a_i^*a_j)_{ij}$.
Indeed, as~$b^*b \leq \|b\|^2$,
    there is some~$c$ with~$c^*c = \|b\|^2 - b^*b$.
Define~$C_{ij} \equiv \frac{1}{\sqrt{n}} ca_j$.
We compute:
$ (C^*C)_{ij} = a_i^*c^*ca_j = a_i^* (\|b\|^2 - b^*b) a_j$
and get the claimed inequality with which
\begin{equation*}
    \begin{split}
    \bigl\| \sum_i (ba_i) \otimes x_i \bigr\|^2
    & = \bigl< x, (M_n\varphi)(\, (a_i^* b^*b a_j)_{ij}\,)x\bigr> \\
    &\leq \|b\|^2 \bigl< x, (M_n\varphi)(\, (a_i^* a_j)_{ij}\,)x\bigr>
    = \|b\|^2 \bigl\| \sum_i a_i \otimes x_i \bigr\|^2
    \end{split}
\end{equation*}
and so~$\varrho_0(b)$ is bounded.
Now define~$\varrho(b) \colon \scrA \to \scrB(\scrH)$
    by~$\varrho(b) \equiv \widehat{\varrho_0(b)}$.
\end{point}
\begin{point}%
We already know~$\varrho$ is a~mu-map.
We want to show it preserves involution: $\varrho(c^*) = \varrho(c)^*$
for all~$c \in \scrA$.
Indeed, for every~$a,b \in \scrA$ and~$x,y \in \scrH$
    we have
    \begin{equation*}
        [\varrho_0(c^*) \, a\otimes x,b \otimes y]
        = [(c^* a)\otimes x,b \otimes y]
        = \left<x, \varphi(a^*cb) y \right>
        = [a\otimes x,\varrho_0(c)\, b \otimes y].
    \end{equation*}
Hence~$
    \left<\varrho(c^*) \eta(a\otimes x), \eta(b\otimes y)\right>=
    \left< \eta(a\otimes x), \varrho(c)\eta(b\otimes y)\right>$.
    As the linear span of~$\eta(a\otimes x)$ is dense in~$\scrK$,
        we find~$\varrho(c^*) = \varrho(c)^*$, as desired.
\end{point}
\begin{point}%
    Let~$V_0 \colon \scrH \to \scrA \odot \scrH$
        be given by~$V_0 x = 1 \otimes x$.
        It's bounded: $\| V_0 x\| = \left<x, \varphi(1^*1) x\right>^{\frac{1}{2}}
        = \|\sqrt{\varphi(1)} x\| \leq \|\sqrt{\varphi(1)}\| \|x\|$.
    Define~$V \equiv \eta \after V_0$.
    For all~$a \in \scrA$ and~$x,y \in \scrH$ we have
            $[a \otimes x, V_0 y]
            = \left<x, \varphi(a^*)y\right>
            = \left<x, \varphi(a)^*y\right>
            = \left<\varphi(a) x, y\right>$.
    Thus~$V^*$ satisfies~$V^* \eta(a \otimes x) = \varphi(a)x$.
    Hence for all~$a \in \scrA$ and~$x \in \scrH$:
    $V^* \varrho(a) V x = V^* \varrho(a) \eta(1 \otimes x)
        = V^* \eta(a\otimes x)
        = \varphi(a)x$.
        So~$\ad_V \after \varrho = \varphi$.

    If~$\varphi$ is unital,
    then~$V^*Vx = V^* (1\otimes x) = \varphi(1)x=x$
        for all~$x \in \scrH$ and so~$V$ is an isometry.
    (Or equivalently~$\ad_V$ is unital.)
\end{point}
\begin{point}%
    It remains to be shown that~$\varrho$
        is normal when~$\varphi$ is.
    Assume~$D \subseteq \scrA$ is a bounded directed
    set of self adjoint elements.
    We need to show~$\sup_{d \in D} \varrho (d) = \varrho(\sup D)$.
    Let~$t$ be in the image of~$\eta$.
    It is sufficient to show~$\left<t, \varrho(\sup D) t\right>
        = \left<t, \sup_{d \in D} \varrho(d)t\right>$.
    We have~$t = \sum_{i=1}^n a_i \otimes x_i$
        for some~$n \in \N$, $a_1, \ldots, a_n \in \scrA$
            and~$x_1,\ldots, x_n \in \scrH$.
    Write~$x$ for the vector~$(x_1, \ldots, x_n)^\T$
        in~$\scrH^{\oplus n}$.
    Then
    \begin{align*}
        \left<t, \varrho(\sup D) t\right>
        &= \bigl< x, (M_n\varphi)\bigl(\, ( a_i^* (\sup D) a_j)_{ij} \,\bigr) x \bigr>_{\scrH^{\oplus n}} \\
        &= \sup_{d\in D} \bigl< x, (M_n\varphi)\bigl(\, ( a_i^* d a_j)_{ij} \,\bigr) x \bigr>_{\scrH^{\oplus n}}\\
        & = \sup_{d\in D} \left<t, \varrho(d)t\right>
            = \bigl<t, \sup_{d \in D} \varrho(d) t\bigr>.
    \end{align*}
    Here we used that~$\left<x, (\,\cdot\,)x\right>$,
        $M_n\varphi$ and 
    $d \mapsto (a_i^*da_j)_{ij}$ are normal.
    \TODO{}
    At the moment it will be a bit tricky to show~$(a_i^*da_j)_{ij}$
    is normal.  Later (\TODO{}) we will have a simpler argument.
    Define~$C_{ij} = \frac{1}{\sqrt{n}} a_j$
        and~$\Delta\colon \scrA \to M_n \scrA$
        by~$\Delta(d)_{ij} = \delta_{ij}$.\TODO{kronecker}.
    Then~$(C^*\Delta(d)C)_{ij} = a_i^* d a_j$.
    Thus by~\TODO{} it is sufficient to show~$\Delta$ is normal.
    Without loss of generality, we may assume~$\scrA$
        is an algebra of operators on a Hilbert spaces~$\scrL$.
    By~\TODO{} 
        $(d)_{d \in D}$ converges ultrastrongly to~$\sup D$.
    In particular it converges in the WOT.
    Similarly
    $(\Delta(d))_{d \in D}$
        converges in the WOT
        to~$\sup_{d \in D} \Delta(d)$.
        Let~$x \equiv (x_1, \ldots, x_n)^\T$ be a vector in~$\scrL^{\oplus n}$.
        Then~$
        (\left<x,  \Delta(d)x\right>)_{d \in D}
        = (\sum_i \left<x_i,  dx_i \right>)_{d \in D}$
        converges to
        $ \sum_i \left<x_i,  (\sup D )x_i \right>
            = \left<x,  \Delta(\sup D )x \right>$
    and so~$(\Delta(d))_{d \in D}$
    also converges in the WOT to $\Delta(\sup D)$.
    Hence~$\Delta(\sup D) = \sup_{d \in D} \Delta(D)$
        and so~$\Delta$ is normal.\qed
\end{point}
\end{point}
\begin{point}%
Stinespring is a generalization of GNS.
It seems tempting to directly prove Stinespring instead of GNS ---
however, we used GNS in the proof of Stinespring to show
that~$\varrho$ is normal, when~$\varphi$ is.
\end{point}
\end{point}
\end{parsec}

\end{document}

% vim: ft=tex.latex
