\documentclass[main]{subfiles}
\begin{document}
\chapter{$C^*$-algebras}
As a warming-up
we develop the essentials of the theory of $C^*$-algebras
in this chapter.
Since we are ultimately interested
in von Neumann algebras in this text
we have evaded
delicate
topics such as tensor products (of $C^*$-algebras), 
quotients, approximate identities,
and $C^*$-algebras without a unit.
Our end goal
for this chapter
is Gelfand's representation theorem,
the fact that every commutative $C^*$-algebra
is isomorphic
to the $C^*$-algebra
$C(X)$ of continuous functions on some compact Hausdorff space~$X$.
We begin easy with the definition and elementary properties
of self-adjoint, positive and invertible elements of a $C^*$-algebra,
and we introduce the spectrum of an element in~\sref{}.
Without breaking a sweat,
we obtain an important technical result
called spectral permanence in~\sref{}
with algebraic manipulations and simple analysis.
However, to reach Gelfand's representation theorem,
we will take three hurdles:
\begin{enumerate}
\item
first we employ complex analysis
to connect the invertible elements of a $C^*$-algebra
with the norm;
\item
then we connect multiplication
and positive elements via the construction of the square root
of positive elements;
\item
and finally we connect the multiplicative states
on a commutative $C^*$-algebra
with the spectrum of its elements
using `maximal Riesz ideals'.
\end{enumerate}
To make this chapter
as accessible as possible,
we have made every effort to identify
and remove
extranious material
from the ordinary development
of the theory
such as the more general theory of Banach algebras
and Banach spaces (and its pathology)
To accommodate these simplifications
we were forced
to take a slightly different path than in the literature,
for while we obtain Gelfand--Mazur's theorem for $C^*$-algebras,
but cannot apply it, see~\sref{}.
\begin{parsec}%
\begin{point}{Definition}%
A \Define{$C^*$-algebra}
is a complex vector space~$\scrA$
endowed with
\begin{enumerate}
\item
a binary operation,
called \Define{multiplication}
(and denoted as such),
which is associative, and linear in both coordinates;
\item
an element~$1$, called \Define{unit},
such that $1\cdot a = a = a\cdot 1$
for all~$a\in \scrA$;
\item
a unary operation $(\,\cdot\,)^*$,
called \Define{involution},
such that $(a^*)^*=a$,
$(ab)^*=b^*a^*$,
$(\lambda a)^* = \bar\lambda a^*$,
and $(a+b)^* = a^*+b^*$
for all~$a,b\in\scrA$ and~$\lambda\in \C$;
\item
a complete \Define{norm} $\|\,\cdot\,\|$
such that
$\|ab\|\leq\|a\|\|b\|$
for all~$a,b\in\scrA$,
and 
\begin{equation*}
\label{eq:Cstar-identity}
\|a^*a\|\ =\ \|a\|^2
\end{equation*}
holds. The latter equality is called the \Define{$C^*$-identity}.
\end{enumerate}
The $C^*$-algebra $\scrA$ is called \Define{commutative}
if $ab=ba$ for all~$a,b\in\scrA$.
\end{point}
\begin{point}{Example}%
The vector space~$\C$ of \Define{complex numbers}
forms a commutative  $C^*$-algebra
in which
multiplication and~$1$
have their usual meaning.
Involution is given by conjugation ($z^*=\bar{z}$),
and norm by modulus ($\|z\|=|z|$).
\end{point}
\begin{point}{Example}%
Let~$X$ be a compact Hausdorff space.
The set $\Define{C(X)}$ of \Define{continuous functions}
from~$\scrA$ to~$\C$
forms a commutative $C^*$-algebra
when addition, (scalar) multiplication, involution and~$1$ are
interpretted coordinatewise \grayed{(e.g.~$(f+g)(x)=f(x)+g(x)$)},
and the norm is taken to be 
$\|f\|=\sup_{x\in X} |f(x)|$
(the \Define{sup-norm}).

\TODO{refer to an appendix on topology.}
\end{point}
\begin{point}{Example}%
A \Define{$C^*$-subalgebra}
of a $C^*$-algebra~$\scrA$
is a subset~$\scrB$ of~$\scrA$,
which is a linear subspace of~$\scrA$,
contains the unit, $1$, is closed under multiplication
and involution, 
and is closed with respect to the norm of~$\scrA$;
such a $C^*$-subalgebra of~$\scrA$
is itself a $C^*$-algebra
when endowed with the operations and norm
of~$\scrA$.
\end{point}
\begin{point}[example-hilb]{Example}%
The vector space~$\scrB(\scrH)$ of bounded operators
on a Hilbert space~$\scrH$ forms a $C^*$-algebra
when endowed with the operator 
norm.
Multiplication is given by composition,
involution by taking the adjoint,
and unit by the identity operator.
A \Define{concrete $C^*$-algebra} or
a \Define{$C^*$-algebra of bounded operators} 
refers to a $C^*$-subalgebra of~$\scrB(\scrH)$.
We elaborate on this example in~\sref{hilb},
and we will see
that every $C^*$-algebra is isomorphic to a $C^*$-algebra
of bounded operators in~\sref{gelfand-naimark}.

\TODO{make parsec/appendix on bounded operators between normed vector
spaces and the operator norm}
\end{point}
\begin{point}{Example}%
Let~$\scrA_i$ be a $C^*$-algebra
for every element~$i$ of some set~$I$.
The \Define{direct product}
of the family $(\scrA_i)_i$
is the $C^*$-algebra
denoted by $\bigoplus_{i\in I}\scrA_i$ on the set
\begin{equation*}
\textstyle
\{\  a\in \prod_{i\in I}\scrA_i\colon\  \sup_{i \in I} \|a(i)\|< \infty \ \}
\end{equation*}
whose operations are defined coordinatewise,
and whose norm is a \Define{sup-norm} given by $\|a\|=\sup_{i}\|a(i)\|$.
\end{point}
\end{parsec}
\TODO{$\|a^*\|=\|a\|$}

\TODO{$\|a^*\|=\|a\|$ and $\|1\|\leq 1$}

\begin{parsec}[hilb]%
\begin{point}%
Let us show in detail
that the bounded operators on a Hilbert space
form a $C^*$-algebra
as described in~\sref{example-hilb}.
Let us recall the definition of ``bounded operators''.
\end{point}
\begin{point}{Definition}%
Let~$\scrX$ and~$\scrY$ be normed
vector spaces.
We say that~$r\in [0,\infty)$
is a \Define{bound} for a linear map (=\Define{operator}) 
$T\colon \scrX\to\scrY$
when  $\|Tx\|\leq r\|x\|$ for all~$x\in \scrX$,
and we say that~$T$ is \Define{bounded}
when there is such a bound.
In that case~$T$ has a least bound,
which is called the \Define{operator norm} of~$T$,
and is denoted by~$\Define{\|T\|}$.
The vector space of bounded operators
from~$\scrX$ to~$\scrY$
is denoted by~$\scrB(\scrX,\scrY)$,
and the vector space of bounded operators
from~$\scrX$ to itself is denoted by~$\scrB(\scrX)$.
\end{point}
\begin{point}[bounded-operators-basic]{Exercise}%
Let~$\scrX$, $\scrY$ and~$\scrZ$ be normed complex vector spaces.
\begin{enumerate}
\item
Show that the operator norm on~$\scrB(\scrX,\scrY)$
is, indeed, a norm.
\item
Let~$T\colon \scrX\to \scrY$ and~$S\colon \scrY\to\scrZ$
be bounded operators.
Show that $ST$ is bounded by~$\|S\|\|T\|$,
so that~$\|ST\|\leq\|S\|\|T\|$.
\item
Show that~$\scrB(\scrX,\scrY)$ 
is complete when~$\scrY$ is complete.
\item
Show that the identity operator $\id\colon \scrX\to \scrX$
is bounded by~$1$.
\end{enumerate}
\end{point}
\begin{point}%
From~\sref{bounded-operators-basic}
it is clear that the complex vector space
of bounded operators~$\scrB(\scrX)$
on a complete normed vector space~$\scrX$
with composition as multiplication
and the identity operator as unit
satisfies all the requirements
to be $C^*$-algebra that do not involve the involution, $(\,\cdot\,)^*$
(that is, $\scrB(\scrX)$ is a \Define{Banach algebra}).
To get an involution,
we need the additional structure
provided by a Hilbert space:
\end{point}
\begin{point}{Definition}%
An \Define{inner product}
on a complex vector space~$V$ 
is a map $\left<\,\cdot\,,\,\cdot\,\right>\colon V\times V\to \C$
such that,
for all~$x,y\in V$,
$\left<x,\,\cdot\,\right>\colon V\to V$ is linear;
$\left<x,x\right>\geq 0$;
and
$\left<x,y\right>=\overline{\left<y,x\right>}$.
We say that the inner product is \Define{definite}
when~$\left<x,x\right>=0\implies x=0$ for~$x\in V$.
A \Define{pre-Hilbert space}~$\scrH$
is a complex vector space endowed with a definite inner product.
We'll shortly see that every such~$\scrH$
carries a norm
given by
 $\|x\|:= \left<x,x\right>^{\nicefrac{1}{2}}$;
if~$\scrH$ is complete with respect to this norm,
we say that~$\scrH$ is a \Define{Hilbert space}.

Let~$\scrH$ and~$\scrK$ be pre-Hilbert spaces.
We say that a bounded operator~$T\colon \scrH\to \scrK$
is \Define{adjoint}
to a bounded operator
$S\colon \scrK\to \scrH$ 
when
\begin{equation*}
\left<Tx,y\right> \ = \ \left<x,Sy\right>
\qquad\text{for all $x\in \scrH$ and $y\in \scrK$.}
\end{equation*}
In that case, we call~$T$ \Define{adjointable}.
It is easy to that such adjointable~$T$ is adjoint to at most one~$S$,
which we denote by~\Define{$T^*$}.
\end{point}
\begin{point}{Exercise}%
Let~$S$ and~$T$ be adjointable operators on a pre-Hilbert space.
\begin{enumerate}
\item
Show that~$T^*$ is adjoint to~$T$ (and so $T^{**}=T$).
\item
Show that~$(T+S)^*=T^*+S^*$
and $(\lambda S)^*=\overline{\lambda}S^*$
for every~$\lambda\in C$.
\item
Show that~$ST$ is adjoint to $T^*S^*$ (and so $(ST)^*=T^*S^*$).
\end{enumerate}
(We will show, of course,
that every bounded operator on a Hilbert space is adjointable.)
\end{point}
\begin{point}[positive-2x2matrix]{Lemma}%
For a positive matrix $A\equiv 
\left(\begin{smallmatrix}p & \overline{c} \\ c & q\end{smallmatrix}\right)$
(i.e.~$\left(
\begin{smallmatrix}\overline{u}&\overline{v}\end{smallmatrix}\right)
A
\left(\begin{smallmatrix}u \\ v \end{smallmatrix}\right) \,\geq \, 0$
for all~$u,v\in \C$),
we have
$p,q\geq 0$, and $\left|c\right|^2 \leq pq$.
\begin{point}{Proof}%
Let~$u,v\in\C$ be given.
We have
\begin{equation*}
0\ \leq\ 
\left(\begin{smallmatrix}\overline{u}&\overline{v}\end{smallmatrix}\right)
A
\left(\begin{smallmatrix}u \\ v \end{smallmatrix}\right)
\ = \ 
\left|u\right|^2 p\,+\, 
\overline{u}v\,\overline{c} \,+\,
u\overline{v}\,c \,+\,
\left|v\right|^2 q.
\end{equation*}
By taking~$u=1$ and $v=0$, we see that~$p\geq 0$,
and similarly $q\geq 0$.

The trick to see that~$\left|c\right|^2\leq pq$
is to
take~$v=1$ and $u=t\overline{c}$ with~$t\in \R$:
\begin{equation*}
0 \ \leq\ p\left|c\right|^2t^2
\,+\,2\left|c\right|^2t 
\,+\, q.
\end{equation*}
If~$p=0$, then~$-2\left|c\right|^2t \leq q $
for all~$t\in \R$,
which implies that~$\left|c\right|^2=0=pq$.

Suppose that~$p>0$.
Then taking~$t=-p^{-1}$ we see that
\begin{equation*}
0 \ \leq\ \left|c\right|^2p^{-1}
\,-\,2\left|c\right|^2p^{-1} 
\,+\, q \ = \ -\left|c\right|^2p^{-1}\,+\,q.
\end{equation*}
Rewriting gives us
 $\left|c\right|^2\leq pq$.\qed
\end{point}
\end{point}
\begin{point}[inner-product-basic]{Exercise}%
Let~$\left<\,\cdot\,,\,\cdot\,\right>$
be an inner product on a vector space~$V$.
Show that
the formula~$\Define{\|x\|}=\smash{\sqrt{\left<x,x\right>}}$
defines a seminorm on~$V$,
that is,
$\|x\|\geq 0$,
$\|\lambda x\|=\left|\lambda\right|\|x\|$,
and---the \Define{triangle inequality}---$\|x+y\|\leq \|x\|+\|y\|$
for all~$\lambda\in \C$ and~$x,y\in V$.

Moreover, prove that~$\|\,\cdot\,\|$
is a norm when~$\left<\,\cdot\,,\,\cdot\,\right>$
is definite,
and for~$x,y\in V$:
\begin{enumerate}
\item
The \Define{Cauchy--Schwarz inequality}:
$\left|\left<x,y\right>\right|^2\,\leq\, \left<x,x\right>
\,\left<y,y\right>$;
\item
\Define{Pythagoras' theorem}:
$\|x\|^2+\|y\|^2\,=\,\|x+y\|^2$ when~$\left<x,y\right>=0$;
\item
The \Define{parallelogram law}:
$\|x\|^2\,+\,
\|y\|^2
\,= \,
\frac{1}{2}(\,\|x+y\|^2\,+\,\|x-y\|^2\,)$;
\item
The \Define{polarization identity}:
$\left<x,y\right> \,=\, \frac{1}{4}\sum_{n=0}^3i^n\|i^nx+y\|^2$.
\end{enumerate}

(Hint: prove the Cauchy--Schwarz inequality
before the triangle inequality
by applying~\sref{positive-2x2matrix} to the matrix
$\smash{\bigl(\begin{smallmatrix}
\smash{\left<x,x\right>} & \smash{\left<x,y\right>} \\
\smash{\left<y,x\right>} & \smash{\left<y,y\right>}
\end{smallmatrix}\bigr)}$.
Then prove $\|x+y\|^2\leq (\|x\|+\|y\|)^2$
using the inequalities~$\left<x,y\right>+\left<y,x\right>
\leq 2\left|\left<x,y\right>\right| \leq 2\|x\|\|y\|$.)
\end{point}
\begin{point}{Lemma}%
For an adjointable operator~$T$ on a pre-Hilbert space~$\scrH$
\begin{equation*}
\|T^*T\|\ =\ \|T\|^2\qquad\text{and}\qquad\|T^*\|\ =\ \|T\|.
\end{equation*}
\begin{point}{Proof}%
If~$T=0$, then~$T^*=0$, and the statements are surely true.

Suppose~$T\neq 0$ (and so~$T^*\neq 0$).
Since $\|Tx\|^2=\left<Tx,Tx\right>=\left<x,T^*Tx\right>
\leq \|x\|\,\|T^*Tx\|\leq \|x\|^2\|T^*T\|$
for every~$x\in \scrH$
by Cauchy--Schwarz,
we have $\|T\|^2\leq \|T^*T\|$.
Since~$\|T^*T\|\leq \|T^*\|\|T\|$
and $\|T\|\neq 0$,
it follows that~$\|T^*\|\leq \|T\|$.
Since by a similar reasoning $\|T\|\leq \|T^*\|$,
we get~$\|T\|=\|T^*\|$.
But then $\|T\|^2\leq \|T^*T\|\leq \|T^*\|\|T\|=\|T\|^2$,
and so $\|T\|^2=\|T^*T\|$.\qed
\end{point}
\end{point}
\begin{point}%
At this point
it is clear that the vector space of adjointable operators
on a Hilbert space forms a $C^*$-algebra.
So it remains to be shown that every bounded operator
is adjointable.
\end{point}
\begin{point}{Definition}
Let~$x$ be an element of a Hilbert space~$\scrH$.
We say that an element~$y$ of a linear subspace~$C$
of~$\scrH$ is a \Define{projection of~$x$ on~$C$}
if
\begin{equation*}
\|x-y\|\,=\,\min\{\,\|x-y'\|\colon \,y'\in C\,\}.
\end{equation*}
\end{point}
\begin{point}{Lemma}%
Let~$\scrH$ be a pre-Hilbert space,
and let $x,e\in\scrH$ with
$\|e\|=1$.

Then~$y=\left<e,x\right>e$ is the unique projection of~$x$ on~$e\C$.
\begin{point}{Proof}%
Let~$y'\in e\C$
with~$y'\neq y$
be given.
To prove that~$y$
is the unique projection of~$x$ on $e\C$
it suffices to show that $\|x-y\|<\|x-y'\|$.
Since~$y'\neq y\equiv \left<e,x\right>e$,
there is~$\lambda\in \C$, $\lambda\neq 0$ 
with $y'=(\lambda+\left<e,x\right>)e$.

Note that $\left<e,y\right>=\left<e,\left<e,x\right>e\right>=
\left<e,x\right>\left<e,e\right>
= \left<e,x\right>$,
and so~$\left<e,x-y\right>=0$.
Then~$y'-y\equiv \lambda e$ and~$x-y$ are orthogonal too,
and thus, by Pythagoras'~theorem (see~\sref{inner-product-basic}),
we have $\|y'-x\|^2
=\|y'-y\|^2+\|y-x\|^2\equiv \left|\lambda\right|^2+\|x-y\|^2
>\|x-y\|^2$, because~$\lambda\neq 0$.
Hence~$\|y'-x\|>\|y-x\|$.\qed
\end{point}
\end{point}
\begin{point}[hilb-projection-basic]{Exercise}%
Let~$y$ be a projection of an element~$x$ of a Hilbert space~$\scrH$
on a linear subspace~$C$.
Show that~$y$ is a projection of~$x$ on $y\C$.
Conclude that~$y$ is the unique projection of~$x$ on~$C$,
and that~$\left<y,x-y\right>=0$.
Show that~$y+c$ is the projection of~$x+c$ on~$C$
for every~$c\in C$.
Conclude that~$\left<y',x-y\right>\equiv\left<y',(x+y'-y)-y'\right>=0$ 
for every~$y'\in C$.
\end{point}
\begin{point}[projection-theorem]{Projection Theorem}%
Let~$C$ be a closed linear subspace
of a Hilbert space~$\scrH$.
Each~$x\in \scrH$
has a unique projection~$y$ on~$C$,
and $\left<y',y\right>=\left<y',x\right>$ for~$y'\in C$.
\begin{point}{Proof}%
We only need to show that there is a projection~$y$
of~$x$ on~$C$,
because~\sref{hilb-projection-basic}
gives us that such~$y$ is unique and satisfies
$\left<y',y\right> = \left<y',x\right>$ for all~$y'\in C$.

Write~$r:=\inf\{\,\|x-y'\|\colon\, y'\in C\,\}$,
and pick a sequence $y_1,y_2,\dotsc \in C$
such that $\|x-y_n\|\rightarrow r$.
We will show that~$y_1,y_2,\dotsc$ is Cauchy.
Let~$\varepsilon >0$
be given,
and pick~$N$ such that $\|y_n-x\|^2\leq r^2+\frac{1}{4}\varepsilon$
for all~$n\geq N$.
Let~$n,m\geq N$ be given.
Then since $\frac{1}{2}(y_n+y_m)$
is in~$C$, we have
$\frac{1}{2}\|y_n+y_m-2x\|\equiv 
\|\frac{1}{2}(y_n+y_m)-x\|\geq r$,
and so by the parallelogram law (see \sref{inner-product-basic}),
\begin{alignat*}{3}
\|y_n-y_m\|^2
\ &\equiv\ 
\|(y_n-x)-(y_m-x)\|^2\\
\ &=\ 
2\|y_n-x\|^2 + 2\|y_m-x\|^2
- \|y_n+y_m-x\|^2\\
\ &\leq\ 
4r^2 + \varepsilon - 4r^2 \ \leq \ \varepsilon.
\end{alignat*}
Hence~$y_1,y_2,\dotsc$ is Cauchy,
and converges to some~$y\in C$,
because~$\scrH$ is complete and~$C$ is closed.
It follows easily that~$\|x-y\|=r$,
and thus~$y$ is the projection of~$x$ on~$C$.\qed
\end{point}
\end{point}
\begin{point}{Riesz Representation Theorem}%
Let~$\scrH$ be a Hilbert space.
For every bounded linear map~$f\colon \scrH\to\C$
there is a unique vector~$x\in \scrH$
with $\left<x,\,\cdot\,\right>=f$.
\begin{point}%
If~$f=0$, then $y=0$ does the job.
Suppose that~$f\neq 0$.
There is~$x'\in\scrH$ with~$f(x')\neq 0$.
Note that~$\ker(f)$ is closed, because~$f$
is bounded.
So by~\sref{projection-theorem},
we know that~$x'$
has a projection~$y$ on~$\ker(f)$,
and $\left<x',z\right>=\left<y,z\right>$
for all~$z\in \ker(f)$.
Then for~$x'':=f(x'-y)^{-1}(x'-y)$,
we have $f(x'')=1$ and~$\left<x'',y'\right>=0$
for all~$y'\in \ker(f)$.

Let~$z\in \scrH$. 
Then $f(\,z-f(z)x''\,)=0$,
so~$z-f(z)x''\in \ker(f)$,
and thus~$0=\left<x'',z-f(z)x''\right>\equiv \left<x'',z\right>-f(z)\|x''\|^2$.
Hence with $x:=x''\|x''\|^{-2}$
we have~$f(z)=\left<x''\|x''\|^{-2},z\right>$
for all~$z\in \scrH$.

Concerning uniqueness, let $x_1,x_2\in \scrH$
with $\left<x_1,z\right>=\left<x_2,z\right>$
for all~$z\in \scrH$ be given.
Taking $z=x_1-x_2$,
we see that $\|x_1-x_2\|^2=\left<x_1-x_2,x_1-x_2\right>=0$,
and so~$x_1=x_2$.\qed
\end{point}
\end{point}
\begin{point}{Exercise}%
Prove that every bounded operator~$T$ on a Hilbert space~$\scrH$
is adjointable, as follows.
Let~$x\in \scrH$ be given.
Prove that~$\left<x,T(\,\cdot\,)\right>\colon \scrH\to \C$
is a bounded linear map.
Let~$Sx$ be the unique vector with $\left<Sx,\,\cdot\,\right>
=\left<x,T(\,\cdot\,)\right>$.
Show that~$x\mapsto Sx$
gives a bounded linear map $S$, which is adjoint to~$T$.
\end{point}
\end{parsec}


\begin{parsec}%
\begin{point}{Definition}%
Given an element $a$ of a $C^*$-algebra $\scrA$, 
\begin{enumerate}
\item we say that $a$ is \Define{self-adjoint} if $a^* =a$, and
\item we write $\Define{\Real{a}}:= \frac{1}{2}(a+a^*)$
and $\Define{\Imag{a}}:=\frac{1}{2i}(a-a^*)$.
\end{enumerate}
The set of self-adjoint elements of~$\scrA$
is denoted by~\Define{$\sa{\scrA}$}.
\end{point}
\begin{point}{Exercise}%
Let~$a$ be an element of a $C^*$-algebra.
\begin{enumerate}
\item 
Show that $\Real{a}$ and $\Imag{a}$ are self-adjoint,
and  $a= \Real{a}+i\Imag{a}$.
\item
Show that $\Real{(a^*)}=\Real{a}$ and $\Imag{(a^*)}=-\Imag{a}$.
\item 
Show that~$a$ is self-adjoint iff $\Real{a}=a$ iff $\Imag{a}=0$.
\item
Show that $a^*a$ is self-adjoint,
and  $a^*a=\Real{a}^2+\Imag{a}^2+i(\Real{a}\Imag{a}-\Imag{a}\Real{a})$.
\item
Give an example of an~$a$ with  $\Real{a}\Imag{a} \neq \Imag{a}\Real{a}$.
\item
Show that $a^*a+aa^* = 2(\Real{a}^2+\Imag{a}^2)$.
\item
The product of self-adjoint elements $b$, $c$ needs not be self-adjoint;
show that, in fact, $bc$ is self-adjoint iff $bc=cb$.
\end{enumerate}
\end{point}
\end{parsec}
\begin{parsec}%
\begin{point}{Notation}%
Recall that (in this text) every $C^*$-algebra~$\scrA$ has a unit, $1$.
Thus, for every scalar $\lambda\in \C$,
we have an element $\lambda\cdot 1$ of~$\scrA$,
which we will simply denote by~$\lambda$.
This should hardly cause any confusion,
for while an expression of an element of~$\scrA$
such as $i+2+5a$ (where $a\in \scrA$) 
may be interpretted in several ways,
the result is always the same.
\end{point}
\begin{point}{Exercise}%
There is, however, one subtle point regarding
the interpretation
of the norm~$\|\lambda\|$ of a
scalar~$\lambda\in \C$ inside a $C^*$-algebra~$\scrA$.
\begin{enumerate}
\item 
Show that $\|\lambda\|\leq \left| \lambda\right|$ (in~$\C$).
\item
Show that $\|1\|=0\neq 1$ when~$\scrA=\{0\}$ is the trivial $C^*$-algebra.
\item
Show that~$\|\lambda\|=\left|\lambda\right|$
when~$\|\lambda\|$ and~$\left|\lambda\right|$
are interpretted as elements of~$\scrA$.
\end{enumerate}
\end{point}
\end{parsec}
\begin{parsec}%
\begin{point}%
Let us now generalize the notion of a positive function
in~$C(X)$
to a positive element of a $C^*$-algebra.
There are several descriptions of
positive functions in~$C(X)$ in terms of the $C^*$-algebra structure
(see~\sref{cx-positive}) on which we can base such a  generalization,
and while we will eventually see that these all yield the same notion
of positive element of a $C^*$-algebra (see~\sref{cstar-positive-final})
we base our definition of positive element (\sref{cstar-positive})
on the description that is perhaps
not most familiar,
but does give us the richest structure at this stage.
\end{point}
\begin{point}[cx-positive]{Exercise}%
Let~$X$ be a compact Hausdorff space.
Show that for self-adjoint $f\in C(X)$, the following are equivalent.
\begin{enumerate}
\item \label{cx-positive-1}
$f(X)\subseteq [0,\infty)$;
\item
$f\equiv g^2$ for some $g\in \sa{C(X)}$;
\item
$f\equiv g^* g$ for some~$g\in C(X)$;
\item
$\|f-t\|\leq t$ for some $t\geq \frac{1}{2}\|f\|$.

\TODO{add picture?}
\end{enumerate}
\begin{point}{Exercise}%
To see how condition~\ref{cx-positive-1}
can be expressed in terms of the $C^*$-algebra structure of~$C(X)$,
prove that  $\lambda\in f(X)$ iff $f-\lambda$
is not invertible.
\end{point}
\end{point}
\begin{point}[cstar-positive]{Definition}%
A self-adjoint element~$a$ of a $C^*$-algebra~$\scrA$ is called
\Define{positive} if $\|a-t\|\leq t$
for some~$t\geq \frac{1}{2}\|a\|$.
We write $a\leq b$ for $a,b\in\scrA$ when $b-a$ is positive,
and we denote the set of positive elements of~$\scrA$
by~$\pos{\scrA}$.
\end{point}
\begin{point}{Lemma}%
Let~$a,b$ be positive elements of a $C^*$-algebra.
Then $a+b$ is positive.
\begin{point}{Proof}
Since~$t\geq 0$,
there is~$t\geq \frac{1}{2}\|a\|$ with $\|a-t\|\leq t$.
Similarly, there is~$s\geq \frac{1}{2}\|b\|$
with $\|b-s\|\leq s$.
Then $\|a+b-(t+s)\|\leq \|a-t\|+\|b-s\|\leq t+s$
and $t+s\geq \frac{1}{2}(\|a\|+\|b\|) \geq \frac{1}{2}\|a+b\|$,
so~$a+b\geq 0$.\qed
\end{point}
\end{point}
\begin{point}{Exercise}%
Let~$\scrA$ be a $C^*$-algebra.
\begin{enumerate}
\item
Show that~$\pos{\scrA}$ is a \emph{cone}:
$0\in \pos{\scrA}$,
$a+b\in \pos{\scrA}$ for all $a,b\in\pos{\scrA}$,
and
$\lambda a\in \pos{\scrA}$  
for all $a\in \pos{\scrA}$ and $\lambda\in [0,\infty)$.
Conclude that~$\leq$ is a partial order on~$\scrA$,
and that~$\sa{\scrA}$ is an ordered vector space.
\item
Show that~$1$ is positive, and  $-\|a\|\leq a \leq \|a\|$
for every self-adjoint element~$a$ of~$\scrA$.
(Thus $1$ is an \emph{order unit} of~$\sa{\scrA}$.)
\item
The behaviour of positive elements may be surprising:
give an example of positive elements $a$ and~$b$
from a $C^*$-algebra
such that $ab$ is not positive.
\item
Given a self-adjoint element~$a$ of~$\scrA$ define
\begin{equation*}
\|a\|_o \ = \ \inf\{\ \lambda\in[0,\infty)\colon \ 
-\lambda\leq a\leq \lambda\ \}.
\end{equation*}
Show that $\|-\|_o$ is a seminorm on~$\sa{\scrA}$,
and that~$\|a\|_o\leq \|a\|$
for all~$a\in\sa{\scrA}$.

Prove that $0\leq a\leq b$ implies that~$\|a\|_o\leq\|b\|_o$
for $a,b\in\sa{\scrA}$.

\item
There is not much more that can easily be
proven about positive elements, at this point,
but don't take my word for it:
try to prove the following facts
about a self-adjoint element~$a$ of~$\scrA$ directly.
\begin{enumerate}
\item $a^2$ is positive;
\item if $a$ is the limit of positive $a_n\in\scrA$,
then $a$ is positive;
\item if $a\geq -\frac{1}{n}$ for all~$n\in \N$, then $a\geq 0$;
\item  $\|a\|=\|a\|_o$.
\end{enumerate}
We will prove these facts
when we return to the positive elements in~\sref{cstar-positive-2}.
\end{enumerate}
\end{point}
\end{parsec}

\begin{parsec}%
\begin{point}%
Let us spend some words
on the morphisms between $C^*$-algebras.
\end{point}
\begin{point}{Definition}
A linear map $f\colon \scrA \to \scrB$
between $C^*$-algebras
is called
\begin{enumerate}
\item
\Define{\textbf{m}ultiplicative}
if $f(ab)=f(a)f(b)$ for all $a,b\in\scrA$;
\item
\Define{\textbf{i}nvolutive}
if $f(a^*)=f(a)^*$ for all~$a\in\scrA$;
\item
\Define{\textbf{p}ositive}
if $f(a)$ is positive
for every positive $a\in\scrA$, and
\item
\Define{\textbf{u}nital}
if $f(1)=1$.
\end{enumerate}
\begin{point}%
We use the bold letters as abbreviations,
so for instance,
$f$ is \Define{pu} if it is positive and unital,
and a \Define{miu-map}
is a multiplicative, involutive, unital linear map between $C^*$-algebras,
(which is usually called a \Define{unital $*$-homomorphism}.)
\end{point}
\begin{point}%
The miu-maps between $C^*$-algebras
form a category which we denote by~$\Cstar{miu}$.
The full subcategory of commutative $C^*$-algebras
is denoted by~$\cCstar{miu}$.
\TODO{products and equalizers in $\Cstar{miu}$}
\end{point}
\end{point}
\end{parsec}

%
% geometric series 
%
\begin{parsec}%
\begin{point}%
Let us first study the invertible elements
of a $C^*$-algebra,
whose role 
is as important as it is technical.
This paragraph culminates in what is essentially
 \emph{spectral permanence} (\sref{spectral-permanence}):
the fact that if an element $a$ of a $C^*$-subalgebra $\scrB$
is invertible in~$\scrA$,
then~$a$ is already invertible in~$\scrB$,
see~\sref{inverse-permanence}.
\end{point}
\begin{point}[geometric]{Lemma}%
Let~$a$ be an element of a $C^*$-algebra~$\scrA$ with~$\|a\|<1$.
Then~$a^\perp=1-a$ has an inverse,
namely~$(a^\perp)^{-1}= \sum_{n=0}^\infty\, a^n$
(norm convergence).
\begin{point}{Proof}%
Note that
$(1-\|a\|)\,(1+\|a\|+\|a\|^2+\dotsb+\|a\|^N) \,=\, 1-\|a\|^{N+1}$,
and so 
\begin{equation*}
\sum_{n=0}^N \|a\|^n \ =\  \frac{1-\|a\|^{N+1}}{1-\|a\|}
\end{equation*}
for every~$N$.
Thus,
since $\|a\|^N$ converges to~$0$
(by~\TODO{} because $\|a\|<1$),
we  get $\sum_{n=0}^\infty \|a\|^n = (1-\|a\|)^{-1}$.

\begin{point}%
Note that $a^N$ norm converges to~$0$,
because $\|a\|^N$ converges to~$0$.
Also (but slightly less obvious),
$\sum_n a^n$ norm converges,
because~$\sum_n \|a\|^n$ converges.
\end{point}
\begin{point}%
Thus, taking the norm limit
on both sides of $(1-a)(1+a+a^2+\dotsb a^N) = 1-a^{N+1}$,
gives us $(1-a)(\sum_n a^n) = 1$.
Since we can derive $(\sum_n a^n)(1-a) = 1$
in a similar manner, 
we see that $\sum_n a^n$ is the inverse of~$1-a$.
\end{point}
\end{point}
\end{point}
\begin{point}[spectrum-bounded]{Exercise}
Let~$a$ be an element of a $C^*$-algebra~$\scrA$.
\begin{enumerate}
\item
Show that $a-\lambda$ is invertible
for every~$\lambda\in\C$ with~$\|a\|< \left|\lambda\right|$.
\item
Show that $a-b$ is invertible
when~$b\in\scrA$ is invertible and $\|a\| < \|b\|$.
\item
Show that $U:=\{\ b\in\scrA\colon\ \text{$b$ is invertible}\ \}$
is an open subset of~$\scrA$.
\end{enumerate}
\end{point}
\begin{point}[harmonic-divergence]{Exercise}%
Let~$a$ be an element of a $C^*$-algebra $\scrA$ with $\|a\|>1$.
\begin{enumerate}
\item
Show that $1,\,a,\,a^2,\,a^3,\,\dotsc$ diverges.
\item
Show that $\sum_n a^n$ diverges.
\end{enumerate}
\end{point}
\begin{point}[cstar-inv-continuous]{Lemma}%
Let~$\scrA$ be a $C^*$-algebra.
The assignment $a\mapsto a^{-1}$
gives a  continuous map
(from $\{\,b\in \scrA\colon\, \text{$b$ is invertible}\,\}$
to~$\scrA$.)
\begin{point}[cstar-inv-continuous-1]{Proof}
First we establish continuity at~$1$:
let~$a\in\scrA$ with $\|1-a\|\leq \frac{1}{2}$ be given;
we claim that~$a$ is invertible,
and~$\|1-a^{-1}\| \leq 2\|1-a\|$.

Indeed, since~$\|1-a\|\leq \frac{1}{2}<1$,
$a$ is invertible by~\sref{geometric},
and $a^{-1}=\sum_{n=0}^\infty (1-a)^n$.
Then~$\|1-a^{-1}\|=\|\sum_{n=1}^\infty (1-a)^n\|\leq \sum_{n=1}^\infty \|1-a\|^n
= \|1-a\|\, (1-\|1-a\|)^{-1}$.
Thus, as $\|1-a\|\leq\frac{1}{2}$,
we get $(1-\|1-a\|)^{-1}\leq 2$,
and so $\|1-a^{-1}\|\leq 2\|1-a\|$.
\begin{point}%
Let~$a$ be an invertible element of~$\scrA$,
and let~$b\in\scrA$ with~$\|a-b\|\leq\frac{1}{2}\|a^{-1}\|$.
We claim that~$b$ is invertible,
and~$\|a^{-1}-b^{-1}\|\leq 2\|a-b\|\,\|a^{-1}\|^2$.

Since $\|a-b\|\leq \frac{1}{2}\|a^{-1}\|$
we have
$\|1-a^{-1}b\|\leq \|a^{-1}\|\,\|a-b\|\leq \frac{1}{2}$.
By~\sref{cstar-inv-continuous-1}, $a^{-1}b$ is invertible,
and $\|1-(a^{-1}b)^{-1}\|\leq 2\|1-a^{-1}b\|\leq 2\|a-b\|\,\|a^{-1}\|$.
Hence $\|a^{-1}-b^{-1}\| = \|(1-(a^{-1}b)^{-1})a^{-1}\|
\leq \|1-(a^{-1}b)^{-1}\|\,\|a^{-1}\|\leq 2 \|a-b\|\,\|a^{-1}\|^2$.

(Based on Kadison--Ringrose Proposition 3.1.6.)
\end{point}
\end{point}
\end{point}
%
%	Towards spectral permanence
%
\begin{point}{Lemma}%
For a self-adjoint element~$a$ from a $C^*$-algebra,
$a-i$ is invertible.
\begin{point}{Proof}%
The trick
is to 
write~$a-i\equiv (a+ni)\,-\,(n+1)i$
for sufficiently large~$n$,
because  
then
$a-i$
is invertible provided that~$n+1 > \|a+ni\|$
by~\sref{spectrum-bounded}.
Indeed, for~$n$ such that~$\|a\|<2n+1$,
we have $\|a+ni\|^2 = \|(a+ni)^*(a+ni)\|
= \|a^2+n^2\|
\leq \|a\|^2+n^2 < 2n+1+n^2 = (n+1)^2$,
and so $\|a+ni\| < n+1$.

(Based on Kadison--Ringrose Proposition 4.1.1(ii).)
\end{point}
\begin{point}[spectrum-self-adjoint-real]{Exercise}%
Let~$a$ be a self-adjoint element of a $C^*$-algebra.
\begin{enumerate}
\item
Show that~$a-\lambda$ is invertible for all $\lambda\in \C\backslash \R$.
\item
Show that $a^2-\lambda$ is invertible for all 
$\lambda\in \C\backslash[0,\infty)$.\\
(Hint: first prove that
 $a^2+1 \equiv (a+i)(a-i)$ is invertible.)

Conclude that $a^n-\lambda$ is invertible for all 
$\lambda\in\C\backslash[0,\infty)$ and \emph{even} $n\in\N$.
\item
Let~$n\in \N$ be \emph{odd}.
Show that $a^n-\lambda$ is invertible
for all~$\lambda\in \C\backslash[0,\infty)$
if and only if $a-\lambda$ is invertible
for all~$\lambda\in \C\backslash[0,\infty)$.\\
(Hint: show that
$a^n+1= \prod_{k=1}^n a+\zeta^{2k+1}$
where $\zeta=e^{\frac{\pi i}{n}}$.)
\end{enumerate}
\end{point}
\end{point}
\begin{point}{Proposition}%
Let~$\scrA$ be a $C^*$-subalgebra
of a $C^*$-algebra $\scrB$.
Let~$a$ be a self-adjoint element of~$\scrA$,
which has an inverse, $a^{-1}$, in~$\scrB$.
Then~$a^{-1}\in\scrA$.
\begin{point}{Proof}%
While we do not know yet that~$a$ is invertible in~$\scrA$,
we do know that~$a+\nicefrac{i}{n}$ 
has an inverse $(a+\nicefrac{i}{n})^{-1}$ in~$\scrA$
by~\sref{spectrum-self-adjoint-real}
for each~$n$
(using that $a$ is self-adjoint.)
Since~$a+\nicefrac{i}{n}$ converges to~$a$ in~$\scrB$ as~$n$ increases,
we see that $(a+\nicefrac{i}{n})^{-1}$ converges to~$a^{-1}$
in~$\scrB$ by~\sref{cstar-inv-continuous}.
Thus, as all~$(a+\nicefrac{1}{n}i)^{-1}$ are in~$\scrA$,
and~$\scrA$ is closed in~$\scrB$,
we see that~$a^{-1}$ is in~$\scrA$.
\end{point}
\begin{point}[inverse-permanence]{Exercise}%
Show that the assumption that~$a$ is self-adjoint
can be dropped. 

(Hint: consider $a^*a$; c.f.~Conway Proposition VIII.1.14.)
\end{point}
\end{point}
\begin{point}{Definition}%
The \Define{spectrum} of an element $a$
of a $C^*$-algebra
is the set \Define{$\spec(a)$}
of complex numbers~$\lambda$
for which~$a-\lambda$ is not invertible.
\begin{point}{Exercise}%
Verify the following examples.
\begin{enumerate}
\item
The spectrum of a continuous function~$f\colon X\to \R$
on a compact Hausdorff space~$X$
being an element of the $C^*$-algebra $C(X)$
is the image of~$f$, that is,
$\spec(f) = \{f(x)\colon x\in X\}$.
\item
The spectrum of a square matrix~$A$
from the $C^*$-algebra $M_n$
is the set of eigenvalues of~$A$.
\end{enumerate}
\end{point}
\begin{point}[spectrum-basic]{Exercise}%
Let~$a$ be an element of a $C^*$-algebra $\scrA$.
\begin{enumerate}
\item
Prove that $\spec(a)\subseteq \R$ when $a$ is self-adjoint
(see~\sref{spectrum-self-adjoint-real}).

The reverse implication also holds,
as we'll see later on in \TODO{}.
\item
Show that $\spec(a^2)\subseteq [0,\infty)$ when $a$ is self-adjoint
(see~\sref{spectrum-self-adjoint-real}).

\item
Show that $|\lambda|\leq \|a\|$ for all~$\lambda\in\spec(a)$
using~\sref{spectrum-bounded}.

In fact, we will see in~\sref{norm-spectrum},
that $\|a\|=\sup\{\left|\lambda\right|\colon \lambda\in \spec(a)\}$.
\item
Show that $\spec(a)$ is closed (using~\sref{spectrum-bounded}).\\
Conclude that~$\spec(a)$ is compact.
\item
Show that $\spec(a+z)=\{\lambda+z\colon \lambda\in\spec(a)\}$
for all~$z\in \C$.
\item
Prove that~$\spec(a^{-1})=\{\lambda^{-1}\colon \lambda\in \spec(a)\}$
if~$a$ is invertible (and~$0\notin \spec(a)$).
\end{enumerate}
\end{point}
\end{point}
\begin{point}%
On first sight,
the spectrum $\spec(a)$
of an element~$a$ of a $C^*$-algebra~$\scrA$ 
depends not only on~$a$,
but also on the surrounding $C^*$-algebra~$\scrA$ for it determines
for which~$\lambda\in\C$ the operator $a-\lambda$ is invertible.
Thus we should perhaps write $\spec_\scrA(a)$ instead
of~$\spec(a)$.
However, such careful bookkeeping turns out 
be unnecessary
by the following result.
\end{point}
\begin{point}[spectral-permanence]{Theorem (Spectral Permanence)}%
Let~$\scrB$ be a $C^*$-subalgebra of a $C^*$-algebra $\scrA$.
Then~$\spec_{\scrA}(a)=\spec_\scrB(a)$
for every element~$a$ of~$\scrB$.
\begin{point}{Proof}%
Let~$a$ be an element of~$\scrB$,
and let~$\lambda\in \C$.
We must show that $a-\lambda$ is invertible in~$\scrA$
iff $a-\lambda$ is invertible in~$\scrB$.
Surely,
if $a-\lambda$ has an inverse $(a-\lambda)^{-1}$ in~$\scrB$,
then~$(a-\lambda)^{-1}$ is also an inverse of~$a-\lambda$ in~$\scrA$,
since~$\scrB\subseteq \scrA$.
The other, non-trivial, direction follows
directly from~\sref{inverse-permanence}.\qed%
\end{point}
\end{point}
\end{parsec}

\begin{parsec}%
\begin{point}%
The next order of business
is to show that the spectrum~$\spec(a)$ of an element~$a$
of a $C^*$-algebra contains enough points, so to speak.
One incarnation of this idea that you might have heard
is that~$\spec(a)$ is non-empty
(see~\sref{spectrum-non-empty}), but
we will need more,
and prove that  $\|a\|=\left|\lambda\right|$
for some~$\lambda\in\spec(a)$.
Somewhat baffling,
the canonical and seemingly
easiest way to derive this fact is by considering the power series
expansion of a cleverly chosen $\scrA$-valued function
(see~\sref{norm-spectrum}).
To this end,
we'll first quickly redevelop some complex analysis
for~$\scrA$-valued functions
(instead of $\C$-valued functions).
\end{point}
\begin{point}{Setting}%
Fix a $C^*$-algebra~$\scrA$ for the remainder of this paragraph.
For brevity,
we'll say that a \Define{function}
is a partially defined map $f\colon \C\to \mathscr{A}$
whose domain of definition $\dom(f)$ is an open subset of~$\C$.
Such a function is called \Define{holomorphic} at a point~$z\in \C$
if $f$ is defined on~$z$ (that is, $z\in \dom(f)$),
and 
\begin{equation*}
\frac{f(x)-f(y)}{x-y}
\end{equation*}
converges (with respect to the norm on~$\scrA$)
to some element~$f'(x)$ of~$\scrA$
as $y\in \dom(f)\backslash\{x\}$
converges to~$x$.

We say that~$f$ is \Define{holomorphic}
if~$f$ is holomorphic at~$x$ for all~$x\in \dom(f)$,
and the function $z\mapsto f'(z)$
with $\dom(f')=\dom(f)$
is its \Define{derivative}.
\end{point}
\begin{point}{Exercise}%
Verify the following examples of holomorphic functions.
\begin{enumerate}
\item
If~$f$ and $g$ are holomorphic functions with $\dom(f)=\dom(g)$,
then $f+g$ and $f\cdot g$ are holomorphic,
and $(f+g)'=f'+g'$ and $(f\cdot g)' = f'g+g'f$.

\item
The function~$f$ given by $f(z)=z$ and~$\dom(f)=\C$
is holomorphic, and $f'(z)=1$ for all $z\in\C$.

\item
Let~$a\in \scrA$. The constant function $f$ given by $f(z)=a$
for all~$z\in \C$ is holomorphic, and $f'(z)=0$ for all~$z\in \scrA$.

\item
Any polynomial,
that is, function~$f$ of the form $f(z)\equiv a_n z^n+\dotsb+a_1 z+a_0$
is holomorphic with $f'(z)=na_nz^{n-1}+\dotsb+2a_2z+a_1$.
\end{enumerate}
\end{point}
\TODO{Definition of polygonal paths}
\TODO{Definition of integration along these}
\begin{point}[goursat]{Goursat's Theorem}%
Let~$f$ be a holomorphic function,
and let~$T$ be a triangle whose interior
is entirely contained in~$\dom(f)$.
Then~$\int_T f = 0$.
\begin{point}[goursat-1]{Proof}%
Note that if~$f$ has a primitive,
that is, $f\equiv g'$ for some holomorphic function~$g$,
then it is clear that~$\int_T f=0$
by \TODO{the fundamental theorem of calculus}.
Although it is true that every holomorphic function
with simply connected domain has a primitive,
this result is not yet available 
(and in fact depends on this very theorem).
Instead we will approximate~$f$
by an affine function
(which does have a primitive)
using the derivative of~$f$.
But since such an approximation only
concerns a single point,
we first need to zoom in.
\begin{point}[goursat-2]%
If we split~$T$ into four similar triangles
$T^\text{i}$, $T^\text{ii}$,
$T^\text{iii}$, $T^\text{iv}$
(see picture~\TODO{})
we have $\smash{\int_Tf = \sum_{n={\text{i}}}^{\text{iv}} \int_{T^n}f}$.
There is $T'$ among
$T^\text{i}$, $T^\text{ii}$,
$T^\text{iii}$, $T^\text{iv}$
with 
 $\|\int_Tf\|\leq 4 \|\int_{T'} f\|$.
Clearly, $\length(T)=2\length(T')$.
Write~$T_0 := T$ and $T_1 := T'$. 

From this it is clear how to
 get a sequence of similar triangles $T_0, T_1, T_2, \dotsc$
with $\|\int_Tf\|\leq 4^n \|\int_{T_n} f\|$,
and $\length(T)=2^n\length(T_n)$.
\end{point}
\begin{point}%
If we pick a point on each triangle~$T_n$ 
we get a Cauchy sequence
that converges to some point~$z_0\in\C$
which lies in (or on) each of the triangles~$T_1,T_2,\dotsc$.
We can approximate $f$ by an affine
function at~$z_0$ as follows.
For $z\in \dom(f)$,
\begin{equation*}
f(z)\ = \ f(z_0)\,+\,f'(z_0)\,(z-z_0)\,+\,r(z)\,(z-z_0),
\end{equation*}
where~$r\colon \dom(f)\to \C$
is given by $r(z)=f'(z_0)-(f(z)-f(z_0))(z-z_0)^{-1}$ for $z\neq z_0$
and $r(z_0)=0$.
We see that~$r(z)$ converges to~$0$ as~$z\to z_0$.

Let~$\varepsilon >0$ be given.
There is~$\delta>0$
such that $z\in\dom(f)$
and $\|r(z)\|\leq \varepsilon$
for all~$z\in \C$ with $\|z-z_0\|<\delta$.
There is~$n$ such that the triangle~$T_n$ is contained
in the ball around~$z_0$ of radius~$\delta$.
Note that $\int_{T_n} f(z_0)+f'(z_0)(z-z_0)\,dz=0$
by the discussion in~\sref{goursat-1}, because
the integrated function is affine.
Thus
\begin{equation*}
\textstyle
\int_{T_n} f \  = \ \int_{T_n}r(z)\,(z-z_0)\,dz.
\end{equation*}
Note that for $z\in T_n$,
we have  $\|z-z_0\|\leq \length(T_n)$,
and $\|r(z)\|\leq \varepsilon$ (because $\|z-z_0\|\leq \delta$),
and so $\|r(z)(z-z_0)\|\leq \varepsilon\,\length(T_n)$.
Thus:
\begin{equation*}
\textstyle
\|\int_{T_n} f\| \  = \ \|\int_{T_n}r(z)\,(z-z_0)\,dz\|
\ \leq\ \varepsilon\length(T_n)^2.
\end{equation*}
Using the inequalities from~\sref{goursat-2},
we get
\begin{equation*}
\textstyle
\|\int_T f\|\ \leq\ 4^n\, \|\int_{T_n} f\|
\ \leq\ \varepsilon \,4^n\,\length(T_n)^2 
\ \equiv\ \varepsilon \length(T)^2.
\end{equation*}
Since~$\varepsilon>0$ was arbitrary,
we see that~$\int_T f=0$.\qed
\end{point}
\end{point}


(Proof is based on~\cite{moore1900}.)
\end{point}%
\begin{point}[cauchy-formula]{Theorem (Cauchy's Integral Formula)}%
Let~$f$ be a holomorphic $\scrA$-valued function.
Let~$p$ be a simple positively oriented 
polygon with $\interior(p)\subseteq\dom(f)$.
Then for $z_0\in \interior(p)$,
\begin{equation*}
f(z_0)\ = \ \frac{1}{2\pi i}\,\int_p \frac{f(z)}{z-z_0}\,dz
\end{equation*}
\begin{point}{Proof}%
Since~$\int_p \frac{f(z_0)}{z-z_0}\,dz
= 2\pi i f(z_0)$ by~\TODO{reason},
it suffices to show that
\begin{equation}
\label{eq:cauchy-formula-1}
\int_p \frac{f(z)-f(z_0)}{z-z_0}\,dz \ = \ 0.
\end{equation}
\begin{point}[cauchy-formula-1]%
Let~$\varepsilon>0$ be given.
Since~$f$ is holomorphic at~$z_0$
we can find $\delta>0$ with
$\|f(z)-f(z_0)\|\leq \|z-z_0\|$
for all~$z\in\dom(f)$ with $\|z-z_0\|\leq \delta$. 
\end{point}
\begin{point}%
To use~\sref{cauchy-formula-1},
we must restrict our attention to a smaller polygon.
Let~$q$ be a simple positively oriented polygon 
with $z_0\in \interior(q)$,  $\overline{q}\subseteq \interior(p)$,
$\length(q)\leq \varepsilon$,
and $\|z_0-z\|\leq \delta$ for all~$z\in \partial q$.
By~\sref{goursat}, we have
\begin{equation}
\label{eq:cauchy-formula-2}
\int_p \frac{f(z)-f(z_0)}{z-z_0}
\ = \ 
\int_q \frac{f(z)-f(z_0)}{z-z_0}.
\end{equation}
By~\sref{cauchy-formula-1}
we have
\begin{equation*}
\left\|\,\int_q \frac{f(z)-f(z_0)}{z-z_0}\,dz\,\right\|
\ \leq \ \length(q)\,\cdot\,
\sup_{z\in\partial q} \,\left\|\,\frac{f(z)-f(z_0)}{z-z_0}\,\right\|
\ \leq \ \varepsilon.
\end{equation*}
Since~$\varepsilon>0$ was arbitrary,
we get Eq.~\eqref{eq:cauchy-formula-1}
from Eq.~\eqref{eq:cauchy-formula-2}.\qed
\end{point}
\end{point}
\end{point}
\begin{point}[taylor]{Proposition}%
Let~$f$ be a holomorphic $\scrA$-valued function.
Let~$p$ be a simple positively oriented polygon 
with $\interior(p) \subseteq \dom(f)$.
Then for all~$w,z\in \interior(p)$
with $\|z-w\|<\inf_{u\in \partial p} \left| u-w \right|$,
we have:
\begin{equation*}
f(z)\ = \ 
\sum_{n=0}^\infty \ \frac{1}{2\pi i}\int_p \frac{f(u)}{(u-w)^{n+1}}\,du
\ (z-w)^n.
\end{equation*} 
\begin{point}{Proof}%
By~\sref{cauchy-formula} we have
\begin{alignat*}{3}
2\pi if(z)\ &=\  \int_p \frac{f(u)}{u-z}\,du
\ =\ 
  \int_p  \frac{f(u)}{u-w}\,\frac{1}{1-\frac{z-w}{u-w}}\,du
\end{alignat*}
Since~$\left|z-w\right|<\left|u-w\right|$
for~$u\in \partial p$,
we get, by~\sref{geometric},
\begin{equation*}
2\pi if(z) \ = \ 
  \int_p \frac{f(u)}{u-w}\, \sum_{n=0}^\infty 
\frac{(z-w)^n}{(u-w)^n}
 \,du\ = \ 
  \sum_{n=0}^\infty \ \int_p   \frac{f(u)}{(u-w)^{n+1}}du \ (z-w)^n,
\end{equation*}
where the interchange of ``$\sum$'' and ``$\int$''
was allowed by~\TODO{to add}.\qed
\end{point}
\end{point}
\begin{point}[norm-spectrum]{Proposition}%
For an element~$a$ of a $C^*$-algebra~$\scrA$,
we have
\begin{equation*}
\|a\|\,=\,\sup\{\,\left|\lambda\right|\colon 
\,\lambda\in \spec(a)\backslash\{0\}\,\}.
\end{equation*}
\begin{point}{Proof}%
Write~$r=
\sup\{\left|\,\lambda\right|\colon\, \lambda\in \spec(a)\backslash\{0\}\,\}$.
Since~$\left|\lambda\right| \leq \|a\|$
for all~$\lambda\in\spec(a)$
(\sref{spectrum-bounded})
we see that~$r\leq \|a\|$.
Thus we only need to show that~$\|a\|\leq r$. 

Let~$\varepsilon>0$ be given.
It suffices to show that~$\|a\|\leq (1+\varepsilon)\,r$.
\begin{point}%
The trick is to consider
the power series expansion
around~$0$ of the holomorphic function~$f$ defined
on~$G:=\{\,z\in \C\colon 1-az\text{ is invertible}\,\}$ 
by  $f(z)=z(1-az)^{-1}$.
\end{point}
\begin{point}%
Indeed, we have $f(z) = \sum_n a^nz^{n+1}$
for all~$z\in \C$ with $\left|z\right|\|a\|<1$
because for such~$z$
we have $\sum_n (az)^n=(1-az)^{-1}$
by~\sref{geometric},
and thus~$f(z)=z(1-az)^{-1}=z\sum_n (az)^n = \sum_n a^nz^{n+1}$.
\end{point}
\begin{point}[norm-spectrum-2]%
Moreover,
we know by~\sref{harmonic-divergence}
that for every $z\in\C$ with $\left|z\right|\|a\|>1$
the series $\sum_n(az)^n$ 
and thus $\sum_n a^n{z}^{n+1}$ diverges.
\end{point}
\begin{point}%
Let~$R$ denote the distance of~$0$ to
the border of~$G$,
that is, 
$R= \inf\{\left|\lambda\right|\in \C\colon \lambda\notin G\}$.
We claim that~$R^{-1}=r$.
To begin,
recall that $\lambda\notin G$ iff $1-a\lambda$ is invertible.
Since clearly~$0\in G$,
we have $\lambda\notin G$ iff $\lambda\neq 0$ and $\lambda^{-1}-a$ 
is invertible, that is, $\lambda^{-1}\in \spec(a)\backslash\{0\}$.
Thus we see that $R^{-1}=r$.
\end{point}
\begin{point}[norm-spectrum-3]%
We know by~\sref{rigid-expansion}
that the expansion $f(z)=\sum_na^nz^{n+1}$
must be valid not only for~$z\in\C$ with $\left|z\right|\|a\|<1$,
but for all~$z\in\C$ with 
$\|z\|< R$,
that is, $\left|z\right|r\equiv \left|z\right| R^{-1} < 1$.
In particular,
for any~$z$ with $\left|z\right|\|a\|>1$ 
(see~\sref{norm-spectrum-2})
we cannot have $\|z\| r <1$,
and so we must have $\left|z\right| r \geq 1$.
Multiplying by~$\|a\|$,
we see that $\|a\|\leq \|a\|\left|z\right|r$.
Since we can choose~$z$ such that  $\left|z\right|\|a\|\leq 1+\varepsilon$,
we see that $\|a\|\leq \|a\|\left|z\right|r \leq (1+\varepsilon)r $,
and so we are done.\qed
\end{point}
\end{point}
\end{point}
\begin{point}[spectrum-non-empty]{Exercise}%
Given an element~$a$ of a $C^*$-algebra show that
\begin{enumerate}
\item $\spec(a)\neq \varnothing$;
\item $\spec(a) =\{\lambda\}$ iff $a=\lambda$ for every $\lambda \in\C$.
\end{enumerate}
\end{point}
\begin{point}{Exercise (Gelfand--Mazur's Theorem for $C^*$-algebras)}%
Prove that if every non-zero element of a $C^*$-algebra~$\scrA$
is invertible, then~$\scrA=\C$ or~$\scrA=\{0\}$.
\end{point}
\begin{point}%
A logical next step
towards Gelfand's representation theorem
is to show that if~$\lambda\in\spec(a)$
for some element~$a$ of a \emph{commutative} $C^*$-algebra~$\scrA$,
then there is a miu-map $f\colon \scrA\to \C$
with~$f(a)=\lambda$.
Here we have moved ourselves in a tight spot
by evading Banach algebras,
because the mentioned result is usually obtained
by finding a maximal ideal~$I$ of~$\scrA$
(by Zorn's Lemma) that contains~$\lambda-a$,
and then forming the \emph{Banach algebra} quotient~$\scrA/I$.
One then applies Gelfand--Mazur's Theorem for \emph{Banach algebras}, 
to see that
$\scrA/I= \C$,
and thereby obtain a miu-map~$f\colon \scrA\to C$ with~$f(a-\lambda)=0$.
The problem here is that while $\scrA/I$
will turn out to be a $C^*$-algebra (indeed, be $\C$)
the formation of the $C^*$-algebra quotient
is non-trivial and depends on Gelfand's representation theorem
(see e.g.~\TODO{conway})
which is the very theorem we are working towards.
The way out of this predicament
is to avoid ideals and quotients of $C^*$- and Banach algebras
altogether,
and instead work 
with order ideals (and what are essentially
 quotients of Riesz and order unit spaces).
To this end,
we develop the theory
of the positive elements of a $C^*$-algebra
farther than is usually done
before Gelfand's representation theorem.
\end{point}
\end{parsec}


\begin{parsec}[cstar-positive-2]%
\begin{point}%
We return to the positive elements 
in a $C^*$-algebra (see~\sref{cstar-positive}).
\end{point}
\begin{point}[real-pos-ineq]{Exercise}%
Show that 
$\left|\,\lambda-t\,\right| \,\leq\, t$ iff  $\lambda \in[0,2t]$,
where $\lambda,t\in\R$.
\end{point}
\begin{point}[pos-spectrum]{Proposition}%
For a self-adjoint element $a$ from a $C^*$-algebra,
and $t\in [0,\infty]$, 
\begin{equation*}
\|a-t\|\,\leq\, t\qquad\iff\qquad \spec(a)\subseteq [0,2t].
\end{equation*}%
\begin{point}{Proof}%
To begin, note that~$\spec(a-t)=\spec(a)-t\subseteq \R$ 
by~\sref{spectrum-basic},
because~$a$ is self-adjoint.
Thus $\|a-t\|=\sup\{\,\left|\lambda-t\right|\colon \lambda\in \spec(a)\,\}$
by~\sref{norm-spectrum}.
Hence $\|a-t\|\leq t$
iff $\left|\lambda-t\right|\leq t$ for all~$\lambda\in\spec(a)$
iff $\spec(a)\subseteq [0,2t]$ (by \sref{real-pos-ineq}).\qed
\end{point}
\begin{point}{Exercise}%
Show
(using~\sref{pos-spectrum} and~\sref{spectrum-basic})
that
for any self-adjoint element $a$ of a $C^*$-algebra~$\scrA$,
the following are equivalent.
\begin{enumerate}
\item 
\label{cstar-pos-1}
$\|a-t\|\leq t$
for some $t\geq \frac{1}{2}\|a\|$;
\item 
\label{cstar-pos-2}
$\|a-t\|\leq t$
for all $t\geq \frac{1}{2}\|a\|$;
\item 
\label{cstar-pos-3}
$\spec(a)\subseteq[0,\infty)$;
\item
$a$ is positive.
\end{enumerate}
We will complete this list in~\sref{cstar-positive-final}.
\end{point}
\end{point}
\begin{point}{Exercise}%
Let~$\scrA$ be a $C^*$-algebra.
\begin{enumerate}
\item
Show that~$\pos{\scrA}$ is closed.
\item
Let~$a$ be a self-adjoint element of~$\scrA$.
Show that
 $-\lambda \leq a\leq \lambda$
iff $\|a\|\leq \lambda$,
for $\lambda\in [0,\infty)$.
Conclude that $\|a\| = \inf\{ \lambda \in \R\colon 
-\lambda \leq a\leq \lambda\}$.
(Thus $\sa{\scrA}$ is a \emph{complete Archimedean order unit space},
see \TODO{}.)

Show that $0\leq a \leq b$ entails $\|a\|\leq \|b\|$
for $a,b\in\sa{\scrA}$.

\item 
Recall that $ab$ need not be positive if~$a,b\geq 0$. However:

Show that $a^2$ is positive for every self-adjoint element~$a$ of~$\scrA$.

Show that $a^n$ is positive for \emph{even} $n\in \N$ and~$a\in\sa{\scrA}$.

Show that $a^n$ is positive iff $a$ is positive for \emph{odd} $n\in \N$
and $a\in\sa{\scrA}$.

Show that $a^n$ is positive
for every positive $a$ from~$\scrA$ and~$n\in \N$.
\item
Let~$a$ be an invertible element of~$\scrA$.
Show that $a\geq 0$ iff $a^{-1}\geq 0$.

\item
Show that a positive element~$a$ of~$\scrA$ is invertible
iff $a\geq \frac{1}{n}$ for some~$n\in \N$.
\end{enumerate}
\end{point}
\begin{point}[prod-spec]{Lemma}%
For elements $a$ and $b$ from a $C^*$-algebra,
we have
\begin{equation*}
\spec(ab)\backslash\{0\}\ =\ \spec(ba)\backslash\{0\}.
\end{equation*}
\begin{point}{Proof}%
Let~$\lambda\in \C$ with $\lambda\neq 0$ be given.
We must show that $\lambda - ab$ is invertible
iff $\lambda - ba$ is invertible.
Suppose that $\lambda-ab$ is invertible.
Then using the equality$a(\lambda-ba)=(\lambda-ab)a$
one sees that $(1+b(\lambda-ab)^{-1}a)(\lambda-ba)=\lambda$.
Since similarly $(\lambda-ba)(1+b(\lambda-ab)^{-1}a)=\lambda$,
we see that $\lambda^{-1}(1+b(\lambda-ab)a)$
is the inverse of~$\lambda-ba$.\qed
\end{point}
\end{point}
\begin{point}[astara-non-neggative]{Lemma}%
We have $a^*a  \leq 0\implies a=0$
for every element~$a$ of a $C^*$-algebra.
\begin{point}{Proof}%
Suppose that $a^*a\leq 0$.
Then~$\spec(a^*a)\subseteq (-\infty,0]$, almost by definition,
and so $\spec(aa^*)\subseteq (-\infty,0]$ by~\sref{prod-spec},
giving $aa^*\leq 0$.
Thus $a^*a+aa^*\leq 0$.

But on the other hand, 
$a^*a+aa^* = 2(\Real{a}^2 + \Imag{a}^2) \geq 0$,
and so~$a^*a+aa^*=0$.
Then $0\geq a^*a=-aa^*\geq 0$ gives $a^*a=0$,
and $a=0$.\qed
\end{point}
\end{point}
\end{parsec}
\begin{parsec}%
\begin{point}%
There is much more that can be said
about the positive elements of a $C^*$-algebra
(as we will in~\sref{sqrt}),
but let us for a moment
turn our attention to the states of a $C^*$-algebra
(see \sref{state}),
which are
essential
in the proof that every $C^*$-algebra
is isomorphic to a $C^*$-algebra of operators on a Hilbert space.
\end{point}
\begin{point}[state]{Definition}%
A \Define{state} of a $C^*$-algebra~$\scrA$
is a piu-map $f\colon \scrA\to \C$.
\begin{point}{Remark}%
\TODO{about piu instead of pu}
\end{point}
\end{point}
\begin{point}
Our program is to show that 
for every self-adjoint element~$a\in \scrA$
of a $C^*$-algebra~$\scrA$,
there is a state~$f$ of~$\scrA$ with $f(a)=\|a\|$ or $f(a)=-\|a\|$.

To obtain such a state
we first find its kernel,
which leads us to the following definitions.
\end{point}
\begin{point}{Definition}%
An \Define{order ideal}
of a $C^*$-algebra~$\scrA$
is a linear subspace~$I$ of~$\scrA$
with $b\in I\implies b^*\in I$
and $b\in I\cap\pos{\scrA}\implies [-b,b]\subseteq I$.
It is called \Define{proper} if~$1\notin I$,
and \Define{maximal} if it is maximal among all proper order ideals.
\end{point}
\begin{point}[order-ideal-basic]{Exercise}%
Let~$\scrA$ be a $C^*$-algebra.
\begin{enumerate}
\item
Show that the kernel of a state is a maximal order ideal.
\item
Let~$I$ be a proper order ideal of~$\scrA$.
Show that there is a maximal 
order ideal~$J$ of~$\scrA$ with $I\subseteq J$.
(Hint: Zorn's Lemma may be useful.)
\item
Let~$a\in \sa{\scrA}$.
Show that 
\begin{equation*}
(a)\ := \ \{\, b\in \scrA\colon\, \exists n,m\in \Z\,
[\ na \,\leq\, \Real{b},\Imag{b}\,\leq\, ma\ ]\,\}
\end{equation*}
is the least order ideal that contains~$a$.

Show that~$1\in (a)$ if and only if $a$ is invertible
and either $0\leq a$ or $a\leq 0$.

\item
Let~$a$ be a self-adjoint element of~$\scrA$ which
is not invertible.
Show that there is a maximal order ideal~$J$
of~$\scrA$
with $a\in J$.

\item
Let~$a$ be a self-adjoint element of~$\scrA$.
Show that  $\|a\|-a$
or $\|a\|+a$ is not invertible.
\end{enumerate}
\end{point}
\begin{point}[maximal-ideal-state]{Lemma}%
Let~$I$ be a maximal ideal of a $C^*$-algebra.
Then there is a state $f\colon \scrA\to \C$
with $\ker(f)=I$.
\begin{point}{Proof}%
Form the quotient vector space $\scrA/I$
with quotient map $q\colon \scrA\to \scrA/I$.
Note that since~$1\notin I$
we have $q(1)\neq 0$
and so we may regard~$\C$ 
to be a linear subspace of~$\scrA/I$
via $\lambda\mapsto q(\lambda)$.
We will, in fact, show that~$\C=\scrA/I$.

But let us first put an order on~$\scrA/I$:
we say that $\mathfrak{a}\in \scrA/I$ is positive
if $\mathfrak{a}\equiv q(a)$ for some~$a\in\pos{\scrA}$,
and write $\mathfrak{a}\leq \mathfrak{b}$ 
if $\mathfrak{b}-\mathfrak{a}$ is positive
for $\mathfrak{a},\mathfrak{b}\in\scrA/I$.
Note that the definition of ``order ideal'' is such
that if both~$\mathfrak{a}$ and $-\mathfrak{a}$ are positive,
then~$\mathfrak{a}=0$.
We leave it to the reader to verify 
that~$\scrA/I$ becomes a partially ordered vector space
with the order defined above.
There is, however,
one detail to which I want to draw you attention,
namely that a scalar $\lambda$ is positive in~$\scrA/I$
iff $\lambda$ is positive in~$\C$.
Indeed, if~$\lambda\geq 0$ in~$\C$,
then $\lambda\geq 0$ in~$\scrA$, and so~$\lambda \geq 0$ in~$\scrA/I$.
On the other hand,
if~$\lambda\geq 0$ in~$\scrA/I$, but~$\lambda\leq 0$ in~$\C$,
then $\lambda\leq 0$ in~$\scrA/I$,
and so $\lambda=0$.
This detail
has the desirable consequence
that once we have shown that~$\scrA/I=\C$,
we automatically get that~$q\colon \scrA\to\C$ is positive.

\begin{point}[pos-hahn-banach-1]%
Let~$a\in \sa{\scrA}$ be given.
Define~$\alpha := \inf\{\,\lambda\in\R\colon\, q(a)\leq \lambda\,\}$.
Note that $-\|a\| \leq \alpha\leq \|a\|$.
We will prove that~$q(a)=\alpha$
by considering the order ideal
\TODO{Double check this proof,
because it is simpler than the one given by Kadison---he first 
shows that $\scrA/I$ is total and Archimedean.}
\begin{equation*}
J\ := \ \{\,b\in \scrA\colon\, \exists m,n\in \Z\,[\ 
m(\alpha-q(a)) \,\leq\, q(\Real{b}),q(\Imag{b})\,\leq\, n(\alpha-q(a))\ ]\,\}.
\end{equation*}
We claim that $1\notin I$.
Indeed, suppose not---towards a contradiction.
Then there is~$n\in \Z$
with $1\leq n(\alpha-q(a))$.
What can we say about~$n$?
If~$n<0$,
then $0\geq \frac{1}{n}\geq \alpha-q(a)$,
so~$\alpha-\frac{1}{n} \leq q(a)$,
but $q(a)\leq \alpha+\varepsilon$
for every~$\varepsilon>0$,
and so~$\alpha-\frac{1}{n}\leq q(a)\leq \alpha-\frac{1}{2n}
\leq \alpha-\frac{1}{n}$,
which implies $\frac{1}{2n}=0$,
which is absurd.
If $n=0$,
then we get $1\leq n(\alpha-q(a))\equiv 0$, which is absurd.
If $n> 0$,
then $\frac{1}{n}\leq \alpha-q(a)$,
or in other words,
 $q(a) \leq \alpha - \nicefrac{1}{n}$,
giving $\alpha \leq \alpha-\nicefrac{1}{n}$
by definition of~$\alpha$,
which is absurd.
Hence~$1\notin J$.

But then since~$I\subseteq J$,
we get~$I=J$, by maximality of~$I$.
Thus, as $\alpha-a\in J$, we have $\alpha-a\in I$,
and so $q(a)=\alpha$, as desired.
\end{point}
\begin{point}%
Let~$a\in \scrA$ be given.
Then~$a=\Real{a}+i\Imag{a}$.
By~\sref{pos-hahn-banach-1},
there are $\alpha,\beta\in \R$ with $q(\Real{a})=\alpha$,
and $q(\Imag{a})=\beta$.
Thus~$q(a)=\alpha+i\beta$.
Hence~$\scrA/I=\C$.
Since the quotient map $q\colon \scrA\to \scrA/I\equiv \C$
is pu, and $\ker(q)=I$, we are done.\qed
\end{point}
\end{point}
\begin{point}{Exercise}%
Let~$a$ be a self-adjoint element of a $C^*$-algebra~$\scrA$.
\begin{enumerate}
\item
Show that there is a state~$f$ of~$\scrA$ with $\left|f(a)\right|=\|a\|$.
\item
Show that~$a$ is positive iff $f(a)\geq 0$ for every state~$f$ of~$\scrA$.
\end{enumerate}
\end{point}
\end{point}
\end{parsec}

\begin{parsec}%
\begin{point}%
The key that unlocks the remaining basic facts 
about the (positive) elements of a  $C^*$-algebra
is the existence of the square root~$\sqrt{a}$ of a positive element~$a$,
and its properties.
For technical reasons,
we will assume $\|a\|\leq 1$,
and construct
 $1-\sqrt{1-a}$ instead of~$\sqrt{a}$.
\end{point}
\begin{point}{Lemma}%
Let $a$ be an element of a $C^*$-algebra $\scrA$
with $0\leq a\leq 1$.
Then there is a unique element~$b\in\scrA$ 
with, $0\leq b\leq 1$,
$ab=ba$,
and~$(1-b)^2 = 1-a$.
To be more specific,
$b$ is the limit of
the sequence $b_0\leq b_1\leq \dotsb$
given by $b_0=0$ and $b_{n+1} = \frac{1}{2}(a+b_n^2)$.
Moreover,
if~$c\in\scrA$ commutes with~$a$, then~$c$ commutes with~$b$,
and if in addition $a\leq 1-c^2$ and $c^*=c$,
we have $b\leq 1-c$.
\begin{point}{Proof}%
When discussing $b_n$ it 
is convenient to write~$b_n \equiv q_n(a)$
where~$q_0,q_1,\dotsc$ are the polynomials over~$\R$ given by
$q_0=0$ and $q_{n+1}=\frac{1}{2}(x + q_n^2)$.
For example,
we have~$b_n\geq 0$, 
because all coefficients of~$q_n$ are all positive,
and $a,a^2,a^3,\dotsc$ are positive by~\sref{cstar-pos-power}.
With a similar argument we can see that
 $b_0 \leq b_1\leq b_2\leq \dotsb$.
Indeed, 
the coefficients of~$q_{n+1}-q_n$
are positive,
by induction,
because
\begin{alignat*}{3}
q_{n+2}-q_{n+1} \ &=\ \textstyle \frac{1}{2}(x+ q_{n+1}^2)
\,-\, \textstyle\frac{1}{2}(x+q_n^2) \\
&=\ \textstyle\frac{1}{2}(q_{n+1}^2- q_n^2) \\
&=\ \textstyle\frac{1}{2}(q_{n+1}+q_n)(q_{n+1}-q_n) \\
&=\ (q_n+\textstyle\frac{1}{2}(q_{n+1}-q_n))(q_{n+1}-q_n),
\end{alignat*}
has positive coefficients
if~$q_{n+1}-q_n$ has positive coefficients,
and $q_1-q_0\equiv \frac{1}{2}x$ clearly has positive coefficients.
Hence~$b_{n+1}-b_{n} = q_{n+1}(a)- q_n(a)$ is positive.
(Note that we have carefully avoided
using the fact here that the product of positive 
commuting elements is positive,
which is not available to us until~\sref{ineq-square-root}.)

Let us now show that~$b_1\leq b_2\leq \dotsb$ converges.
Let~$n\geq N$ from~$\N$ be given.
Since the coefficients of $q_n-q_N$ are positive,
and $\|a\|\leq 1$,
the triangle inequality gives us
$\|b_n-b_N\|\equiv \|(q_n-q_N)(a)\|\leq q_n(1)-q_N(1)$,
and
so it suffices to 
show that the ascending sequence
 $q_0(1)\leq q_1(1)\leq \dotsb$
of real numbers
converges,
c.q.~is bounded.
Indeed,
we have $q_n(1)\leq 1$,
by induction,
because $q_{n+1}(1)\equiv \frac{1}{2}(1+q_n(1)^2)
\leq 1$ if $q_n(1)\leq 1$,
and clearly $0\equiv q_0(1)\leq 1$.

Let~$b$ be the limit of $b_0\leq b_1\leq\dotsb$.
Then~$b$ being the limit of positive elements
is positive
(see~\sref{positive-norm-closed}),
and if $c\in \scrA$ commutes with~$a$,
then $c$ commutes with all powers of~$a$,
and therefore with all~$b_n$,
and thus with~$b$.
Further, 
from the recurrence relation $q_{n+1} = \frac{1}{2}(a+q_n^2)$
we get $b=\frac{1}{2}(a+b^2)$,
and so $-a = -2b+b^2$, 
giving us  $(1-b)^2 = 1-2b+b^2 = 1-a$.

Let us prove that~$b\leq 1$.
To begin, note that~$\|b_n\|\leq 1$ for all~$n$, by induction,
because, $0\equiv \|b_0\|\leq 1$,
and if $\|b_n\|\leq 1$, then $\|b_{n+1}\|\leq \frac{1}{2}(\|a\|+\|b_n\|^2)
\leq 1$, since $\|a\|\leq 1$.
Since~$b_n\geq 0$, we get $-1\leq b_n\leq 1$ for all~$n$,
and so $b\leq 1$.

\begin{point}[square-commuting-monotone]%
Let us take a step back for the moment.
From what we have proven so far
we see that each positive $c\in\scrA$
is of the form $c\equiv d^2$ for some positive~$d\in\scrA$
which commutes with all~$e\in \scrA$ that commute with~$c$.

From this we can see that $c_1c_2\geq 0$
for  
 $c_1,c_2 \in\pos{\scrA}$
with $c_1c_2 = c_2c_1$.
Indeed, writing $c_i\equiv d_i^2$ with $d_i$ as above,
we have $d_1c_2=c_2d_1$ (because $c_1c_2=c_2c_1$), and thus 
$d_1d_2=d_2d_1$. It follows that $d_1d_2$ is self-adjoint,
and $c_1c_2 = (d_1d_2)^2$. Hence $c_1c_2\geq 0$.

We will also need the following corollary.
For~$c,d\in\pos{\scrA}$ with $c\leq d$ and $cd=dc$,
we have $c^2\leq d^2$.
Indeed, $d^2-c^2 \equiv d(d-c)+c(d-c)$
is positive by the previous paragraph.
\end{point}
\begin{point}[ineq-square-root]%
Let~$c\in\sa{\scrA}$ be such that~$ca=ac$ and  $a\leq 1-c^2$.
We must show that $b\leq 1-c$.
Of course,
since~$b$ is the limit of $b_1,b_2,\dotsc$,
it suffices to show that~$b_n\leq 1-c$,
and we'll do this by induction.
Since $0\leq c^2 \leq 1-a$,
 we have $\|c\|^2\leq \|1-a\|\leq 1$,
and so $-1\leq c\leq 1$.
Thus $b_0\equiv 0\leq 1-c$.
Now, suppose that~$b_n\leq 1-c$ for some~$n$.
Then $b_{n+1} = \frac{1}{2}(a+b_n^2)
\leq \frac{1}{2}( (1-c^2)+(1-c)^2) = 1-c$,
where we have used that $b_n^2 \leq (1-c)^2$,
because $b_n\leq 1-c$
by~\sref{square-commuting-monotone}.
\begin{point}%
We can now show that~$b$ is unique.
Let~$b'\in \scrA$ with $0\leq b'\leq 1$,
 $b'a=ab'$ and $(1-b')^2=1-a$ be given;
we must prove that $b'=b$.
Note that $b'\leq 1$,
because $\|1-b'\|^2=\|1-a\|\leq 1$,

From $a=1-(1-b')^2$,
we immediately get $b \leq 1-(1-b')=b'$ by~\sref{ineq-square-root}.
For the other direction,
note that
$(1-b')^2= (1-b)^2 \equiv (1-b'+(b'-b))^2 = (1-b')^2+2(1-b')(b'-b)+(b'-b)^2$,
which gives $0=2(1-b')(b'-b)+(b'-b)^2$.
Now, since~$1-b'$ and $b'-b$ are positive,
and commute, we see that $(1-b')(b'-b)$ is positive 
by~\sref{ineq-square-root}, and so 
 $0=2(1-b')(b'-b)+(b'-b)^2\geq (b'-b)^2 \geq 0$,
which entails $(b'-b)^2=0$, and so $\|(b'-b)^2\|=\|b'-b\|^2=0$,
yielding $b=b'$.\qed
\end{point}
\end{point}
\TODO{Thank Sander}
\end{point}
\end{point}
\begin{point}[sqrt]{Exercise}%
Let~$a$ be a positive element of a $C^*$-algebra~$\scrA$.
Show that there is a unique 
positive element of~$\scrA$
denoted by $\Define{\sqrt{a}}$ with $\smash{\sqrt{a}^2}=a$
and $a\sqrt{a}=\sqrt{a}a$.
Show that if~$c\in\scrA$ commutes with~$a$,
then $c\sqrt{a}=\sqrt{a}c$,
and if in addition $c^*=c$ and $c^2\leq a$,
then $c\leq \sqrt{a}$.
Using this, verify the following facts.
\begin{enumerate}
\item
If~$a,b\in \scrA$ are positive,
and~$ab=ba$,
then $ab\geq 0$.

\item
Let~$a\in\pos{\scrA}$.
If $b,c\in \sa{\scrA}$ commute with~$a$,
then $b\leq c$ implies $ab\leq ac$.
\end{enumerate}
\end{point}
\begin{point}{Definition}
Given a self-adjoint element~$a$ of a $C^*$-algebra $\scrA$,
we write
\begin{equation*}
\textstyle
\Define{\left|a\right|}\ :=\ \sqrt{a^2}
\qquad
\Define{\pos{a}}\ :=\ \frac{1}{2}(\left|a\right| + a)
\qquad
\Define{a_{-}}\ :=\ \frac{1}{2}(\left|a\right| - a).
\end{equation*}
We call $a_+$ the \Define{positive part} of~$a$,
and $a_-$ the \Define{negative part}.
\end{point}
\begin{point}{Exercise}%
Let~$a$ be a self-adjoint element of a
 $C^*$-algebra $\scrA$.
\begin{enumerate}
\item
Show that $-\left|a\right| \leq a \leq \left| a \right|$,
and $\|\,\left|a\right|\,\|= \|a\|$.
\item
Prove that $a_+$ and $a_-$ are positive,  $a=a_+-a_-$
and $a_+a_-=a_-a_+=0$.
\item
One should not read too much into the notation
$\left|\,\cdot\,\right|$
in the non-commutative case:
give an example of
self-adjoint elements~$a$ and~$b$ of a $C^*$-algebra with
 $\left|a+b\right|\nleq \left|a\right|+ \left|b\right|$.

(Hint: one may take  
$a=\frac{1}{2}\left(\begin{smallmatrix}1 & 1 \\ 1 & 1\end{smallmatrix}\right)$
and $b=-\left(\begin{smallmatrix}1 & 0 \\ 0 & 0 \end{smallmatrix}\right)$.)
\end{enumerate}
\end{point}
\begin{point}[astara-positive]{Lemma}%
Given an element $a$ of a $C^*$-algebra $\scrA$,
we have $a^*a\geq 0$.
\begin{point}{Proof}%
Writing $b:=a((a^*a)_-)^{\nicefrac{1}{2}}$,
we have $b^*b=-((a^*a)_-)^2\leq 0$,
and so $b=0$ by~\sref{astara-non-neggative}.
Hence~$a^*a_-=0$ giving us $a^*a=(a^*a)_+\geq 0$.\qed
\end{point}
\end{point}
\begin{point}[cstar-positive-final]{Exercise}%
Round up our results regarding positive elements
to 
prove that
the following are equivalent
for a self-adjoint element $a$ of a $C^*$-algebra.
\begin{enumerate}
\item 
$a$ is positive, that is,  $\|a-t\|\leq t$
for some $t\geq \frac{1}{2}\|a\|$;
\item
$\|a-t\|\leq t$ for all~$t\geq \frac{1}{2}\|a\|$;
\item
$a\equiv b^2$ for some self-adjoint $b\in\scrA$;
\item
$a\equiv c^* c$ for some $c\in\scrA$;
\item
$\spec(a)\subseteq [0,\infty)$.
\end{enumerate}
\end{point}
\end{parsec}

\begin{parsec}%
\begin{point}%
The interaction between the multiplication and order
on a $C^*$-algebra can be subtle
as we have seen in~\TODO{}.  However,
when the $C^*$-algebra is commutative
almost all peculiarities disappear,
as we see in this paragraph.
This is to be expected
as we will see that any commutative $C^*$-algebra
is isomorphic to a $C^*$-algebra
of functions on a compact Hausdorff space.
\end{point}
\begin{point}{Exercise}%
Let~$\scrA$ be a \emph{commutative} $C^*$-algebra.
Let~$a,b,c\in\sa{\scrA}$.
\begin{enumerate}
\item
Show that $\left| a\right|$ is the supremum of~$a$ and~$-a$
in~$\sa{\scrA}$.
\item
Show that if~$a$ and $b$ have a supremum, $a\vee b$, in $\sa{\scrA}$,
then~$c+a\vee b$ is the supremum of~$a+c$ and $b+c$.
\item
Show that~$\sa{\scrA}$ is a \Define{Riesz space},
that is,  a lattice ordered vector space.\\
(Hint: prove that $\frac{1}{2}(a+b+\left|a-b\right|)$
is the supremum of~$a$ and~$b$ in~$\sa{\scrA}$.)
\end{enumerate}
\end{point}
\begin{point}[riesz-decomposition-lemma]{Exercise}%
Prove the \Define{Riesz decomposition lemma}:\\
For positive elements~$a,b,c$ of a commutative $C^*$-algebra~$\scrA$
with~$c\leq a+b$
we have $c\equiv a'+b'$
where  $0\leq a'\leq a$ and $0\leq b'\leq b$.
\end{point}
\end{parsec}

\begin{parsec}%
\begin{point}%
Now that we have have a firm grip
on the positive elements of a $C^*$-algebra
we turn to what is perhaps the most important
fact about commutative $C^*$-algebras:
that they are isomorphic to $C^*$-algebras
of continous functions on a compact Hausdorff space,
via the \emph{Gelfand representation}.

\TODO{Show that $\spec(\scrA)$ is a compact Hausdorff space.}
\end{point}
\begin{point}{Setting}%
In this paragraph,
$\scrA$ is a commutative $C^*$-algebra.
\end{point}
\begin{point}[gelfand-representation]{Definition}%
The \Define{spectrum} of~$\scrA$,
denoted by \Define{$\spec(\scrA)$},
is the set of all miu-maps $f\colon \scrA\to \C$.
We endow~$\spec(\scrA)$
with the topology of pointwise convergence.

The \Define{Gelfand representation}
of~$\scrA$
is the miu-map~$\gamma\colon \scrA\to C(\spec(\scrA))$
given by $\gamma(a)(f)=f(a)$.
\begin{point}[gelfand-representation-basic]{Exercise}%
Verify that 
 the map $\spec(\scrA)\to \C,\ f\mapsto f(a)$ is indeed
continuous for every~$a\in\scrA$,
and that~$\gamma$ miu.
\end{point}
\end{point}
\begin{point}%
Our program for this paragraph is to show that
the Gelfand representation~$\gamma$ is 
a miu-isomorphism.
In fact,
we will show that it gives the unit
of an equivalence between the category of commutative $C^*$-algebras
and the opposite of the category of compact Hausdorff spaces.
The first hurdle we take is the injectivity of~$\gamma$
--- that there are sufficiently many
points in the spectrum of a commutative $C^*$-algebra,
so to speak ---,
and involves Zorn's Lemma,
naturally,
but also the following special type of order ideal.
\end{point}
\begin{point}{Definition}%
A \Define{Riesz ideal} of~$\scrA$
is an order ideal~$I$
such that $a\in I\cap\sa{\scrA}\implies \left|a\right|\in I$.
A \Define{maximal Riesz ideal}
is a proper Riesz ideal which is maximal among
proper Riesz ideals.
\begin{point}{Remark}%
\TODO{Make remark on why Riesz ideals instead of $*$-ring ideals.}
\end{point}
\end{point}
\begin{point}[riesz-ideal-ring-ideal]{Lemma}%
Let~$I$ be a Riesz ideal of~$\scrA$.
For all~$a\in \scrA$ and $x\in I$ we have $ax\in I$.
\begin{point}{Proof}%
Since~$x=\Real{x}+i\Imag{x}$,
it suffices to show that~$a\Real{x}\in I$ and $a\Imag{x}\in I$.
Note that~$\Real{x},\Imag{x}\in I$,
so we might as well assume that~$x$ is self-adjoint to begin with.
Similarly, using that
 $\pos{x}\in I$ (because $\pos{x}=\frac{1}{2}(\left|x\right|+x)$
and~$\left|x\right|\in I$) and $x_-\in I$,
we can reduce the problem to the case that~$x$ is positive.
We may also assume that~$a$ is self-adjoint.
Now, since~$x\geq 0$ and $-\|a\|\leq a\leq \|a\|$,
we have $-\|a\|x \leq ax\leq \|a\|x$
by~\sref{sqrt},
and so~$ax\in I$,
because $\|a\|x\in I$.\qed
\end{point}
\end{point}
\begin{point}[riesz-ideal-basic]{Exercise}%
Verify the following facts about Riesz ideals.
\begin{enumerate}
\item
The least Riesz ideal that contains a self-adjoint element~$a$
of~$\scrA$ is
\begin{equation*}
(a)_m\ :=\ \{\,b\in \scrA\colon\, 
\exists n\in \N\,[\ \left|\Real{b}\right|,\,\left|\Imag{b}\right|
\,\leq\, n\left|a\right| \ ]\,\}.
\end{equation*}
Moreover,  $(a)_m=\scrA$ iff $a$ is invertible,
and we have~$(a)=(a)_m$ when~$a\geq 0$
(where $(a)$ is the least order ideal that contains~$a$,
see~\sref{order-ideal-basic}).
For non-positive~$a$, however, we may have~$(a)\neq (a)_m$.
\item
$I+J$ is a Riesz ideal of~$\scrA$
when $I$ and~$J$ are Riesz ideals. (Hint: use~\sref{riesz-decomposition-lemma}.)
But~$I+J$ might not be an order ideal
when~$I$ and~$J$ are order ideals.

\item
Each proper Riesz ideal is contained in a maximal Riesz ideal.
\end{enumerate}
\end{point}
\begin{point}[maximal-riesz-ideal-maximal-order-ideal]{Lemma}%
A maximal Riesz ideal~$I$ of~$\scrA$
is a maximal order ideal.
\begin{point}{Proof}%
Let~$J$ be a proper order ideal with $I\subseteq J$.
We must show that $J=I$.
Let~$a\in J$ be given;
we must show that $a\in I$.
Since~$\Real{a},\Imag{a}\in J$,
it suffices to show that~$\Real{a},\Imag{a}\in I$,
and so we might as well assume that~$a$ is self-adjoint
to begin with.
Similarly,
since~$\left|a\right|\in J$,
and it suffices to show that~$\left|a\right|\in I$,
because then $-\left|a\right|\leq a\leq \left|a\right|$
entails $a\in I$,
we might as well assume that~$a$ is positive.

Note that the least ideal~$(a)$ that contains~$a$
is also a Riesz ideal by~\sref{riesz-ideal-basic}.
Hence  $I+(a)$ is a Riesz ideal by~\sref{riesz-ideal-basic}
Since~$a\in J$, we have $(a)\subseteq J$,
and so~$I+(a)\subseteq J$ is proper.
It follows that $a\in I+(a)=I$ by maximality of~$I$.\qed
\end{point}
\end{point}
\begin{point}[riesz-ideal-miu-map]{Lemma}%
Let~$I$ be a maximal Riesz ideal of~$\scrA$.
Then there is a miu-map $f\colon \scrA\to \C$
with $\ker(f)=I$.
\begin{point}{Proof}%
Since~$I$ is a maximal order ideal 
by~\sref{maximal-riesz-ideal-maximal-order-ideal},
there is a piu-map $f\colon \scrA\to \C$
with~$\ker(f)=I$ by~\sref{maximal-ideal-state}.
It remains to be shown that~$f$ is multiplicative.
Let~$a,b\in \scrA$ be given;
we must show that $f(ab)=f(a)f(b)$.
Surely, since~$f$ is unital,
we have $f(b-f(b))=f(b)-f(b)=0$,
an so $b-f(b)\in \ker(f)\equiv I$.
Now, since~$I$ is a Riesz ideal,
we have $a(b-f(b))\in I\equiv \ker(f)$ by~\sref{riesz-ideal-ring-ideal},
and so~$0=f(\,a(b-f(b))\,)=f(ab)-f(a)f(b)$.
Hence~$f$ is multiplicative.\qed
\end{point}
\end{point}
\begin{point}[spectrum-miu]{Lemma}%
For each self-adjoint~$a\in\scrA$ 
we have $\spec(a)=\{f(a)\colon f\in\spec(\scrA)\}$.
\begin{point}{Proof}%
Let~$f\in \spec(\scrA)$,
and~$a\in\sa{\scrA}$.
If~$a-f(a)$ would have had an inverse~$b$ in~$\scrA$,
then~$f(b)$ would be an inverse
of~$f(\,a-f(a)\,)=0$ in~$\C$,
which does not exist. Thus~$a-f(a)$ is not invertible,
that is, $f(a)\in \spec(a)$.

Let~$\lambda\in\spec(a)$.
We must find~$f\in \spec(\scrA)$ with~$f(a)=\lambda$.
Since~$a-\lambda$ is not invertible
and self-adjoint	,
the Riesz ideal $(a-\lambda)_m$ is proper
and so there is a maximal Riesz ideal~$I$ that contains~$a-\lambda$
(see~\sref{riesz-ideal-basic}).
By~\sref{riesz-ideal-miu-map}
there is a miu-map $f\in \spec(\scrA)$ with~$\ker(f)=I$.
For this~$f$,
we have, $f(\lambda-a)=0$
(since $\lambda-a\in I$),
and so~$f(a)=\lambda$.\qed

\TODO{What about non-self-adjoint~$a$?}
\end{point}
\end{point}
\begin{point}[gelfand-representation-isometry]{Exercise}%
Prove that $\|\gamma(a)\|=\|a\|$
for each~$a\in\scrA$.
(Hint: first assume that~$a$ is self-adjoint,
and use \sref{spectrum-miu} and~\sref{norm-spectrum}.
For the general case,
use the $C^*$-identity.)

Conclude that the Gelfand representation $\gamma\colon \scrA\to C(\spec(\scrA))$
is injective,
and that its range $\{\gamma(a)\colon a\in\scrA\}$
is a $C^*$-subalgebra of~$C(\spec(\scrA))$.
\end{point}
\begin{point}%
To show that~$\gamma$ is surjective,
we use the following special case of
the Stone--Weierstra\ss{} theorem. 
\end{point}
\begin{point}[stone-weierstrass]{Theorem}%
Let~$X$ be a compact Hausdorff space,
and let~$\scrS$ be a $C^*$-subalgebra of~$C(X)$
which `separates the points of~$X$',
that is, for all~$x,y\in X$
there is~$f\in \scrS$ with $f(x)\neq f(y)$.
Then~$\scrS=C(X)$.
\begin{point}{Proof}%
Let~$g\in \pos{C(X)}$ and $\varepsilon >0$.
To prove that~$\scrS=C(X)$,
it suffices to show that~$g\in \scrS$,
and for this,
it suffices to find~$f\in \scrS$ with $\|f-g\|\leq \varepsilon$,
because~$\scrS$ is closed.
It is convenient to assume that~$g(x)> 0$ for all~$x\in X$,
which we may, without loss of generality,
by replacing~$g$ by~$1+g$.

\begin{point}[stone-weierstrass-1]%
Let~$x,y\in X$.
We know there is~$f\in \scrS$ with $f(x)\neq f(y)$.
Note that we can assume that~$f(x)=0$ (by replacing~$f$ by~$f-f(x)$),
and that~$f$ is self-adjoint (by replacing~$f$
by either~$\Real{f}$ or~$\Imag{f}$),
and that~$f$ is positive
(by replacing~$f$ by~$f_+$ or~$f_-$),
and that~$f(y)=g(y)>0$
(by replacing $f$ by $\frac{g(y)}{f(y)} f$),
and that~$f\leq g(y)$
(by replacing $f$ by $f\wedge g(y)$).
\end{point}
\begin{point}%
Let~$y\in X$ be given.
We will show that there is~$f\in\scrS$
with $0\leq f\leq g+\varepsilon$
and~$f(y)=g(y)$.
Indeed,
since~$g$ is continuous
there is an open neighbourhood~$V$ of~$y$
with~$g(y) \leq  g(x)+\varepsilon$
for all~$x\in V$.
For each~$x\in X\backslash V$ there is $f_x \in [0,f(y)]_{\scrS}$
with $f_x(x)=0$ and~$f_x(y)=g(y)$ by~\sref{stone-weierstrass-1}.
Since the open subsets
$U_x := \{\,z\in X\colon f_x(z)\leq \varepsilon\,\}$
with~$x\in X\backslash V$
form an open cover of the closed (and thus compact) subset $X\backslash V$,
there are $x_1,\dotsc,x_N\in X\backslash U$
with $U_{x_1}\cup\dotsb\cup U_{x_N}=X$.
Define $f:=f_{x_1}\wedge \dotsb \wedge f_{x_N}$.
Then~$f\in \scrS$, $0\leq f\leq g(y)$, $f(y)=g(y)$,
and $f(x)\leq \varepsilon$
for every~$x\in X\backslash U$.

We claim that $f\leq g+\varepsilon$.
Indeed,
if~$x\in X\backslash U$,
then $f(x)\leq \varepsilon\leq g(x)+\varepsilon$.
If~$x\in U$,
then $f(x)\leq g(y)\leq g(x)+\varepsilon$
(by definition of~$U$).
Hence $f\leq g+\varepsilon$.
\end{point}
\begin{point}%
Thus for each~$y\in X$
there is $f_y\in \scrS$ with $0\leq f_y \leq g+\varepsilon$
and~$f_y(y)=g(y)$.
Since~$f_y$ is continuous at~$y$,
and~$f_y(y)=g(y)$,
there is an open neighbourhood~$U_y$ of~$y$
with $g(y)-\varepsilon\leq f_y(x)$
for all~$x\in U_y$.
Since these open neighbourhoods cover~$X$,
and~$X$ is compact,
there are $y_1,\dotsc,y_N\in X$
with $U_{y_1}\cup\dotsb\cup U_{y_N} = X$.
Define $f:=f_{y_1}\vee \dotsb\vee f_{y_N}$.
Then~$f\in\scrS$,
and $g-\varepsilon \leq f\leq g+\varepsilon$,
giving $\|f-g\|\leq \varepsilon$.\qed
\end{point}
\end{point}
\end{point}
\begin{point}[spectrum-calg-compact-hausdorff]{Lemma}%
The spectrum $\spec(\scrA)$ of~$\scrA$ is a compact Hausdorff space.
\begin{point}{Proof}%
Since for each~$a\in \scrA$
and~$f\in \spec(\scrA)$
we have  $\|f(a)\|\leq \|a\|$ 
by~\TODO{},
we see that~$f(a)$ is an element of the compact set
$\{\,z\in \C\colon\, \left|z\right|\leq\|a\|\,\}$,
and so~$\spec(\scrA)$ is a subset of
\begin{equation*}
\textstyle
\prod_{a\in \scrA}\, \{\,z\in \C\colon\, \left|z\right|\leq \|a\|\,\},
\end{equation*}
which is a compact Hausdorff space
(by Tychonoff's theorem \TODO{ref}, under the product topology
it inherits
from the space of all functions $\scrA\to \C$).
So to prove that~$\spec(\scrA)$
is a compact Hausdorff space,
it suffices to show that~$\spec(\scrA)$
is closed.
In other words,
we must show that if~$f\colon \scrA\to \C$
is the pointwise limit of a net of miu-maps $(f_i)_i$,
then~$f$ is a miu-map as well.
But this is easily achieved
using the continuity of addition, involution and multiplication on~$\C$,
because, for instance, 
for~$a,b\in\scrA$, we have $f(ab)
= \lim_i f_i(ab)=\lim_i f_i(a)f_i(b)
 = (\lim_i f_i(a))\,(\lim_i f_i(b))
= f(a) \,f(b)$.\qed
\end{point}
\end{point}
\begin{point}[gelfand]{Gelfand's Representation Theorem}%
For a commutative $C^*$-algebra~$\scrA$,
the Gelfand representation, 
 $\gamma\colon \scrA\to C(\spec(\scrA))$
defined in~\sref{gelfand-representation}
is a miu-isomorphism.
\begin{point}{Proof}%
We already know that~$\gamma$ is an injective miu-map
(see~\sref{gelfand-representation-basic} 
and~\sref{gelfand-representation-isometry}).
So to prove that~$\gamma$ is a miu-isomorphism,
it remains to be shown that~$\gamma$ is surjective.
Since~$\spec(\scrA)$ is a compact Hausdorff space 
(by~\sref{spectrum-calg-compact-hausdorff}),
and~$\gamma(\scrA)\equiv \{\gamma(a)\colon a\in \scrA\}$
is a $C^*$-subalgebra of~$C(\spec(\scrA))$
(by~\sref{gelfand-representation-isometry}),
it suffices to show that~$\gamma(\scrA)$
separates the points of~$\spec(X)$
by~\sref{stone-weierstrass}.
This is obvious,
because
for~$f,g\in \spec(\scrA)$ with~$f\neq g$
there is~$a\in \scrA$ with~$f(a)\equiv \gamma(a)(f)
\neq \gamma(a)(g)\equiv g(a)$.\qed
\end{point}
\end{point}
\end{parsec}

\begin{parsec}%
\begin{point}%
As a cherry on the cake,
we use Gelfand's representation theorem~\sref{gelfand}
to get equivalence between the categories $\op{(\cCstar{miu})}$
and~$\CH$.

To set the stage,
we extend $X\mapsto C(X)$ to a functor
$\CH\to \op{(\cCstar{miu})}$
by sending a continuous function~$f\colon X\to Y$
to the miu-map $C(f)\colon C(Y)\to C(X)$
given by~$C(f)(g)=g\circ f$ for $g\in C(Y)$,
and we extend $\scrA\mapsto \spec(\scrA)$
to a functor $\spec\colon \op{(\cCstar{miu})}\to \CH$
by sending a miu-map $\varphi \colon \scrA\to\scrB$
to the continuous map~$\spec(\varphi)\colon \spec(\scrB)\to\spec(\scrA)$
given by~$\spec(\varphi)(f)=f\circ \varphi$.

The Gelfand representations $\gamma_\scrA\colon \scrA\to C(\spec(\scrA))$
form a natural isomorphism
from $C\circ \spec$ to the idenity functor on~$\op{(\cCstar{miu})}$.
So to get an equivalence,
it suffices to find a natural isomorphism
from the identity on~$\CH$ to~$\spec\circ C$,
which is provided by the following lemma.
\end{point}
\begin{point}{Lemma}%
Let~$X$ be a compact Hausdorff space,
and let~$\tau \colon C(X)\to \C$ be a miu-map.
Then there is~$x\in X$ with $\tau(f)=f (x)$
for all~$f\in C(X)$.
\begin{point}{Proof}%
Define
$Z\,= \, \{\,x\in X\colon \ h(x)\neq 0\text{ for some~$h\in \pos{C(X)}$
with $\tau(h)=0$}\,\}$.
We'll prove~$X\backslash Z$ contains
exactly one point, $x_0$, and $\tau(f)=f(x_0)$ for all~$f$.
\begin{point}%`
To see that~$X\backslash Z$ contains no more than one point,
let~$x,y\in X$ with $x\neq y$ be given;
we will show that either~$x\in Z$ or~$y\in Z$.
By the usual topological trickery,
we can find~$f,g\in \pos{C(X)}$
with $fg=0$, $f(x)=1$ and~$g(y)=1$.
Then~$0=\tau(fg)=\tau(f)\,\tau(g)$,
so either~$\tau(f)=0$ (and~$x\in Z$), or~$\tau(g)=0$
(and~$y\in Z$).

That~$X\backslash Z$ is non-empty
follows from the following result (by taking~$f=1$).
\end{point}
\begin{point}[multiplicative-state-on-cx-1]%
For~$f\in \pos{C(X)}$
with~$f(x)> 0 \implies x\in Z$ for all~$x\in X$
we have~$\tau(f)=0$.
Indeed, for each~$x\in X$ with~$f(x)>0$
(and so~$x\in Z$)
we can find~$h\in \pos{C(X)}$
with $\tau(h)=0$ and~$h(x)\neq 0$.
Then~$f(x)< g(x)$
and~$\tau(g)=0$
for $g:=(\frac{f(x)}{g(x)}+1)h$.
By compactness,
we can find $g_1,\dotsc,g_N\in \pos{C(X)}$
with~$\tau(g_n)=0$,
such that for every~$x\in X$
there is~$n$ with $g(x)<f_n(x)$.
Then writing $g:=g_1\vee \dotsb \vee g_N$,
we have $0\leq f\leq g$ and~$\tau(g)=0$
(by~\TODO{miu-map preserves infima}).
It follows that~$\tau(f)=0$.
\end{point}
\begin{point}%
We now know that~$X\backslash Z$ contains exactly
one point, say~$x_0$.
To see that~$\tau(f)=f(x_0)$
for~$f\in C(X)$,
write $g:=(f-f(x_0))^*(f-f(x_0))$
and note that $g(x)>0\implies x\neq x_0\implies  x\in Z$.
Thus by~\sref{multiplicative-state-on-cx-1},
we get $0=\tau(g)=\left|\tau(f)-f(x_0)\right|^2$,
and so $\tau(f)=f(x_0)$.\qed
\end{point}
\end{point}
\begin{point}{Exercise}%
Let~$X$ be a compact Hausdorff space.
Show that for every~$x\in X$
the map $\delta_x\colon C(X)\to \C,\ f\mapsto f(x)$
is miu.
Show that the map $X\to \spec(C(X)),\ x\mapsto \delta_x$
is a continuous bijection
onto a compact Hausdorff space,
and thus a homeomorphism.
\end{point}
\end{point}
\end{parsec}


%
%  Cauchy--Schwarz for positive functionals
%
\begin{parsec}[cstar-cs]%
\begin{point}%
We will now prove that every $C^*$-algebra
is isomorphic
to a $C^*$-algebra
of bounded operators.
We will generalize this construction
in Chapter~\TODO{Paschke}.
\end{point}
\begin{point}{Lemma}%
Let~$\scrA$ be a $C^*$-algebra,
and let~$f\colon \scrA\to \C$ be a pi-map.\\
Then, for all~$a,b\in\scrA$,\quad
$\left|f(a^*b)\right|^2 \ \leq\ f(a^*a)\,f(b^*b)$.
\begin{point}{Proof}%
Apply~\sref{cs}
to the inner product on~$\scrA$ given 
by $\left<a,b\right>=f(a^*b)$.
(Note that $f(a^*a)\equiv \left<a,a\right>\geq 0$,
because $a^*a$ is positive
by~\sref{astara-positive}.)\qed
\end{point}
\end{point}
\end{parsec}
\end{document}
