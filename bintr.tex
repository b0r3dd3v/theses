\documentclass[b]{subfiles}
\begin{document}

\chapter{Introduction}
\begin{parsec}[bas-first-parsec]%
\begin{point}%
This thesis is a mathematical study of quantum computing consisting
    of two related, but independant parts:
    chapter~\ref{chapter1} on \emph{dilations} and~\ref{chapter2}
    ``diamond, andthen, dagger'' on \emph{effectus theory}.
Both parts can be read separetely
    and feature a comprehensive introduction of their own.
For now, we focus on what they have in common.
\begin{point}%
Both parts study the category of von Neumann algebras
    (with ncpsu-maps\footnote{%
        An abbreviation for contractive normal completely
        positive linear maps.} between them).
A von Neumann algebra models a (possibly infinite-dimensional)
    quantum datatype, for instance:
\begin{center}
    \begin{tabular}{llll}
        $M_2$ & a qubit & $M_3 \otimes M_3$ & an (ordered) pair of qutrits \\
        $M_3$ & a qutrit & $\C^2$ & a bit \\
        $M_2 \oplus M_3$ & a qubit or a qutrit
        & $\scrB(\ell^2)$ & quantum integers
\end{tabular}
\end{center}
The ncpsu-maps between them represent physical
    (but not necessarily computable) operations
        in the opposite direction:
\begin{enumerate}
\item
    \emph{Measure a qubit in the computational basis} \quad
            ($\mathrm{qubit} \to \mathrm{bit}$)\\
        $\C_2 \rightarrow M_2,$ \quad 
        $(\lambda,\mu) \ \mapsto\ \lambda \ketbra{0}{0} + \mu \ketbra{1}{1}$
\item
    \emph{Discard the second qubit} \quad
            ($\mathrm{qubit} \times \mathrm{qubit}\to \mathrm{qubit}$) \\
        $M_2 \rightarrow M_2 \otimes M_2$, \quad
        $a \ \mapsto \ a \otimes 1$
\item
    \emph{Apply a Hadamard-gate to a qubit} \quad
            ($\mathrm{qubit} \to \mathrm{qubit}$) \\
        $M_2 \rightarrow M_2$, \quad
        $a \ \mapsto \ H^*a H \otimes 1$, \quad where~$H = \frac{1}{\sqrt{2}}\begin{spmatrix}
            1 & 1 \\ 1 & -1
        \end{spmatrix}$
\item
    \emph{Initialize a qutrit as 0} \quad
            ($\mathrm{empty} \to \mathrm{qutrit}$) \\
        $M_3 \rightarrow \C$, \quad
        $a \ \mapsto \ \bra{0}a\ket{0}$
\end{enumerate}
\end{point}
\begin{point}%
The first part of this thesis starts with a concrete question:
    does every ncpsu-map between von Neumann algebras
    admit a Stinespring-like dilation?
We will see that indeed it does.
The insights attained pondering this question lead naturally,
        as it so often does in mathematics, to other new results:
    for instance, we learn more about  the tensor product of
    dilations~\sref{paschke-tensor}
        and the pure maps associated to them~\sref{paschke-pure}.
But perhaps the most interesting by-catch
    are the contributions to the theory
    of self-dual Hilbert C$^*$-modules over a von Neumann algebra,
    which underly these dilation.
\end{point}
\begin{point}%
In contrast, the second part of this thesis starts
    with the abstract (or rather: vague) question:
    what is unique about the category of von Neumann algebras?
The goal is to find a set of reasonable axioms,
    which picks out the category of von Neumann algebras
    among all other categories.
In other words: a reconstruction of quantum theory with categorical axioms.
    Unfortunately, we do not reach this goal here.\footnote{%
        Building on this thesis,
            van de Wetering recently found a reconstruction
            of finite-dimensional quantum theory.\cite{wetering}}
However, by studying von Neumann algebras
    in this dogmatic straitjacket
    we are forced, as again is often the case in mathematics,
    upon several new notions and results,
    which are of independant interest,
    such as, for instance,~$\diamond$-adjointness,
    pure maps and the existence of their dagger.
\end{point}
\end{point}
\begin{point}{Twin theses}%
A substantial part of the research presented here,
    has been performed in close collaboration with
    my twin brother Abraham Westerbaan.
Our results are interwoven to such a degree,
    that seperating them would be difficult
    and, more importantly, detrimental to the clarity of presentation.
Thus we decided to write our theses as two volumes in a series:
    Abraham's thesis~\cite{bram} covers the preliminaries on von Neumann
    algebras and, among other results, an axiomatization of
        the sequential product~$b\mapsto \sqrt{a}b\sqrt{a}$
        --- the present thesis
            builds upon those results,
            in its study of dilations and effectus theory.
\end{point}
\begin{point}{Preliminaries}%
For the first part of this thesis,
    we only assume knowledge of, what is considered to be,
    the basic theory of von Neumann algebras.
Unfortunately, this `basic theory' takes most introductory texts
    close to a thousand pages to develop.
This situation has led my twin brother to ambitiously
     redevelop the basic theory of von Neumann algebras
     tailored to the needs our theses (and others
     in the field of quantum computing).
The result is impressive:
    in his thesis, Abraham presents~\cite{bram}
    a clear, comprehensive and self-contained development of the basic
    theory of von Neumann algebras in less than 200 pages.
For a traditional treatment,
    I can wholeheartedly recommend the books of Kadison and Ringrose \cite{kr}.
For the second part,
    we will only require knowledge of category theory
        that is covered in any introductory text, like \cite{awodey}.
\end{point}
\begin{point}[bintr-example]{Writing style}%
Page numbers have been replaced by paragraph numbers:
    for instance, \sref{bintr-example} points to this paragraph.
Numbers below~\sref{bas-first-parsec}
    refer to paragraphs in my twin brother's thesis~\cite{bram}.
Quite unconventionally, this thesis includes exercises.
Most exercises replace straight-forward, but repetitive, proofs.
Other exercises, marked with an asterix *,
    are harder and explore tangents.
In both cases, these exercises aim to aid the reader;
    not in the least by reducing clutter.
\end{point}
\begin{point}{Advertisements}%
This thesis is based on the publications~\cite{wwpaschke,westerbaan2016universal,cho2015quotient,statesofconvexsets}.
During my doctoral research,
    I also published the following papers,
    which although somewhat related,
    would've made this thesis too broad in scope if included.
\begin{enumerate}
    \item In \emph{A Kochen--Specker System has at least 22 Vectors} \cite{uijlen2016kochen},
        Uijlen and I justify the title building upon the work
        of Arends, Ouaknine and Wampler.
\item 
    In \emph{Unordered Tuples in Quantum Computation} \cite{bags}
        Furber and I argue that~$M_3 \oplus \C$ is the right algebra
        modelling an unordered pair of qubits.  Then we compute
        the algebra for an unordered~$n$-tuple of~$d$-level sytems.
\item
    In \emph{Picture-perfect QKD} \cite{kissinger2017picture}
        Kissinger, Tull and I show how to graphically derive a security
        bound on quantum key distribution using the continuity of Stinespring's dilation.
\item
    In \cite{schwabe2016solving}, Schwabe and I
            work out the details of a quantum circuit
            to break binary~$\mathcal{MQ}$ using Grover's algorithm.
\end{enumerate}
Next, I would like to mention some other approaches
        to study quantum theory mathematically,
        which are related in spirit and of which I am aware.
% In the last decade, the category of Hilbert spaces
%     with bounded operators has been studied extensively by dozens
%     of authors, which has led to (several) rich graphical
%     calculi. \cite{abramsky2004categorical}
% Instead
% 
%         \cite{dvurecenskij2013new}
\end{point}
\begin{point}{Acknowledgements}%
    
\end{point}
\end{parsec}

\end{document}

% vim: ft=tex.latex
