
\chapter{Introduction}
\begin{parsec}{1340}[bas-first-parsec]%
\begin{point}{10}%
This thesis is a mathematical study of quantum computing, concentrating
    on two related, but independent topics.
First up are \emph{dilations}, covered in chapter~\ref{chapter1}.
    In chapter \ref{chapter2}
    ``diamond, andthen, dagger''
    we turn to the second topic: \emph{effectus theory}.
Both chapters, or rather parts, can be read separately
    and feature a comprehensive introduction of their own.
For now, we focus on what they have in common.
\begin{point}{20}%
Both parts study the category of von Neumann algebras
    (with ncpsu-maps\footnote{%
        An abbreviation for \Define{n}ormal \Define{c}ompletely \Define{p}ositive contractive (i.e.~\Define{s}ub\Define{u}nital) linear maps.} between them).
A von Neumann algebra is a model for a (possibly infinite-dimensional)
    quantum datatype, for instance:
\begin{center}
    \begin{tabular}{lllll}
        $M_2$ & a qubit &\quad\qquad& $M_3 \otimes M_3$ & an (ordered) pair of qutrits \\
        $M_3$ & a qutrit && $\C^2$ & a bit \\
        $M_2 \oplus M_3$ & a qubit or a qutrit &
        & $\scrB(\ell^2)$ & quantum integers
\end{tabular}
\end{center}
The ncpsu-maps between these algebras represent physical
    (but not necessarily computable) operations
        in the opposite direction
        (as in the `Heisenberg picture'):
\begin{enumerate}
\item
    \emph{Measure a qubit in the computational basis} \quad
            ($\mathrm{qubit} \to \mathrm{bit}$)\\
        $\C_2 \rightarrow M_2,$ \quad 
        $(\lambda,\mu) \ \mapsto\ \lambda \ketbra{0}{0} + \mu \ketbra{1}{1}$,
        \quad using braket notation\footnote{%
            See e.g.~\cite[fig.~2.1]{nielsen2002quantum},
            so here
            $\ket{0}\equiv\begin{spmatrix}1 \\ 0 \end{spmatrix}$,
            $\bra{0}\equiv\begin{spmatrix}1 & 0 \end{spmatrix}$,
            $\ket{1}\equiv\begin{spmatrix}0 \\ 1 \end{spmatrix}$ and
                $\ketbra{0}{0} = \begin{spmatrix} 1 & 0
                \\ 0 & 0 \end{spmatrix}$.}
\item
    \emph{Discard the second qubit} \quad
            ($\mathrm{qubit} \times \mathrm{qubit}\to \mathrm{qubit}$) \\
        $M_2 \rightarrow M_2 \otimes M_2$, \quad
        $a \ \mapsto \ a \otimes 1$
\item
    \emph{Apply a Hadamard-gate to a qubit} \quad
            ($\mathrm{qubit} \to \mathrm{qubit}$) \\
        $M_2 \rightarrow M_2$, \quad
        $a \ \mapsto \ H^*a H $, \quad where~$H = \frac{1}{\sqrt{2}}\begin{spmatrix}
            1 & 1 \\ 1 & -1
        \end{spmatrix}$
\item
    \emph{Initialize a qutrit as 0} \quad
            ($\mathrm{empty} \to \mathrm{qutrit}$) \\
        $M_3 \rightarrow \C$, \quad
        $a \ \mapsto \ \bra{0}a\ket{0}$
\end{enumerate}
\end{point}
\spacingfix{}
\begin{point}{30}%
The first part of this thesis starts with a concrete question:
    does the famous  Stinespring dilation theorem~\cite{stinespring},
    a cornerstone of quantum theory,
    extend to every ncpsu-map between von Neumann algebras?
Roughly, this means that every ncpsu-map  splits
    in two maps of a particularly nice form.
We will see that indeed it does.
The insights attained pondering this question lead naturally,
        as they so often do in mathematics, to other new results:
    for instance, we learn more about  the tensor product of
    dilations (\sref{paschke-tensor})
        and the pure maps associated to them (\sref{paschke-pure}).
But perhaps the mathematically
    most interesting by-catch of pondering the question
    are the contributions (see \sref{overview-dils}) to the theory
    of self-dual Hilbert C$^*$-modules over a von Neumann algebra,
    which underlie these dilations.
\end{point}
\begin{point}{40}%
In contrast to the first part of this thesis,
    the second part starts
    with the abstract (or rather: vague) question:
    what characterizes the category of von Neumann algebras?
The goal is to find a set of reasonable axioms
    which pick out the category of von Neumann algebras
    among all other categories.
In other words: a reconstruction of quantum theory with categorical axioms
    (presupposing von Neumann algebras are the right model of quantum theory).
    Unfortunately, we do not reach this goal here.\footnote{%
        Building on this thesis,
            van de Wetering recently found a reconstruction
            of finite-dimensional quantum theory \cite{wetering}.}
However, by studying von Neumann algebras
    with this self-imposed restriction,
    one is led to, as again is often the case in mathematics,
    upon several new notions and results,
    which are of independent interest,
    such as, for instance,~$\diamond$-adjointness,
    pure maps and the existence of their dagger.
\end{point}
\end{point}
\begin{point}{50}{Twin theses}%
A substantial part of the research presented here
    has been performed in close collaboration with
    my twin brother Abraham Westerbaan.
Our results are interwoven to such a degree
    that separating them is difficult
    and, more importantly, detrimental to the clarity of presentation.
Therefore we decided to write our theses as two consecutive volumes
    (with consecutive numbering, more on that later).
    Abraham's thesis~\cite{bram} is the first volume
    and covers the preliminaries on von Neumann
    algebras and, among other results, an axiomatization of
        the sequential product~$b\mapsto \sqrt{a}b\sqrt{a}$.
The present thesis is the second volume and builds upon
    the results of Abraham's thesis,
            in its study of dilations and effectus theory.
Because of this clear division,
    a significant amount of Abraham's original work appears in my thesis
        and, vice versa, my work appears in his.
However, the large majority of the
        new results in each of our theses is our own.
\end{point}
\begin{point}{60}{Preliminaries}%
For the first part of this thesis,
    only knowledge is presumed of, what is considered to be,
    the basic theory of von Neumann algebras.
Unfortunately, this `basic theory' takes most introductory texts
    close to a thousand pages to develop.
This situation has led my twin brother to ambitiously
     redevelop the basic theory of von Neumann algebras
     tailored to the needs of our theses (and hopefully others
     in the field).
In his thesis, Abraham presents~\cite{bram}
    a clear, comprehensive and self-contained development of the basic
    theory of von Neumann algebras in less than 200 pages.
For a traditional treatment,
    I can wholeheartedly recommend the books of Kadison and Ringrose \cite{kr}.
For the second part of this thesis,
    only elementary category theory is required
    as covered in any introductory text, like e.g.~\cite{awodey}.
\end{point}
\begin{point}{70}[bintr-example]{Writing style}%
This thesis is divided into points,
    each a few paragraphs long.
Points are numbered with Roman numerals in the margin:
    this is point~\sref{bintr-example}.
Points  are organized into groups,
    which are assigned Arabic numerals,
    that appear at the start of the group in the margin.
The group of this point is~\sref{bas-first-parsec}.
Group numbers below~\sref{bas-first-parsec}
    appear in my twin brother's thesis~\cite{bram}.
Page numbers have been replaced by the
    group numbers that appear on the spread.\footnote{%
        A spread are two pages that are visible at the same time.}
This unusual system allows for
    very short references:
    instead of writing `see Theorem 2.3.4 on page 123',
    we simply write `see \sref{existence-paschke}'.
Even better: `see the middle of the proof of Theorem 2.3.4 on page 123'
    becomes the more precise `see \sref{paschke-uniqueness}'.

Somewhat unconventionally, this thesis includes exercises.
Most exercises replace straight-forward, but repetitive, proofs.
Other exercises, marked with an asterisk *,
    are harder and explore tangents.
In both cases, these exercises are not meant to force the reader into the
    role of a pupil, but rather to aid her, by reducing clutter.
Solutions to these exercises (and errata)
    can be found at the end of the online version of this text
    available on arXiv under
    \href{https://arxiv.org/abs/1803.01911}{1803.01911}.
\end{point}
\begin{point}{80}{Advertisements}%
This thesis is based on the publications~\cite{wwpaschke,westerbaan2016universal,cho2015quotient,statesofconvexsets}.
During my doctoral research,
    I also published the following papers,
    which although somewhat related,
    would've made this thesis too broad in scope if included.
\begin{enumerate}
    \item In \emph{A Kochen--Specker System has at least 22 Vectors} \cite{uijlen2016kochen},
        Uijlen and I justify the title building upon the work
        of Arends, Ouaknine and Wampler.
\item 
    In \emph{Unordered Tuples in Quantum Computation} \cite{bags}
        Furber and I argue that~$M_3 \oplus \C$ is the right algebra
        modeling an unordered pair of qubits.  Then we compute
        the algebra for an unordered~$n$-tuple of~$d$-level systems.
\item
    In \emph{Picture-perfect QKD} \cite{kissinger2017picture}
        Kissinger, Tull and I show how to graphically derive a security
        bound on quantum key distribution using the continuity of Stinespring's dilation.
\item
    In \cite{schwabe2016solving}, Schwabe and I
            work out the details of a quantum circuit
            to break binary~$\mathcal{MQ}$ using Grover's algorithm.
\end{enumerate}
In the second part of this thesis,
    we
 use the notion of effectus (introduced by Jacobs \cite{newdirections}),
    solely for studying quantum theory.
Effectus theory, however, has more breadth to it,
 see e.g.~\cite{jacobs2017quantum,
cho2017disintegration,
adams2015type,
jacobs2016hyper,
jacobs2017channel,
jacobs2017formal,
cho2017efprob,
jacobs2017probability,
jacobs2017recipe,
jacobs2016effectuses,
jacobs2016affine,
jacobs2017distances,
jacobs2015effect}.
For a long, but comprehensive introduction, see \cite{effintro}.
\begin{point}{90}%
Next, I would like to mention some other recent approaches
    to the mathematical study of quantum theory
        which are related in spirit and of which I am aware.
First, there are colleagues who also study particular categories
    related to quantum theory.
Spearheaded by Coecke and Abramsky \cite{abramsky2004categorical},
    dozens of authors around Oxford
    have developed insightful graphical calculi
    by studying the category of Hilbert spaces
    categorically \cite{coecke2017picturing}.
Not explicitly categorically (but certainly in spirit)
    is the work around operational probabilistic theories,
    see e.g.~\cite{DAriano2016}.
Besides these approaches (and that of effectus theory),
    there are many other varied categorical forages
    into quantum theory, e.g.~\cite{kornell2012,rennela2017infinite,staton,furber2013kleisli}.
Then there are authors who pick out different structures from quantum theory
    to study.
Most impressive is the work of Alfsen and Shultz,
    who studied and axiomatized the state
    spaces of many operator algebras, including von Neumann algebras~\cite{alfsen2012}.
Recently, the poset of commutative subalgebras of a fixed operator algebra
    has received attention,
    see e.g.~\cite{heunen2015domains,bert,heunen2012bohrification}.
Finally,
    tracing back to von Neumann himself,
    the lattice of projections (representing sharp predicates)
    and the poset of effects (fuzzy predicates),
    has have been studied by many, see e.g.~\cite{dvurecenskij2013new}.
\end{point}
\end{point}
\begin{point}{91}{About the cover}
One of the most familiar structures
    in operator algebras (the mathematical field of this thesis)
    is perhaps the infinite-dimensional
    Hilbert space~$\ell^2(\N)$.
What would its unit sphere look like?
The cover depicts an approximation:
    a 12-dimensional spherical shell split into two.
On the outside the pieces are completely reflective ---
    on the inside they are dark blue and only mostly reflective.
The pieces are sat between two (11-dimensional) hyperplanes
    with a somewhat reflective checkerboard pattern.
    The planes are cropped at a certain radius from the origin
        turning them into \emph{hyperdisks}.
The picture is rendered using a homebrew\footnote{%
    Source code is available at~\url{https://github.com/westerbaan/ndpt}.}
    \emph{path tracer},
    which simulates how rays of light would bounce to end up
    in the camera.
The calculations involved (such as the one to determine how a ray bounces
    from an~$n$-dimensional sphere)
    are all conveniently expressed and performed using
    inner products.
The primary rays cast by the camera have the biggest presence
    in the first three dimensions, only ten percent in the fourth,
    five percent in the fifth et cetera.
This causes more higher-dimensional oddities to appear in the image
    where rays have travelled further or bounced farther in the
    higher dimensions.
\end{point}
\begin{point}{100}{Funding}
My research has been supported by the
European Research Council under grant agreement \textnumero~320571.
\end{point}
\begin{point}{110}{Acknowledgments}%
Progress in theoretical fields
    is often dramatized
    as the fruit of personal struggles of  \emph{einzelg\"angers}
    grown in deep focus and isolation.
But what is missing in this depiction is that
    often the seeds were sown
    in casual conversation among colleagues.
I would like to acknowledge two seeds I'm aware of.
First, it was Sam Staton who suggested looking at a universal property
    for instruments, which led to quotients and comprehension.
Secondly, Chris Heunen insisted,
    against my initial skepticism,
    that Stinespring's dilation must have a
    universal property.
I've also had the immense pleasure of discussing research with
    Robin Adams,
    Guillaume Allais,
    Henk Barendregt,
    Kenta Cho,
    Joan Daemen,
    Robert Furber,
    Bart Jacobs,
    Bas Joosten,
    Martti Karvonen,
    Aleks Kissinger,
    Mike Koeman,
    Hans Maassen,
    Joshua Moerman,
    Teun van Nuland,
    Dusko Pavlovic,
    Arnoud van Rooij,
    Frank Roumen,
    Marc Schoolderman,
    Peter Schwabe,
    Peter Selinger,
    Alexandra Silva,
    Michael Skeide,
    Pawel Sobocinski,
    Sean Tull,
    Sander Uijlen,
    Marloes Venema,
    Abraham Westerbaan,
    John van de Wetering and
    Fabio Zanasi.
I'm particularly grateful to Arnoud, Abraham and John
    for proofreading this thesis.
I'm honored to have been received by Dusko
    for a two-month research visit,
    on which he, Jason, Jon, Mariam, Liang-Ting, Marine, Muzamil, Pawel
    and Peter-Michael made me feel at home on the other side of the planet.
On this side of the planet,
    it is hard to imagine a friendlier group of colleagues.
I'm much obliged to Marko van Eekelen,
    Wouter Geraedts,
    Engelbert Hubbers,
    Carlo Meijer and
    Joeri de Ruiter
    with whom I've performed security audits as a side-gig,
    which allowed for an extension of my position.
I'm grateful to my supervisor, Bart Jacobs,
    who showed me a realistic optimism,
    which is in such short supply these days.
I'm touched by the patience of my friends, (extended) family and Annelies,
    during the final stretch of writing this thesis.

Last, but not least, I would like to share my deep appreciation
    for the (under)graduate education
    I've been fortunate enough to receive at the Radboud Universiteit.
Only now, a few years later,
    I can see how rare it is,
    not only for its uncompromised treatment of the topics at hand,
    but also for the personal attention to each student.
Of all the great professors,
Henk Barendregt,
    Mai Gehrke, Arnoud van Rooij and Wim Veldman
    have been of particular personal influence to me --- I owe them a lot.
\end{point}
\end{parsec}


% vim: ft=tex.latex
