\documentclass[b]{subfiles}
\begin{document}

\chapter{Introduction}
\begin{parsec}%
\begin{point}%
This thesis is a foundational study of quantum computing consisting
    of two related, but independant parts:
    chapter~\ref{chapter1} on \emph{dilations} and~\ref{chapter2}
    ``diamond, andthen, dagger'' on \emph{effectus theory}.
Both parts can be read separetely
    and feature a comprehensive introduction of their own.
For now, we focus on what they have in common.
\begin{point}%
Both parts study the category of von Neumann algebras
    (with ncpsu-maps\footnote{%
        An abbreviation for contractive normal completely
        positive linear maps.} between them).
A von Neumann algebra models a (possibly infinite-dimensional)
    quantum datatype, for instance:
\begin{center}
    \begin{tabular}{llll}
        $M_2$ & a qubit & $M_3 \otimes M_3$ & an (ordered) pair of qutrits \\
        $M_3$ & a qutrit & $\C^2$ & a bit \\
        $M_2 \oplus M_3$ & a qubit or a qutrit
        & $\scrB(\ell^2)$ & quantum integers
\end{tabular}
\end{center}
The ncpsu-maps between them represent physical
    (but not necessarily computable) operations
        in the opposite direction:
\begin{enumerate}
\item
    \emph{Measure a qubit in the computational basis} \quad
            ($\mathrm{qubit} \to \mathrm{bit}$)\\
        $\C_2 \rightarrow M_2,$ \quad 
        $(\lambda,\mu) \ \mapsto\ \lambda \ketbra{0}{0} + \mu \ketbra{1}{1}$
\item
    \emph{Discard the second qubit} \quad
            ($\mathrm{qubit} \times \mathrm{qubit}\to \mathrm{qubit}$) \\
        $M_2 \rightarrow M_2 \otimes M_2$, \quad
        $a \ \mapsto \ a \otimes 1$
\item
    \emph{Apply a Hadamard-gate to a qubit} \quad
            ($\mathrm{qubit} \to \mathrm{qubit}$) \\
        $M_2 \rightarrow M_2$, \quad
        $a \ \mapsto \ H^*a H \otimes 1$, \quad where~$H = \frac{1}{\sqrt{2}}\begin{spmatrix}
            1 & 1 \\ 1 & -1
        \end{spmatrix}$
\item
    \emph{Initialize a qutrit as 0} \quad
            ($\mathrm{empty} \to \mathrm{qutrit}$) \\
        $M_3 \rightarrow \C$, \quad
        $a \ \mapsto \ \bra{0}a\ket{0}$
\end{enumerate}
\end{point}
\begin{point}%
The first part of this thesis starts with a concrete question:
    does every ncpsu-map between von Neumann algebras
    admit a Stinespring-like dilation?
We will see that indeed it does.
The insights attained pondering this question lead naturally,
        as it so often does in mathematics, to other new results:
    for instance, we learn more about  the tensor product of
    dilations~\sref{paschke-tensor}
        and the pure maps associated to them~\sref{paschke-pure}.
But perhaps the most interesting by-catch
    are the contributions to the theory
    of self-dual Hilbert C$^*$-modules over a von Neumann algebras,
    which underly these dilation.
\end{point}
\begin{point}%
In contrast, the second part of this thesis starts
    with the abstract (or rather: vague) question:
    what is unique about the category of von Neumann algebras?
The goal is to find a set of reasonable axioms,
    which picks out the category of von Neumann algebras
    among all other categories.
In other words: a reconstruction of quantum theory with categorical axioms.
    Unfortunately, we do not reach this goal.\footnote{%
        Building on this thesis,
            van de Wetering recently found a reconstruction
            of finite-dimensional quantum theory.\cite{wetering}}
However, by studying von Neumann algebras
    in this dogmatic straitjacket
    we are forced, as again is often the case in mathematics,
    upon several new notions and results,
    which are of independant interest,
    such as, for instance,~$\diamond$-adjointness,
    pure maps and the existence of their dagger.
\end{point}
\begin{point}%
On the topics of this thesis,
    I collaborated 
\end{point}
\end{point}
\begin{point}{Acknowledgements}%
    
\end{point}
\end{parsec}

\end{document}

% vim: ft=tex.latex
