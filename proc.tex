\documentclass[a]{subfiles}
\begin{document}
\chapter{Processes}
\section{Corners and Filters}
\begin{parsec}%
\begin{point}{Definition}%
Given an projection~$e$ of a von Neumann algebra~$\scrA$,
the \Define{corner} of~$e$
is the subset~$e\scrA e$ of~$\scrA$ 
(consisting of the elements of~$\scrA$
of the form~$eae$ with~$a\in\scrA$).
In this context,
the obvious map~$e\scrA e\to\scrA$
is called the \Define{inclusion}
and the map $a\mapsto eae,\ \scrA\to e\scrA e$
is called the \Define{projection}.
\end{point}
\begin{point}[corner-basic]{Exercise}%
Let~$e$ be a projection from a von Neumann algebra~$\scrA$.
\begin{enumerate}
\item
Show that~$a\in\scrA$ 
is an element of~$e\scrA e$ iff~$eae=a$
iff $\ceilr{a}\cup\ceill{a} \leq e$.
\item
Show that the corner~$e\scrA e$
is closed under addition, (scalar) multiplication,
and involution.
\item
Show that~$e$ is a unit for~$e\scrA e$,
that is, $ea=ae=a$ for all~$a\in e\scrA e$.
\item
Show that~$e\scrA e$ is norm and ultraweakly closed.\\
(Hint: use the fact that $e(\,\cdot\,)e\colon \scrA\to\scrA$
is normal and bounded.)
\item
Show that~$e\scrA e$ --- 
endowed with the addition, (scalar) multiplication,
involution and norm from~$\scrA$,
and with~$e$ as its unit ---  is a $C^*$-algebra.
\item
Show that the supremum of a bounded directed
set~$D$ of self-adjoint elements of~$e\scrA e$
computed in~$\scrA$
is itself in~$e\scrA e$,
and, in fact, the supremum of~$D$ in~$e\scrA e$.
\item
Show that the inclusion $e\scrA e\to\scrA$
is an ncpsu-map.
\item
Deduce from this that the restriction of an np-map
$\omega\colon \scrA\to\C$ to
a map $e\scrA e\to\C$
is an np-map.

Conclude that~$e\scrA e$ is a von Neumann algebra.
\item
Show that the projection $a\mapsto eae,\ \scrA\to e\scrA e$
is an ncpu-map.
\item
Show that every np-map $\omega\colon e\scrA e\to\C$
is the restriction
of the np-map $\omega(e(\,\cdot\,)e)\colon \scrA\to\C$.
Deduce from this that the ultraweak topology of~$e\scrA e$
coincides (on $e\scrA e$) with the ultraweak topology on~$\scrA$.
Show that the ultrastrong topologies on~$e\scrA e$ and~$\scrA$
coincide in a similar fashion.
\end{enumerate}
\end{point}
\begin{point}[ad-ncp]{Exercise}%
Let~$a$ be an element of a von Neumann algebra~$\scrA$,
and let~$p$ and~$q$ be projections
of~$\scrA$ with $a^*pa\leq q$.
\begin{enumerate}
\item
Show that $a^*ba\in q\scrA q$
for every~$b\in p\scrA p$.
\item
Show that~$a^*(\,\cdot\,)a$
gives an ncp-map $p\scrA p\to q\scrA q$.
\end{enumerate}
\end{point}
\end{parsec}%
\begin{parsec}%
\begin{point}{Definition}%
Let~$p$ be an effect of a von Neumann algebra~$\scrA$.
A \Define{corner} of~$p$ is an ncp-map $\pi\colon \scrA\to\scrC$
to a von Neumann algebra~$\scrC$
with~$\pi(p^\perp)=0$
which is initial among such maps 
in the sense
that every ncp-map $f\colon \scrA\to\scrB$
with~$f(p^\perp)=0$
factors as $f=g\circ\pi$
for some unique ncp-map $g\colon \scrC\to\scrB$.
\end{point}
\begin{point}{Proposition}%
Given an effect~$p$ of a von Neumann algebra~$\scrA$,
and a partial isometry~$u$ of~$\scrA$
with $\floor{p}=uu^*$,
the map $\pi\colon \scrA\to u^*u \scrA u^*u$
given by~$\pi(a)=u^*au$ is a corner of~$p$.
\begin{point}{Proof}%
By~\sref{ad-ncp}, $\pi$ is an ncp-map.
To see that~$\pi(p^\perp)\equiv u^*p^\perp u =0$,
note that since~$u^*u=u^*\,u u^*\,u$,
we have $0=u^*(uu^*)^\perp u =u^*\smash{\floor{p}}^\perp u
= u^*\ceil{\smash{p^\perp}} u$,
and so
$0 = \ceil{u^* \ceil{\smash{p^\perp}} u }
=\ceil{u^* p^\perp u}$
by~\sref{ceil-fundamental},
giving~$u^*p^\perp u=0$
by~\sref{ceil-basic}.


Let~$\scrB$ be a von Neumann algebra,
and let~$f\colon \scrA\to\scrB$ be an ncp-map
with $f(p^\perp)=0$.
To show that~$\pi$ is a corner,
we must show that there is a unique ncp-map
$g\colon u^*u \scrA u^*u\to\scrB$
with $f=g\circ \pi$.
Uniqueness follows
from the observation that~$\pi$ is surjective.
Concerning existence,
define~$g:= f\circ \zeta$,
where $\zeta\colon  u^*u\scrA u^*u\to \scrA$
is the ncp-map given by~$\zeta(a)=uau^*$
for~$a\in\scrA$ (see~\sref{ad-ncp}),
so that it is immediately clear that~$g$ is an ncp-map.
It remains to be shown~$f=g\circ \pi$,
that is,
$f(a)=f(uu^*\,a\,uu^*)$ for all~$a\in\scrA$.
This follows from~\sref{cp-comprehension}
because~$f(\smash{(uu^*)^\perp})=0$,
since~$\ceil{\smash{f(\,\smash{(uu^*)^\perp}\,)}}
=\ceil{\smash{f(\smash{\floor{p}}^\perp)}}
=\ceil{\smash{f(\ceil{\smash{p^\perp}})}}
= \ceil{\smash{f(p^\perp)}}=\ceil{0}=0$.\qed
\end{point}
\end{point}
\end{parsec}
\begin{parsec}%
\begin{point}{Definition}%
A \Define{filter}
is an ncp-map $c\colon \scrC\to\scrA$
between von Neumann algebras
such that every ncp-map $f\colon \scrB\to\scrA$
with~$f(1)\leq c(1)$
factors as $f=c\circ g$
for some unique ncp-map $g \colon \scrB\to\scrC$.
\end{point}
\begin{point}{Proposition}%
Given an element~$b$ of a von Neumann algebra~$\scrA$,
the map $c\colon \scrA\to \ceill{b}\!\scrA\!\ceill{b}$
given by~$c(a)=b^*ab$
is a filter.
\begin{point}{Proof}%
Note that~$c(a)=b^*ab=\ceill{b}b^*ab\ceill{b}$
is an element of~$\ceill{b}\!\scrA\!\ceill{b}$
for each~$a\in\scrA$,
because~$b\ceill{b}=b$
and (so) $\ceill{b}b^*=b^*$.
Further, $c$ is an ncp-map by~\TODO{...}.
Thus it remains to be shown
that given~$f\colon \scrB\to\scrA$
with~$f(1)\leq g(1)\equiv b^*b$
there is a unique ncp-map $g\colon \scrB\to \ceill{b}\!\scrA\!\ceill{b}$
with~$f=c\circ g$.

To establish uniqueness of such~$g$, 
we let~$g_1,g_2\colon \scrB\to\ceill{b}\!\scrA\!\ceill{b}$
with~$c\circ g_1=c\circ g_2$ be given,
and shall prove that~$g_1=g_2$.
It suffices to show that~$g_1(a)=g_2(a)$
for every positive~$a\in\scrA$
(because an arbitrary element of~$\scrA$ can be written
as a linear combination of positive elements).
But since~$b^* g_1(a) b = c(g_1(a)) = c(g_2(a))= b^* g_2(a) b$,
this follows from~\TODO{...},
considering that~$g_1(a)$ and~$g_2(a)$
are positive, 
and~$\ceil{g_i(a)} \leq \ceill{b}$
(because $g_i(a)\in \ceill{b}\!\scrA\!\ceill{b}$).

\TODO{existance}
\end{point}
\end{point}
\end{parsec}
\begin{parsec}
\TODO{$uu^*\scrA uu^*\to\scrA, a\mapsto u^* a u$
is a corner
when~$u$ is a partial isometry.}
\end{parsec}

\section{To be moved}
\begin{parsec}%
\begin{point}[uwc-bounded]{Proposition}%
An ultraweakly continuous linear map $f\colon \scrA\to\scrB$
between von Neumann algebras is bounded.
\begin{point}[uwc-bounded-1]{Proof}%
Suppose for now that~$\scrB=\C$
(so that the norm topology and ultraweak topology on~$\scrB$ coincide).
Then any sequence $a_1,a_2,\dotsc$
in~$\scrA$ which converges in the norm to~$0$,
also converges ultraweakly to~$0$,
and so~$f(a_1),\,f(a_2),\,\dotsc$ converges
to~$0$ (because~$f$ is ultraweakly continuous).
It follows that~$f$ is continuous with respect to the norm
on~$\scrA$, and therefore bounded.
\begin{point}%
Let~$\scrB$ again be arbitrary.
Writing~$\Omega$ for the set of npu-maps $\omega\colon \scrB\to\C$
we have~\TODO{NO! this is not true}
$\|f(a)\|=\sup_{\omega\in\Omega} \left|\omega(f(a))\right|$
for all~$a\in\scrA$
by \TODO{}.
Hence it suffices to show that $\omega\circ f$
is bounded for each~$\omega\in\Omega$,
because then $\|f(a)\|\leq (\sup_{\omega\in\Omega}\|\omega\circ f\|)\,\|a\|$
for all~$a\in \scrA$,
and~$\sup_{\omega\in \Omega} \|\omega\circ f\|<\infty$
by the Principle of Uniform Boundedness (\sref{pub}).
Now,~$\omega\circ f$ is bounded by~\sref{uwc-bounded-1} 
(where~$\omega\in\Omega$),
because $\omega\circ f$ is ultraweakly continuous
(because $\omega$ and~$f$ are ultraweakly continuous).\qed
\end{point}
\end{point}
\end{point}
\begin{point}[vn-dense-pos]{Lemma}%
Let~$\scrS$ be an ultrastrongly dense $C^*$-subalgebra
of a von Neumann algebra~$\scrA$.
Let~$D$ be a norm dense subset of~$\scrS$.
Then~$\{\,d^*d\colon\,d\in D\,\}$
is ultrastrongly dense in~$\pos{\scrA}$.
\begin{point}{Proof}%
By Kaplansky's Density Theorem (\sref{kaplansky})
$\pos{\scrS}$ 
is ultrastrongly dense in~$\pos{\scrA}$,
so it suffices to show that $\{\, d^*d\,\colon d\in D\}$
is ultrastrongly dense in~$\pos{\scrS}$.
Given~$a\in \pos{\scrS}$,
there is a sequence $d_1,d_2,\dotsc \in D$
that norm converges to~$\sqrt{a}$,
so that~$d_1^*d_1,\,d_2^*d_2,\,\dotsc$
converges to~$\sqrt{a}^*\sqrt{a}\equiv a$
with respect to the norm, and thus ultrastrongly too.\qed
\end{point}
\end{point}
\end{parsec}%
\section{Tensor Product}
\TODO{complete}
\begin{parsec}%
\begin{point}{Definition}%
Let~$\scrA$ and~$\scrB$ be von Neumann algebras.
\begin{enumerate}
\item
The \Define{ultraweak tensor product topology}
on the algebraic tensor product~$\scrA\odot\scrB$
is the least topology that makes
all linear maps of the form $\sum_n \omega_n
\colon \scrA \odot \scrB\to\C$
continuous,
where $\omega_1,\omega_2,\dotsc \colon \scrA\to \C$
are linear maps which
are \emph{simple} and \emph{positive} (see below),
and $\sum_n \omega_n(1) <\infty$.

A linear map $\omega\colon \scrA\odot \scrB\to\C$
is 
\begin{enumerate}
\item
\Define{simple}
iff $\omega$ is of the form $\omega \equiv \sum_{n=1}^N \sigma_n\odot\tau_n$
for some ultraweakly continuous linear maps 
$\sigma_1,\dotsc,\sigma_N\colon \scrA\to\C$ and 
$\tau_1,\dotsc,\tau_N\colon \scrB\to\C$;
\item
\Define{positive}
iff $\sum_{i,j}\omega(\,a_i^*a_j\,\otimes\, b_i^*b_j\,)$
is positive 
for all $a_1,\dots,a_N\in\scrA$
and~$b_1,\dotsc,b_N\in\scrB$.
\end{enumerate}
\item
A bilinear map $\beta\colon \scrA\times \scrB\to\scrC$
to a von Neumann algebra~$\scrC$
is called \Define{normal}
when the unique extension $\beta_\odot \colon \scrA\odot \scrB\to \scrC$
is continuous with respect to the ultraweak tensor product topology 
on~$\scrA\odot\scrB$
and the ultraweak topology on~$\scrC$.
\end{enumerate}
\end{point}
\begin{point}{Theorem}%
Given von Neumann algebras~$\scrA$ and~$\scrB$
there is a von Neumann algebra~$\Define{\scrA\otimes\scrB}$
(called the \Define{tensor product} of~$\scrA$ and~$\scrB$)
and a normal
bilinear map $\Define{\otimes} \colon \scrA\times \scrB\to
\scrA\otimes \scrB$
with the property, that for every 
normal bilinear map $\beta\colon \scrA\times \scrB\to\scrC$
to a von Neumann algebra~$\scrC$
there is a unique ultraweakly continuous
map $\Define{\beta_\otimes}\colon \scrA\otimes \scrB
\to \scrC$ with $\beta_\otimes(a\otimes b) = \beta(a,b)$
for all~$a\in \scrA$ and $b\in\scrB$.
Moreover,
$\beta_\otimes$ is
\begin{enumerate}
\item
multiplicative
iff $\beta(a_1,b_1)\beta(a_2,b_2)=\beta(a_1a_2,b_1b_2)$
for all~$a_1,a_2\in\scrA$ and $b_1,b_2\in\scrB$;
\item
involution preserving
iff $\beta(a,b)^* = \beta(a^*,b^*)$ for all~$a\in\scrA$
and $b\in\scrB$;
\item
unital iff $\beta(1,1)=1$;
subunital iff $\beta(1,1)\leq 1$;
\item
positive iff 
$\sum_{i,j} \beta(a_i^*a_j,b_i^*b_j) \geq 0$
for all~$a_1,\dotsc,a_N\in\scrA$,$b_1,\dotsc,b_N\in\scrB$;
\item
completely positive iff 
the matrix $(\ \beta(a_i^*a_j,b_i^*b_j)\ )_{i,j}\geq 0$
is positive
for all~$a_1,\dotsc,a_N\in\scrA$
and $b_1,\dotsc,b_N\in\scrB$;
\end{enumerate}
Given ultraweakly continuous maps $f\colon \scrA\to\scrA'$
and $g\colon \scrB\to\scrB'$
to von Neumann algebras $\scrA'$ and~$\scrB'$,
the assignment $(a,b)\mapsto f(a)\otimes g(b)$
gives a normal bilinear map 
$\beta\colon \scrA\times \scrB\to \scrA'\otimes \scrB'$.
Then $\Define{f\otimes g}:=\beta_\otimes\colon \scrA\otimes \scrB
\to\scrA'\otimes \scrB'$
is
\begin{enumerate}
\item multiplicative
when $f$ and~$g$ are multiplicative;
\item involution preserving
when $f$ and~$g$ are involution preserving;
\item unital when $f$ and~$g$ are unital;
	subunital when $f$ and~$g$ are subunital;
\item completely positive when $f$ and~$g$ are completely positive.
\end{enumerate}
\end{point}
\end{parsec}
\end{document}
