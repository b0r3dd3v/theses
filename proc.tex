\documentclass[a]{subfiles}
\begin{document}
\chapter{Processes}
\section{Measurement}
\begin{parsec}
\begin{point}%
We now turn to the study of
maps on a von Neumann algebra~$\scrA$
of the form
$a\mapsto \sqrt{p}a\sqrt{p}\colon\,\scrA\to\scrA$,
where~$p$ is an effect of~$\scrA$,
that represent measurement of~$p$,
and are called \emph{assert maps} in~\TODO{cite Bart}
--- the importance of these maps 
to any logical description of
quantum computation is not easily overstated.

On the effects of~$\scrA$
these maps are studied~\TODO{...} in the guise
of the binary operation
$p\ast q=\sqrt{p} q \sqrt{p}$
called the \emph{sequential product}.
The main result of this section
is an axiomatisation of this operation
in terms of properties
of the underlying assert maps.

Our first observation
to this end
is that any assert map factors as
\begin{equation*}
\xymatrix@C=10em{
\scrA
\ar[r]^-{\pi\colon a\mapsto \ceil{p}a\ceil{p}}
&
\ceil{p}\!\scrA\!\ceil{p}
\ar[r]^-{c\colon a\mapsto \sqrt{p}a\sqrt{p}}
&
\scrA
},
\end{equation*}
where both~$\pi$ and~$c$ obey a universal property:
$c$ is a \emph{filter} of~$p$, see~\sref{filter},
and~$\pi$ is a \emph{corner} of~$\ceil{p}$, see~\sref{corner}.
Such maps
that are the composition of a filter and a corner
will be called \emph{pure}, see~\sref{pure},
Since not only assert maps turn out to be pure, but also maps of the form
$b^*(\,\cdot\,)b\colon \scrA\to\scrA$ for an arbitrary element~$a$
of~$\scrA$,
we need one more property of assert maps, namely
that
\begin{equation*}
	\sqrt{p}\,e_1\,\sqrt{p}\ \leq\  e_2^\perp 
	\qquad\iff\qquad
	\sqrt{p}\,e_2 \,\sqrt{p}\ \leq\  e_1^\perp
\end{equation*}
for all projections~$e_1$ and~$e_2$ of~$\scrA$---which we'll
describe by saying that $\sqrt{p}(\,\cdot\,)\sqrt{p}\colon \scrA\to\scrA$
is \emph{purely self-adjoint}.
Judging only by the name
it may not surprise you that the map $b(\,\cdot\,)b\colon \scrA\to\scrA$
where~$b\in \scrA$ is self-adjoint (but not necessarily positive)
turns out to be purely self-adjoint too,
so that as a final touch we introduce the notion
of \emph{purely positive} maps $f\colon \scrA\to\scrA$
that are simply maps of the form~$f\equiv gg$ for some purely self-adjoint~$g$.

The main technical result, then, of this section
is that any purely positive map $f\colon\scrA\to\scrA$
is of the form~$f=\sqrt{p}(\,\cdot\,)\sqrt{p}$
where~$p=f(1)$;
and, accordingly, our axioms 
(in~\sref{uniqueness-sequential-product})
that uniquely
determine the sequential product~$\ast$
on the effects of a von Neumann algebra~$\scrA$ are:
for every effect~$p$ of~$\scrA$,
\begin{enumerate}
\item
$p\ast 1=p$,
\item
$p\ast q = f(q)$
for all~$q\in [0,1]_\scrA$
for some pure map~$f\colon \scrA\to\scrA$,
\item
$p=q\ast q$ for some $q$ from~$[0,1]_\scrA$,
\item
$p \ast (p \ast q) = (p\ast p)\ast q$
for all~$q\in[0,1]_\scrA$,
\item
$p \ast e_1 \leq e_2^\perp\iff
p \ast e_2 \leq e_1^\perp$
for all projections $e_1,e_2$ of~$\scrA$.
\end{enumerate}%
While I would certainly not like
to undersell the results mentioned above
I suspect that the notion of purety exposed along the way
will turn out to be of far greater significance.
I'd like to advertise
here that we have already discovered that purity can be described in
wildly different terms---definitely a good omen, if any---in
that a map~$f\colon \scrA\to\scrB$ is pure when in its 
\emph{Paschke dilation} \TODO{add ref.}
$\xymatrix{\scrA
	\ar[r]|-\varrho
&
	\scrP\ar[r]|-c
&
\scrB}$
the map $\varrho$ is surjective.
Because of the faith I've developed for our notion of purity I've allowed myself
to address some theoretical questions concerning it
here that are not required for the main results of this thesis,
but suppose a general interest in purity:
I'll show that every pure map~$f\colon\scrA\to\scrB$
is extreme among the ncp-maps~$g\colon \scrA\to\scrB$ with~$f(1)=g(1)$,
and, in fact, enjoys the possibly stronger property 
of being~\emph{rigid} (see~\sref{rigid} and~\sref{pure-is-rigid}).
\end{point}
\end{parsec}
\subsection{Corner and Filter}
\begin{parsec}%
\begin{point}{Definition}%
Given an projection~$e$ of a von Neumann algebra~$\scrA$,
the \Define{corner} of~$e$
is the subset~$e\scrA e$ of~$\scrA$ 
(consisting of the elements of~$\scrA$
of the form~$eae$ with~$a\in\scrA$).
In this context,
the obvious map~$e\scrA e\to\scrA$
is called the \Define{inclusion}
and the map $a\mapsto eae,\ \scrA\to e\scrA e$
is called the \Define{projection}.
\end{point}
\begin{point}[corner-basic]{Exercise}%
Let~$e$ be a projection from a von Neumann algebra~$\scrA$.
\begin{enumerate}
\item
Show that~$a\in\scrA$ 
is an element of~$e\scrA e$ iff~$eae=a$
iff $\ceilr{a}\cup\ceill{a} \leq e$.
\item
Show that the corner~$e\scrA e$
is closed under addition, (scalar) multiplication,
and involution.
\item
Show that~$e$ is a unit for~$e\scrA e$,
that is, $ea=ae=a$ for all~$a\in e\scrA e$.
\item
Show that~$e\scrA e$ is norm and ultraweakly closed.\\
(Hint: use the fact that $e(\,\cdot\,)e\colon \scrA\to\scrA$
is normal and bounded.)
\item
Show that~$e\scrA e$ --- 
endowed with the addition, (scalar) multiplication,
involution and norm from~$\scrA$,
and with~$e$ as its unit ---  is a $C^*$-algebra.
\item
Show that the supremum of a bounded directed
set~$D$ of self-adjoint elements of~$e\scrA e$
computed in~$\scrA$
is itself in~$e\scrA e$,
and, in fact, the supremum of~$D$ in~$e\scrA e$.
\item
Show that the inclusion $e\scrA e\to\scrA$
is an ncpsu-map.
\item
Deduce from this that the restriction of an np-map
$\omega\colon \scrA\to\C$ to
a map $e\scrA e\to\C$
is an np-map.

Conclude that~$e\scrA e$ is a von Neumann algebra.
\item
Show that the projection $a\mapsto eae,\ \scrA\to e\scrA e$
is an ncpu-map.
\item
Show that every np-map $\omega\colon e\scrA e\to\C$
is the restriction
of the np-map $\omega(e(\,\cdot\,)e)\colon \scrA\to\C$.
Deduce from this that the ultraweak topology of~$e\scrA e$
coincides (on $e\scrA e$) with the ultraweak topology on~$\scrA$.
Show that the ultrastrong topologies on~$e\scrA e$ and~$\scrA$
coincide in a similar fashion.
\end{enumerate}
\end{point}
\begin{point}[ad-ncp]{Exercise}%
Let~$a$ be an element of a von Neumann algebra~$\scrA$,
and let~$p$ and~$q$ be projections
of~$\scrA$ with $a^*pa\leq q$.
\begin{enumerate}
\item
Show that $a^*ba\in q\scrA q$
for every~$b\in p\scrA p$.
\item
Show that~$a^*(\,\cdot\,)a$
gives an ncp-map $p\scrA p\to q\scrA q$.
\end{enumerate}
\end{point}
\end{parsec}%
\begin{parsec}%
\begin{point}[corner]{Definition}%
Let~$p$ be an effect of a von Neumann algebra~$\scrA$.
A \Define{corner} of~$p$ is an
ncp-map $\pi\colon \scrA\to\scrC$
to a von Neumann algebra~$\scrC$
with~$\pi(p^\perp)=0$
which is initial among such maps 
in the sense
that every ncp-map $f\colon \scrA\to\scrB$
with~$f(p^\perp)=0$
factors as $f=g\circ\pi$
for some unique ncp-map $g\colon \scrC\to\scrB$.

While most corners
that we'll deal with are unital,
there are also corners which are not unital
(see~\sref{non-unital-corner}),
so when we write ``corner'' we shall
always mean a ``unital corner''
unless explicitly stated otherwise.
\end{point}
\begin{point}{Proposition}%
Given an effect~$p$ of a von Neumann algebra~$\scrA$,
and a partial isometry~$u$ of~$\scrA$
with $\floor{p}=uu^*$,
the map $\pi\colon \scrA\to u^*u \scrA u^*u$
given by~$\pi(a)=u^*au$ is a corner of~$p$.
\begin{point}{Proof}%
By~\sref{ad-ncp}, $\pi$ is an ncp-map.
To see that~$\pi(p^\perp)\equiv u^*p^\perp u =0$,
note that since~$u^*u=u^*\,u u^*\,u$,
we have $0=u^*(uu^*)^\perp u =u^*\smash{\floor{p}}^\perp u
= u^*\ceil{\smash{p^\perp}} u$,
and so
$0 = \ceil{u^* \ceil{\smash{p^\perp}} u }
=\ceil{u^* p^\perp u}$
by~\sref{ceil-fundamental},
giving~$u^*p^\perp u=0$
by~\sref{ceil-basic}.


Let~$\scrB$ be a von Neumann algebra,
and let~$f\colon \scrA\to\scrB$ be an ncp-map
with $f(p^\perp)=0$.
To show that~$\pi$ is a corner,
we must show that there is a unique ncp-map
$g\colon u^*u \scrA u^*u\to\scrB$
with $f=g\circ \pi$.
Uniqueness follows
from the observation that~$\pi$ is surjective.
Concerning existence,
define~$g:= f\circ \zeta$,
where $\zeta\colon  u^*u\scrA u^*u\to \scrA$
is the ncp-map given by~$\zeta(a)=uau^*$
for~$a\in\scrA$ (see~\sref{ad-ncp}),
so that it is immediately clear that~$g$ is an ncp-map.
It remains to be shown~$f=g\circ \pi$,
that is,
$f(a)=f(uu^*\,a\,uu^*)$ for all~$a\in\scrA$.
This follows from~\sref{cp-comprehension}
because~$f(\smash{(uu^*)^\perp})=0$,
since~$\ceil{\smash{f(\,\smash{(uu^*)^\perp}\,)}}
=\ceil{\smash{f(\smash{\floor{p}}^\perp)}}
=\ceil{\smash{f(\ceil{\smash{p^\perp}})}}
= \ceil{\smash{f(p^\perp)}}=\ceil{0}=0$.\qed
\end{point}
\end{point}
\end{parsec}
\begin{parsec}%
\begin{point}[filter]{Definition}%
A \Define{filter}
is an ncp-map $c\colon \scrC\to\scrA$
between von Neumann algebras
such that every ncp-map $f\colon \scrB\to\scrA$
with~$f(1)\leq c(1)$
factors as $f=c\circ g$
for some unique ncp-map $g \colon \scrB\to\scrC$.
\end{point}
\begin{point}[canonical-filter]{Proposition}%
Given an element~$d$ of a von Neumann algebra~$\scrA$,
the map $c\colon \ceilr{d}\!\scrA\!\ceilr{d}\to\scrA$
given by~$c(a)=d^*ad$
is a filter.
\begin{point}{Proof}%
Note that~$c$ is an ncp-map by~\sref{ad-ncp}.
Let~$\scrB$ be a von Neumann algebra,
and let~$f\colon \scrB\to\scrA$ be an ncp-map
with $f(1)\leq c(1)$.
To show that~$c$ is a filter,
we must show that there is a unique ncp-map
$g\colon \scrB\to
\ceilr{d}\!\scrA\!\ceilr{d}$
with~$f=c\circ g$.
Uniqueness of~$g$ follows from the observation
that~$c$ is injective by~\sref{mult-cancellation}.

To establish the existence of such~$g$,
note that~$f(b)$ is an element of~$d^*\scrA d$,
when~$b$ is positive
by~\sref{sequential-douglas}
because~$0\leq f(b)\leq \|b\|f(1)\leq \|b\| c(1)=\|b\|d^*d$,
and thus for arbitrary~$b\in\scrB$ too
(being a linear combination
of positive elements).
We can thus define $g\colon \scrB\to \ceilr{d}\!\scrA\!\ceilr{d}$
by~$g(b)=d^*\backslash f(b)/d$
for all~$b\in\scrB$.
It is clear that~$g$ is linear and positive,
and~$c\circ g=f$.

To see that~$g$ is normal,
note that
$d^*\backslash\,\cdot\,/d\colon
d^*(\scrA)_1 d\to\scrA$
is ultrastrongly continuous by~\sref{div-usc},
as is~$f$ by~\sref{cp-uscont}
(also) as map from~$(\scrB)_1$ to~$d^*(\scrA)_1 d$,
so that~$g$ is ultrastrongly continuous on~$(\scrB)_1$,
and therefore normal by~\sref{p-uwcont}.

Finally, $g$ is completely positive
by~\sref{ncp-uwlim},
because it is by~\sref{div-approx}
the pointwise ultrastrong limit
of the by~\sref{ad-ncp} completely positive maps
$(\sum_{n=1}^Nt_n)^* \,f(\,\cdot\,)\,(\sum_{n=1}^N t_n)$,
where~$t_1,t_2,\dotsc$
is an approximate pseudoinverse of~$d$.\qed
\end{point}
\end{point}
\end{parsec}
\begin{parsec}%
\begin{point}{Definition}%
Let~$\scrA$ be a von Neumann algebra.
\begin{enumerate}
\item
Given a positive element~$p$
of~$\scrA$
we denote
by $\Define{c_p}\colon \ceil{p}\!\scrA\!\ceil{p}\to\scrA$
the \Define{standard filter} for~$p$
given by~$c_p(a)=\sqrt{p}a\sqrt{p}$
for all~$a\in\ceil{p}\!\scrA\!\ceil{p}$.
\item
Given an effect~$p$ of~$\scrA$
we denote
by $\Define{\pi_p}\colon \scrA\to\floor{p}\!\scrA\!\floor{p}$
the \Define{standard corner} of~$p$
given by~$\pi_p(a)=\floor{p}\!a\!\floor{p}$.
\end{enumerate}
\end{point}
\begin{point}[filter-basic]{Exercise}%
Let~$c\colon \scrC\to\scrA$ be a filter,
where~$\scrC$ and~$\scrA$ are von Neumann algebras.
\begin{enumerate}
\item
Show that, writing~$p:=f(1)$,
there is a unique
ncp-map $\alpha \colon \scrC\to \ceil{p}\!\scrA\!\ceil{p}$
with $c = c_p \circ \alpha$;
and that this~$\alpha$ is a unital ncp-isomorphism.
\item
Show that~$c$ is injective
(by proving first that~$c_p$ is injective
using~\sref{mult-cancellation}).

Conclude that~$c$
is faithful, and that~$c$ is mono in~$\W{CP}$.
\item
Show that~$c$ is bipositive
(by proving first that~$c_p$
is positive using~\sref{sequential-douglas}).
\end{enumerate}
\end{point}
\begin{point}[filters-composition]{Exercise}%
Show that the composition~$d\circ c$
of filters~$c\colon\scrC\to\scrD$
and~$d\colon \scrD\to\scrA$ 
between von Neumann algebras
is again a filter.
\end{point}
\begin{point}[corner-basic]{Exercise}%
Let~$p$ be an effect of a von Neumann algebra~$\scrA$,
and let~$\pi\colon \scrA\to\scrC$ be a corner of~$p$.
\begin{enumerate}
\item
Show that there is a unique ncp-map
$\beta \colon \floor{p}\!\scrA\!\floor{p}\to\scrC$
with~$\pi = \beta\circ \pi_p$;
and that this~$\beta$ is unital and an ncp-isomorphism.
\item
Show that~$\pi$ is surjective, and that~$\pi$ is epi in~$\W{cp}$.
\end{enumerate}
\end{point}
\begin{point}[corners-floor]{Exercise}%
Show that an ncpu-map $\pi\colon \scrA\to\scrB$
between von Neumann algebras
is a corner for an effect~$p$ of~$\scrA$
iff~$\pi$ is a corner for~$\floor{p}$;
in which case~$\ceil{\pi}=\floor{p}$.

Thus a corner~$\pi$ is a corner for~$\ceil{\pi}$.
\end{point}
\begin{point}[corners-composition]{Exercise}%
Show that the composition~$\tau\circ \pi$
of corners~$\pi\colon \scrA\to\scrB$
and~$\tau\colon \scrB\to\scrC$
between von Neumann algebras
is again a corner.\\
(Hint:
prove
and use the inequality
$\ceil{\tau}\leq \ceil{\smash{\pi(\ceil{\tau\circ \pi}^\perp)}}^\perp$.)
\end{point}
\begin{point}[filter-corner]{Theorem}%
Given an ncp-map $f\colon\scrA\to\scrB$
between von Neumann algebras,
a projection~$e$ of~$\scrA$
with~$\ceil{f}\leq e$,
and a positive element~$p$
of~$\scrB$ with~$f(1) \leq p$,
there is a unique ncp-map
$g \colon e\scrA e
\to \ceil{p}\!\scrB\!\ceil{p}$
such that
\begin{equation*}
\xymatrix{
\scrA
\ar[r]^-f
\ar[d]_{\pi_e}
&
\scrB
\\
e\scrA e
\ar[r]_g
& 
\ceil{p}\!\scrB\!\ceil{p}
\ar[u]_{c_p}
}
\end{equation*}
commutes,
and it is given by
$g(a)=\sqrt{p}\backslash f(a)/\!\sqrt{p}$
for all~$a\in e\scrA e$.
\begin{point}{Proof}%
Uniqueness of~$g$ follows from the facts
that~$\pi_e$ is epi and~$c_p$ is mono
in~$\W{cp}$,
see~\sref{corner-basic} and~\sref{filter-basic}.

Concerning existence, 
since~$\pi_e$ is a corner of~$e$,~\sref{corner},
and~$\ceil{f}\leq e$,
or in other words, $f(e^\perp)=0$,
there is a unique ncp-map $h\colon e\scrA e\to \scrB$
with $h \circ \pi_e = f$.
Note that~$h(a)=f(a)$ for all~$a$ from~$e\scrA e$.

As~$h(1)=h(\pi_e(1))=f(1)\leq p=c_p(1)$,
and~$c_p$ is a filter,~\sref{filter},
there is a unique ncp-map
$g\colon e\scrA e \to p \scrB p$
with $c_p\circ g = h$,
which is (by the proof of \sref{canonical-filter}) given by
$g(a)=\sqrt{p}\backslash h(a)/\sqrt{p}
\equiv \sqrt{p}\backslash f(a)/\sqrt{p}$
for all~$a$ from~$e\scrA e$.
Then~$c_p\circ g\circ \pi_e = h\circ \pi_e = f$.\qed
\end{point}
\end{point}
\begin{point}[square-f]{Corollary}%
Given an ncp-map $f\colon \scrA\to\scrB$
between von Neumann algebras
there is a unique ncp-map $\Define{[f]}\colon 
\ceil{f}\!\scrA\!\ceil{f}
\to
\ceil{f(1)}\!\scrB\!\ceil{f(1)}$
such that 
\begin{equation*}
\xymatrix@C=4em{
\scrA
\ar[r]^-f
\ar[d]_{\pi_{\ceil{f}}}
&
\scrB
\\
\ceil{f}\!\scrA\!\ceil{f}
\ar[r]_-{[f]}
& 
\ceil{f(1)}\!\scrB\!\ceil{f(1)}
\ar[u]_{c_{f(1)}}
}
\end{equation*}
commutes;
and it is given by~$[f](a)=\sqrt{f(1)}\backslash f(a)/\!\sqrt{f(1)}$
for all~$a$ from $\ceil{f}\!\scrA\!\ceil{f}$.

Moreover, 
$[f]$ is unital and faithful.
\end{point}
\begin{point}{Example}%
For any faithful unital ncp-map $f\colon \scrA\to \scrB$
we have~$[f]=f$.
Such map need not be an isomorphism;
as one may take $f\colon (\lambda,\mu)\mapsto \lambda+\mu,
\C^2\to\C$.
\end{point}
\begin{point}[ad-pure]{Example}%
In the concrete case
that $f\equiv a^*(\,\cdot\,)a \colon
s\scrA s\to t\scrA t$,
where~$a$ is an element
of a von Neumann algebra,
and $s$ and~$t$ are projections of~$\scrA$
with
$\ceilr{a}\leq s$
and~$\ceill{a}\leq t$,
the map~$[f]$ 
is closely related to the
polar decomposition $a\equiv [a]\sqrt{a^*a}
= \sqrt{aa^*}[a]$ of~$a$,
where $[a]=a/\sqrt{a^*a}$
(see~\sref{polar-decomposition}).

Indeed,
since  $\ceil{f}=\ceilr{a}$,
$f(1)=a^*a$,
and~$[f]\equiv \sqrt{a^*a}\backslash a^*(\,\cdot\,)a/\sqrt{a^*a}
\equiv [a](\,\cdot\,)[a]^*$,
the picture becomes:
\begin{equation*}
\xymatrix@C=10em{
s\scrA s
\ar[r]^-{f\,=\,a^*\,(\,\cdot\,)\,a}
\ar[d]_{\pi_{\ceilr{a}}}
&
t\scrA t
\\
\ceilr{a}\!\scrA\!\ceilr{a}
\ar[r]_-{[f] \,=\,  [a]\,(\,\cdot\,)\,[a]^*}
& 
\ceill{a}\!\scrA\!\ceill{a}
\ar[u]_{c_{a^*a}}
}
\end{equation*}
Note that in this example
$[f]$ is an ncpu-isomorphism,
because~$[a]$ is a partial isometry
with initial projection~$\ceill{a}$
and final projection~$\ceilr{a}$.
Thus one can think of the diagram above
as an isomorphism theorem of sorts,
which applies only to certain  ncp-maps
that'll be called \emph{pure} in a moment (see~\sref{pure-fundamental}).
\end{point}
\end{parsec}
\subsection{Isomorphism}
\begin{parsec}%
\begin{point}%
In case you were wondering,
the ncpu-isomorphism
we encounted in~\sref{ad-pure}
is simply a nmiu-isomorphism 
(see~\sref{iso}), which follows
from the following characterization of multiplicativity.
\end{point}
\begin{point}[gardner]{Proposition}%
For an ncpu-map $f\colon \scrA\to\scrB$
between von Neumann algebras
the following are equivalent.
\begin{enumerate}
\item
\label{gardner-1}
$f$ is multiplicative.
\item
\label{gardner-2}
$f(a)f(b)=0$
for all $a,b\in\scrA_+$ with $ab=0$.
\item
\label{gardner-3}
$\ceil{f(p)}\ceil{f(q)}=0$
for all projections $p$ and~$q$ of~$\scrA$ with $pq=0$.
\item
\label{gardner-4}
$f$ maps projections to projections.
\end{enumerate}
\begin{point}{Proof}%
(Based in part on the work of Gardner in~\cite{gardner}).
\begin{point}{\sref{gardner-1}$\Longrightarrow$\sref{gardner-4}}%
	is rather obvious.
\end{point}
\begin{point}{\sref{gardner-4}$\Longrightarrow$\sref{gardner-3}}%
Let~$p$ and~$q$ be projections of~$\scrA$ with~$pq=0$.
Then~$p\leq q^\perp$, and so~$f(p)\leq f(q^\perp)=f(q)^\perp$,
which implies that $\ceil{f(p)}\ceil{f(q)}
=f(p)f(q)=0$ since~$f(p)$ and~$f(q)$ are projections.
\end{point}
\begin{point}{\sref{gardner-3}$\Longrightarrow$\sref{gardner-2}}%
Let~$a$ and~$b$ be positive elements of~$\scrA$ with~$ab=0$.
We must show that~$f(a)f(b)=0$,
and for this it suffices to show that
$\ceil{f(a)}\ceil{f(b)}=0$,
because $f(a)f(b)=f(a)\ceil{f(a)}\ceil{f(b)}f(b)$.
Since~$ab=0$,
we have~$\ceil{a}\ceil{b}=0$ by~\sref{mult-cancellation}
and so $\ceil{f(a)}\ceil{f(b)}=\ceil{f(\ceil{a})}\ceil{f(\ceil{b})}=0$.
\end{point}
\begin{point}{\sref{gardner-2}$\Longrightarrow$\sref{gardner-1}}%
We must show that~$f(a)f(b)=f(ab)$
for all~$a,b\in \scrA$.
Since the linear span of projections is norm-dense in~$\scrA$,
it suffices to show that $f(a)f(e)=f(ae)$
for any $a\in\scrA$ and a projection~$e$ of~$\scrA$.
Given such~$a$ and~$e$,
we on the one hand have $ae^\perp\, e=0$,
so that~$f(ae^\perp)f(e)=0$,
that is, $f(a)f(e)=f(ae)f(e)$;
and on the other hand
we have $ae\,e^\perp=0$,
so that~$f(ae)f(e^\perp)=0$,
that is, $f(ae)=f(ae)f(e)$;
so that we reach~$f(ae)=f(a)f(e)$ as sum total,
and the result that~$f$ is multiplicative.\qed
\end{point}
\end{point}
\end{point}
\begin{point}[iso]{Theorem}%
An ncpsu-isomorphism $f\colon \scrA\to\scrB$
between von Neumann algebras 
(so both~$f$ and~$f^{-1}$ are ncpsu-maps)
is an nmiu-isomorphism.
\begin{point}{Proof}%
Since~$f^{-1}(1)\leq 1$
and so~$1=f(f^{-1}(1))\leq f(1)\leq 1$,
we see that~$f(1)=1$, so both $f$ and $f^{-1}$ are unital.
It remains to be shown that~$f$ and~$f^{-1}$ are multiplicative.
Since by~\sref{projection-order-sharp} an effect element~$a$ of~$\scrA$
is a projection iff~$0$ is the infimum of~$a$ and~$a^\perp$,
and~$f$ (as ncpu-isomorphism) preserves all the structure 
mentioned in this characterization of being a projection,
we see that~$f$ maps projections to projections,
and is thus multiplicative, by~\sref{gardner}.
It follows automatically that~$f^{-1}$ is multiplicative too.\qed
\end{point}
\end{point}
\end{parsec}
\subsection{Purity}
\begin{parsec}%
\begin{point}{Definition}%
Filters, corners,
and their compositions we'll call \Define{pure}.
\end{point}
\begin{point}{Exercise}%
Show that the following maps are pure.
\begin{enumerate}%
\item
An ncp-isomorphism between von Neumann algebras.
\item
The identity map~$\id\colon \scrA\to\scrA$
on a von Neumann algebra~$\scrA$.
\item
The map $a^*\,(\,\cdot\,)\,a\colon \scrA\to\scrA$
for any element~$a$ of a von Neumann algebra~$\scrA$.
\end{enumerate}
\end{point}
\begin{point}[pure-fundamental]{Proposition}%
For an ncp-map $f\colon \scrA\to\scrB$ between von Neumann algebras
the following are equivalent.
\begin{enumerate}
\item 
\label{pure-fundamental-1}
	$f$ is pure, i.e., $f$ is the composition
	of (perhaps many) filters and corners.
\item
\label{pure-fundamental-2}
	$f = c\circ \pi$ for a filter $c\colon \scrC\to\scrB$
	and a corner $\pi\colon \scrA\to\scrC$.
\item
\label{pure-fundamental-3}
	$[f]$ from~\sref{square-f} is an ncpu-isomorphism.
\end{enumerate}
\begin{point}{Proof}%
Note that \ref{pure-fundamental-3}$\Longrightarrow$\ref{pure-fundamental-2}
and \ref{pure-fundamental-2}$\Longrightarrow$\ref{pure-fundamental-1}
are rather obvious.
\begin{point}{\ref{pure-fundamental-1}$\Longrightarrow$%
\ref{pure-fundamental-2}}%
Calling $f$ \emph{properly pure}
when~$f\equiv c\circ \pi$
for some filter~$c$ and corner~$\pi$,
we must show that every pure map is properly pure.
For this it suffices to show that the composition of properly
pure maps is again properly pure;
which,
since filters are closed under composition
(by~\sref{filters-composition}),
and corners are closed under composition
(by~\sref{corners-composition}),
boils down to proving that the composition
$\pi\circ c$ of a corner~$\pi$ and a filter~$c$
is properly pure.
Since~$\pi\equiv \alpha\circ \pi_{\ceil{\pi}}$
and~$c\equiv c_{c(1)}\circ \beta$
for ncpu-isomorphisms~$\alpha$ and~$\beta$
(see~\sref{filter-basic}
and~\sref{corner-basic})
it suffices to show that
$f:=\pi_{s} c_{p}$ is properly pure
for a positive element~$p$ and a projection~$s$
of a von Neumann algebra~$\scrA$.
Since such~$f$ is of the form $f=s\sqrt{p}(\,\cdot\,)\sqrt{p}s
\colon \ceil{p}\!\scrA\!\ceil{p}\to s\scrA s$,
we know by~\sref{ad-pure}
that~$[f]$ is an ncpu-isomorphism,
and thus that~$f\equiv c_{f(1)}\circ [f]\circ \pi_{\ceil{f}}$ is properly pure.
\end{point}
\begin{point}{\ref{pure-fundamental-2}$\Longrightarrow$%
\ref{pure-fundamental-3}}%
Recall that $[f]$
is by definition the unique ncp-map
with~$f = c_{f(1)} [f] \pi_{\ceil{f}}$,
see~\sref{square-f}.
Note that since~$f=c\circ \pi$,
we have~$\ceil{f}=\ceil{\pi}$ (because~$\ceil{c}=1$, \TODO{...}),
and~$f(1)=c(1)$ (because~$\pi(1)=1$).
Since there are ncpu-isomorphisms~$\alpha$ and~$\beta$
with $\pi= \alpha \pi_{\ceil{\pi}}$ and  $c=c_{c(1)} \beta$,
we see that~$f=c_{c(1)} \beta\alpha \pi_{\ceil{\pi}}$,
and so~$[f]=\beta\alpha$
by definition of~$[f]$,
which implies that~$[f]$ is an ncpu-isomorphism.\qed
\end{point}
\end{point}
\end{point}
\begin{point}[special-pure-maps]{Exercise}%
Use~\sref{pure-fundamental} to show that 
\begin{enumerate}
\item
a faithful pure map is a filter,
\item
a unital pure map is a corner, and
\item
a unital and faithful pure map is an ncpu-isomorphism.
\end{enumerate}
\end{point}
\end{parsec}
\subsection{Contraposition}
\begin{parsec}%
\begin{point}{Definition}%
Given an ncp-map $f\colon \scrA\to\scrB$
between von Neumann algebras
we define
$\Define{f^\bullet}\colon \Proj(\scrA)\to \Proj(\scrB)$
by~$f^\bullet(e)=\ceil{f(e)}$
for all~$e\in\Proj(\scrA)$.
\end{point}
\begin{point}{Proposition}%
Given an ncp-map $f\colon \scrA\to\scrB$
between von Neumann algebras
and a projection~$e$ from~$\scrB$
there is a least projection~$\Define{f_\bullet(e)}$ from~$\scrA$ 
with~$\ceil{f(\,f_\bullet(e)^\perp\,)}\leq e^\perp$,
namely~$f_\bullet(e) =\ceil{\,ef(\,\cdot\,)e\,}$
(being the carrier 
of the ncp-map $ef(\,\cdot\,)e$ from~\TODO{...});
giving a map $\Define{f_\bullet}\colon \Proj(\scrB)\to\Proj(\scrA)$.

\TODO{refer to ITET}
\begin{point}{Proof}%
Since by definition $\ceil{\,ef(\,\cdot\,)e\,}$
is the greatest projection~$s$ of~$\scrA$
with $ef(s^\perp)e=0$ (see~\sref{carrier});
and~$ef(s^\perp )e=0$ iff~$\ceil{f(s^\perp)}
\leq\ceil{e(\,\cdot\,)e}^\perp\equiv
e^\perp$;
the projection
$\ceil{\,ef(\,\cdot\,)e\,}$
satisfies the description of~$f_\bullet(e)$.\qed
\end{point}
\end{point}
\begin{point}{Exercise}%
Let~$f\colon \scrA\to\scrB$ be an ncp-map between von Neumann algebras.
\begin{enumerate}
\item
Show that $f^\bullet(s)\leq t^\perp$
iff $f_\bullet(t)\leq s^\perp$,
for all~$s\in\Proj(\scrA)$ and~$t\in\Proj(\scrB)$.
\item
Show that $f^\bullet(\,\bigcup E\,)
= \bigcup_{e\in E} f^\bullet(e)$
for every set of projections~$E$ from~$\scrA$.

Show that~$f_\bullet(\,\bigcap E\,)
= \bigcap_{e\in E} f_\bullet(e)$
for all~$E\subseteq \Proj(\scrB)$.
\end{enumerate}
\end{point}
\begin{point}{Exercise}%
Show that for ncp-maps $f,g\colon\scrA\to\scrB$
between von Neumann algebras $f^\bullet = g^\bullet$
iff $f_\bullet = g_\bullet$.
In that case we say that $f$ and~$g$ are \Define{equivalent}.
\begin{point}%
Show that for ncp-maps $f\colon \scrA\to\scrB$
and~$g\colon \scrB\to\scrA$ we have
$f^\bullet=g_\bullet$ iff $f_\bullet = g^\bullet$
iff $\ceil{f(s)}\leq t^\perp\iff \ceil{g(t)}\leq s^\perp$
for all projections $s$ from~$\scrA$ and~$t$ from~$\scrB$.

In that case we say that~$f$ and~$g$ are \Define{contraposed}.
\end{point}
\end{point}
\begin{point}[equivalent-examples]{Examples}%
\begin{enumerate}
\item
Given an element~$a$ of a von Neumann algebra~$\scrA$
the maps $a^*(\,\cdot\,)a$ and~$a(\,\cdot\,)a^*$
on~$\scrA$ are contraposed.

If~$p$ and~$q$ are projections of~$\scrA$
with $a^*pa\leq q$
(as in~\sref{ad-ncp}),
then the maps
$a^*(\,\cdot\,)a \colon p\scrA p\to q\scrA q$
and~$a(\,\cdot\,)a^*\colon q\scrA q \to p\scrA p$
are contraposed.

In particular,
the standard corner $\pi_s\colon \scrA\to s \scrA s$
and the standard filter $c_s\colon s\scrA s\to \scrA$
for a projection~$s$ from~$\scrA$
are contraposed.
\item
An ncp-isomorphism $f\colon \scrA\to\scrB$
between von Neumann algebras
is contraposed to its inverse~$f^{-1}\colon \scrB\to\scrA$.
\item
There may be many maps equivalent to a given ncp-map $f\colon \scrA\to\scrB$
between von Neumann algebras:
show that~$(zf)^\bullet = f^\bullet$
for every positive central element~$z$ of~$\scrB$ with~$\ceil{z}=1$.
\end{enumerate}
\end{point}
\begin{point}{Exercise}%
Let $\xymatrix{
	\scrA\ar[r]|-{f}&
	\scrB\ar[r]|-{g}&
\scrC}$
be ncp-maps between von Neumann algebras~$\scrA$,
$\scrB$ and~$\scrC$.
\begin{enumerate}
\item
Show that $(g\circ f)^\bullet = g^\bullet\circ f^\bullet$
(using~\sref{ncp-ceil}),
and $(g\circ f)_\bullet = f_\bullet\circ g_\bullet$.

\item
Assuming that $f$ is equivalent 
to an ncp-map $f'\colon \scrA\to\scrB$
and~$g$ is equivalent to
and ncp-map~$g'\colon \scrB\to\scrC$,
show that~$g\circ f$ is equivalent to~$g'\circ f'$.
\item
Assuming that $f$ is contraposed to
an ncp-map $f'\colon \scrB\to\scrA$
and~$g$ is contraposed to
an ncp-map $g'\colon \scrC\to\scrB$,
show that~$g\circ f$ is contraposed to~$f'\circ g'$.
\end{enumerate}
\end{point}
\begin{point}[bullet-sum]{Proposition}%
Given ncp-maps~$f,g\colon \scrA\to\scrB$
between von Neumann algebras
\begin{equation*}
(f+g)^\bullet(s) \,=\, f^\bullet(s)\, \cup\, g^\bullet(s)
\qquad
\text{and}
\qquad (f+g)_\bullet(t)\, =\, f_\bullet(t) \,\cup\, g_\bullet(t)
\end{equation*}
for all~$s\in \Proj(\scrA)$ and~$t\in \Proj(\scrB)$.
\begin{point}[bullet-sum]{Proof}%
Note that $(f+g)^\bullet(s)
=  \ceil{(f+g)(s)}
= \ceil{f(s)+g(s)}
= \ceil{f(s)}\cup \ceil{g(s)}
= f^\bullet(s) \cup g^\bullet(s)$
by~\sref{ceil-basic}.
Since~$(f+g)_\bullet(t)\leq s^\perp$
iff~$f^\bullet(s)\cup g^\bullet(s)\equiv (f+g)^\bullet(s)\leq t^\perp$
iff both $f^\bullet(s)\leq t^\perp$ and~$g^\bullet(s)\leq t^\perp$
iff both $f_\bullet(t)\leq s^\perp$ and~$g_\bullet(t)\leq s^\perp$
iff~$f_\bullet(t)\cup g_\bullet(t)\leq s^\perp$,
we see that $(f+g)_\bullet(t)=f_\bullet(t)\cup g_\bullet(t)$.\qed
\end{point}
\end{point}
\begin{point}[carrier-f-dagger-f]{Lemma}%
Given contraposed
maps~$f\colon \scrA\to\scrB$
and~$g\colon \scrB\to\scrA$ between von Neumann algebras,
we have $\ceil{f}=\ceil{gf}$.
\begin{point}{Proof}%
$\ceil{gf}=(gf)_\bullet(1)
= f_\bullet(g_\bullet(1))
= g^\bullet(\ceil{g})
= g^\bullet(1)=f_\bullet(1)=\ceil{f}$.\qed
\end{point}
\end{point}
\end{parsec}
\subsection{Rigidity}
\begin{parsec}%
\begin{point}%
We now turn a remarkable property shared
by pure and nmiu-maps.
\end{point}
\begin{point}[rigid]{Definition}%
We say that a ncp-map $f\colon \scrA\to\scrB$
between von Neumann algebras is \Define{rigid}
when the only ncp-map $g\colon \scrA\to\scrB$
with $g(1)=f(1)$ and $\ceil{f(p)}=\ceil{g(p)}$ for all
projections~$p$ from~$\scrA$ is~$f$ itself.
\end{point}
\begin{point}{Proposition}%
A rigid map $f\colon \scrA\to\scrB$
between von Neumann algebras
is extreme among the maps $g\colon \scrA\to\scrB$
with $g(1)=f(1)$.
\begin{point}{Proof}%
Given $f\equiv \lambda g_1 + \lambda^\perp g_2$
where~$\lambda\in(0,1)$ 
and~$g_1,g_2\colon \scrA\to\scrB$
are ncp-maps with $g_i(1)=f(1)$,
we must show that~$f=g_1=g_2$.
Note that for every projection~$s$
of~$\scrA$
we have~$f^\bullet(s) = (\lambda g_1+\lambda^\perp g_2)^\bullet(s)
= g_1^\bullet(s)\cup g_2^\bullet(s)$
by~\sref{bullet-sum} and~\sref{equivalent-examples};
and in particular~$g_1^\bullet(s)\leq f^\bullet(s)$.
Then for $h:=\lambda  g_1 + \lambda^\perp f$
we have $h(1)=f(1)$
and~$h^\bullet(s) = g_1^\bullet(s)\cup f^\bullet(s)
= f^\bullet(s)$,
so that~$\lambda g_1 + \lambda^\perp f \equiv 
h=f = \lambda g_1 +\lambda^\perp g_2$ by rigidity of~$f$;
and thus~$f=g_2$.
Similarly, $f=g_1$.\qed%
\end{point}
\end{point}
\begin{point}[nmiu-rigid]{Proposition}%
A nmiu-map $\varrho\colon \scrA\to\scrB$
between von Neumann algebras is rigid.
\begin{point}{Proof}%
Let~$g\colon \scrA\to\scrB$
be an ncpu-map
with~$\ceil{\varrho(p)}=\ceil{g(p)}$
for every projection~$p$ of~$\scrA$.
To show that~$\varrho$ is rigid,
we must show that~$g=\varrho$,
and for this, it suffices to prove that $g(p)=\varrho(p)$
for every projection~$p$ of~$\scrA$ (by~\TODO{...}).
To this end, we'll show that~$g$ is multiplicative,
because then~$g$ maps projections to projections,
so that $g(p)=\ceil{g(p)}=\ceil{\varrho(p)}=\varrho(p)$
for every projection~$p$ of~$\scrA$.
We'll show that $g$ is multiplicative
using~\sref{gardner}
by proving that
$\ceil{g(p)}\ceil{g(q)}=0$
for projections $p$ and~$q$ of~$\scrA$
with~$pq=0$.
Indeed,
$\ceil{g(p)}\ceil{g(q)}=\ceil{\varrho(p)}\ceil{\varrho(q)}
=\varrho(p)\varrho(q)=\varrho(pq)=\varrho(0)=0$.\qed
\end{point}
\end{point}
\begin{point}[canonical-quotient-rigid]{Lemma}%
Given an element~$b$ of a von Neumann algebra~$\scrA$
the ncp-map $a\mapsto b^* a b,\ \ceilr{b}\!\scrA\!\ceilr{b}\to\scrA$
is rigid.
\begin{point}{Proof}%
Let~$g\colon \ceilr{b}\!\scrA\!\ceilr{b}\to\scrA$
be an ncp-map with~$g(1)=b^*b$
and $\ceil{b^*pb}=\ceil{g(p)}$ for every projection~$p$ 
of $\ceilr{b}\!\scrA\!\ceilr{b}$.
To prove that~$c:=b^*(\,\cdot\,) b
\colon \ceilr{b}\!\scrA\!\ceilr{b}\to \scrA$ is rigid,
we must show that~$g=c$.
Since~$c$ is a filter
(by~\sref{canonical-filter})
and~$g(1)=b^*b$
there is a unique ncp-map~$h\colon \ceilr{b}\!\scrA\!\ceilr{b}
\to\ceilr{b}\!\scrA\!\ceilr{b}$
with~$g=c\circ h$.
Our task then is to show that~$h=\id$,
and for this it suffices to show that,
for all~$a\in\ceilr{b}\!\scrA\!\ceilr{b}$,
\begin{equation}
\label{filter-rigid-1}
e_n\, h(\,e_n\, a\, e_n\,)\, e_n
\ = \ e_n \, a\,  e_n
\end{equation}
for some sequence of projections $e_1,e_2,\dotsc$
of~$\ceilr{b}\!\scrA\!\ceilr{b}$
that converges ultrastrongly to~$\ceilr{b}$,
because---as multiplication is jointly ultrastrongly continuous
on bounded sets (by~\TODO{...})---the left-hand side of the equation above 
converges ultrastrongly to~$g(a)$,
while the right-hand side converges ultrastrongly to~$a$.
We'll take $e_N := \sum_{n=1}^N \ceill{t_n}$,
where~$t_1,t_2,\dotsc$
is an approximate pseudoinverse for~$b$,
because $\ceilr{b} = \sum_n\ceill{t_n}$.

Since the identity on~$e_n \scrA e_n$ is rigid
by~\sref{nmiu-rigid},
it suffices (for~\eqref{filter-rigid-1})
to show that 
$e_n h(e_n) e_n = e_n$
and 
$\ceil{e_nh(p)e_n} = p$
for every projection $p$ from $e_n\scrA e_n$.
Writing~$s_N:=\sum_{n=1}^N t_n$,
we have $bs_n = e_n$,
and so
$
\ceil{e_n h(p) e_n}
=
\ceil{s_n^* b^* h(p) b s_n}
=
\ceil{s_n^*g(p) s_n}
=
\ceil{s_n^* \ceil{g(p)} s_n}
=
\ceil{s_n^* \ceil{b^* p b} s_n}
=
\ceil{s_n^* b^* p b s_n}
=
\ceil{e_n p e_n }
$
for every 
projection~$p$ from~$\ceilr{b}\!\scrA\!\ceilr{b}$.
In particular, $\ceil{e_n h(p)e_n} = p$
when~$p$ is from~$e_n \scrA e_n$;
and we see $\ceil{e_n h(e_n^\perp) e_n}=\ceil{e_n e_n^\perp e_n}=0$
when we take~$p=e_n^\perp$,
so that~$e_n h(e_n^\perp) e_n =0$,
which yields $e_nh(e_n)e_n = e_n$.\qed
\end{point}
\end{point}
\begin{point}[pure-is-rigid]{Theorem}%
Every pure map between von Neumann algebras is rigid.
\begin{point}{Proof}%
Let~$f\colon \scrA\to\scrB$ be a pure map between von Neumann algebras,
and let~$g\colon \scrA\to\scrB$ be an ncp-map
with $f(1)=g(1)$
and $f^\bullet = g^\bullet$.
To show that~$f$
is rigid,
we must prove that~$f=g$.
We know by~\sref{square-f}
that $f$ can be written as $f\equiv c_{f(1)} \circ [f]\circ \pi_{\ceil{f}}$,
and that~$c_{f(1)}$ is rigid,
by~\sref{canonical-quotient-rigid},
which we'll use shortly.
Towards this end,
note that since~$f^\bullet = g^\bullet$,
we have $f_\bullet = g_\bullet$,
and so $\ceil{f}=f_\bullet(1)=g_\bullet(1)=\ceil{g}$.
As~$\pi_{\ceil{f}}$ is a corner of~$\ceil{f}=\ceil{g}$,
there is a unique ncp-map $h\colon \ceil{f}\!\scrA\!\ceil{f}\to\scrB$
with $h\circ \pi_{\ceil{f}} =g$. 
Since then
$h^\bullet \circ \pi_{\ceil{f}}^\bullet
= g^\bullet = f^\bullet 
= c_{f(1)}^\bullet
\circ [f]^\bullet \circ \pi_{\ceil{f}}^\bullet$,
 and $\pi_{\ceil{f}}^\bullet$ is clearly surjective,
we get~$h^\bullet = c_{f(1)}^\bullet\circ [f]^\bullet$,
and thus  $(h\circ [f]^{-1})^\bullet = c_{f(1)}^\bullet$,
using here that~$[f]$ is invertible,
because~$f$ is pure.
Now,
using that~$c_{f(1)}$
is rigid,
and $h([f]^{-1}(1))=h(1)=h(\pi_{\ceil{f}}(1))=g(1)=f(1)=c_{f(1)}(1)$,
we get~$h\circ [f]^{-1}=c_{f(1)}$,
which yields
$g=h\circ \pi_{\ceil{f}} 
=h\circ [f]^{-1}\circ [f]\circ \pi_{\ceil{f}}
= c_{f(1)} \circ [f] \circ \pi_{\ceil{f}} = f$,
and thus~$f$ is rigid.\qed
\end{point}
\end{point}
\end{parsec}
\subsection{Pure positivity}
\begin{parsec}%
\begin{point}{Definition}%
We'll call a map $f\colon \scrA\to\scrA$
between von Neumann algebras
\begin{enumerate}
\item
\Define{purely self-adjoint}
if~$f$ is pure and contraposed to itself
($f^\bullet = f_\bullet$), and
\item
\Define{purely positive}
if~$f\equiv gg$
for some purely self-adjoint map
$g\colon \scrA\to\scrA$.
\end{enumerate}
\end{point}
\begin{point}[purely-positive-examples]{Examples}%
Let~$\scrA$ be a von Neumann algebra.
\begin{enumerate}
\item
Given a self-adjoint element~$a$ of~$\scrA$,
the map~$a(\,\cdot\,)a\colon \scrA\to\scrA$ is 
purely self-adjoint.
\item
Given a positive element~$a$ of~$\scrA$,
the map $a(\,\cdot\,)a\colon \scrA\to\scrA$
is purely positive.
\end{enumerate}
\end{point}
\begin{point}[purely-positive-basic]{Exercise}%
Let~$f\colon \scrA\to\scrA$
be an ncp-map,
where~$\scrA$ is a von Neumann algebra.
\begin{enumerate}
\item
Show that~$\ceil{f}=\ceil{f(1)}$ when~$f$
is purely self-adjoint.
\item
Assuming~$f$ is purely self-adjoint,
show that~$ff$ is purely self-adjoint,
and show that~$\ceil{ff}=\ceil{f}$ (cf.~\sref{carrier-f-dagger-f}).
\item
Show that~$f$ is purely self-adjoint
when~$f$ is purely positive.
\end{enumerate}
\end{point}
\end{parsec}
\begin{parsec}%
\begin{point}%
We now turn to the question
to what extend a filter~$c$ is determined by
its action~$c^\bullet\colon e\mapsto \ceil{c(e)}$ on projections;
we will see in~\TODO{}
that two filters $c_1$ and~$c_2$
are equivalent, $c_1^\bullet = c_2^\bullet$,
if and only if~$c_1(1)$ and~$c_2(1)$
are equal up to some central elements,
that is, \emph{centrally similar}.
\end{point}
\begin{point}{Definition}%
We say that positive elements $p$ and~$q$ of a von Neumann algebra~$\scrA$
are \Define{centrally similar}
if~$cp=dq$ for some positive central elements~$c$ and~$d$ of~$\scrA$
with~$\ceil{p}\leq \ceil{c}$
and~$\ceil{q}\leq \ceil{d}$.
\end{point}
\begin{point}[centrally-similar-basic]{Exercise}%
Let~$p$ and~$q$ be positive elements
of a von Neumann algebra~$\scrA$.
\begin{enumerate}
\item
Show that when~$p$ and~$q$ are centrally similar,
every element~$a$ of~$\scrA$ that commutes
with~$p$ commutes with~$q$ too;
and in particular, $pq=qp$.
\item
Show that when~$p$ and~$q$ are centrally similar,
$\ceil{p}=\ceil{q}$.
\item
Show that when~$p$ and~$q$ commute,
and both $\frac{p\wedge q}{p}$ 
and~$\frac{p\wedge q}{q}$
are central,
$p$ and~$q$ are centrally similar.
\item
Show that when~$p$ and~$q$ are pseudoinvertible,
$p$ and~$q$ are centrally similar iff
$pq^{\sim 1}$ is central
iff $qp^{\sim 1}$ is central
iff both $(p\wedge q)p^{\sim 1}$
and~$(p\wedge q)q^{\sim 1}$ are central.
\item
Assuming that $p$ and~$q$ commute
and $e_1 \leq e_2 \leq \dotsb$
are projections commuting with~$p$ and~$q$
with~$\bigcup_n e_n=\ceil{p}$
such that the~$e_np$ and~$e_nq$
are pseudoinvertible,
and centrally similar,
show that $p$ and~$q$ are centrally similar
on the 
grounds that both  $\frac{p\wedge q}{p}$
and~$\frac{p\wedge q}{q}$ are central.

(Hint:
$\smash{e_n \frac{p\wedge q}{p} = \frac{(e_np)\wedge(e_nq)}{e_np}}$
are central,
and
converge ultraweakly to $\frac{p\wedge q}{p}$.)
\end{enumerate}
\end{point}
\begin{point}[centrally-similar-fundamental]{Lemma}%
Suppose that $\ceil{q \, \vartheta(e)\, q}\leq e$
and~$\ceil{q \,\vartheta(e^\perp)\, q} \leq e^\perp$,
where~$e$ is a projection of a von Neumann algebra~$\scrA$,
$q$ is a positive element of~$\scrA$,
and~$\vartheta\colon \scrA\to\scrA$ is a miu-map.
Then~$eq=qe$ and~$\vartheta(e)=e$.
\begin{point}{Proof}%
We have $\vartheta(e)qe=\vartheta(e)q$,
because~$e\geq  \ceil{q\,\vartheta(e)\,q}
\equiv \ceill{\vartheta(e)q}$ (see~\sref{ceill-basic}).
Similarly, $\vartheta(e^\perp)qe^\perp = \vartheta(e^\perp)q$,
because $e^\perp \geq \ceil{q\,\vartheta(e^\perp)\,q}
\equiv \ceill{\vartheta(e^\perp)q}$,
and so~$\vartheta(e^\perp)qe=0$,
which implies $\vartheta(e)qe=qe$.
Thus~$qe=\vartheta(e)qe=\vartheta(e)q$,
and so $q^2e=q\vartheta(e)q$ is self-adjoint,
which gives us that $q^2e=(q^2e)^*=eq^2$.
Since~$q^2$ commutes with~$e$,
$q=\smash{\sqrt{q^2}}$ commutes
with~$e$ too (see~\sref{sqrt}).
Finally, $\vartheta(e)q=qe=eq$
and~$\ceil{q}=1$
imply that~$\vartheta(e)=e$ by~\sref{mult-cancellation}.\qed
\end{point}
\end{point}
\begin{point}[centrally-similar-corollary]{Corollary}%
A positive element~$q$
of a von Neumann algebra~$\scrA$
with~$\ceil{q}=1$
is central provided
there is a miu-map~$\vartheta\colon \scrA\to\scrA$
with $\ceil{q\,\vartheta(e)\,q}\leq e$
for every projection~$e$ from~$\scrA$;
and in that case~$\vartheta=\id$.
\end{point}
\begin{point}[positive-quotients-centrally-similar]{Proposition}%
Positive elements~$p$ and~$q$
of a von Neumann algebra~$\scrA$
with~$\ceil{p}=\ceil{q}=1$
are centrally similar 
when there is a miu-isomorphism
$\vartheta\colon \scrA\to\scrA$
with~$\ceil{pep}=\ceil{q\,\vartheta(e)\,q}$
for all projections~$e$ of~$\scrA$;
and in that case  $\vartheta=\id$.
\begin{point}{Proof}%
Let~$e$ be a projection from~$\scrA$ with~$ep=pe$.
Since~$1=\ceil{p}=\ceil{\smash{p^2}}$
we have $e=\ceil{e\ceil{\smash{p^2}}e}
=\ceil{e\smash{p^2} e}=\ceil{pep}=\ceil{q\,\vartheta(e)\,q}$.
Since~$e^\perp$ commutes with~$p$ too,
we get~$e^\perp = \ceil{\smash{q\,\vartheta(e^\perp)\,q}}$
by the same token;
and thus~$eq=qe$ and~$\vartheta(e)=e$ 
by~\sref{centrally-similar-fundamental}.
Since~$p$ is the norm limit
of linear combinations of such projections~$e$,
we get $pq=qp$ and~$\vartheta(p)=p$.

Since~$p$ and~$q$ commute,
we can find a sequence
of projections~$e_1\leq e_2 \leq \dotsb$
that commute with~$p$ and~$q$
with~$\bigcup_n e_n =\ceil{p}$
and such that $pe_n$ and~$qe_n$
are pseudoinvertible --- one may,
for example,
take $e_N:=\sum_{n=1}^N \ceil{t_n}$
where~$t_1,t_2,\dotsc$
is an approximate pseudoinverse
of~$p\wedge q$ (see~\sref{approximate-pseudoinverse}).
Note that to prove that~$p$ and~$q$ are centrally similar,
it suffices to show that $pe_n$ and~$qe_n$ are centrally similar,
by~\sref{centrally-similar-basic}.
Further, to prove that~$\vartheta(a)=a$
for some~$a\in\scrA$,
it suffices to show that~$\vartheta( e_n a e_n  ) = e_n a e_n$,
because $e_n a e_n$ converges ultraweakly to~$a$
by~\TODO{}.
Note that~$\vartheta(e_n)=e_n$,
because~$e_np=pe_n$,
and so~$\vartheta$ maps~$e_n\scrA e_n$ into~$e_n \scrA e_n$.

Thus, by considering~$e_n \scrA e_n$ 
instead of~$\scrA$,
and the restriction of~$\vartheta$ to~$e_n\scrA e_n$
instead of~$\vartheta$,
and~$pe_n$ and~$qe_n$
instead of~$p$ and~$q$,
we reduce the problem to the case that~$p$ and~$q$ are invertible;
and so we may assume without loss of generality that~$p$ and~$q$
are invertible to start with.
Given a projection~$e$ from~$\scrA$
we have $\ceil{p^{-1}q \,\vartheta(e)\, q p^{-1}}
= \ceil{p^{-1}\ceil{q\,\vartheta(e)\,q}p^{-1}}
= \ceil{p^{-1}\ceil{pep}p^{-1}}=e$;
so 
by~\sref{centrally-similar-corollary},
we get that
$\vartheta=\id$
and
$p^{-1}q$ is central;
and so
$p$ and~$q$ are centrally similar (by~\sref{centrally-similar-basic}).
\qed
\end{point}
\end{point}
\begin{point}[faithful-positive-map-uniqueness]{Proposition}%
A faithful positive map $f\colon \scrA\to\scrA$
on a von Neumann algebra~$\scrA$
is of the form~$f=\sqrt{p}(\,\cdot\,)\sqrt{p}$
where $p:=f(1)$.
\begin{point}{Proof}%
Note that~$f$,
being faithful and pure,
is a filter
(by~\sref{special-pure-maps}),
and thus of the form $f\equiv \sqrt{p}\,\vartheta(\,\cdot\,)\,\sqrt{p}$
for some isomorphism~$\vartheta\colon \scrA\to\scrA$.
Our task then is to show that~$\vartheta=\id$,
and for this
it suffices, by~\sref{positive-quotients-centrally-similar},
to find some positive~$q$ in~$\scrA$ with~$\ceil{q}=1$
and~$f^\bullet(e)\equiv\ceil{\sqrt{p}\,\vartheta(e)\,\sqrt{p}}
= \ceil{qeq}$ for all projections~$e$ in~$\scrA$.

Since~$f$ is positive,
we have~$f\equiv \xi \xi$ for some self-adjoint
map~$\xi\colon \scrA\to\scrA$.
Since~$1=\ceil{f}=f_\bullet(1)=
\xi_\bullet(\xi_\bullet(1))\leq \xi_\bullet(1)=\ceil{\xi}$
we have~$\ceil{\xi}=1$,
and so, $\xi$, being pure and faithful,
is a filter (by~\sref{special-pure-maps}).
Furthermore,
as~$\tilde\xi:=\sqrt{\xi(1)}(\,\cdot\,)\sqrt{\xi(1)}\colon \scrA\to\scrA$
is a filter of~$\xi(1)$ too,
there is an isomorphism~$\alpha\colon \scrA\to\scrA$
with $\xi=\tilde\xi\alpha$.
Now, $ {\tilde\xi}^\bullet \alpha^\bullet
={\xi}^\bullet={\xi}_\bullet
=\alpha_\bullet\tilde \xi_\bullet
= (\alpha^\bullet)^{-1}{\tilde\xi}^\bullet$
implies~${\tilde\xi}^\bullet = \alpha^\bullet 
{\tilde \xi}^\bullet \alpha^\bullet$,
and 
$f^\bullet= (\xi\xi)^\bullet
= {\tilde \xi}^\bullet\alpha^\bullet{\tilde \xi}^\bullet\alpha^\bullet
={\tilde \xi}^\bullet{\tilde \xi}^\bullet=(\tilde \xi\tilde \xi)^\bullet$.
In other words,
$\ceil{\sqrt{p}\,\vartheta(e)\,\sqrt{p}}
=f^\bullet(e)=(\tilde\xi\tilde\xi)^\bullet(e)
= \ceil{\xi(1)\,e\,\xi(1)}$
for all projections~$e$ of~$\scrA$,
which implies that~$\vartheta=\id$
by~\sref{positive-quotients-centrally-similar},
and hence that~$f=\sqrt{p}\,(\,\cdot\,)\,\sqrt{p}$.\qed
\end{point}
\end{point}
\end{parsec}
\begin{parsec}%
\begin{point}%
To strip 
from~\sref{faithful-positive-map-uniqueness}
the assumption 
that~$f$
is faithful 
we employ this device:
\end{point}
\begin{point}[chevron-f]{Definition}%
Given a ncp-map $f\colon \scrA\to\scrB$
between von Neumann algebras
we denote by
$\Define{\left<f\right>}\colon \ceil{f}\!\scrA\!\ceil{f}
\to \ceil{f(1)}\!\scrB\!\ceil{f(1)}$
the unique ncp-map
such that 
\begin{equation*}
\xymatrix@C=6em{
\scrA
\ar[r]^f
\ar[d]_{\pi_{\ceil{f}}}
&
\scrB
\\
\ceil{f}\!\scrA\!\ceil{f}
\ar[r]^{\left<f\right>}
&
\ceil{f(1)}\!\scrB\!\ceil{f(1)}
\ar[u]_{c_{\ceil{f(1)}}}
}
\end{equation*}
commutes.
(Compare this with the definition of~$[f]$ in~\sref{square-f}.)
\end{point}
\begin{point}[chevron-f-basic]{Exercise}%
Let~$f\colon \scrA\to\scrB$ be an ncp-map.
\begin{enumerate}
\item
Show that~$\left<f\right>
= \pi_{\ceil{f(1)}}\circ f \circ c_{\ceil{f}}$
(using, perhaps, that $\pi_{\ceil{f}}\circ c_{\ceil{f}}=\id$).
\item
Show that
$\left<f\right> = \pi_{\ceil{f(1)}} \circ c_{f(1)}\circ [f]$.

(Thus $\left<f\right>\!(a) = 
\sqrt{f(1)}\ [f]\!(a)\ \sqrt{f(1)}$
for all $a$ from~$\ceil{f}\!\scrA\!\ceil{f}$.)
\item
Show that~$\left<f\right>$
is faithful,
and~$\left<f\right>\!(1)=f(1)$.
\item
Assuming that~$f$ is pure,
show that~$\left<f\right>$ is pure,
and hence a filter (by~\sref{special-pure-maps}).
\end{enumerate}
\end{point}
\begin{point}[chevron-f-purely-positive]{Exercise}%
Let~$f\colon \scrA\to\scrA$
be an ncp-map, where~$\scrA$ is a von Neumann algebra.
\begin{enumerate}
\item
Suppose that~$f$ is purely self-adjoint.

Recall that~$\ceil{f}=\ceil{f(1)}$,
and so $\left<f\right>\colon \ceil{f}\!\scrA\!\ceil{f}
\to \ceil{f}\!\scrA\!\ceil{f}$.

Prove that~$\left<f\right>$
is purely self-adjoint.
\item
Suppose again that~$f$ is purely self-adjoint,
and recall from~\sref{purely-positive-basic} that $f^2$
is purely self-adjoint, and~$\ceil{f^2} = \ceil{f}$.
Show that $\left<f^2\right> = \left<f\right>^2$.
\item
Assuming that~$f$ is purely positive,
show that~$\left<f\right>$ is purely positive.
\end{enumerate}
\end{point}
\begin{point}[positive-map-uniqueness]{Theorem}%
Given a positive element~$p$ of a von Neumann algebra~$\scrA$
there is a unique purely positive map $f\colon \scrA\to\scrA$
with~$f(1)=p$,
namely~$f=\sqrt{p}(\,\cdot\,)\sqrt{p}$.
\begin{point}{Proof}%
We've already seen in~\sref{purely-positive-examples}
that $f=\sqrt{p}(\,\cdot\,)\sqrt{p}\colon \scrA\to\scrA$
is a purely positive map with~$f(1)=p$.
Concerning uniqueness,
(given arbitrary~$f$)
the map~$\left<f\right>\colon \ceil{p}\!\scrA\!\ceil{p}
\to \ceil{p}\!\scrA\!\ceil{p}$
from~\sref{chevron-f}
is purely positive by~\sref{chevron-f-purely-positive},
and faithful by~\sref{chevron-f-basic},
and so of the form
$\left<f\right>=\sqrt{p}(\,\cdot\,)\sqrt{p}$
by~\sref{faithful-positive-map-uniqueness}
(since~$\left<f\right>\!(1)=f(1)=p$);
implying that
$f= c_{\ceil{p}}\circ\left<f\right>\circ \pi_{\ceil{p}}
= \sqrt{p}\ceil{p}(\,\cdot\,)\ceil{p}\sqrt{p}
= \sqrt{p}(\,\cdot\,)\sqrt{p}$.\qed
\end{point}
\end{point}
\begin{point}{Corollary (``Square Root Axiom'')}%
Given a positive element~$p$ of a von Neumann algebra~$\scrA$
there is a unique positive map~$g\colon \scrA\to\scrA$
with~$g(g(1))=p$, namely
$g=\sqrt[4]{p}\,(\,\cdot\,)\,\sqrt[4]{p}$.
\begin{point}{Proof}%
Any such positive map~$g\colon \scrA\to\scrA$ with~$g(g(1))=p$
will be of the form
$g=\smash{\sqrt{g(1)}\,(\,\cdot\,)\,\sqrt{g(1)}}$
by~\sref{positive-map-uniqueness};
so that~$p=g(g(1))=g(1)^2$
implies that~$g(1)=\sqrt{p}$
by~\sref{sqrt},
thereby giving~$g=\sqrt[4]{p}\,(\,\cdot\,)\,\sqrt[4]{p}$.\qed
\end{point}
\end{point}
\end{parsec}
\begin{parsec}%
\begin{point}[uniqueness-sequential-product]{Theorem}%
On the effects of every von Neumann algebra~$\scrA$
there is a unique binary operation~$\ast$
such that for all~$p$ from~$[0,1]_\scrA$,
\begin{enumerate}
\item \label{ax1}
$p\ast 1 = p$,
\item\label{ax2}
$p\ast q = f(q)$
for all~$q$ from~$[0,1]_\scrA$
for some pure map~$f\colon \scrA\to\scrA$,
\item\label{ax3}
$p\ast (p\ast q)=(p\ast p)\ast q$
for all~$q$ from $[0,1]_\scrA$,
\item\label{ax4}
$p=q\ast q$ for some~$q$ from~$[0,1]_\scrA$,
\item\label{ax5}
$p \ast e_1 \leq e_2^\perp
\iff p\ast e_2 \leq e_1^\perp$
for all projections~$e_1,e_2$ from~$\scrA$;
\end{enumerate}
namely, the sequential product,
given by
$p\ast q = \sqrt{p}q\sqrt{p}$
for all~$p, q$ from~$[0,1]_\scrA$.
\begin{point}{Proof}%
Let~$p$ be from $[0,1]_\scrA$ be given.
Pick~$p'$ from~$[0,1]_\scrA$
with $p = p'\ast p'$
using~\ref{ax4},
and find a pure map~$f\colon \scrA\to\scrA$
with~$f(q)=p'\ast q$ for all~$q$ from $[0,1]_\scrA$
using~\ref{ax2}.
Then~$f$ is purely self-adjoint by~\ref{ax5},
and so~$ff$ is purely positive.
Since~$f(f(1))=p'\ast (p'\ast 1)
= p'\ast p'=p$ by~\ref{ax1},
$ff=\sqrt{p}(\,\cdot\,)\sqrt{p}$
by~\sref{positive-map-uniqueness},
so $p\ast q
= (p'\ast p')\ast q
= p'\ast (p' \ast q) = f(f(q))=\sqrt{p}q\sqrt{p}$
for all~$q\in [0,1]_\scrA$ by~\ref{ax3}.\qed
\end{point}
\end{point}
\begin{point}{Exercise}%
We'll show  that none of the axioms
from~\sref{uniqueness-sequential-product}
can be omitted.
\begin{enumerate}
\item
Show that
$p\ast q := \ceil{p}q\ceil{p}$
satisfies all axioms of~\sref{uniqueness-sequential-product}
except~\ref{ax1}.
\item
Show that  $p\ast q := \floor{p}q\floor{p}\ +\ \smash{\sqrt{p-\floor{p}}\,q\,
\sqrt{p-\floor{p}}}$
satisfies all axioms except~\ref{ax2}.
\item
Show that if for every effect~$p$ of~$\scrA$
we pick a unitary~$u_p$ from~$\ceil{p}\!\scrA\!\ceil{p}$
then~$\ast$ given by
$p\ast q= \sqrt{p}u_p^* \,q\, u_p\sqrt{p}$
satisfies~\ref{ax1} and~\ref{ax2}.

Show that this~$\ast$ obeys~\ref{ax3} when~$u_p^2=u_{p^2}$,
and~\ref{ax4} when $pu_p=u_p p$,
and~\ref{ax5} when~$u_p^*=u_p$.

Conclude that when $u_p$ is defined by $u_p:=g(p)$,
where~$g\colon [0,1]\to\{-1,1\}$
is any Borel function with $g(\nicefrac{2}{3})=1$
and~$g(\nicefrac{4}{9})=-1$
the operation~$\ast$ (defined by~$u_p$ as above) satisfies
all conditions of~\sref{uniqueness-sequential-product} except~\ref{ax3}.
\item
\TODO{...}
\item
Show that there is a Borel
function~$g\colon[0,1]\to S^1$
with $g(\nicefrac{1}{2})\neq 1$
and~$g(\lambda^2)=g(\lambda)^2$ for all~$\lambda\in [0,1]$,
and that~$\ast$ given by~$p\ast q = \sqrt{p} g(p)^* \,q \,g(p)\sqrt{p}$
satisfies all conditions of~\sref{uniqueness-sequential-product}
except~\ref{ax5}.




\end{enumerate}
\end{point}
\end{parsec}


\section{Tensor product}
\subsection{Definition and existence}
\begin{parsec}%
\begin{point}{Definition}%
A bilinear map $\beta\colon \scrA\times \scrB\to\scrC$
between von Neumann algebras is
\begin{enumerate}
\item
\Define{\textbf{u}nital}
when~$\beta(1,1)=1$,
\item
\Define{\textbf{m}ultiplicative}
if~$\beta(ab,cd)=\beta(a,c)\beta(b,d)$
for all~$a,b\in\scrA$, $c,d\in\scrB$,
\item
\Define{\textbf{i}nvolution preserving}
if~$\beta(a,b)^*=\beta(a^*,b^*)$
for all~$a\in\scrA$, $b\in\scrB$.
\end{enumerate}
We abbreviate these properties as in~\sref{maps},
and say, for instance, that $\beta$ is \Define{miu-bilinear}
when it is unital, multiplicative and involution preserving.

This list is extended in~\TODO{add ref.}.
\end{point}
\begin{point}[tensor]{Definition}%
A miu-bilinear map $\gamma\colon \scrA\times\scrB\to\scrT$
between von Neumann algebras
is a \Define{tensor product} 
of~$\scrA$ and~$\scrB$
when it obeys the following conditions.
\begin{enumerate}
\item
\label{tensor-1}
The range of~$\gamma$ generates~$\scrT$.
\TODO{reformulate}
\item
\label{tensor-2}
For all np-functionals
$\sigma\colon \scrA\to\C$ and~$\tau\colon \scrB\to\C$
there is a (due to~\ref{tensor-1} unique)
np-functional $\Define{\gamma(\sigma,\tau)}\colon \scrT\to\C$
with, for all~$a\in\scrA$, $b\in\scrB$,
\begin{equation*}
	\gamma(\sigma,\tau)(\,\gamma(a,b)\,)\ =\ \sigma(a)\,\tau(b).
\end{equation*}
\item
\label{tensor-3}
These $\gamma(\sigma,\tau)$,
called \Define{product functionals},
are center separating.
\end{enumerate}
\end{point}
\begin{point}{Remark}%
This compact definition of the tensor product
leaves three questions unanswered:
whether such a tensor product of two von Neumann algebras
always exists,
whether it has some  universal property,
and whether it is unique in some way.
We'll presently address all three questions.
\end{point}
\end{parsec}
\begin{parsec}%
\begin{point}%
We'll start with the existence
of a tensor product of von Neumann algebras.
\end{point}
\begin{point}[special-tensor]{Theorem}%
Let~$\scrA$
and~$\scrB$
be von Neumann algebras
of bounded operators
on Hilbert spaces~$\scrH$ and~$\scrK$,
respectively.
Given $A\in\scrA$ and~$B\in\scrB$
we can form $A\otimes B\colon \scrH\otimes \scrK\to\scrH\otimes \scrK$
as in~\TODO{...} giving 
a bilinear map $\otimes\colon \scrA\times \scrB\to \scrB(\scrH\otimes\scrK)$.
Letting~$\scrT$ 
be the von Neumann subalgebra
of $\scrB(\scrH\otimes\scrK)$
generated by the range of~$\otimes$,
the restriction~$\gamma\colon \scrA\times \scrB\to\scrT$
of~$\otimes$
is a tensor product of~$\scrA$ and~$\scrB$.
\begin{point}{Proof}%
We need to check that the three conditions of~\sref{tensor} hold.
\begin{point}{Condition~\ref{tensor-1}}%
The range of~$\gamma$ being the same as the range of~$\otimes$
generates~$\scrT$ 
simply by the way~$\scrT$ was defined.
\end{point}
\begin{point}{Condition~\ref{tensor-2}}%
Let~$\sigma\colon \scrA\to\C$
and~$\tau\colon \scrB\to\C$ be np-maps.
We must find an np-functional~$\omega$ on~$\scrT$
with $\omega(A\otimes B)=\sigma(A)\tau(B)$
for all~$A\in \scrA$, $B\in \scrB$.
Note that by~\sref{normal-functional}
$\sigma$ and~$\tau$ are of the form
$\sigma\equiv\sum_n \left<x_n,(\,\cdot\,)x_n\right>$
and~$\tau\equiv\sum_n \left<y_n,(\,\cdot\,)y_n\right>$
for some~$x_1,x_2,\dotsc\in\scrH$
and~$y_1,y_2,\dotsc\in\scrK$
with~$\sum_n\|x_n\|^2<\infty$
and~$\sum_m\|y_m\|^2 <\infty$.
So as~$\sum_{n,m}\|x_n\otimes y_m\|^2\equiv\sum_n \|x_n\|^2\,
\sum_m\|y_m\|^2 <\infty$,
we can define
an np-functional~$\omega$ on~$\scrT$
by $\omega(T):= \sum_{n,m} \left<x_n\otimes y_m,
T\,x_n\otimes y_m\right>$;
which does the job:  $\omega(A\otimes B)
= \sum_{n,m} \left<x_n, Ax_n\right> \left<y_m,By_m\right>
= \sigma(A)\tau(B)$
for all~$A\in\scrA$ and~$B\in\scrB$.
\end{point}
\begin{point}{Condition~\ref{tensor-3}}%
It remains to be shown that the
product functionals
on~$\scrT$ are center separating,
and we'll show, in fact, that they form a faithful collection.
These functionals are---as we've just seen---all
of the form
$\sum_{m,n}\left<x_n\otimes y_n,(\,\cdot\,)\,x_n\otimes y_m\right>$
for some $x_1,x_2,\dotsc\in\scrH$
and~$y_1,y_2,\dotsc\in\scrK$
(and, conversely, it's easily seen
that a functional
of that form
is a product functional).
It suffices,
then,
to show that the subset of
product functionals
of the  form $\left<x\otimes y,(\,\cdot\,)x\otimes y\right>$
where~$x\in\scrH$ and~$y\in\scrK$ is faithful.
To this end,
let~$T\in\scrT_+$
with $\left< x\otimes y, Tx\otimes y\right>=0$
for all~$x\in\scrH$ and~$y\in\scrK$ be given
in order to show that~$T=0$.
Note that since
$\|\sqrt{T}\,x\otimes y\|^2=\left<x\otimes y,T\,x\otimes y\right>
=0$, and so~$\sqrt{T}\,x\otimes y=0$
for all~$x\in\scrH$, $y\in \scrK$,
we have~$\sqrt{T}=0$ (since the linear
span of the $x\otimes y$ is dense in~$\scrH\otimes\scrK$),
and thus~$T=0$.\qed
\end{point}
\end{point}
\end{point}
\begin{point}{Exercise}%
Given von Neumann algebras~$\scrA$ and~$\scrB$
construct a tensor product~$\gamma\colon \scrA\times \scrB\to\scrT$
of~$\scrA$ and~$\scrB$
using~\sref{ngns} and~\sref{spacial-tensor}.
\end{point}
\end{parsec}
\subsection{Universal property}
\begin{parsec}%
\begin{point}%
Before we bring our categorical faculties to bear
upon the tensor product of von Neumann algebras
it is perhaps appropriate
to briefly recall the algebraic tensor product
of vector spaces~$V$ and~$W$ first ---
it is a vector space~$V\odot W$
equipped
with a bilear mapping~$\odot\colon V\times W\to V\odot W$
which is  universal  
in the sense that for every bilinear mapping~$\beta\colon V\times W\to Z$
into some vector space~$Z$
there is a unique linear map~$\beta_\odot\colon V\odot W\to Z$
with~$\beta_\odot(v\odot w)=\beta(v,w)$
for all~$v\in V$ and~$w\in W$.
This property uniquely determines the algebraic tensor product in the sense
that for any bilinear map~$\mathbin{\tilde\odot}\colon
 V\times W\to V\mathbin{\tilde\odot} W$
 into a vector space~$V\mathbin{\tilde \odot} W$
which shares this property
there is a unique linear isomorphism $\varphi\colon V\odot W\to V
\mathbin{\tilde \odot} W$
with $\varphi(v\odot w) = v\mathbin{\tilde\odot} w$
for all~$v\in V$ and~$w\in W$.

In fact, one may take this property as a neat abstract 
definition of the algebraic
tensor product.
However, to  see that the darn thing actually exists,
one still needs a concrete description
such as this one:
take given a basis~$B$ of~$V$ and a basis~$C$ of~$W$
the bilinear map $\odot$ on~$V\times W$
to the vector space $(B\times C)\cdot \C$ with basis~$B\times C$
determined by~$b\odot c = (b,c)$
for~$b\in B$ and~$c\in C$.
This shows us not only that the algebraic tensor product
exists,
but also 
that~$\odot$ is injective (among other things).

This is all, of course, well known; the interesting thing here is that
with some work one can see that
a tensor product
$\gamma\colon \scrA\times \scrB\to \scrT$
of von Neumann algebras
$\scrA$ and~$\scrB$
has a similar universal property!
We'll see that any bilinear map $\beta\colon \scrA\times \scrB\to\scrC$
into a von Neumann algebra~$\scrC$
which is sufficiently regular
extends uniquely along~$\gamma$ to a ultraweakly continuous
map $\beta_\gamma \colon  \scrT\to\C$,
where regular will mean 
that the extension $\beta_\odot\colon \scrA\odot\scrB\to\scrC$
from the \emph{algebraic} tensor product
is ultraweakly continuous
and bounded
with respect to the norm and ultraweak topology
induced on~$\scrA\odot\scrB$ by~$\scrT$
via~$\gamma$.

To prevent a circular description
here
we'll first describe the norm and ultraweak topology
that the tensor product induces on~$\scrA\odot \scrB$
directly,
which turns out to be independent (as it should)
from the choice of~$\gamma$.
This description
is essentially based
on the fact that the product functionals on~$\scrT$
are center separating;
and that this determines both norm and 
ultraweak topology
is just a general 
observation concerning
center separating sets, as we saw in~\sref{vn-center-separating-fundamental}.
\end{point}
\begin{point}[tensor-extra]{Definitions}%
Let~$\scrA$ and~$\scrB$ be von Neumann algebras.
\begin{enumerate}
\item
A \Define{basic functional}
is 
a map $\omega \colon \scrA\odot\scrB\to\C$
with
$\omega\equiv (\sigma\odot \tau)(t^*(\,\cdot\,)t)$
for some np-maps
$\sigma\colon \scrA\to\C$, $\tau\colon \scrB\to\C$,
and
$t\in \scrA\odot\scrB$.

A \Define{simple functional}
is a finite sum of basic functionals.
\item
Each basic functional $\omega \colon \scrA\odot\scrB\to\C$
gives us an operation~$\Define{[\,\cdot\,,\,\cdot\,]_\omega}$,
that will turn out to be an inner product in~\sref{basic-state-inner-product}
 by
$\Define{[s,t]_\omega}:=\omega(s^*t)$
(cf.~\sref{state-inner-product}),
and an associated semi-norm
denoted by~$\Define{\|t\|_\omega}:=[t,t]_\omega^{\smash{\nicefrac{1}{2}}} 
= \omega(t^*t)^{\nicefrac{1}{2}}$.

The \Define{tensor product norm}
on~$\scrA\odot \scrB$
is the norm (see~\sref{tensor-product-norm})
given by
\begin{equation*}
	\textstyle
	\Define{\|t\|}\ =\ \sup_\omega \|t\|_\omega,
\end{equation*}
where~$\omega$ ranges over all basic functionals
on~$\scrA\odot\scrB$
with~$\omega(1)\leq 1$.
\item
Note that having endowed~$\scrA\odot \scrB$
with the tensor product norm
we can speak of bounded functionals on~$\scrA\odot \scrB$,
and the operator norm on them;
and note that
the basic and simple  functionals are bounded.

The \Define{ultraweak tensor product topology}
is the least topology on~$\scrA\odot\scrB$
that makes all operator norm limits
of simple functionals continuous.
\item
A bilinear map $\beta\colon \scrA\times \scrB\to\scrC$
to a von Neumann algebra~$\scrC$
is called
\begin{enumerate}
\item
(continues the list from~\TODO{...})
\item
\Define{bounded}
when the unique extension $\beta_\odot \colon \scrA\odot \scrB\to \scrC$
is bounded,
\item 
\Define{\textbf{n}ormal}
when~$\beta_\odot$
is continuous with respect to the ultraweak tensor product topology 
on~$\scrA\odot\scrB$
and the ultraweak topology on~$\scrC$,
\item
\Define{\textbf{c}ompletely \textbf{p}ossitive}
when~$\sum_{i,j,k}c_k^*\,\beta(a_i^*a_j,b_i^*b_j)\,c_k\geq 0$
for all tuples $a_1,\dotsc,a_N\in\scrA$,
$b_1,\dotsc,b_N\in\scrB$,
and $c_1,\dotsc,c_N\in\scrC$.
\end{enumerate}
\end{enumerate}
\end{point}
\begin{point}[product-state-positive]{Lemma}%
Given $C^*$-algebras~$\scrA$ and~$\scrB$
we have~$(\sigma\odot \tau) (t^*t)\geq 0$
for all  $t\in\scrA\odot\scrB$
and p-maps  $\sigma\colon \scrA\to\C$
and~$\tau\colon \scrB\to\C$.
\begin{point}{Proof}%
Note that writing~$t\equiv \sum_n a_n \odot b_n$,
where~$a_1,\dotsc,a_N\in \scrA$, $b_1,\dotsc,b_N\in \scrB$,
we have
$(\sigma\odot\tau)(t^*t)
= \sum_{n,m} \sigma(a_n^*a_m)\,\tau(b_n^*b_m)$.
Since~$A:=(a_n^*a_m)$
is a positive matrix over~$\scrA$,
and~$\sigma\colon \scrA\to\C$
is completely positive (by~\sref{cp-commutative}),
the matrix~$(M_N\sigma)(A)$ is positive,
and thus of the form~$(M_N\sigma)(A) = C^*C$
for some $N\times N$-matrix  $C\equiv(c_{nm})$,
giving us that~$\sigma(a_n^*a_m)=\sum_k \overline{c}_{kn} c_{km}$
for all~$n,m$.
Since by the same token there is an $N\times N$-matrix
$D\equiv(d_{nm})$
with
$\tau(b_n^* b_m ) = \sum_\ell \overline{d}_{\ell n} d_{\ell m}$
we get
$(\sigma\odot\tau)(t^*t)
= \sum_{n,m} \sigma(a_n^*a_m)\,\tau(b_n^*b_m)
= \sum_{n,m,k,\ell } \overline{d}_{k n}
c_{k m} \overline{d}_{\ell n} d_{\ell m}
= \sum_{k,\ell}\,\left|\,\sum_{n} c_{k n} d_{\ell n}\,\right|^2
\ \geq\  0$.\qed
\end{point}
\end{point}
\begin{point}[basic-state-inner-product]{Exercise}%
Use~\sref{product-state-positive} 
to show that
 $[\,\cdot\,,\,\cdot\,]_\omega$
from~\sref{tensor-extra}
is an inner product.
\end{point}
\begin{point}{Lemma}%
Product functionals on~$\scrA\odot\scrB$
formed from 
separating
collections~$\Omega$ and~$\Xi$ 
of linear functionals
on $C^*$-algebras~$\scrA$ and~$\scrB$,
respectively,
are seperating
in the sense that given~$t\in\scrA\odot\scrB$
the condition that $(\sigma\odot \tau)(t)=0$
for all~$\sigma\in \Omega$ and~$\tau\in\Xi$
entails that~$t=0$.
\begin{point}{Proof}%
Write~$t\equiv \sum_n a_n\odot b_n$
for some  $a_1,\dotsc,a_N\in\scrA$
and~$b_1,\dotsc,b_N\in \scrB$.
Note that by replacing them if necessary
we may assume that the~$a_1,\dotsc,a_N$
are linearly independent.
Let~$\tau\in\Xi$ be given.
Since~$0=(\sigma\odot \tau)(t)
= \sum_n\sigma(a_n)\tau(b_n)
= \sigma(\,\sum_n a_n\tau(b_n)\,)$
for all $\sigma$ from the separating collection~$\Omega$,
we have~$0=\sum_n a_n\tau(b_n)$,
and so---$a_1,\dotsc,a_N$ being linearly independent---we get
 $0=\tau(b_1)=\dotsb = \tau(b_N)$.
Since this holds for any~$\tau$
in the separating collection~$\Xi$
we get~$0=b_1=\dotsb=b_N$,
and thus~$t=\sum_n a_n\odot b_n=0$.\qed
\end{point}
\end{point}
\begin{point}{Exercise}%
Show that the tensor product norm
from~\sref{tensor-extra}
is, indeed, a norm.
\end{point}
\begin{point}[product-functional]{Exercise}%
Note that given np-functionals 
$\sigma \colon \scrA\to\C$
and~$\tau\colon \scrB\to\C$
on von Neumann algebras,
the functional $\sigma\odot\tau\colon \scrA\odot\scrB\to\C$
is ultraweakly continuous and bounded,
almost by definition.

Show that $f\odot g$ is
bounded and ultraweakly continuous too
for all~$f\in\scrA_*$ and~$g\in\scrB_*$
(perhaps using~\sref{luws}).
\end{point}
\begin{point}[tensor-basic]{Exercise}%
We're going to show that 
the ultraweak tensor product topology
and tensor product norm from~\sref{tensor-extra}
actually describe the norm and ultraweak topology
on~$\scrA\odot\scrB$ induced by a tensor product
$\scrA\times\scrB\to\scrT$ (via~$\gamma_\odot$)
by establishing
the two closely related facts that
$\gamma_\odot\colon\scrA\odot\scrB\to\scrT$
is an isometry
and an ultraweak embedding,
and that certain
functionals 
$\omega\colon \scrA\odot\scrB\to\C$
can be extended uniquely to~$\scrT$ along~$\gamma_\odot$.
\begin{enumerate}
\item
Show using~\sref{vn-center-separating-fundamental}
that the collection~$\Omega$
of np-functionals
on~$\scrT$ of the form 
$\gamma(\sigma,\tau)(\gamma_\odot(s)^*(\,\cdot\,)\gamma_\odot(s))$,
where~$\sigma\colon \scrA\to\C$,
$\tau\colon \scrB\to\C$
are np-functionals
and~$s\in\scrA\odot\scrB$,
is order separating,
and that
every np-functional on~$\scrT$
is the operator norm limit of finite sums
of functionals from~$\Omega$.

Show that~$\omega\circ \gamma_\odot$ is a basic functional
(see \sref{tensor-extra})
for every~$\omega\in\Omega$,
and that every basic functional is of this form
for some unique~$\omega\in\Omega$.

\item
Show that the subset~$\Omega_1$ of~$\Omega$
of unital maps is order separating,
and so determines the norm on~$\scrT$
via~$\|a\|^2=\|a^*a\| = \sup_{\omega\in\Omega_1} \omega(a^*a)$
for all~$a\in\scrT$
(see~\sref{order-separating-norm}).

Prove that $\|\gamma_\odot(s)\|
=\sup_{\omega\in\Omega_1} \omega(s^*s)^{\nicefrac{1}{2}}
=\sup_{\omega\in\Omega_1}
\|s\|_{\omega\circ \tau_\odot}=\|s\|$
for all~$s\in\scrA\odot\scrB$,
and 
conclude
that~$\gamma_\odot$ is an isometry.

\item
Show that~$\|f\circ\gamma_\odot\|\leq \|f\|$
for every $f\in\scrT_*$,
and deduce from this
that when~$\omega\colon \scrT\to\C$
is an np-functional
its restriction $\omega\circ\gamma_\odot$
is the operator norm limit
of simple functionals on~$\scrA\odot\scrB$
implying that~$\omega\circ\gamma_\odot$---and
thus $\gamma_\odot$ itself---is ultraweakly continuous.

\item
In order to show that~$\gamma_\odot$ is an ultraweak embedding,
we'll first need that we have
in fact the equality $\|f\circ \gamma_\odot\|=\|f\|$
for all~$f\in\scrT_*$.

In order to show this,
recall 
(from \sref{polar-decomposition-of-functional})
that there is a partial isometry~$u$ in~$\scrT$
with~$f(u)=\|f\|$ (see~\sref{functional-norm}). 

Show that given~$\varepsilon>0$ 
there is a net~$(s_\alpha)_\alpha$
in~$\scrA\odot\scrB$
with $\|s_\alpha\|\leq 1+\varepsilon$ for all~$\alpha$
such that $\gamma_\odot(s_\alpha)$
converges ultrastrongly to~$t$ as~$\alpha\to\infty$
(cf.~\sref{dense-subalgebra}).

Deduce that $\|f\|=f(u)=\left|f(u)\right|
=\lim_\alpha \left|f(\gamma_\odot(s_\alpha))\right|
\leq \|f\circ \gamma_\odot\| (1+\varepsilon)$,
and
conclude that~$\|f\|=\|f\circ\gamma_\odot\|$.

\item
Show that any functional $\omega'\colon \scrA\odot\scrB\to\C$
that is the operator norm limit of simple functionals
on~$\scrA\odot\scrB$
can be extended uniquely along~$\gamma_\odot$
to an np-functional on~$\scrT$
(using the fact that the operator
norm limit of np-functionals is an np-functional again,
see~\TODO{...}).

Deduce from this that~$\gamma_\odot$ is a ultraweak topological embedding.

(Note that by~\sref{vn-extension} any 
bounded ultraweakly continuous functional
on~$\scrA\odot\scrB$ can be extended uniquely
to a normal functional on~$\scrT$.)
\end{enumerate}
\end{point}
\begin{point}[tensor-universal-property]{Theorem}%
A tensor product~$\gamma\colon \scrA\times\scrB\to\scrT$
of von Neumann algebras~$\scrA$ and~$\scrB$
has this universal property:
for every normal bounded bilinear map $\beta\colon \scrA\times \scrB\to\scrC$
to a von Neumann algebra~$\scrC$
there is a unique ultraweakly continuous
map  $\Define{\beta_\gamma}\colon \scrT
\to \scrC$ with $\beta_\gamma\circ \gamma  = \beta$.
Moreover, $\|\beta_\gamma\|=\|\beta_\odot\|$.
\begin{point}{Proof}%
Since $\beta_\odot\colon \scrA\odot\scrB\to\scrC$
is ultraweakly continuous and bounded,
and
$\scrA\odot\scrB$
can by~\sref{tensor-basic} be
considered an ultraweakly dense $*$-subalgebra
of~$\scrT$ via~$\gamma_\odot$,
the theorem follows from~\sref{vn-extension}
except for some trivial details.\qed
\end{point}
\end{point}
\end{parsec}%
\begin{parsec}%
\begin{point}%
Concerning completely positive
bilinear maps.
\end{point}
\begin{point}{Exercise}%
Show that a mi-bilinear
map $\beta\colon \scrA\times\scrB\to\scrC$ between von Neumann 
algebras is completely positive.
\end{point}
\begin{point}{Notation}%
Given a bilinear map
$\beta\colon \scrA\times\scrB\to\scrC$
between von Neumann algebras,
we define
$\Define{M_N\beta}\colon M_N\scrA\times M_N\scrB\to M_N\scrC$
by  $(M_N\beta)(A,B) = 
(\beta(A_{ij},B_{ij}))_{ij}$
for each~$N$.
\end{point}
\begin{point}[cp-bilinear]{Exercise}%
Show that for a bilinear map $\beta\colon \scrA\times\scrB\to\scrC$
between von Neumann algebras
the following are equivalent.
\begin{enumerate}
\item
$\beta$ is completely positive.
\item
$M_N\beta$ is completely positive
for each~$N$.
\item
$(M_N\beta)(A,B)\geq 0$
for all~$A\in M_N(\scrA)_+$, $B\in M_N(\scrB)_+$ and~$N$.
\end{enumerate}
Deduce 
as a corollary that $h\circ \beta \circ (f\times g)$
is completely positive
when~$f\colon \scrA'\to\scrA$,
$g\colon \scrB'\to\scrB$
and~$h\colon \scrC\to\scrC'$
are cp-maps between von Neumann algebras.
\end{point}
\end{parsec}
\begin{parsec}%
\begin{point}[tensor-universal-property-extra]{Exercise}%
Let $\gamma\colon \scrA\times \scrB\to\scrT$
be a tensor product
of von Neumann algebras,
  $\beta\colon \scrA\times \scrB\to\scrC$
 a normal bounded bilinear map,
 and~$\beta_\gamma\colon \scrT\to\scrC$
 its extension along~$\gamma_\odot$
from~\sref{tensor-universal-property}.
Show that
\begin{enumerate}
\item
$\beta_\gamma$
is multiplicative iff~$\beta$ is multiplicative
(see~\sref{tensor-extra});
\item 
$\beta_\gamma$ 
is involution preserving iff~$\beta$ is involution preserving;
\item
$\beta_\gamma$ is unital
iff $\beta$ is unital;
\item
$\beta_\gamma$ is positive
iff
$\sum_{i,j} \beta(a_i^*a_j,b_i^*b_j) \geq 0$
for all tuples $a_1,\dotsc,a_N$
from~$\scrA$ and $b_1,\dotsc,b_N$ from $\scrB$;
\item
$\beta_\gamma$ is completely positive iff
$\beta$ is completely positive.
\end{enumerate}
\TODO{Check if some hints need to be added}

\end{point}
\begin{point}[tensor-uniqueness]{Exercise}%
Show that the tensor product of von Neumann algebras~$\scrA$
and~$\scrB$ is unique in
the sense
that when~$\gamma\colon \scrA\times \scrB\to\scrT$
and~$\gamma'\colon \scrA\times \scrB\to\scrT'$
are tensor products of~$\scrA$ and~$\scrB$,
then there is a unique
nmiu-isomorphism $\varphi\colon \scrT\to\scrT'$
with $\varphi(\tau(a,b))=\tau'(a,b)$
for all~$a\in\scrA$ and~$b\in\scrB$.
\end{point}
\end{parsec}
\subsection{Functoriality}
\begin{parsec}%
\begin{point}{Proposition}%
Given ncp-maps $f\colon \scrA\to\scrC$
and $g\colon \scrB\to\scrD$
between von Neumann algebras
there is a unique ncp-map
$\Define{f\otimes g}\colon \scrA\otimes\scrB\to\scrC\otimes\scrD$
with 
\begin{equation*}
	(f\otimes g)(a\otimes b) \ =\ f(a)\otimes f(b)
\end{equation*}
for all~$a\in\scrA$ and~$b\in\scrB$.
Moreover,
\begin{enumerate}
\item
$f\otimes g$ is multiplicative 
when~$f$ and~$g$ are multiplicative;
\item
$f\otimes g$ is involution preserving
when~$f$ and~$g$ are involution preserving; and
\item
$f\otimes g$ is (sub)unital 
when~$f$ and~$g$ are (sub)unital.
\end{enumerate}
\begin{point}{Proof}%
As uniqueness of~$f\otimes g$ is rather obvious,
we leave it at that.
To establish
existence of~$f\otimes g$,
it suffices to show
that the bilinear map~$\beta\colon \scrA\times\scrB\to\scrC\otimes\scrD$
given by~$\beta(a,b)=f(a)\otimes g(b)$,
which is completely positive by~\sref{cp-bilinear},
is bounded and normal;
because then we may take $f\otimes g:=\beta_\otimes$
as in~\sref{tensor-universal-property}
and all
the stated properties of~$f\otimes g$ will then  follow 
with the very least of effort
from~\sref{tensor-universal-property-extra}.

To see that~$\beta$ is bounded,
we'll prove that~$\|\beta_\odot(s)\| \leq \|f\|\|g\| \|s\|$
given an element~$s$ of~$\scrA\otimes \scrB$,
and for this
it suffices (by the definition
of the tensor product norm, \sref{tensor-extra}) 
to show that
$\omega(\beta_\odot(s)^*\beta_\odot(s))
\leq \|f\|^2\|g\|^2\|s\|^2$
given a basic functional~$\omega$
on~$\scrA\odot\scrB$ with~$\omega(1)\leq 1$.
We'll prove in a moment that
$\|\omega\circ\beta_\odot\|\leq \|f\|\|g\|$
and~$\beta_\odot(s)^*\beta_\odot(s)\leq
\|f\|\|g\|\beta_\odot(s^*s)$,
because with these two claims
we get 
$\omega(\beta_\odot(s)^*\beta_\odot(s))
\leq \omega(\beta_\odot(s^*s))
\leq \|f\|\|g\| \|\omega\circ \beta_\odot\| \|s\|^2
\leq \|f\|^2\|g\|^2\|s\|^2$
 --- which is the result desired.

Concerning the first
promise, that $\|\omega\circ \beta_\odot\|\leq \|f\|\|g\|$,
note that writing
$\omega\equiv (\sigma\odot\tau)(t^*(\,\cdot\,)t)$,
where~$\sigma$ and~$\tau$
are np-maps on~$\scrC$ and~$\scrD$, respectively,
and~$t\equiv\sum_{ij} c_i \odot d_i$
is from~$\scrC\odot\scrD$,
we have 
\begin{equation*}
	\omega\circ\beta_\odot
= \textstyle \sum_{ij} \sigma(c_i^* f(\,\cdot\,)c_j)\,\odot\,
\tau(d_i^* g(\,\cdot\,)d_j),
\end{equation*}
and so~$\omega\circ \beta_\odot$
is ultraweakly continuous and bounded
by~\sref{product-functional},
because the $\sigma(c_i^*f(\,\cdot\,)c_j)$
and~$\tau(d_i^*g(\,\cdot\,)d_j)$
are bounded ultraweakly continuous functionals.
Although the bound for~$\omega\circ\beta_\odot$ 
thus obtained is in all probability nowhere near~$\|f\|\|g\|$,
it does allow
us by~\sref{tensor-universal-property}
to extend~$\omega\circ\beta_\odot$
to an ultraweakly continuous 
functional $\omega':=(\omega\circ\beta)_\otimes$ on~$\scrC\otimes \scrD$
with the same norm, $\|\omega'\|=\|\omega\circ\beta_\odot\|$.
Since this extension~$\omega'$
is completely positive
(because~$\beta$ and thus~$\omega\circ \beta$ are completely positive,
see~\sref{cp-bilinear})
its norm is by~\sref{cp-russo-dye} given by~$\|\omega'\|=\omega'(1)\equiv
\omega(f(1)\otimes g(1))\leq \|f\|\|g\|$,
where we used that~$\omega(1)\leq 1$.
Thus $\|\omega\circ \beta_\odot\|=\|\omega'\| \leq \|f\|\|g\|$,
as was claimed.

Incidentally,
since each~$\omega\circ \beta_\odot$
is ultraweakly continuous,
so is~$\beta_\odot$,
and thus~$\beta$ is normal.
The only thing that remains
is to make good on our last promise,
that~$\beta_\odot(s)^*\beta_\odot(s)\leq 
\|f\|\|g\|\beta_\odot(s^*s)$.
To this end,
write $s\equiv \sum_i a_i\odot b_i$,
and consider
the matrices~$A$ and~$B$ given by
\begin{equation*}
A\ :=\  \begin{pmatrix}
a_1 & a_2 & \dotsb & a_n \\
0 & 0 & \dotsb & 0 \\
\vdots & \vdots & \ddots&\vdots \\
0 & 0 & \dotsb & 0
\end{pmatrix}
\qquad\qquad
B\ :=\  \begin{pmatrix}
b_1 & b_2 & \dotsb & b_n \\
0 & 0 & \dotsb & 0 \\
\vdots & \vdots & \ddots&\vdots \\
0 & 0 & \dotsb & 0
\end{pmatrix},
\end{equation*}
and the cp-map $h\colon M_n(\scrC\otimes \scrD)\to \scrC\otimes \scrD$
(see~\TODO{vector state of Hilbert $\scrB$-modules})
given by~$h(C)= \left<(1,\dotsc,1), C(1,\dotsc,1)\right> =\sum_{ij}C_{ij}$.
We make these arrangements
so that we
may apply the inequality
$(M_nf)(A)^*  (M_nf)(A)
\leq \|(M_nf)(1)\| (M_nf)(A^*A)$
easily derived from~\sref{cp-cs}.
Indeed,
noting also~$\|(M_nf)(1)\|=\|f(1)\|=\|f\|$,
we have
\begin{alignat*}{3}
	\beta_\odot(s)^*\beta_\odot(s) 
\ &=\ \textstyle \sum_{ij}f(a_i)^*f(a_j)\otimes g(b_i)^*g(b_j)\\
\ &=\ h(\ 
(M_nf)(A)^* (M_nf)(A) \ \ (M_n \otimes)\ \ 
(M_ng)(B)^*(M_ng)(B) \ ) \\
\ &\leq\  
\|f\|\|g\| 
h(\ (M_nf)(A^*A) \ \ (M_n \otimes)\ \ 
(M_ng)(B^*B) \ ) 
\\
\ &=\  
\|f\|\|g\| \,\textstyle \sum_{ij} f(a_i^*a_j)\otimes g(b_i^*b_j) \\
\ &= \ \|f\|\|g\|\,\beta_\odot(s^*s),
\end{alignat*}
which concludes this proof.\qed
\end{point}
\end{point}
\begin{point}{Exercise}%
Show that the assignments
$(\scrA,\scrB)\mapsto \scrA\otimes \scrB$,
and~$(f,g)\mapsto f\otimes g$
give a bifunctor~$\otimes\colon \Cat{C}\times\Cat{C}
\to\Cat{C}$
where~$\Cat{C}$
can be~$\W{miu}$, $\W{cp}$, $\W{cpu}$
and $\W{cpsu}$.
\end{point}
\begin{point}[tensor-injective]{Proposition}%
Given injective nmiu-maps
$f\colon \scrA\to\scrC$ and~$g\colon \scrB\to\scrD$,
the nmiu-map $f\otimes g\colon \scrA\otimes \scrB\to
\scrC\otimes\scrD$ is injective.
\begin{point}{Proof}%
The trick
is to consider the von Neumann subalgebra~$\scrT$
generated by
the elements of~$\scrC\otimes \scrD$
of the form~$f(a)\otimes g(b)$
where~$a\in\scrA$ and~$b\in\scrB$,
and to show that  the miu-bilinear map
$\gamma\colon \scrA\times\scrB\to\scrT$
given by~$\gamma(a,b)=f(a)\otimes g(b)$
is a tensor product of~$\scrA$ and~$\scrB$.
Indeed,
if this is achieved,
then
there is, by~\sref{tensor-uniqueness},
a unique nmiu-map $\varphi\colon \scrA\otimes\scrB\to\scrT$
with $\varphi(a\otimes b)=\gamma(a,b)
=f(a)\otimes g(b)$,
so that the following diagram commutes.
\begin{equation*}
\xymatrix@C=4em{
\scrA\times \scrB
\ar[rr]^-{f\times g}
\ar[rd]^-\gamma
\ar[d]_-\otimes
&
&
\scrC\times\scrD 
\ar[d]^-\otimes
\\
\scrA\otimes\scrB
\ar[r]|-\varphi
&
\scrT
\ar[r]|-\subseteq
&
\scrC\otimes\scrD
}
\end{equation*}
The map on the bottom side of this rectangle above is
none other than~$f\otimes g$,
and is thus,
being
the composition of the isomorphism~$\varphi$
with the inclusion $\scrT\subseteq \scrC\otimes\scrD$,
injective.

It remains to be show that~$\gamma$
is a tensor product,
that is, obeys the conditions from~\sref{tensor}.
Condition~\ref{tensor-1}
holds simply by definition of~$\scrT$.
To see that~$\gamma$
obeys condition~\ref{tensor-2},
let np-functionals
$\tilde\sigma\colon \scrA\to\C$
and $\tilde\tau\colon \scrB\to\C$ be given;
we must find an np-functional $\gamma(\tilde\sigma,
\tilde\tau)$ on~$\scrT$
with $\gamma(\tilde\sigma,\tilde\tau)(a\otimes b)
= \gamma(a,b)$.

By ultraweak permanence
$\tilde\sigma$ and~$\tilde\tau$
can be extended along~$f$ and~$g$, respectively,
see~\sref{functional-extension},
giving us np-functionals $\sigma\colon \scrC\to\C$
and $\tau\colon\scrD\to\C$
with~$\tilde\sigma = \sigma\circ f$
and $\tilde\tau = \tau\circ g$.
Now simply take $\gamma(\tilde\sigma,\tilde\tau)$
to be the restriction
of $\sigma\otimes \tau$
to~$\scrT$,
which does the job.

Finally,
concerning
condition~\ref{tensor-3},
let~$z$ be a central projection of~$\scrT$
with $\gamma(\tilde\sigma,\tilde\tau)(z)=0$
for all~$\tilde\sigma$
and~$\tilde\tau$ of aforementioned type.
We must show that~$z=0$,
and for this
it suffices to show that
$(\sigma\otimes \tau)(z)=0$
for all  np-functionals
$\sigma$ and~$\tau$ on~$\scrC$ and~$\scrD$,
respectively.
Since for such~$\sigma$ and~$\tau$
we have
$\gamma(\tilde\sigma,\tilde\tau)
(\gamma(a,b))
 = 
 \sigma(f(a))\,\tau(g(b))
 = (\sigma\otimes\tau)(\gamma(a,b))$
 for all~$a\in\scrA$ and~$b\in\scrB$,
 we have $\gamma(\tilde\sigma,\tilde\tau)
 (t) = (\sigma\otimes \tau)(t)$
 for all~$t\in\scrT$,
 and, in particular,
 $0=\gamma(\tilde\sigma,\tilde\tau)(z)
 = (\sigma\otimes\tau)(z)$.
 Hence~$z=0$.\qed
\end{point}
\end{point}
\end{parsec}%
\begin{parsec}%
\begin{point}[product-functional-norm]{Lemma}%
Given von Neumann algebras~$\scrA$ and~$\scrB$,
we have $\|f\otimes g\|=\|f\|\|g\|$
for all~$f\in\scrA_*$ and~$g\in\scrB_*$.
\begin{point}{Proof}%
The trick is to use the polar decomposition
for normal functionals, \sref{polar-decomposition-of-functional}.
On its account we can find partial isometries $u\in\scrA$
and $v\in\scrB$
such that
$f(u(\,\cdot\,))$
and~$g(v(\,\cdot\,))$ are positive,
and
$f\equiv f(uu^*(\,\cdot\,))$,  $g\equiv g(vv^*(\,\cdot\,))$.
Then $u\otimes v$ is a partial isometry
such that $(f\otimes g)((u\otimes v)(\,\cdot\,))$
is positive,
and $f\otimes g
= (f\otimes g)(\,(u\otimes v)\, (u\otimes v)^*\,(\,\cdot\,)\,)$
so that $\|f\otimes g\|=(f\otimes g)(u\otimes v)
= f(u)g(v)=\|f\|\|g\|$ by~\sref{functional-norm}.\qed
\end{point}
\end{point}
\begin{point}[tensor-simple-facts]{Exercise}%
There are some easily obtained facts
concerning the tensor product~$\scrA\otimes\scrB$
of von Neumann algebras
that nevertheless deserve explicit mention.
\begin{enumerate}
\item
Show that~$a\otimes b\geq 0$
for all~$a\in\scrA_+$ and~$b\in\scrB_+$;
and conclude that $a_1\otimes b_1 \leq a_2\otimes b_2$
for all
$a_1\leq a_2$ from~$\scrA$ and $b_1\leq b_2$ from~$\scrB$.
\item
Show that $\|a\otimes b\| = \|a\|\|b\|$
for all~$a\in\scrA$ and~$b\in\scrB$.

Conclude that $\otimes\colon \scrA\times\scrB\to\scrA\otimes \scrB$
is norm continuous.

(Warning: as~$\otimes$ is not linear
this is not entirely trivial.)
\item
Show that $\otimes\colon \scrA_*\times\scrB_*\to(\scrA\otimes\scrB)_*$
is norm continuous
(using~\sref{product-functional-norm}).
\item
Show that~$\otimes\colon \scrA\times \scrB\to \scrA\otimes \scrB$
is ultraweakly continuous.

(Hint: since we
already know that~$\otimes_\odot\colon \scrA\odot\scrB\to\scrA\otimes\scrB$
is ultraweakly continuous, by~\sref{tensor-basic},
an equivalent question 
is whether~$\odot\colon \scrA\times\scrB\to\scrA\odot\scrB$
is ultraweakly continuous,
which may be boiled down
to the fact
that $(a,b)\mapsto \sum_{ij} \sigma(a_i^* a a_j)\,\tau(b_i^* b b_j)\colon
\ \scrA\times \scrB\to\C$ is ultraweakly continuous,
where~$\sigma$ and~$\tau$ are np-functionals
on~$\scrA$ and~$\scrB$, respectively,
and
$a_1,\dotsc,a_n\in\scrA$, and~$b_1,\dotsc,b_n\in\scrB$.)

\end{enumerate}
\end{point}
\begin{point}[tensor-generation]{Proposition}
Let~$\scrA$ and~$\scrB$ 
be von Neumann algebras.
\begin{enumerate}
\item
\label{tensor-generation-1}
If~$S$ and~$T$ are ultraweakly dense subsets of~$\scrA$
and~$\scrB$, respectively,
then~$\{\,s\otimes t\colon\,s\in S,\,t\in T\,\}$
is ultraweakly dense in~$\scrA\otimes\scrB$.
\item
\label{tensor-generation-2}
If~$\Omega$ and~$\Theta$ are center separating collections
of np-functionals on~$\scrA$ and~$\scrB$, respectively,
then~$\{\,\omega\otimes\vartheta\colon\, 
\omega\in\Omega,\,\vartheta\in\Theta\,\}$
is center seperating for~$\scrA\otimes\scrB$.
\end{enumerate}
\begin{point}{Proof}%
Concerning~\ref{tensor-generation-1},
since the elements of~$\scrA\otimes\scrB$
of the form $a\otimes b$
lie ultraweakly dense in~$\scrA\otimes\scrB$ 
where~$a\in\scrA$ and~$b\in\scrB$,
it suffices to show that such element~$a\otimes b$
is the ultraweak limit of elements of the form~$s\otimes t$
where~$s\in S$ and~$t\in T$.
This is indeed the case
as there are nets $(s_\alpha)_\alpha$
and~$(t_\beta)_\beta$ in~$S$ and~$T$
that converge to~$a$ and~$b$, respectively,
and so, 
because~$\otimes$ is ultraweakly continuous by~\sref{tensor-simple-facts},
we see that~$s_\alpha\otimes t_\beta$
converges ultraweakly to~$a\otimes b$
as~$\alpha,\beta\to\infty$.

Concerning~\ref{tensor-generation-2},
let~$t$ be a positive element of~$\scrA\otimes \scrB$
with $(\omega\otimes \vartheta)(s^*ts)=0$
for all~$\omega\in\Omega$, $\vartheta\in\Theta$,
and~$s\in\scrA\otimes \scrB$;
we must show that~$t=0$.
For this it suffices
to show that~$(\sigma\otimes\tau)(t)=0$
for all np-functionals  $\sigma\colon \scrA\to\C$
and 
$\tau\colon\scrB\to\C$
(since the product functionals $\sigma\otimes\tau$
form a faithful collection, by~\TODO{...}).
Now, since~$\Omega$ is center separating
such $\sigma$ may by~\sref{vn-center-separating-fundamental} be obtained
as operator norm limit of finite sums
of functionals of the
form~$\omega(a^*(\,\cdot\,)a)$
where~$\omega\in\Omega$ and~$a \in\scrA$.
Since a np-functional $\tau\colon \scrB\to\C$ 
can be obtained in a similar fashion
from~$\Theta$,
and~$\otimes \colon \scrA_*\otimes \scrB_*\to(\scrA\otimes\scrB)_*$
is operator norm continuous (by~\sref{tensor-simple-facts}),
we see that a product functional~$\sigma\otimes \tau$
can be obtained as the operator norm limit
of finite sums of functionals
of the form $\omega(a^*(\,\cdot\,)a)\,\otimes\,
\vartheta(b^*(\,\cdot\,)b)
\,\equiv\, (\omega\otimes\vartheta)(\,(a\otimes b)^*\,(\,\cdot\,)
\,(a\otimes b)\,)$;
and since those functionals 
map~$t$ to~$0$,
by assumption,
we conclude that~$(\sigma\otimes \tau)(t)=0$ too.\qed
\end{point}
\end{point}
\end{parsec}
\subsection{Monoidal structure}
\begin{parsec}%
\begin{point}%
Up to this point
we have only written about the tensor product~$\scrA\otimes \scrB$
of \emph{two} von Neumann algebra
(to save ink),
but all of it,
as you will no doubt have observed already,
can be easily adapted 
to deal with
a tensor product
$\otimes\colon \scrA_1\times\dotsc\scrA_n
\to\scrA_1\otimes \dotsb\otimes \scrA_n$
of a tuple $\scrA_1,\dotsc,\scrA_n$ of von Neumann algebras,
which will then, of course, be a multilinear map
instead of a bilinear map, etc..

What is less obvious
is that there should be any relation
between 
$(\scrA\otimes\scrB)\otimes \scrC$
and~$\scrA\otimes\scrB\otimes\scrC$;
but there is.
\end{point}
\begin{point}{Proposition}%
Given von Neumann algebras~$\scrA$, $\scrB$ and~$\scrC$,
the trilinear map $\gamma\colon (a,b,c)\mapsto (a\otimes b)\otimes c,\ 
\scrA\times\scrB\times\scrC \to (\scrA\otimes \scrB)\otimes\scrC$
is a tensor product.
\begin{point}{Proof}%
We need to verify the three conditions
from~\sref{tensor} (adapted
to trilinear maps).
The first condition,
that the elements of the form~$(a\otimes b)\otimes c$
generate $(\scrA\otimes \scrB)\otimes \scrC$
follows
by~\sref{tensor-generation}
since
the elements of the form~$a\otimes b$
generate~$\scrA\otimes \scrB$
(and~$\scrC$ generates~$\scrC$).
The second condition
is met by defining
$\gamma(\sigma,\tau,\upsilon):= (\sigma\otimes\tau)\otimes\upsilon$
for all np-functionals
$\sigma\colon \scrA\to\C$,
$\tau\colon \scrB \to\C$
and~$\upsilon\colon \scrC\to\C$.
Finally,
these product functionals
$\gamma(\sigma,\tau,\upsilon)$
are center separating by~\sref{tensor-generation}
because
the functionals on~$\scrA\otimes \scrB$
of the form $\sigma\otimes\tau$
are center separating (and so is
the set of all np-functionals on~$\scrC$),
which was the third condition.\qed
\end{point}
\end{point}
\begin{point}{Corollary}%
There is a unique nmiu-isomorphism
\begin{equation*}
		\alpha\colon \, \scrA\otimes(\scrB\otimes \scrC)
\longrightarrow  (\scrA \otimes \scrB)\otimes\scrC,
\end{equation*}
with $\alpha(a\otimes(b\otimes c))=(a\otimes b)\otimes c$
for all~$a\in\scrA$, $b\in\scrB$, $c\in\scrC$,
for any von Neumann algebras
$\scrA$, $\scrB$, $\scrC$.
\end{point}
\begin{point}{Exercise}%
Show that~$\W{miu}$, $\W{cp}$, $\W{cpu}$
and~$\W{cpsu}$ endowed with the tensor product
are monoidal categories with~$\C$ as unit.
\end{point}
\end{parsec}
\section{Quantum Lambda Calculus}
\begin{parsec}%
\begin{point}%
In this section
we lay the foundation
for a model
of the quantum lambda calculus
built from von Neumann algebras.
We will not venture
to describe the quantum lambda calculus
in all its details here,
nor will we fully describe the model
(as we did in~\TODO{Kenta and Bram}),
but we will instead
focus on the two key ingredients,
the interpretation of ``$\bang$'' and ``$\limp$''
---
with them the expert
can easily produce the model.

Nevertheless
I will try to give an impression for those who are not familiar 
with the quantum lambda calculus.
It is a type theory
proposed by Selinger and Valiron in~\TODO{...}
to describe programs for quantum computers
especially designed
to include 
not only
function types ($\limp$)
and
classical data types (such as~$\bit$),
but also quantum data types
(such as $\qbit$),
so that there can be a term such as
$\mathsf{new}\colon \bit\limp \qbit$
that represents the program
that initializes a qubit in the given state.
There are of course also terms
such as
$\mathsf{0}\colon \bit$
and~$\mathsf{1}\colon \bit$,
so that~$\mathsf{meas}\,\mathsf{0}\colon \qbit$
represents a qbit in state~$\ket{0}$.
The addition of quantum data 
to a type theory
is a very delicate matter
for if one were to allow
for example
in this system
a variable to be used twice
(a thing usually beyond dispute)
it would not take much more
to construct
a program
that duplicates the contents
of a qubit,
which is prohibited.

Still, classical data
such as a bit
can be duplicated freely,
so to accommodate this 
the type~$\bang \bit$ is used.
More precisely,
the type
$\bang A$
represents that part of the type of~$A$ that is duplicable,
so that~$\bang \bit$
is the proper type for a bit,
and $\bang\qbit$ is empty.
For example,
the term that represents the measurement
of a qubit is~$\mathsf{meas}\colon \qbit\limp \bang\bit$,
where the~$\bang$ indicates that the bit resulting from the measurement
may be duplicated freely.

The model
we alluded to
assigns to each type~$A$ a von Neumann algebra~$\sem{A}$,
e.g.~$\sem{\qbit}=M_2$
and~$\sem{\bit}=\C^2$.
A term~$t:A$
is interpretted as an npsu-functional $\sem{t:A}\colon \sem{A}\to\C$,
so for example $\sem{\mathsf{0}\colon \bit}\colon
(x,y)\mapsto x\colon \C^2\to\C$.
When~$t:A$ has free variables $x_1:B_1,\dotsc,x_N:B_N$
the interpretation 
becomes an ncpsu-map $\sem{t}\colon \sem{A}\to\sem{B_1}\otimes 
\dotsb \otimes \sem{B_N}$,
so for example,
\begin{equation*}
	\sem{\, x\colon \qbit \vdash \mathsf{meas} x\,}
\colon (x,y)\mapsto
\smash{\bigl(
	\begin{smallmatrix}x& 0 \\ 0 & y
	\end{smallmatrix}\bigr)}
\colon \C^2\to M_2.
\end{equation*}
In short, there are no surprises here.
As said, the difficulty lies
in the definition
of $\sem{!A}$ and~$\sem{A\limp B}$,
for which we will provide the following
three ingredients.
\begin{itemize}
\item
The fact (due to Kornell, see~\TODO{...})
that
the category 
$\op{(\W{miu})}$
is monoidal closed,
that is,
that
for every von Neumann algebra~$\scrB$,
the functor
$\scrB\otimes (\,\cdot\,)\colon 
\W{miu}\to \W{miu}$
has a left adjoint
$(\,\cdot\,)^{*\scrB}$.

\item
The following two adjunctions.
\begin{equation*}
\xymatrix@C=6em{
	\Cat{Set}
	\ar@/^1.0em/[r]^-{\ell^\infty}
	\ar@{}[r]|-{\perp}
	&
	\op{(\W{miu})}
	\ar@/^1.0em/[r]^-{\subseteq}
	\ar@/^1.0em/[l]^-{\nsp:=\W{miu}(-,\C)}
	\ar@{}[r]|-{\perp}
	&
	\op{(\W{cpsu})}
	\ar@/^1.0em/[l]^-{\mathcal{F}}
}
\end{equation*}
\end{itemize}
\end{point}
The interpretation of~$\sem{!A}$
and~$\sem{A\limp B}$ will then be
\begin{equation*}
\sem{!A}\ =\ 
\linf(\nsp(\sem{A}))
\qquad\text{and}\qquad
\sem{A\limp B}
\ = \ 
\mathcal{F}(\sem{B})^{*\sem{A}}.
\end{equation*}
Note that~$\sem{!A}$
will always be a discrete von Neumann algebra
no matter how complicated~$\sem{A}$ may be,
so that although this does the job
perhaps a more interpretation of~$\bang$
may be chosen as well.
This is not the case:
in the next
section we'll show that
any von Neumann algebra
that carries 
a $\otimes$-monoid structure
(such as $\sem{!A}$ must)
is commutative and discrete,
and that~$\linf(\nsp(\scrA))$
is moreover the free $\otimes$-monoid
on~$\scrA$.
\end{parsec}
\begin{parsec}%
\begin{point}%
We'll need the following two results
from the literature
on von Neumann algebras---I 
had no time to incorperate their proofs in this text.
\end{point}
\begin{point}[intersection-tensor]{Proposition}%
Given Hilbert spaces~$\scrH$ and~$\scrK$,
and von Neumann subalgebras~$\scrA_1$ and~$\scrA_2$
of~$\scrB(\scrH)$
and von Neumann subalgebras~$\scrB_1$ and~$\scrB_2$
of~$\scrB(\scrK)$,
we have 
\begin{equation*}
(\scrA_1 \otimes  \scrB_1)\,\cap\,
(\scrA_2 \otimes \scrB_2)
= (\scrA_1\cap\scrA_2)\,\otimes\,
(\scrB_1\cap \scrB_2).
\end{equation*}
Here
$\scrA_1 \otimes \scrB_1$
denotes not just any tensor product of~$\scrA_1$ and~$\scrB_1$,
see~\TODO{refer to notation},
but instead
the ``concrete'' tensor product 
of~$\scrA_1$ and~$\scrB_1$:
the least von Neumann subalgebra of~$\scrB(\scrH\otimes\scrK)$
that contains all operators
of the form $A\otimes B$ where~$A\in\scrA_1$
and~$B\in\scrB_1$.
\begin{point}{Proof}%
See Corollary IV.5.10 of~\cite{Takesaki1}.\qed
\end{point}
\end{point}
\begin{point}{Proposition}%
\TODO{On commutative von Neumann algebras.}
\begin{point}{Proof}%
\TODO{Add reference.}
\end{point}
\end{point}
\end{parsec}
\subsection{First Adjunction}
\subsection{Second Adjunction}
\subsection{Free Exponential}
	
\begin{parsec}%
\begin{point}%
We'll prove Kornell's result (from~\TODO{Quantum Collections})
that the functor
$\scrB\otimes(\,\cdot\,)\colon
\W{miu}\to\W{miu}$
has a left adjoint~$(\,\cdot\,)^{*\scrB}$
for every von Neumann algebra~$\scrB$.
Kornell original proof is rather complex,
and so is ours, unfortunately,
but we've managed
to peel off one layer of complexity from the original 
proof
by way of Freyd's Adjoint Functor Theorem,
reducing the problem
to the facts that
$\scrB\otimes(\,\cdot\,)\colon \W{miu}\to\W{miu}$
preserves products,  equalizers,
and satisfies the solution set condition.
\TODO{preservation of products}
\end{point}
\begin{point}[vn-generation-bound]{Lemma}%
If a von Neumann algebra~$\scrA$
is generated by~$S\subseteq \scrA$,
then
\begin{equation*}
\#\scrA \ \leq\  2^{2^{\#\mathbb{C}+\#S}}.
\end{equation*}
\begin{point}{Proof}%
Note that the
$*$-subalgebra of~$\scrA$
generated by~$S$
is dense in~$\scrA$.
Since every element of~$S'$
may be formed
from the infinite set $S\cup \mathbb{C}$ using 
the finitary operations
of
addition, multiplication,
and
involution,
we have~$\#S'\leq \#\mathbb{C}+ \#S$.
Since every element of~$\scrA$
is the ultraweak limit of a filter
(see \cite[\S12]{willard1970})
on~$S'$
of which there no more than~$2^{2^{\#S'}}$,
we conclude~$\#\scrA \leq 2^{2^{\#\mathbb{C}+\#S}}$.\qed
\end{point}
\end{point}
\begin{point}[vn-gns-bound]{Lemma}%
A von Neumann algebra~$\scrA$
can be faithfully represented
on a Hilbert space which contains no more
than~$2^{\#\scrA}$ vectors.
\begin{point}{Proof}%
If~$\scrA=\{0\}$,
then the result is obvious,
so let us assume that~$\scrA\neq \{0\}$.
Then~$\scrA$ is infinite,
and so~$\aleph_0 \cdot \#\scrA  = \#\scrA$. 

Let~$\Omega$ be the set of np-functionals on~$\scrA$.
Recall that 
by the GNS-construction (see~\sref{ngns})
$\mathscr{A}$
can be faithfully represented on
the Hilbert space
$\scrH_\Omega\equiv \bigoplus_{\omega\in\Omega} \scrH_\omega$.
Since every element of~$\scrH_\omega$
is the limit of a sequence of elements from~$\scrA$,
we have $\#\scrH_\omega \leq \aleph_0^{\#\scrA} \leq (2^{\aleph_0})^{\#\scrA} 
= 2^{\#\scrA}$,
because $\aleph_0\cdot \#\scrA=\#\scrA$.
Since every normal state is a map $\omega\colon \scrA\to\C$,
we have $\#\Omega\leq \#\C^{\#\scrA}=(2^{\aleph_0})^{\#\scrA}
= 2^{\#\scrA}$, because $\aleph_0 \cdot \#\scrA = \#\scrA$.
Hence $\#\scrH = \sum_{\omega\in\Omega} \#\scrH_\omega
\leq 2^{\#\scrA}\cdot2^{\#\scrA}
=2^{\#\scrA}$.\qed
\end{point}
\end{point}

\begin{point}[equalizer-lemma]{Lemma}%
Every nmiu-map $h\colon \scrD\to\scrA\otimes \scrC$,
where~$\scrA$, $\scrC$ and~$\scrD$ are von Neumann algebras,
factors
as 
$\smash{\xymatrix@C=3em{\scrD
\ar[r]|-{\tilde{h}}
& 
\tilde\scrA\otimes \scrC
\ar[r]|-{\iota\otimes\id}
&
\scrA \otimes \scrC
}}$,
where~$\smash{\tilde\scrA}$
is a von Neumann algebra,
and~$\iota$ and~$\tilde{h}$
are nmiu-maps,
such that
for all nmiu-maps $f,g\colon \scrA\to\scrB$
into some von Neumann algebra~$\scrB$
with $(f\otimes\id)\circ h = (g\otimes \id)\circ h$
we have $f\circ \iota = g\circ \iota$.

Moreover,
$\tilde\scrA$
can be generated by
less than~$\#\scrD\,\cdot\,2^{\#\scrC}$ elements.
\begin{point}{Proof}%
Assume (without loss of generality)
that
$\scrC$
is a von Neumann algebra of operators on
a Hilbert space~$\scrH$
with no more than $2^{\#\scrC}$ vectors, see~\sref{vn-gns-bound}.

For every vector~$\xi\in \scrH$
let
$r_\xi\colon \scrA\otimes \scrC\to\scrA$
be the unique np-map
given by~$r_\xi(a\otimes c) = \left<\xi,c\xi\right> a$
for all~$a\in\scrA$ and~$c\in \scrC$
(see~\sref{tensor-universal-property}
and~\sref{tensor-universal-property-extra}),
and
let~$\tilde\scrA$
be the least Neumann subalgebra
of~$\scrA$
that contains $S:=\smash{ \bigcup_{\xi \in\scrH} r_\xi(h(\scrD))}$,
and let~$\iota\colon \tilde\scrA\to\scrA$
be the inclusion
(so~$\iota$ is nmiu).
Note that~$S$ (which generates~$\tilde\scrA$)
has no more than $\#\scrD\cdot\#\scrH\leq \#\scrD \cdot 2^{\#\scrC}$
elements.

Let~$f,g\colon \scrA\to\scrB$
be nmiu-maps
into a von Neumann algebra~$\scrB$
such that $(f\otimes\id)\circ h = (g\otimes \id)\circ h$.
We must show that~$f\circ \iota = g\circ \iota$.
By definition of~$\tilde\scrA$
(and the fact that~$f$ and~$g$ are nmiu),
it suffices to show that $f\circ r_\xi\circ h=g\circ r_\xi\circ h$
for all~$\xi\in\scrH$.
Note that given such~$\xi$,
we have $f\circ r_\xi = r_\xi' \circ (f\otimes\id)$,
where~$r_\xi'\colon \scrB\otimes\scrC\to\scrB$
is the np-map
given by~$r_\xi'(b\otimes c)=\left<\xi,c\xi\right>b$.
Since similarly,
$g\circ r_\xi = r_\xi'\circ (g\otimes \id)$,
we get~$f\circ r_\xi \circ h
= r_\xi' \circ (f\otimes \id)\circ h
= r_\xi' \circ (g\otimes \id)\circ h
= g\circ r_\xi \circ h$.

It remains only to 
be shown that $h(\scrD) \subseteq \tilde\scrA\otimes \scrC$,
because we may then simply let~$\tilde{h}$
be the restriction of~$h$ to~$\tilde\scrA\otimes \scrC$.
It is enough to prove that
$h(\scrD) \subseteq \tilde\scrA 
\otimes  \bsp(\scrH)$,
because 
$\tilde\scrA \otimes  \scrC
\,=\, (\tilde\scrA\otimes \bsp(\scrH))\,\cap\,
(\scrA\otimes \scrC)$
(see~\sref{intersection-tensor})
and we already know that $h(\scrD)\subseteq \scrA\otimes \scrC$.
Let $(e_k)_k$ be orthonormal basis of $\scrH$.
Since $1= \sum_k \ket{e_k}\!\bra{e_k}$
in $\bsp(\scrH)$,
we have, for all~$d\in\scrD$,
\begin{align*}
h(d) \ &=\ \textstyle\bigl(\sum_k 1\otimes \ket{e_k}\!\bra{e_k}\bigr)
\ h(d)\ \bigl(\sum_\ell 1\otimes \ket{e_\ell}\!\bra{e_\ell}\bigr)\\
\ &=\ \textstyle \sum_k\sum_\ell\ 
 (\,1\otimes \ket{e_k}\!\bra{e_k}\,) \ h(d)\  
 (\,1\otimes \ket{e_\ell}\!\bra{e_\ell}\,).
\end{align*}
We are done
if we can prove that,
for all~$\xi,\zeta\in\scrH$,
\begin{equation}
\label{eq:main-equaliser-todo}
(\,1\otimes \ket{\xi}\!\bra{\xi}\,) \ 
h(d)\  (\,1\otimes \ket{\zeta}\!\bra{\zeta}\,)
\ \in\ \tilde\scrA\otimes \bsp(\scrH).
\end{equation}
By an easy computation, we see that,
for all $e \,\in\,\scrA\otimes \scrC$
	of the form $e\equiv a\otimes c$,
\begin{equation*}
\label{eq:polarisation-equaliser}
(\,1\otimes \ket{\xi}\!\bra{\xi}\,) \ e
\  (\,1\otimes \ket{\zeta}\!\bra{\zeta}\,)
\ =\ 
\frac{1}{4}\sum_{k=0}^3i^k \,r_{i^k\xi+\zeta}(e)\otimes \ket\xi\!\bra\zeta.
\end{equation*}
It follows that the equation above holds for all~$e\in \scrA\otimes \scrC$.
Choosing $e=h(d)$
we see that~\eqref{eq:main-equaliser-todo}
holds,
because
$r_{i^k\xi+\zeta}(h(d)) \in\tilde\scrA$.\qed
\end{point}
\end{point}
\TODO{equalizers of nmiu-maps}
\begin{point}{Proposition}%
Let $e\colon\scrE\to\scrA$ be an equaliser of
nmiu-maps $f,g\colon\scrA\to\scrB$ 
between von Neumann algebras.
Then $e\otimes\id\colon\scrE\otimes\scrC\to\scrA\otimes\scrC$
is an equaliser of 
$f\otimes\id$ and $g\otimes\id$
for every von Neumann algebra~$\scrC$.
\begin{point}{Proof}%
Let~$h\colon\scrD\to\scrA\otimes \scrC$
be a nmiu-map 
with $(f\otimes\id)\circ h=(g\otimes\id)\circ h$.
We must show that there is a unique
 nmiu-map $k\colon\scrD\to\scrE\otimes\scrC$
such that  $h=(e\otimes\id)\circ k$.
Note that since the equalizer map~$e$ 
is injective,
$e\otimes\id\colon\scrE\otimes\scrC\to\scrA\otimes\scrC$
is injective (by~\sref{tensor-injective})
and thus uniqueness of~$k$ is clear.
Concerning existence,
by~\sref{equalizer-lemma},
$h$ factors as
$\smash{\xymatrix@C=3em{\scrD
\ar[r]|-{\tilde{h}}
& 
\tilde\scrA\otimes \scrC
\ar[r]|-{\iota\otimes\id}
&
\scrA \otimes \scrC
}}$
where~$\tilde{h}$ and~$\iota$ are nmiu-maps,
and moreover,
we have $f\circ \iota = g\circ \iota$.
Since~$e$ is an equalizer of~$f$ and~$g$,
there is a unique nmiu-map $\tilde\iota\colon \tilde\scrA\to \scrE$
with~$e\circ \tilde\iota = \iota$.
Now, define $k:=(\tilde \iota\otimes \id)\circ \tilde{h}\colon 
\scrD\to \scrE\otimes \scrC$.
Then~$(e\otimes \id)\circ k = 
((e\circ\tilde \iota)\otimes \id)\circ \tilde{h}
= (\iota\otimes \id)\circ \tilde{h}
= h$.\qed
\end{point}
\end{point}


\begin{point}{Theorem}%
The functor $(-)\otimes\scrA\colon\W{miu}\to\W{miu}$
has a left adjoint
for every von Neumann algebra~$\scrA$.
\begin{point}{Proof}%
	The category $\W{miu}$ is (small-)complete,
and
$(-)\otimes\scrA\colon\W{miu}\to\W{miu}$
preserves (small-)products and equalisers.
Thus,
by Freyd's (General)
Adjoint Functor Theorem~\cite[Thm.~V.6.2]{maclane1978},
it suffices to check the following Solution Set Condition
(where we've used that $\W{miu}$
is locally small).
\begin{itemize}
\item
	For each $\scrB\in\W{miu}$, there is a small subset $\mathcal{S}$ of objects in $\W{miu}$
such that every arrow $h\colon \scrB\to\scrC\otimes\scrA$
can be written as a composite $h=(t\otimes\id_{\scrA})\circ f$ for some $\scrD\in \mathcal{S}$,
$f\colon\scrB\to\scrD\otimes \scrA$, and $t\colon \scrD\to\scrC$.
\end{itemize}
Let $\scrB$ be an arbitrary von Neumann algebra.
We claim that the following set $\mathcal{S}$ satisfies the required condition:
\[
	\mathcal{S}=
\{\scrD\mid
\text{$\scrD$ is a von Neumann algebra on~$\kappa$}\},
\quad\text{where}\quad\kappa=2^{2^{\#\C\cdot\#\scrB\cdot2^{\#\scrA}}}.
\]
Indeed,
suppose that $h\colon \scrB\to\scrC\otimes\scrA$ is given.
By \sref{equalizer-lemma},
$h$ factors
as~$\xymatrix{\scrB\ar[r] & \tilde\scrC\otimes 
\scrA\ar[r]|-{\iota\otimes \id} & \scrC\otimes\scrA }$
where~$\tilde\scrC$
is a von Neumann algebra
generated by no more than~$\#\scrB\cdot2^{\#\scrA}$
elements.
It follows that~$\tilde\scrC$ has no
more than~$\kappa$ elements (by~\sref{vn-generation-bound}).
Thus we may assume without loss of generality
that~$\tilde\scrC$
is a subset of~$\kappa$,
that is,~$\tilde\scrC\in\mathcal{S}$.\qed
\end{point}
\end{point}
\end{parsec}

\section{Duplicators and Monoids}
\subsection{Duplicators}
\TODO{intro}
\begin{parsec}%
\begin{point}{Definition}
A von Neumann algebra~$\mathscr{A}$
is \Define{duplicable}
if there is a \Define{duplicator} on~$\mathscr{A}$,
that is,
an npsu-map
$\delta\colon \mathscr{A}\otimes \mathscr{A}\to\mathscr{A}$
with a \Define{unit} $u\in [0,1]_\scrA$ 
satisfying
\begin{equation*}
\delta(a\otimes u)\ =\ a\ = \ \delta(u\otimes a)
\quad\text{for all~$a\in\mathscr{A}$.}
\end{equation*}
\end{point}

\noindent
The unit $u$ can be identified with
a positive subunital map $\tilde{u}\colon \C\to \scrA$ via $\tilde{u}(\lambda)=\lambda u$.
The definition is motivated by the fact that the interpretation of $\bang A$
must carry a commutative monoid structure in $\W{miu}$.
The condition is weaker, requiring the maps to be only positive subunital,
and dropping associativity and commutativity.
Nevertheless this is sufficient to prove:
\begin{point}[duplicable]{Theorem}
A von Neumann algebra~$\mathscr{A}$
is duplicable if and only if
$\mathscr{A}$ is nmiu-isomorphic to $\linf(X)$ for some set~$X$.
In that case, the duplicator $(\delta,u)$
is unique, given by
$\delta(a\otimes b) = a\cdot b$ and $u=1$.
\end{point}

\noindent
Thus, to interpret duplicable types,
we can really only use von Neumann algebras of the form $\linf(X)$.
It also follows that a von Neumann algebra
is duplicable precisely when it is a (commutative) monoid
in $\W{miu}$, or in the 
symmetric monoidal category $\W{cpsu}$ of von Neumann algebras
and normal completely positive subunital (CPsU) maps.

\label{sec:characterisation-dup-vna}
We will prove Theorem~\ref{duplicable} in this section.
Let us give
a rough sketch of the proof.
To show that every duplicable von Neumann algebra
is MIU-isomorphic to~$\linf(X)$ for some set~$X$,
we first prove that~$\mathscr{A}$
is Abelian (in Lemma~\ref{lem:uniqueness-duplicator}).
This reduces the situation
to a measure theoretic problem,
because~$\mathscr{A} \cong L^\infty(X)$
for some appropriate (i.e.~localisable) measure space~$X$.
For simplicity,
we only consider
measure spaces with~$\mu(X)<\infty$
(which are localisable).
This restriction turns out to be harmless
(in the proof of Theorem~\ref{duplicable}).
The measure space~$X$ splits
in a discrete~$D$ and a continuous part~$C$,
so~$X\cong D\oplus C$
(see Lemma~\ref{lem:measure-space-continuous-discrete}).
Since~$L^\infty(D)\cong \linf(D')$
for some set~$D'$,
and $L^\infty(X)\cong \linf(D')\oplus L^\infty(C)$,
our task will be  to show that~$L^\infty(C)=\{0\}$.
Since~$L^\infty(X)$ is duplicable,
we will see that~$L^\infty(C)$ is duplicable as well
(see Corollary~\ref{cor:duplicable-product}).
Thus the crux of the proof is that~$L^\infty(C)$
cannot be duplicable unless~$\mu(C)=0$
(see Lemma~\ref{lem:continuous-measure-space}).

\label{SS:abelian}
\begin{point}{Lemma}
\label{lem:unit-duplicator}
Let~$\delta$
be a duplicator 
with unit~$u$
on a von Neumann algebra~$\mathscr{A}$.
Then~$u=1$ and~$\delta(1\otimes 1)=1$.
\end{point}
\begin{point}{Proof}
Since~$1=\delta(u\otimes 1)\leq \delta(1\otimes 1) \leq 1$,
we have $\delta(u^\perp\otimes 1)=0$.
But then~$u^\perp=0$, and thus~$u=1$,
because  $u^\perp = \delta(u^\perp \otimes u)
\leq \delta(u^\perp \otimes 1) = 0$.
Hence~$u=1$.
Thus~$1=\delta(1\otimes u)=\delta(1\otimes 1)$.
\end{point}

The following consequence
of Tomiyama's theorem
is based on Lemma~8.3 of~\cite{ndlmcs}.
\begin{point}{Lemma}
\label{lem:sef-instrument}
Let~$\mathscr{A}$ be a unital $C^*$-algebra,
and let~$f\colon \mathscr{A}\oplus\mathscr{A}\to \mathscr{A}$
be a positive unital map 
with $f(a,a)=a$ for all~$a\in \mathscr{A}$.
Then $p:=f(1,0)$ is central,
and
for all~$a,b\in\mathscr{A}$
	we have
$f(a,b) \,=\, ap\,+\, bp^\perp$.
\end{point}
\begin{point}{Proof}
By Tomiyama's theorem we have,
for all~$a,b,c,d\in\mathscr{A}$,
\begin{equation*}
a \,f(c,d)\, b \ = \ f(\,acb\,,\,adb\,).
\end{equation*}
In particular,
for all~$a\in\mathscr{A}$,
we have $ap=af(1,0)=f(a,0)=f(1,0)a=pa$.
Thus~$p$ is central.
Similarly, $f(0,b)=bp^\perp$
for all~$b\in\mathscr{A}$.
Then~$f(a,b)=f(a,0)+f(0,b)=ap+bp^\perp$
for all~$a,b\in\mathscr{A}$.
\end{point}

\begin{point}{Lemma}
\label{lem:uniqueness-duplicator}
Let $\delta\colon\mathscr{A}\otimes \mathscr{A}\to\mathscr{A}$
be a duplicator on a von Neumann algebra~$\mathscr{A}$.
Then~$\mathscr{A}$ is Abelian and~$\delta(a\otimes b)=a\cdot b$
for all~$a,b\in\mathscr{A}$.
\end{point}
\begin{point}{Proof}
We must show that all~$a\in\mathscr{A}$ are central.
It suffices
to show that all~$p\in [0,1]_\mathscr{A}$ are central
(by the usual reasoning).
Similarly, 
we only need to prove that $\delta(a\otimes p) = a\cdot p$
for all~$a\in\mathscr{A}$ and $p\in [0,1]_\mathscr{A}$.

Let~$p\in[0,1]_\mathscr{A}$ be given.
Define~$f\colon \mathscr{A}\oplus\mathscr{A}\to\mathscr{A}$
by $f(a,b) = \delta(a\otimes p+b\otimes p^\perp)$
for all~$a,b\in\mathscr{A}$.
Then~$f$ is positive, unital,
$f(1,0)=p$,
and 
$f(a,a)=a$
for all~$a\in \mathscr{A}$.
Thus, by Lemma~\ref{lem:sef-instrument},
we get that~$p$ is central,
and  $f(a,b)=ap+bp^\perp$ for all~$a,b\in\mathscr{A}$.
Then~$a\cdot p=f(a,0)=\delta(a \otimes p)$.
\end{point}
\begin{point}{Remark}
The special
case of Lemma~\ref{lem:uniqueness-duplicator}
in which~$\delta$ is \emph{completely} positive
can be found
in the literature,
see for example
Theorem~6 of~\cite{Maassen2010}
(in which~$\mathscr{A}$ is also finite dimensional).
\end{point}
\begin{point}{Corollary}
\label{cor:duplicability-multiplication}
Let~$\mathscr{A}$
be a von Neumann algebra.
Then~$\mathscr{A}$
is duplicable
iff there is
a normal positive linear map $\delta\colon\mathscr{A}\otimes\mathscr{A}
\to \mathscr{A}$
with $\mu(a\otimes b)=a\cdot b$ 
for all~$a,b\in\mathscr{A}$.
\end{point}
\begin{point}{Corollary}
\label{cor:duplicable-product}
Von Neumann algebras~$\mathscr{A}$ and~$\mathscr{B}$
are duplicable
when  $\mathscr{A}\oplus \mathscr{B}$ is duplicable.
\end{point}
\begin{point}{Proof}
Let $\delta\colon (\mathscr{A}\oplus\mathscr{B})\otimes
(\mathscr{A}\oplus\mathscr{B})\longrightarrow
\mathscr{A}\oplus\mathscr{B}$
be a duplicator on~$\mathscr{A}\oplus\mathscr{B}$.
By Lemma~\ref{lem:uniqueness-duplicator}
we know that~$\mathscr{A}\oplus\mathscr{B}$
is Abelian,
and that $\delta((a_1,b_1)\otimes (a_2,b_2))
= (a_1a_2,b_1b_2)$
for all $a_1,a_2\in\mathscr{A}$
and~$b_1,b_2\in\mathscr{B}$.

Let~$\kappa_1\colon \mathscr{A}\to\mathscr{A}\oplus\mathscr{B}$
be the normal MIU-map
given by~$\kappa_1(a)=(a,0)$ for all~$a\in\mathscr{A}$.
Let~$\delta_\mathscr{A}$ be the composition of
$\xymatrix@C=3em{
\mathscr{A}\otimes\mathscr{A}
\ar[r]|-{\kappa_1\otimes\kappa_1}
&
(\mathscr{A}\oplus\mathscr{B})
\otimes
(\mathscr{A}\oplus\mathscr{B})
\ar[r]|-{\delta}
&
\mathscr{A}\oplus\mathscr{B}
\ar[r]|-{\pi_1}
&
\mathscr{A}
}$.
Then~$\delta_\mathscr{A}$ is normal, positive,
and
$\delta_\mathscr{A}(a_1\otimes a_2)
\,=\,  \pi_1(\delta((a_1,0)\otimes (a_2,0))) 
\,=\, \pi_1(a_1a_2,0)\,=\,a_1a_2$
for all~$a_1,a_2\in\mathscr{A}$.
Thus, by Corollary~\ref{cor:duplicability-multiplication},
$\mathscr{A}$
is duplicable.
\end{point}

We will now work towards
the proof that
if~$C$ is a finite measure space,
then
$L^\infty(C)$
cannot be duplicable 
unless~$\mu(C)=0$,
see Lemma~\ref{lem:continuous-finite-measure-space-not-duplicable}.
Let us first fix some terminology
from measure theory (see~\cite{Fremlin2000}).

\begin{point}{Definition}
\label{def:measure-space}
Let~$X$ be a measure space.
\begin{enumerate}
\item
The measurable subsets of~$X$
are denoted by~$\Sigma_X$.
\item
A measurable subset~$A$ of~$X$
is \textbf{atomic}
if $0<\mu(A)<\infty$,
and $\mu(A')=\mu(A)$ for all~$A'\in\Sigma_X$
with~$A'\subseteq A$ and~$\mu(A')>0$.

\item
$X$ is \textbf{discrete} if $X$ is covered by atomic measurable subsets.

\item
$X$ is \textbf{continuous}
if~$X$ contains no atomic subsets.
\end{enumerate}
\end{point}
\begin{point}{Definition}
Given a measure space~$X$
with~$\mu(X)<\infty$, 
let
$L^\infty(X)$
denote the von Neumann algebra
of bounded functions $f\colon X\to\mathbb{C}$.
Two such functions $f$ and~$g$ are identified in $L^\infty(X)$
when $f(x)=g(x)$ for almost all~$x\in X$.
Multiplication, addition, involution in~$L^\infty(X)$
are all computer coordinatewise,
and the norm~$\|f\|$ of~$f\in L^\infty(X)$
is the least number $r>0$ such that $|f(x)|\leq r$
for almost all~$x\in X$.
(For more details,
see e.g.~Example~IX/7.2 of~\cite{conway2007}.)
\end{point}

The following lemma,
which will be very useful,
is a variation on
Zorn's Lemma
that does not require the axiom of choice.
\begin{point}{Lemma}
\label{lem:measure-zorn}
Let~$X$ be a measure space with $\mu(X)<\infty$.
Let~$\mathcal{S}$
be a collection of measurable subsets of~$X$
such that~$\bigcup_n A_n\in\mathcal{S}$
for all $A_1\subseteq A_2 \subseteq \dotsb$
in~$\mathcal{S}$.
Then for all~$A\in\mathcal{S}$,
there is~$B\in\mathcal{S}$
with $A\subseteq B$
which is maximal in the sense
that $\mu(B')=\mu(B)$
for all~$B'\in\mathcal{S}$ with $B\subseteq B'$.
\end{point}
\begin{point}{Proof}
The trick is to consider for every $C\in\Sigma_X$
the quantity
$\beta_C \ = \ \sup\{\,\mu(D)\mid C\subseteq D
\text{ and }D\in \mathcal{S}\,\}$.

Note that $\mu(C)\leq \beta_C \leq \mu(X)$
for all~$C\in\Sigma_X$,
	and $\beta_{C_2}\leq \beta_{C_1}$
	for all~$C_1,C_2\in\Sigma_X$ with $C_1 \subseteq C_2$.
To prove this lemma, it suffices to find~$B\in\Sigma_X$
with $A\subseteq B$ and~$\mu(B)=\beta_B$.

Define~$B_1:= B$.
Pick~$B_2\in\mathcal{S}$
such that $B_1 \subseteq B_2$
and~$\beta_{B_1}-\mu(B_2)\leq \nicefrac{1}{2}$.
Pick~$B_3\in\mathcal{S}$
such that $B_2\subseteq B_3$
and~$\beta_{B_2}-\mu(B_3) \leq \nicefrac{1}{3}$.
Proceeding in this fashion,
we get a sequence $B\equiv B_1\subseteq B_2 \subseteq \dotsb$
in~$\mathcal{S}$
with $\beta_{B_{n}}-\mu(B_{n+1})\leq \nicefrac{1}{n}$
for all~$n\in\mathbb{N}$.
Define~$B:=\bigcup_n B_n$.
Then~$B\in \mathcal{S}$.
Moreover,
\begin{equation*}
\mu(B_1)\,\leq\, \mu(B_2)\,\leq\,
\dotsb \,\leq\,\mu(B)\,\leq\, \beta_B \,\leq\, \dotsb
\,\leq\, \beta_{B_2}\,\leq\, \beta_{B_1}.
\end{equation*}
Since for every~$n\in\mathbb{N}$
we have both $\mu(B_{n+1})\leq \mu(B)\leq \beta_B \leq \beta_{B_n}$
and $\beta_{B_n}- \mu(B_{n+1}) \leq \nicefrac{1}{n}$,
we get $\beta_B-\mu(B)\leq \nicefrac{1}{n}$,
and so~$\beta_B = \mu(B)$.
\end{point}

\begin{point}{Lemma}
\label{lem:measure-space-continuous-discrete}
Let~$X$ be a measure space with~$\mu(X)<\infty$.
Then there is a measurable subset~$D\subseteq X$
such that~$D$ is discrete
and~$X\backslash D$ is continuous.
\end{point}
\begin{point}{Proof}
Since clearly the countable union
of discrete measurable subsets of~$X$
is again discrete,
there is by Lemma~\ref{lem:measure-zorn}
a discrete measurable subset~$D$ of~$X$
which is maximal in the sense that~$\mu(D')=\mu(D)$
for every discrete measurable subset~$D'$ of~$X$ with $D\subseteq D'$.
To show that~$X\backslash D$ is continuous,
we must prove that~$X\backslash D$
contains no atomic measurable subsets.
If~$A\subseteq X\backslash D$ is an atomic measurable subset
of~$X$,
then~$D\cup A$
is a discrete measurable
subset of~$X$
which contains~$D$,
and $\mu(D\cup A)=\mu(D)\cup \mu(A) > \mu(D)$.
This contradicts the  maximality of~$D$.
Thus~$X\backslash D$ is continuous.
\end{point}
\begin{point}{Lemma}
\label{lem:continuous-measure-space}
Let~$X$ be a continuous measure space
with~$\mu(X)<\infty$.
Then for every~$r\in [0,\mu(X)]$
there is a measurable subset~$A$ of~$X$ with $\mu(X)=r$.
\end{point}
\begin{point}{Proof}
Let us quickly get rid of the case that~$\mu(X)=0$.
Indeed, then~$r=0$, and so~$A=\varnothing$ will do.
For the remainder, assume that~$\mu(X)>0$.

For starters, we show that for every~$\varepsilon >0$
and~$B\in\Sigma_X$ with~$\mu(B)>0$
there is~$A\in\Sigma_X$ with $A\subseteq B$
and  $0<\mu(A)<\varepsilon$.
Define~$A_1 := B$.
Since~$\mu(B)>0$,
and~$A_1$ is not atomic (because~$X$ is continuous)
there is~$A\in\Sigma_X$ with $A\subseteq A_1$ 
and $\mu(A)\neq \mu(A_1)$.
Since~$\mu(A)+\mu(A_1\backslash A)=\mu(A_1)$,
either $0<\mu(A)\leq \frac{1}{2}\mu(A_1)$
or $0<\mu(X\backslash A)\leq \frac{1}{2}\mu(A_1)$.
In any case,
there is~$A_2\subseteq A_1$
with $A_2\in\Sigma_X$
and $0<\mu(A_2)\leq \frac{1}{2}\mu(A_1)$.
Similarly,
since~$A_2$ is not atomic (because~$X$ is continuous),
there is~$A_3\subseteq A_2$
with~$A_3\in\Sigma_X$ and $0<\mu(A_3)\leq \frac{1}{2}\mu(A_2)$.
Proceeding in a similar fashion,
we obtain a sequence $B\equiv A_1 \supseteq A_2\supseteq \dotsb$
of measurable subsets of~$X$
with $0<\mu(A_n)\leq 2^{-n}\mu(X)$.
Then, for every $\varepsilon >0$
there is~$n\in\mathbb{N}$
such that $0<\mu(A_n)\leq \varepsilon$ and~$A_n\subseteq B$.

Now, 
let us prove that there is~$A\in\Sigma_X$ with $\mu(A)=r$.
By Lemma~\ref{lem:measure-zorn}
there is a measurable
subset~$A$ of~$X$
with $\mu(A)\leq r$
and which is maximal
in the sense that $\mu(A')=\mu(A)$
for all~$A'\in\Sigma_X$
with $\mu(A)\leq r$ and~$A\subseteq A'$.
In fact, we claim that~$\mu(A)=r$.
Indeed, suppose that~$\varepsilon := r-\mu(A)>0$
towards a contradiction.
By the previous discussion,
there is~$C\in\Sigma_X$ with $C\subseteq X\backslash A$
such that $\mu(C)\leq \varepsilon$.
Then~$A\cup C$ is measurable,
and $\mu(A\cup C)=\mu(A)+\mu(C)\leq \mu(A)+\varepsilon\leq r$,
which contradicts the maximality of~$A$.
\end{point}

\begin{point}{Lemma}
\label{lem:continuous-finite-measure-space-not-duplicable}
Let~$X$ be a continuous measure space
with~$\mu(X)<\infty$.
If~$L^\infty(X)$ is duplicable,
then~$\mu(X)=0$.
\end{point}
\begin{point}{Proof}
Suppose that~$L^\infty(X)$ is duplicable
and~$\mu(X)>0$
towards a contradiction.
Let~$\delta$
be a duplicator
on~$L^\infty(X)$.
By Lemma~\ref{lem:uniqueness-duplicator}
we know that~$\delta(f\otimes g)=f\cdot g$ for all~$f,g\in L^\infty(X)$.

Let~$\omega\colon L^\infty(X)\to \mathbb{C}$
be given by~$\omega(f)=\frac{1}{\mu(X)}\int f \,d\mu$
for all~$f\in L^\infty(X)$.
Then~$\omega$ is normal, positive and unital.
Also, $\omega$ is faithful,
or in other words,
for all~$f\in L^\infty(X)$
with $f\geq 0$ and $\omega(f)=0$
we have $f=0$.
It is known
(see e.g.~Corollary 5.12 of~\cite{Takesaki1})
that  there is a  faithful normal positive unital linear map
$\omega\otimes \omega\colon L^\infty(X)\otimes L^\infty(X)\to \mathbb{C}$
with~$(\omega\otimes \omega)(f\otimes g) = \omega(f)\cdot \omega(g)$
for all~$f,g\in L^\infty(X)$.
We will use
$\omega\otimes \omega$ to tease out a contradiction,
but first we will need a second ingredient.

Since~$X$ is continuous,
we may partition~$X$ into two measurable
subsets of equal measure 
with the aid of Lemma~\ref{lem:continuous-measure-space},
that is,
there are measurable subsets $X_{1}$ and~$X_{2}$
of~$X$ with $X=X_{1}\cup X_{2}$, $X_{1}\cap X_{2}=\varnothing$,
and
$\mu(X_{1})=\mu(X_{2})=\frac{1}{2}\mu(X)$.
Similarly, $X_{1}$ 
can be split into two measurable subsets, $X_{11}$ and $X_{12}$,
of equal measure, and so on.
In this way,
we obtain for every word~$w$ over the alphabet~$\{1,2\}$
--- in symbols, $w\in \{1,2\}^*$ ---
a measurable subset~$X_w$ of~$X$
such that $X_w = X_{w1}\cup X_{w2}$,
$X_{w1}\cap X_{w2}=\varnothing$,
and $\mu(X_{w1})=\mu(X_{w2})=\frac{1}{2}\mu(X_w)$.
It follows that~$\mu(X_w)=\frac{1}{2^{\#w}}\mu(X)$,
where~$\#w$ is the length of the word~$w$.

Now, let~$p_w = \mathbf{1}_{X_w}$ 
be the indicator function of~$X_w$
for every~$w\in \{1,2\}^*$.
Let~$w\in \{1,2\}^*$ be given.
Then~$p_w$ is a projection in~$L^\infty(X)$,
and~$\omega(p_w)=2^{-\#w}$.
Moreover, $p_w = p_{w1}+p_{w2}$,
and so
\begin{alignat*}{3}
p_w\otimes p_w 
\ &=\  
p_{w1}\otimes p_{w1} \,+\,
p_{w1}\otimes p_{w2} \,+\,
p_{w2}\otimes p_{w1} \,+\,
p_{w2}\otimes p_{w2}\\
\ &\geq\ 
p_{w1}\otimes p_{w1} \,+\,
p_{w2}\otimes p_{w2}.
\end{alignat*}
Thus, if we define 
$q_N\ :=\ \sum_{w\in \{1,2\}^N}\,p_w\otimes p_w$
for every natural number~$N$,
where~$\{1,2\}^N$ is the set of words over~$\{1,2\}$ of length~$N$,
then we get a descending sequence $q_1\geq q_2\geq q_3\geq \dotsb$
of projections in~$L^\infty(X)\otimes L^\infty(X)$.
Let~$q$ be the infimum of $q_1\geq q_2 \geq \dotsb$ 
in the set of self-adjoint elements of~$L^\infty(X)\otimes
L^\infty(X)$.
Do we have~$q=0$ ?

On the one hand,
we claim that $\delta(q)=1$, and so~$q\neq 0$.
Indeed,
$\delta(p_w\otimes p_w)=p_w\cdot p_w = p_w$
for all~$w\in \{1,2\}^N$.
Thus $\delta(q_N) = \sum_{w\in \{1,2\}^N}  \delta(p_w\otimes p_w)
= \sum_{w\in\{1,2\}^N} p_w=1$ for all~$N\in \mathbb{N}$.
Hence $\delta(q)=\bigwedge_n \delta(q_N) = 1$,
because~$\delta$ is normal.
On the other hand,
we claim that $(\omega\otimes \omega)(q)=0$,
and so~$q=0$ since~$\omega\otimes \omega$ is 
faithful and $q\geq 0$.
Indeed,
$(\omega\otimes\omega)(q_N)=
\sum_{w\in\{1,2\}^N} \omega(p_w)\cdot\omega(p_w)
= \sum_{w\in\{1,2\}^N} 2^{-N}\cdot 2^{-N} = 2^{-N}$
for all~$N\in \mathbb{N}$,
	and so $(\omega\otimes\omega)(q)
=\bigwedge_N (\omega\otimes\omega)(q_N) = \bigwedge_N 2^{-N}=0$.
Thus, $q=0$ and $q\neq 0$, which is impossible.
\end{point}

\begin{point}{Lemma}
\label{lem:atomic-measure-space}
Let~$A$ be an atomic measure space.
Then~$L^\infty(A)\cong \mathbb{C}$.
\end{point}
\begin{point}{Proof}
Let~$f\in L^\infty(A)$ be given.
It suffices to show that
there is~$z\in \mathbb{C}$
such that
$f(x)=z$ for almost all~$x\in A$.
Moreover, we only need to consider the case
that~$f$ takes its values in~$\mathbb{R}$
(because we may split~$f$ in its real and imaginary parts,
and in turn split these in positive and negative parts).

Since,
writing  $I_n = (n,n+1]$,
we have $\mu(A) = \sum_{n\in\mathbb{Z}} f^{-1}(I_n)$,
and~$A$ is atomic,
there a (unique)~$n\in \mathbb{N}$
with  $\mu(A)=\mu(f^{-1}(I_n))$.
Similarly,
writing 
$J_{1} = (n,\frac{2n+1}{2}]$
and $J_{2}=(\frac{2n+1}{2},n+1]$,
we have
$\mu(A) = \mu(J_{1}) + \mu(J_{2})$,
and so there is a (unique)  $m\in \{1,2\}$
with $\mu(A)=\mu(f^{-1}(J_{m}))$.

Continuing in this way,
we can find real numbers $s_1 \leq s_2 \leq \dotsb \leq t_2 \leq t_1$
with $t_n-s_n \leq 2^{-n}$
and $\mu(A)=\mu( f^{-1}(\ (s_n,t_n]\ ))$
for all~$n\in \mathbb{N}$.
Of course,
we also have~$\mu(A)=\mu(f^{-1}(\ [s_n,t_n]\ ))$,
and so
\begin{equation*}
\textstyle
\mu(A) \ =\  \bigwedge_n \mu(f^{-1}(\ [s_n,t_n]\ ))
\ =\  \mu(f^{-1}(\ \bigcap_n [s_n,t_n]\ )).
\end{equation*}
Since~$t_n-s_n \to 0$ as~$n\to \infty$,
there is a real number~$\lambda\in \mathbb{R}$
with $\{\lambda \} = \bigcap_n[s_n,t_n]$,
and so~$\mu(A)=f^{-1}(\{\lambda\})$.
Hence~$f(x)=\lambda$ for almost all~$x\in A$.
\end{point}
\begin{point}{Lemma}
\label{lem:measure-space-partition}
Let~$X$ be a measure space with~$\mu(\mathscr{A})<\infty$.
Then for every partition~$\mathcal{A}$
of~$X$
consisting of measurable subsets,
we have~$L^\infty(X)\cong \bigoplus_{A\in\mathcal{A}} L^\infty(A)$.
\end{point}
\begin{point}{Proof}
Note that since~$\sum_{A\in\mathcal{A}} \mu(A)=\mu(X)$,
the set $\mathcal{A}' = \{A\in \mathcal{A}\mid \mu(A)>0\}$
is countable.
We will also
need the fact that~$A_0 = \bigcup\{ A\in\mathcal{A}\mid \mu(A)=0\}$
is negligible.
To see this,
note that
$X\backslash A_0$ is 
the union of~$\mathcal{A}'$
and thus measurable.
Further, we have $\mu(X\backslash A_0) = 
\sum_{A\in \mathcal{A}'} \mu(A)
= \sum_{A\in\mathcal{A}} \mu(A)=\mu(X)$,
and thus~$\mu(A_0)=0$.

Let~$A\in \mathcal{A}$
and~$f \in L^\infty(X)$ be given.
Then the restriction $f|A\colon A\to \mathbb{C}$
is an element of~$L^\infty(A)$.
It is not hard to see that $f\mapsto f|_A$
gives a MIU-map
$R_A\colon L^\infty(X)\to L^\infty(A)$.
Let~$R\colon L^\infty(X)\to \bigoplus_{A\in \mathcal{A}} L^\infty(A)$
be given by $R(f) = (R_A(f))_{A\in \mathcal{A}}$.
We claim that~$R$ is bijective (and thus a normal MIU-isomorphism).

To show that~$R$ is injective,
let $f\in L^\infty(X)$ with~$R(f)=0$ be given.
Then for every~$A\in \mathcal{A}$
there is a negligible subset~$N_A$ of~$A$ with $f(x)=0$
for all~$x\in A\backslash N_A$.
Thus~$f(x)=0$ for all~$x\in X\backslash\bigcup_{A\in \mathcal{A}} N_A$.
Since~$N := \bigcup_{A\in\mathcal{A}} N_A 
\subseteq A_0 \cup \bigcup_{A\in\mathcal{A}'} N_A$,
we see that~$N$ is negligible,
and so~$f(x)=0$ for almost all~$x\in X$.
Thus $f=0$ in~$L^\infty(X)$,
and thus~$R$ is injective.

To show that~$R$ is surjective,
let $f\in \bigoplus_{A\in \mathcal{A}} L^\infty(A)$
be given,
and define~$g\colon X\to \mathbb{C}$
by $g(x)=f_A(x)$
for all~$A\in\mathcal{A}$ and~$x\in A$.
We claim that~$g\in L^\infty(X)$,
and then clearly $R(g)=f$.
To begin, we show that $g$ is measurable.
Let~$U$ be a measurable subset of~$\mathbb{C}$.
We must show that~$g^{-1}(U)$ is measurable.
We have
\begin{equation*}
\textstyle 
g^{-1}(U)\ = \ \bigcup_{A\in \mathcal{A}} f^{-1}_A(U)
\ = \ \bigcup_{A\in \mathcal{A}'} \, f^{-1}_A(U)
\ \,\cup\ \,\bigcup_{A\in \mathcal{A}\backslash \mathcal{A}'}\, f^{-1}_A(U).
\end{equation*}
Note that~$\bigcup_{A\in\mathcal{A}'}f_A^{-1}(A)$
is measurable
(being a countable union of measurable sets),
and that $\bigcup_{A\in\mathcal{A}\backslash\mathcal{A}'} f_A^{-1}(A)$
is negligible
(being a subset of the negligible set~$A_0$).
Thus~$g^{-1}(U)$ is measurable.
Hence~$g$ is measurable.
It remains to be shown
that $g$ is essentially bounded,
that is, that there is $r>0$ such that $|f(x)|\leq r$
for almost all~$r\in X$.
For every~$A\in\mathcal{A}$
there is a negligible
subset~$N_A$ of~$A$
such that $|f_A(x)|\leq \|f_A\|$
for all~$x\in A\backslash N_A$.
Since~$\|f\|=\sup_{A\in\mathcal{A}}\|f_A\|$,
we have $|g|\leq \|f\|$
for all~$x\in X\backslash (\bigcup_{A\in\mathcal{A}}N_A)$.
Since~$N:=\bigcup_{A\in\mathcal{A}}N_A\,\subseteq\,
A_0\cup\bigcup_{A\in\mathcal{A}'}A$,
we see that~$N$ is negligible,
and thus that~$g$ is essentially bounded.
Of course, $R(g)=f$, and thus~$R$ is surjective.
\end{point}

\begin{point}{Corollary}
\label{cor:discrete-ell-x}
For every discrete 
measure space~$X$ with~$\mu(X)<\infty$
there is a  set~$Y$ with $L^\infty(X)\cong \linf(Y)$.
\end{point}

We are now ready to give the proof
the main result of this paper.
\begin{point}{Proof}[Proof of Theorem~\ref{duplicable}]
We have already seen that $\linf(X)$
can be  equipped with a commutative monoid
structure in~$\W{miu}$
for any set~$X$,
and is thus duplicable.
Conversely,
let~$\delta\colon \mathscr{A}\otimes\mathscr{A}\to\mathscr{A}$
be a duplicator with unit~$u$ on a von Neumann algebra~$\mathscr{A}$.
By Lemma~\ref{lem:unit-duplicator}, we know that~$u=1$,
and by Lemma~\ref{lem:uniqueness-duplicator},
we know that~$\mathscr{A}$
is Abelian 
and~$\delta(a\otimes b)=a\cdot b$
for all~$a,b\in \mathscr{A}$.
Thus, the only thing that remains to be shown
is that~$\mathscr{A}$ is MIU-isomorphic to $\linf(Y)$
for some set~$Y$.

It is known that 
any Abelian von Neumann algebra must be of the form $L^\infty(X)$,
where~$X$ is a  measure space.
Moreover, $X$ can be taken to be a \emph{localisable} measure space,
but we will not need the general theory of localisable measure spaces here.
Instead,
we can get away with using the fact
(obtained 
by inspecting the proof of Proposition~1.18.1
of~\cite{Sakai1998}),
that there is a family of measure spaces $(X_i)_{i\in I}$
with $\mathscr{A}\cong \bigoplus_{i\in I} L^\infty(X_i)$
and $\mu(X_i)<\infty$ for all~$i\in I$.
To prove that~$\mathscr{A}\cong \ell^\infty(Y)$
for some set~$Y$ it suffices
to show that there is for every~$i\in I$
 a set~$Y_i$
with $L^\infty(X_i)\cong \ell^\infty(Y_i)$,
because then 
\begin{equation*}
\textstyle \mathscr{A}\ \cong \ 
\bigoplus_{i\in I} \ell^\infty(Y_i)\ \cong\ 
\ell^\infty\bigl(\,\bigcup_{i\in I} Y_i\,\bigr).
\end{equation*}

Let~$i\in I$ be given.
We must find a set~$Y$ 
such that~$L^\infty(X_i)\cong \ell^\infty(Y)$.
Since~$\mathscr{A}\cong L^\infty(X_i)\,\oplus\,\bigoplus_{j\neq i} 
L^\infty(X_j)$ is duplicable,
$L^\infty(X_i)$ is duplicable
by  Corollary~\ref{cor:duplicable-product}.
By Lemma~\ref{lem:measure-space-continuous-discrete},
there is a measurable subset~$D$ of~$X_i$ such that~$D$
is discrete, and $C:=X\backslash D$ is continuous.
We have~$L^\infty(X_i)\cong L^\infty(D)\oplus L^\infty(C)$
by Lemma~\ref{lem:measure-space-partition},
and
so $L^\infty(D)$ and~$L^\infty(C)$
are duplicable
(again by Corollary~\ref{cor:duplicable-product}).

By Lemma~\ref{lem:continuous-finite-measure-space-not-duplicable},
$L^\infty(C)$
can only be duplicable if~$\mu(C)=0$,
and so~$L^\infty(C)\cong \{0\}$.
On the other hand,
since~$D$ is discrete,
we have~$L^\infty(D)\cong \ell^\infty(Y)$
for some set~$Y$
(by Corollary~\ref{cor:discrete-ell-x}).
Thus we have
\begin{equation*}
L^\infty(X_i)
\ \cong\  L^\infty(D)\,\oplus\, L^\infty(C)
\ \cong\  \ell^\infty(Y)\,\oplus\, \{0\}
\ \cong\  \ell^\infty(Y).
\end{equation*}
Hence~$\mathscr{A}\cong \ell^\infty(Z)$
for some set~$Z$.
\end{point}

\subsection{Monoids}
\label{sec:monoids-in-vna}

We further justify our choice,
$\sem{!A} = \linf(\nsp(\sem{A}))$,
by proving that $\linf(\nsp(\mathscr{A}))$
is the free (commutative) monoid on~$\mathscr{A}$ in~$\W{miu}$.
As a corollary, we also obtain that $\linf(\W{cpsu}(\mathscr{A},\mathbb{C}))$
is the free (commutative) monoid on~$\mathscr{A}$
in~$\W{cpsu}$.


Let $(\Cat{C},\otimes,I)$ be a symmetric monoidal category (SMC).
A \textbf{monoid} in $\Cat{C}$ is an object $A\in\Cat{C}$ with
a `multiplication' map
$m\colon A\otimes A\to A$
and a `unit' map $u\colon I\to A$
satisfying the associativity and the unit law,
i.e.\ making the following diagrams commute.
\[
\xymatrix@R-.5pc{
(A\otimes A)\otimes A
\ar[d]_{\alpha}
\ar[rr]^-{m\otimes \id}
&&
A\otimes A
\ar[d]^{m}
\\
A\otimes (A\otimes A)
\ar[r]_-{\id\otimes m}
&
A\otimes A
\ar[r]_-{m}
&
A
}
\qquad
\xymatrix@R-.5pc{
I\otimes A
\ar[dr]_{\lambda}
\ar[r]^-{u\otimes \id}
&
A\otimes A
\ar[d]^{m}
&
\ar[l]_-{\id\otimes u}
A\otimes I
\ar[dl]^{\rho}
\\
&
A
&
}
\]
Here $\alpha,\lambda,\rho$ respectively
denote the associativity isomorphism, and
the left and the right unit isomorphism.
A monoid $(A,m,u)$ is \textbf{commutative} if
$m\circ \gamma=m$,
where $\gamma\colon A\otimes A\to A\otimes A$ is the symmetry isomorphism.
A \textbf{monoid morphism} between monoids $(A_1,m_1,u_1)$
and $(A_2,m_2,u_2)$ is an arrow $f\colon A_1\to A_2$
that satisfies $m_2\circ (f\otimes f)=f \circ m_1$
and $u_2=f \circ u_1$.
We denote the category of monoids
and monoid morphisms in $\Cat{C}$
by $\Mon(\Cat{C},\otimes,I)$ or simply by $\Mon(\Cat{C})$.
We write $\CMon(\Cat{C})\subseteq\Mon(\Cat{C})$ for
the full subcategory of commutative monoids.

% The notion of \emph{(commutative) comonoids} in an SMC
% is defined in the dual manner.

First we characterise duplicable von Neumann algebras
in terms of monoids.

\begin{point}{Proposition}
\label{prop:dup-vna-is-monoid}
Let $\scrA$ be a von Neumann algebra.
The following are equivalent.
\begin{enumerate}
\item
$\scrA$ is duplicable.
\item
$\scrA$ carries a monoid structure in $(\W{miu},\otimes,\C)$.
\item
$\scrA$ carries a monoid structure in $(\W{cpsu},\otimes,\C)$.
\end{enumerate}
In that case, $\scrA$ is a commutative monoid
(both in $\W{miu}$ and $\W{cpsu}$),
and the monoid structure $(m\colon\scrA\otimes\scrA\to\scrA,u\colon\C\to\scrA)$
is a duplicator on $\scrA$
(when $u$ is identified with $u(1)\in\scrA$).
By Theorem~\ref{duplicable}, a monoid structure on $\scrA$ is unique.
\end{point}
\begin{point}{Proof}
(1 $\Rightarrow$ 2)
By Theorem~\ref{duplicable},
$\scrA$ is MIU-isomorphic to $\linf(X)$ for some set $X$.
By Corollary~\ref{cor:linfX-cmon},
$\linf(X)$ carries a commutative monoid structure in $\W{miu}$.
Thus we can equip $\scrA$ with a commutative monoid structure in $\W{miu}$
via the isomorphism $\scrA\cong\linf(X)$.

(2 $\Rightarrow$ 3) Trivial.

(3 $\Rightarrow$ 1)
Let $m\colon\scrA\otimes\scrA\to\scrA$ and $u\colon\C\to\scrA$ be a monoid structure
on $\scrA$ in $\W{cpsu}$.
Then $m(u(1)\otimes a)=
(m\circ (u\otimes \id))(1\otimes a)=
\lambda(1\otimes a)=1\cdot a=a$.
and similarly $m(a\otimes u(1))=a$.
Thus $\scrA$ is duplicable via $m$ and $u(1)$.
\end{point}

It follows that there is no distinction
between (commutative) monoids in $\W{miu}$ and in $\W{cpsu}$.

\begin{point}{Proposition}
\label{prop:cmon-mon-vNA}
$\CMon(\W{miu})
=\Mon(\W{miu})
=\CMon(\W{cpsu})
=\Mon(\W{cpsu})$.
\end{point}
\begin{point}{Proof}
By Proposition~\ref{prop:dup-vna-is-monoid},
$\Mon(\W{miu})$ and $\Mon(\W{cpsu})$ have the same objects,
and $\CMon(\W{miu})=\Mon(\W{miu})$ and $\CMon(\W{cpsu})=\Mon(\W{cpsu})$.
Let $f\colon (\scrA_1,m_1,u_1)\to (\scrA_2,m_2,u_2)$
be a morphism in $\Mon(\W{cpsu})$.
Then $f(1)=(f\circ u_1)(1)=u_2(1)=1$
and $f(ab)=(f\circ m_1)(a\otimes b)=(m_2\circ (f\otimes f))(a\otimes b)=f(a)f(b)$,
using the fact that the monoid structure is a unique duplicator.
Thus $f$ is a normal MIU-map, and
we are done since $\Mon(\W{miu})\subseteq \Mon(\W{cpsu})$.
\end{point}

It turns out that $\linf(\nsp(\mathscr{A}))$,
our interpretation of the $\bang$ operator in the quantum lambda calculus,
is exactly the free (commutative) monoid on $\scrA$ in $\W{miu}$.

\begin{point}{Theorem}
\label{thm:free-monoid-in-vNAMIU}
Let~$\mathscr{A}$
be a von Neumann algebra,
and let~$\eta\colon \mathscr{A}\to\linf(\nsp(\mathscr{A}))$
be the normal MIU-map
given by~$\eta(a)(\varphi)= \varphi(a)$.
Then $\linf(\nsp(\mathscr{A}))$
is the free (commutative) monoid
on~$\mathscr{A}$
in~$\W{miu}$ via~$\eta$.
\end{point}
\begin{point}{Proof}
Let~$\mathscr{B}$
be a monoid 
on~$\W{miu}$,
and let~$f\colon \mathscr{A}\to\mathscr{B}$
be a normal MIU-map
We must show that
there is a unique
monoid morphism
$g\colon \linf(\nsp(\mathscr{A}))
\rightarrow \mathscr{B}$
such that~$g\circ \eta = f$.

By Theorem~\ref{duplicable},
we may assume that~$\mathscr{B}=\linf(Y)$
for some set~$Y$.
Since~$\nsp\colon \op{(\W{miu})}\to \Cat{Set}$
is left adjoint
to~$\linf\colon \Cat{Set} \to \op{(\W{miu})}$
with unit~$\eta$ (Lemma~\ref{lem:nsp-linf-ssm}),
there is a unique map $h\colon Y\to \nsp(\mathscr{A})$
with $\linf(h)\circ \eta = f$.
Since~$\linf$ is full and faithful
by Corollary~\ref{cor:linf-ff},
the only thing that remains to be shown is that~$\linf(h)$
is a monoid morphism.
Indeed it is,
since the monoid multiplication
on~$\linf(\nsp(\mathscr{A}))$
and~$\linf(Y)$
is given by ordinary multiplication,
which is preserved by~$\linf(h)$,
being a MIU-map.
\end{point}

\begin{point}{Corollary}
Let~$\mathscr{A}$
be a von Neumann algebra.
Then $\linf(\W{cpsu}(\mathscr{A},\C))$
is the free (commutative) monoid
on~$\mathscr{A}$ in~$\W{cpsu}$.
\end{point}
\begin{point}{Proof}
Theorem~\ref{thm:free-monoid-in-vNAMIU} asserts that
$\linf\circ\nsp$ is a left adjoint to
the forgetful functor $\Mon(\W{miu})\to\W{miu}$.
By Prop.~\ref{prop:cmon-mon-vNA},
the forgetful functor $\Mon(\W{cpsu})\to\W{cpsu}$
factors through $\W{miu}$ as:
\[
	\text{todo: correct}
%\xymatrix{
%\Mon(\W{cpsu})
%\ar@{=}[r]
%&
%\Mon(\W{miu})
%\ar[r]^-{\bot}&
%\ar@/_3ex/[l]_{\linf\circ\nsp}
%\W{miu}
%\ar@{^{ (}->}[r]^-{\bot}&
%\ar@/_3ex/[l]_{\mathcal{F}}
%\W{cpsu}
%}
\]
where $\mathcal{F}$ is the adjoint functor of Lemma~\ref{lem:adjoint-to-incl}.
Thus the free monoid on $\scrA$ in $\W{cpsu}$ is given by:
\[
(\linf\circ\nsp\circ\mathcal{F})(\scrA)
=
\linf(\W{miu}(\mathcal{F}\scrA,\C))
\cong
\linf(\W{cpsu}(\scrA,\C))
\enspace.
\]
\end{point}

Finally we observe that duplicable von Neumann algebras and
monoids in $\W{miu}$ (or in $\W{cpsu}$) are simply identified with sets.
Let $\dW{miu}\subseteq\W{miu}$ denote the full subcategory
consisting of duplicable von Neumann algebras.

\begin{point}{Proposition}
	$\Mon(\W{miu})\cong\dW{miu}\simeq\op{\Cat{Set}}$.
\end{point}
\begin{point}{Proof}
($\Mon(\W{miu})\cong\dW{miu}$)
By Proposition~\ref{prop:dup-vna-is-monoid},
we have the forgetful functor $U\colon\Mon(\W{miu})\to\dW{miu}$
that is bijective on objects.
We need to prove that the functor is full,
namely that any normal MIU-map between monoids
is a monoid morphism.
The monoid structure is preserved by normal MIU-maps
since the monoid structure is a duplicator
by Proposition~\ref{prop:dup-vna-is-monoid},
and the duplicator is given by the multiplication and unit
of the von Neumann algebra by Theorem~\ref{duplicable}.

($\dW{miu}\simeq\op{\Cat{Set}}$)
We have a functor $\linf\colon\op{\Cat{Set}}\to\dW{miu}$,
which is full and faithful by Corollary~\ref{cor:linf-ff},
and also essentially surjective
by Theorem~\ref{duplicable}.
\end{point}
\end{parsec}
\section{To be moved}
\begin{parsec}%
\begin{point}{Lemma}%
\TODO{Adapt this lemma to something of the kind:
if $\times$ has the universal property of the tensor product
from~\sref{tensor-universal-property},
then~$\otimes$ is a tensor product.}
Given a tensor product  $\otimes \colon \scrA\times \scrB\to\scrA\otimes \scrB$
of von Neumann algebras~$\scrA$ and~$\scrB$,
$\otimes_\odot\colon \scrA\odot\scrB\to \scrA\otimes \scrB$
has ultraweakly dense range.
\begin{point}{Proof}%
Since
$\otimes_\odot(\scrA\odot \scrB)$,
the range of~$\otimes_\odot$,
is easily seen to be a $*$-subalgebra
of~$\scrA\otimes \scrB$,
its closure---let us denote it by~$\scrT$---is a
von Neumann subalgebra of~$\scrA\otimes \scrB$
by~\TODO{}.
Our task is, of course, to show that~$\scrT=\scrA\otimes \scrB$.
Since the restriction $\tau\colon \scrA\times\scrB\to 
\scrT$ of~$\otimes$ is clearly bounded,
and normal (by ultraweak permanence, \TODO{...}),
there is by the defining property of the tensor product
a unique ultraweakly continuous map 
$\varrho\colon \scrA\otimes \scrB\to\scrT$
with $\varrho(a\otimes b)= a\otimes b$ for all~$a\in\scrA$
and~$b\in \scrB$.
It follows that~$\varrho(t)=t$ for all~$t\in \scrT$,
or, in other words, $\varrho\circ \iota = \id$
where~$\iota\colon \scrT\to\scrA\otimes \scrB$
denotes the inclusion (being a nmiu-map).
On the other hand,
$\iota\circ \varrho\colon \scrA\otimes \scrB\to\scrA\otimes \scrB$
is an ultraweakly continuous bounded
linear map with $(\iota\circ \varrho)(a\otimes b)=a\otimes b$
for all~$a\in\scrA$ and~$b\in\scrB$,
as is~$\id\colon \scrA\otimes \scrB\to\scrA\otimes\scrB$;
so that we may conclude~$\iota\circ \varrho=\id$.
Hence~$\iota\colon \scrT\to\scrA\otimes\scrB$
is surjective, and~$\scrT=\scrA\otimes \scrB$.\qed
\end{point}
\end{point}
\begin{point}[uwc-bounded]{Proposition}%
An ultraweakly continuous linear map $f\colon \scrA\to\scrB$
between von Neumann algebras is bounded.
\begin{point}[uwc-bounded-1]{Proof}%
The problem reduces to the case that~$f$ is involution preserving
by writing~$f$ as
$f\equiv \Real{f}+i\Imag{f}$,
where~$\Real{f}(a)=\Real{f(\Real{a})}
+ i\Real{f(\Imag{a})}$
and~$\Imag{f}(a)=\Imag{f(\Real{a})}+i\Imag{f(\Imag{a})}$
for~$a\in\scrA$,
since clearly~$f$ is bounded when both~$\Real{f}$ and~$\Imag{f}$
are bounded,
and both $\Real{f}$ and~$\Imag{f}$ 
are ultraweakly continuous and involution preserving.

Note that such an involution preserving map~$f$
is bounded
when its restriction $g\colon \Real{\scrA}\to\Real{\scrB}$
to self-adjoint elements
is bounded,
because then $\|f(a)\|=\|g(\Real{a})+ig(\Imag{a})\|\leq
\|g\| (\|\Real{a}\|+\|\Imag{a}\|)
\leq 2\|g\| \|a\|$ for all~$a\in\scrA$.

The point of this reduction
to self-adjoint elements
is that writing~$\Omega$ for the 
set of npu-maps $\omega\colon \scrB\to\C$
the equality
$\|f(a)\|=\sup_{\omega\in\Omega} \left|\omega(f(a))\right|$
holds by~\TODO{...} for all self-adjoint~$a\in\scrA$,
but not necessarily for arbitrary~$a\in\scrA$.
Thus, to prove that~$f$
is bounded it suffices to show that $\omega\circ f$
is bounded for each~$\omega\in\Omega$,
because then $\|f(a)\|\leq (\sup_{\omega\in\Omega}\|\omega\circ f\|)\,\|a\|$
for all~$a\in \Real{\scrA}$,
and~$\sup_{\omega\in \Omega} \|\omega\circ f\|<\infty$
by the principle of uniform boundedness (\sref{pub}).

To see that such a functional $\sigma:=\omega\circ f$,
where~$\omega\in\Omega$,
is bounded,
note that any sequence~$a_1,a_2,\dotsc$
in~$\scrA$ that converges to~$0$
with respect to the norm on~$\scrA$
also converges to~$0$ ultraweakly,
and so
(as both~$\omega$ and~$f$ are ultraweakly continuous)
$\sigma(a_1),\,\sigma(a_2),\,\dotsc$
converges to~$0$ in~$\C$ ultraweakly,
and thus with respect to the norm on~$\C$.\qed
\end{point}
\end{point}
\begin{point}[ncp-uwlim]{Proposition}%
Given von Neumann algebras~$\scrA$
and~$\scrB$
the pointwise ultraweak limit
$f\colon \scrA\to\scrB$
of a net of  positive linear maps $f_\alpha\colon \scrA\to\scrB$
is positive, and, 
\begin{enumerate}
\item
$f$ is completely positive provided
that the $f_\alpha$ are completely positive, and
\item
$f$ is normal provided that the $f_\alpha$ are normal
and the ultraweak convergence of the~$f_\alpha$ to~$f$
is uniform on~$[0,1]_\scrA$.
\end{enumerate}
\begin{point}{Proof}%
Since given~$a\in \scrA$ the element~$f(a)$
is the ultraweak limit of the positive elements~$f_\alpha(a)$,
and therefore positive (by~\sref{vn-positive-basic}),
we see that~$f$ is positive.

Suppose that each~$f_\alpha$ is completely positive.
To show that~$f$ is completely positive,
we must prove, given~$a_1,\dotsc,a_n\in\scrA$
and~$b_1,\dotsc,b_n\in\scrB$,
that 
the element $\sum_{i,j} b_i^* f(a_i^*a_j)b_j$
of~$\scrB$
is positive.
And indeed it is,
being the ultraweak limit of
the positive elements $\sum_{i,j} b_i^* f_\alpha (a_i^* a_j)b_j$,
because  $f_\alpha(a_i^* a_j)$
converges ultraweakly to~$f(a_i^* a_j)$,
and~$b_i^*(\,\cdot\,)b_j\colon \scrB\to\scrB$
is ultraweakly continuous
(\sref{mult-uws-cont})
for any~$i$ and~$j$.

If the~$f_\alpha$ 
are normal,
and converge uniformly on~$[0,1]_\scrA$ ultraweakly
to~$f$,
then~$f$ is ultraweakly continuous
on~$[0,1]_\scrA$
(because the uniform limit of continuous functions is continous),
and thus normal (by~\sref{p-uwcont}).\qed
\end{point}
\end{point}
\end{parsec}
\begin{parsec}%
\begin{point}[vn-dense-pos]{Lemma}%
Let~$\scrS$ be an ultrastrongly dense $C^*$-subalgebra
of a von Neumann algebra~$\scrA$.
Let~$D$ be a norm dense subset of~$\scrS$.
Then~$\{\,d^*d\colon\,d\in D\,\}$
is ultrastrongly dense in~$\pos{\scrA}$.
\begin{point}{Proof}%
By Kaplansky's Density Theorem (\sref{kaplansky})
$\pos{\scrS}$ 
is ultrastrongly dense in~$\pos{\scrA}$,
so it suffices to show that $\{\, d^*d\,\colon d\in D\}$
is ultrastrongly dense in~$\pos{\scrS}$.
Given~$a\in \pos{\scrS}$,
there is a sequence $d_1,d_2,\dotsc \in D$
that norm converges to~$\sqrt{a}$,
so that~$d_1^*d_1,\,d_2^*d_2,\,\dotsc$
converges to~$\sqrt{a}^*\sqrt{a}\equiv a$
with respect to the norm, and thus ultrastrongly too.\qed
\end{point}
\end{point}
\end{parsec}%
\end{document}

% vim: ft=tex.latex
