\documentclass[main]{subfiles}
\begin{document}
\begin{parsec}%
\begin{point}{Definition}%
A \define{$C^*$-algebra}
is a complex vector space~$\scrA$
endowed with
\begin{enumerate}
\item
a binary operation,
called \define{multiplication}
(and denoted as such),
which is associative, and linear in both coordinates;
\item
an element~$1$, called \define{unit},
such that $1\cdot a = a = a\cdot 1$
for all~$a\in \scrA$;
\item
a unary operation $(\,\cdot\,)^*$,
called \define{involution},
such that $(a^*)^*=a$,
$(ab)^*=b^*a^*$,
$(\lambda a)^* = \bar\lambda a^*$,
and $(a+b)^* = a^*+b^*$
for all~$a,b\in\scrA$ and~$\lambda\in \C$;
\item
a complete \define{norm} $\|\,\cdot\,\|$
such that
$\|ab\|\leq\|a\|\|b\|$
for all~$a,b\in\scrA$,
and 
\begin{equation*}
\label{eq:Cstar-identity}
\|a^*a\|\ =\ \|a\|^2
\end{equation*}
holds, which is called the \define{$C^*$-identity}.
\end{enumerate}
The $C^*$-algebra $\scrA$ is called \define{commutative}
if $ab=ba$ for all~$a,b\in\scrA$.
\end{point}
\end{parsec}
\begin{parsec}%
\begin{point}{Examples}%
The set~$\C$ of \define{complex numbers}
forms a commutative  $C^*$-algebra
in which addition, (scalar) multiplication, and~$1$
have their usual meaning.
Involution is given by conjugation ($z^*=\bar{z}$),
and norm by modulus ($\|z\|=|z|$).
\begin{point}%
Let~$X$ be a compact Hausdorff space.
The set $\define{C(X)}$ of \define{continuous functions}
from~$\scrA$ to~$\C$
forms a commutative $C^*$-algebra
when addition, (scalar) multiplication, involution and~$1$ are
interpretted coordinatewise \grayed{(e.g.~$(f+g)(x)=f(x)+g(x)$)},
and the norm is taken to be 
$\|f\|=\sup_{x\in X} |f(x)|$
(the \define{sup-norm}).
\begin{point}%
The set~$\scrB(\scrH)$ of bounded operators
on a Hilbert space~$\scrH$ forms a $C^*$-algebra.
\end{point}
\end{point}
\end{point}
\end{parsec}
\begin{parsec}%
\begin{point}{Definition}
An element $a$ of a $C^*$-algebra $\scrA$ is called
\begin{enumerate}
\item \define{self-adjoint} if $a^* =a$;
\item \define{positive}
if $a\equiv b^*b$ for some $b\in \scrA$;
\item a \define{projection} if $a=a^*a$;
\item \define{central} if $ab=ba$ for all~$b\in\scrA$.
\end{enumerate}
\end{point}
\end{parsec}
%
% morphisms between C^*-algebras
%
\begin{parsec}
\begin{point}{Definition}
A linear map $f\colon \scrA \to \scrB$
between $C^*$-algebras
is called
\begin{enumerate}
\item
\define{\textbf{m}ultiplicative}
if $f(ab)=f(a)f(b)$ for all $a,b\in\scrA$;
\item
\define{\textbf{i}nvolutive}
if $f(a^*)=f(a)^*$ for all~$a\in\scrA$;
\item
\define{\textbf{p}ositive}
if $f(a)$ is positive
for every positive $a\in\scrA$, and
\item
\define{\textbf{u}nital}
if $f(1)=1$.
\end{enumerate}
\end{point}
\begin{point}%
We use the bold letters as abbreviations,
so for instance,
$f$ is \define{pu} if it is positive and unital,
and a \define{miu-map}
is a multiplicative, involutive, unital linear map between $C^*$-algebras,
(which is usually called a \define{unital $*$-homomorphism}.)
\end{point}
\end{parsec}
%
%
%
\begin{parsec}[ad-monotone]%
\begin{point}{Exercise}%
Show that~$a^*ba$
is positive
when~$b$ is a positive of a $C^*$-algebra~$\scrA$,
and $a\in \scrA$ is any element.
\end{point}
\end{parsec}

%
% geometric series 
%
\begin{parsec}[geometric]%
\begin{point}{Lemma}%
Let~$a$ be an element of a $C^*$-algebra~$\scrA$ with~$\|a\|<1$.
Then~$a^\perp=1-a$ has an inverse,
namely~$(a^\perp)^{-1}= \sum_{n=0}^\infty\, a^n$
(norm convergence).
\end{point}
\begin{point}{Proof}%
Note that
$(1-\|a\|)\,(1+\|a\|+\|a\|^2+\dotsb+\|a\|^N) \,=\, 1-\|a\|^{N+1}$,
and so 
\begin{equation*}
\sum_{n=0}^N \|a\|^n \ =\  \frac{1-\|a\|^{N+1}}{1-\|a\|}
\end{equation*}
for every~$N$.
Thus,
since $\|a\|^N$ converges to~$0$
(by~\TODO{} because $\|a\|<1$),
we  get $\sum_{n=0}^\infty \|a\|^n = (1-\|a\|)^{-1}$.

\begin{point}%
Note that $a^N$ norm converges to~$0$,
because $\|a\|^N$ converges to~$0$.
Also (but slightly less obvious),
$\sum_n a^n$ norm converges,
because~$\sum_n \|a\|^n$ converges.
\end{point}
\begin{point}%
Thus, taking the norm limit
on both sides of $(1-a)(1+a+a^2+\dotsb a^N) = 1-a^{N+1}$,
gives us $(1-a)(\sum_n a^n) = 1$.
Since we can derive $(\sum_n a^n)(1-a) = 1$
in a similar manner, 
we see that $\sum_n a^n$ is the inverse of~$1-a$.
\end{point}
\end{point}
\begin{point}[spectrum-bounded]{Exercise}
Deduce from the result above that $a-\lambda$ is invertible
when~$|\lambda|>\|a\|$,
where~$a$ is an element of a $C^*$-algebra~$\scrA$,
and~$\lambda\in\C$.
\end{point}
\end{parsec}
%
%  Cauchy--Schwarz for positive functionals
%
\begin{parsec}[cstar-cs]%
\begin{point}{Lemma}%
Let~$\scrA$ be a $C^*$-algebra,
and let~$f\colon \scrA\to \C$ be a positive linear map.\\
Then, for all~$a,b\in\scrA$,\quad
$\left|f(a^*b)\right|^2 \ \leq\ f(a^*a)\,f(b^*b)$.
\end{point}
\begin{point}{Proof}%
\TODO{give; NB: it is not trivial that~$f$ is involutive}
\end{point}
\end{parsec}




\begin{parsec}[gelfand]%
\begin{point}{Gelfand's Representation Theorem}%
Let~$\scrA$ be a commutative $C^*$-algebra.
\begin{enumerate}
\item
The \define{spectrum} of~$\scrA$, \define{$\spec(\scrA)$},
the set 
of miu-maps $\scrA\to \C$
endowed with the topology of pointwise convergence,
is a compact Hausdorff space.

\item
The \define{Gelfand transform},
the map $\gamma\colon \scrA\to C(\spec(\scrA))$
given by $\gamma(a)(\varphi)=\varphi(a)$,
is an isomorphism.
\end{enumerate}
\end{point}
\end{parsec}
%
% square root
%
\begin{parsec}[cstar-square]%
\begin{point}{Corollary}%
For each positive element~$a$ of a $C^*$-algebra~$\scrA$,
there is a unique positive element~\define{$\sqrt{a}$} of~$\scrA$
with $(\sqrt{a})^2 = a$.

\begin{point}[cstar-square-commutes]%
Moreover,
if $b\in \scrA$ commutes with~$a$,
then~$b$ commutes with~$\sqrt{a}$.
\end{point}
\end{point}
\end{parsec}
%
%
%
\begin{parsec}%
\begin{point}{Exercise}
Show that a positive map between $C^*$-algebras
is involutive.
\end{point}
\end{parsec}


%
% von Neumann algebras
%
\begin{parsec}[vna]%
\begin{point}{Definition}%
A $C^*$-algebra~$\scrA$
is a \define{von Neumann algebra}
when
\begin{enumerate}
\item
every bounded directed subset~$D$
of self-adjoint elements of~$\scrA$ (so $D\subseteq \sa{\scrA}$) 
has a supremum $\bigvee D$ in $\sa{\scrA}$, and
\item
if $a$ is a positive element of~$\scrA$
with $\omega(a)=0$ for every \emph{normal} (see below) positive 
linear map $\omega\colon \scrA\to \C$,
then~$a=0$.
\end{enumerate}
\begin{point}%
A positive linear map $\omega\colon \scrA\to \C$
is called \define{normal}
if $\omega(\bigvee D) = \bigvee_{d\in D} \omega(d)$
for every bounded directed subset of self-adjoint elements of~$D$
which has a supremum $\bigvee D$ in $\sa{\scrA}$.
\end{point}%
\begin{point}%
The \define{ultraweak topology} on $\scrA$
is the least topology on~$\scrA$
that makes all normal positive linear maps $\omega\colon \scrA\to \C$
continuous.
\end{point}
\end{point}
\end{parsec}
%
% multiplication turns suprema into ultraweak limits
%
\begin{parsec}[vanishing-effects]%
\begin{point}{Lemma}%
Let~$(x_\alpha)_\alpha$ be 
a net of effects of a von Neumann algebra~$\scrA$
which converges ultraweakly to~$0$.
Then $(x_\alpha a)_\alpha$ converges ultraweakly
to $0$ for every~$a\in\scrA$.
\end{point}
\begin{point}{Proof}%
Let~$\omega\colon \scrA\to \C$ be a normal positive linear map.
We have, for each~$\alpha$,
\begin{alignat*}{3}
\left|\,\omega(x_\alpha a)\,\right|^2
\ &=\ 
\left|\, \omega(\,\sqrt{x_\alpha}\,\sqrt{x_\alpha}\,a\,)\, \right|^2
\qquad&&\text{since $x_\alpha\geq 0$}\\
\ &\leq\ 
\omega(x_\alpha)\  \omega(\,a^* x_\alpha a\,) 
\qquad&&\text{by \sref{cstar-cs}}\\
\ &\leq\ 
\omega(x_\alpha)\ \omega(a^*a)
\qquad&&\text{since $x_\alpha\leq 1$}.
\end{alignat*}
Thus,
since $(\omega(x_\alpha))_\alpha$
converges to~$0$,
we see that $(\omega(x_\alpha a))_\alpha$
converges to~$0$,
and so $(x_\alpha a)_\alpha$ converges ultraweakly to~$0$.
\end{point}
\end{parsec}
\begin{parsec}%
\begin{point}{Exercise}
Let~$D$ be a bounded directed set of self-adjoint
elements of a von Neumann algebra~$\scrA$,
and let~$a\in \scrA$.
\begin{point}[vna-supremum-uwlimit]%
Show that the net~$(d)_{d\in D}$ converges ultraweakly to~$\bigvee D$.
\end{point}
\begin{point}[vna-supremum-mult]%
Use~\sref{vanishing-effects}
to show that $(da)_d$ converges ultraweakly to~$(\bigvee D)a$,
and that~$(a^*d)_d$ converges ultraweakly to~$a^* (\bigvee D)$.
\end{point}
\begin{point}[vna-supremum-commutes]%
Show that if~$ad=da$ for all~$d\in D$,
then $a(\bigvee D) = (\bigvee D)a $.
\end{point}
\end{point}
\end{parsec}
%
%  ad is normal
%
\begin{parsec}[ad-normal]%
\begin{point}{Proposition}%
Let~$a$ be an element of a von Neumann algebra~$\scrA$.
Then~$\bigvee_{d\in D} a^*\,d\,a = a^*\,(\bigvee D)\, a$
for every bounded directed subset~$D$ of self-adjoint
elements of~$\scrA$.
\end{point}
\begin{point}[ad-normal-1]{Proof}%
If~$a$ is invertible,
then the (by~\sref{ad-monotone}) order preserving map $b\mapsto a^*ba$
has an order preserving inverse (namely $b\mapsto (a^{-1})^* b a^{-1}$),
and therefor preserves all suprema.
\begin{point}%
The general case reduces to the case that~$a$ 
is invertible
in the following way.
There is (by~\sref{spectrum-bounded})
 $\lambda>0$ such that $\lambda+a$ is invertible.
Then as $d$ increases 
\begin{equation*}
a^*\,d\,a \ \equiv\  (\lambda+a)^*\,d\,(\lambda+a) \,-\,
 \lambda^2 \,-\, \lambda a^*d \,-\, \lambda da
\end{equation*}
converges ultraweakly
to~$a^* \,(\bigvee D)\,a$,
because $(\ (\lambda+a)^*\,d\,(\lambda+a)\ )_d$
converges ultraweakly to $(\lambda+a)^*\,(\bigvee D)\,(\lambda+a)$
by~\sref{ad-normal-1} and~\sref{vna-supremum-uwlimit},
and $(a^*d+da)_d$ converges ultraweakly to $a^*(\bigvee D)+(\bigvee D)a$
by~\sref{vna-supremum-mult}.
Since~$(a^*da)_d$ converges to~$\bigvee_{d\in D} a^*d a$ too,
we conclude that~$\bigvee_{d\in D} a^* \,d\, a = a^*\,(\bigvee D)\,a$.
\end{point}
\end{point}
\end{parsec}



\begin{parsec}[ad-contraposed]%
\begin{point}{Lemma}%
Let~$a$ be an element of a $C^*$-algebra~$\scrA$
with $\|a\|\leq 1$,
and let~$p$ and~$q$ be projections on~$\scrA$.
Then 
$a^* p a \leq q^\perp$
iff $paq=0$
iff  $aqa^*\leq p^\perp$.
\end{point}
\begin{point}{Proof}%
Suppose that~$a^*pa\leq q^\perp$.
Then we have $q a^*pa q \leq qq^\perp q = 0$
(see \sref{ad-monotone})
and so $paq=0$,
because $\|paq\|^2=\|(paq)^*paq\|=0$
by the $C^*$-identity.
Applying $(\,\cdot\,)^*$,
we get $qa^*p=0$, and so both $qa^* = qa^*p^\perp$
and $aq = p^\perp aq$, giving
us $aqa^* = p^\perp a q a^* p^\perp 
\leq p^\perp$,
where we used that $aqa^*\leq aa^*\leq \|aa^*\|=\|a\|^2\leq 1$.
By a similar reasoning,
we get $aqa^*\leq p^\perp \implies paq=0\implies a^*pa\leq q^\perp$.
\end{point}
\begin{point}[projection-above-effect]{Corollary}%
Let~$a$ be an effect of a $C^*$-algebra~$\scrA$,
and let~$p$ be a projection from~$\scrA$.
Then $a\leq p$ iff $\sqrt{a}p=p$ iff $p\sqrt{a}=p$
iff $\sqrt{a}p^\perp = 0$ iff $p^\perp\sqrt{a}=0$ 
iff  $ap=p$ iff $pa=p$ iff $ap^\perp=0$ iff $p^\perp a=0$
iff $a^2 \leq p$
iff $\sqrt{a}\leq p$.
\end{point}
\end{parsec}

%
% Projections
%
\begin{parsec}%
\begin{point}{Proposition}%
Above every effect~$b$ of a von Neumann algebra~$\scrA$,
there is a smallest projection, \define{$\ceil{b}$},
called the \define{ceiling} of~$b$,
 given by $\ceil{b}=\bigvee_{n=0}^\infty b^{\nicefrac{1}{2^n}}$.
\begin{point}[vna-ceil-commutes]%
Moreover, if $a\in \scrA$ commutes with $b$,
then~$a$ commutes with~$\ceil{b}$.
\end{point}
\end{point}
\begin{point}{Proof}
Let~$p$ denote the supremum of~$0\leq b\leq b^{\nicefrac{1}{2}}\leq
b^{\nicefrac{1}{4}}\leq\dotsb\leq 1$.
\begin{point}[vna-ceil-point-1]%
To begin,
note that if~$a\in \scrA$
commutes with~$b$,
then~$a$ commutes with~$p$.
Indeed, for such~$a$ we have~$a\sqrt{b}=\sqrt{b}a$
by~\sref{cstar-square-commutes},
and so $a b^{\nicefrac{1}{2^n}} = b^{\nicefrac{1}{2^n}} a$
for each~$n$
by induction.
Thus~$ap=pa$ by~\sref{vna-supremum-commutes}.
\end{point}
\begin{point}%
Let us prove that~$p$ is a projection, c.q.~$p^2=p$. 
Since~$p\leq 1$, we already have $p^2\equiv \sqrt{p}p\sqrt{p}\leq p$
by~\sref{ad-monotone},
and so we only need to show that $p\leq p^2$. We have:
\begin{alignat*}{3}
 p^2 \ &=\  \textstyle \bigvee_m \sqrt{p} \,b^{\nicefrac{1}{2^m}} \,\sqrt{p}
\qquad&&\text{by \sref{ad-normal}} \\
&=\ \textstyle\bigvee_m b^{\nicefrac{1}{2^{m+1}}}\, p\,
b^{\nicefrac{1}{2^{m+1}}} 
\qquad&&\text{by \sref{vna-ceil-point-1} and \sref{cstar-square-commutes}} \\
&=\ \textstyle \bigvee_m \bigvee_n \, 
b^{\nicefrac{1}{2^{m+1}}}\, b^{\nicefrac{1}{2^n}}\,
b^{\nicefrac{1}{2^{m+1}}} \qquad && \text{by \sref{ad-normal}}
\end{alignat*}
Thus $p^2 \geq b^{\nicefrac{1}{2^k}}$
for each~$k$ (taking $n=m=k+1$,)
and so~$p^2 \geq p$.
\end{point}
\begin{point}%
It remains to be shown that~$p$ is the \emph{least} projection
above~$b$.
Let~$q$ be a projection in~$\scrA$ with $b\leq q$;
we must show that~$q\leq p$.
We have $b^{\nicefrac{1}{2}}\leq q$
by~\sref{projection-above-effect},
and so $b^{\nicefrac{1}{2^n}}\leq q$ for each~$n$ by induction.
Hence $p\leq q$.
\end{point}
\end{point}
\end{parsec}

%
%	floor
%
\begin{parsec}[vna-floor]%
\begin{point}{Proposition}%
Below every effect~$b$ of a von Neumann algebra~$\scrA$,
there is greatest projection, $\floor{b}$,
called the \define{floor} of~$b$,
given by~$\floor{b} = \bigwedge_{n=0}^\infty b^{2^{n}}$.
\begin{point}%
Moreover, if~$a\in \scrA$ commutes with~$b$,
then~$b$ commutes with~$\floor{b}$.
\end{point}
\end{point}
\begin{point}{Proof}%
Let~$p$ denote the infimum of $1\geq b\geq b^2 \geq b^4 \geq  \dotsb \geq 0$.
\begin{point}[vna-floor-point-1]%
If~$a\in \scrA$ commutes with~$b$,
then~$a$ commutes with~$p$.
Indeed, such~$a$ commutes with~$b^2$ (because
$ab^2 = bab = b^2a$,)
and so~$a$ commutes with~$b^{2^n}$ for each~$n$ by induction.
Thus~$a$ commutes with~$p\equiv\bigwedge_n b^{2^n}$ 
(by a variation on~\sref{vna-supremum-commutes}.)
\end{point}
\begin{point}%
To see that~$p$ is a projection, c.q.~$p^2=p$,
we only need to show that~$p\leq p^2$,
because we get $p^2\equiv \sqrt{p}\,p\,\sqrt{p}\leq p$
from $p\leq 1$ (using~\sref{ad-monotone}.)
Now, since
\begin{alignat*}{3}
p^2 \ &=\ \textstyle \bigwedge_m\  \sqrt{p}\, b^{2^m} \sqrt{p}\qquad
&&\text{by a variation on~\sref{ad-normal}}\\
&=\ \textstyle \bigwedge_m \ b^{2^{m-1}} p\, b^{2^{m-1}}\qquad
&&\text{by~\sref{vna-floor-point-1} and~\sref{cstar-square-commutes}}\\
&=\ \textstyle \bigwedge_m \bigwedge_n \ 
b^{2^{m-1}}\, b^{2^n}\, b^{2^{m-1}}\qquad
&&\text{by~\sref{ad-normal},}
\end{alignat*}
and $p\leq b^{2^{m-1}}\, b^{2^n}\,b^{2^{m-1}}$
for all~$n,m$, we get~$p\leq p^2$.
\end{point}
\end{point}
\end{parsec}



\end{document}
