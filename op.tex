\documentclass[main]{subfiles}
\begin{document}
\begin{parsec}%
\begin{point}{Definition}%
A \define{$C^*$-algebra}
is a complex vector space~$\scrA$
endowed with
\begin{enumerate}
\item
a binary operation,
called \define{multiplication}
(and denoted as such),
which is associative, and linear in both coordinates;
\item
an element~$1$, called \define{unit},
such that $1\cdot a = a = a\cdot 1$
for all~$a\in \scrA$;
\item
a unary operation $(\,\cdot\,)^*$,
called \define{involution},
such that $(a^*)^*=a$,
$(ab)^*=b^*a^*$,
$(\lambda a)^* = \bar\lambda a^*$,
and $(a+b)^* = a^*+b^*$
for all~$a,b\in\scrA$ and~$\lambda\in \C$;
\item
a complete \define{norm} $\|\,\cdot\,\|$
such that
$\|ab\|\leq\|a\|\|b\|$
for all~$a,b\in\scrA$,
and 
\begin{equation*}
\label{eq:Cstar-identity}
\|a^*a\|\ =\ \|a\|^2
\end{equation*}
holds, which is called the \define{$C^*$-identity}.
\end{enumerate}
The $C^*$-algebra $\scrA$ is called \define{commutative}
if $ab=ba$ for all~$a,b\in\scrA$.
\end{point}
\end{parsec}
\begin{parsec}%
\begin{point}{Examples}%
The set~$\C$ of \define{complex numbers}
forms a commutative  $C^*$-algebra
in which addition, (scalar) multiplication, and~$1$
have their usual meaning.
Involution is given by conjugation ($z^*=\bar{z}$),
and norm by modulus ($\|z\|=|z|$).
\begin{point}%
Let~$X$ be a compact Hausdorff space.
The set $\define{C(X)}$ of \define{continuous functions}
from~$\scrA$ to~$\C$
forms a commutative $C^*$-algebra
when addition, (scalar) multiplication, involution and~$1$ are
interpretted coordinatewise \grayed{(e.g.~$(f+g)(x)=f(x)+g(x)$)},
and the norm is taken to be 
$\|f\|=\sup_{x\in X} |f(x)|$
(the \define{sup-norm}).
\begin{point}%
The set~$\scrB(\scrH)$ of bounded operators
on a Hilbert space~$\scrH$ forms a $C^*$-algebra.
\end{point}
\end{point}
\end{point}
\end{parsec}
\begin{parsec}%
\begin{point}{Definition}
An element $a$ of a $C^*$-algebra $\scrA$ is called
\begin{enumerate}
\item \define{self-adjoint} if $a^* =a$;
\item \define{positive}
if $a\equiv b^*b$ for some $b\in \scrA$;
\item a \define{projection} if $a=a^*a$;
\item \define{central} if $ab=ba$ for all~$b\in\scrA$.
\end{enumerate}
\end{point}
\end{parsec}
%
% morphisms between C^*-algebras
%
\begin{parsec}
\begin{point}{Definition}
A linear map $f\colon \scrA \to \scrB$
between $C^*$-algebras
is called
\begin{enumerate}
\item
\define{\textbf{m}ultiplicative}
if $f(ab)=f(a)f(b)$ for all $a,b\in\scrA$;
\item
\define{\textbf{i}nvolutive}
if $f(a^*)=f(a)^*$ for all~$a\in\scrA$;
\item
\define{\textbf{p}ositive}
if $f(a)$ is positive
for every positive $a\in\scrA$, and
\item
\define{\textbf{u}nital}
if $f(1)=1$.
\end{enumerate}
\end{point}
\begin{point}%
We use the bold letters as abbreviations,
so for instance,
$f$ is \define{pu} if it is positive and unital,
and a \define{miu-map}
is a multiplicative, involutive, unital linear map between $C^*$-algebras,
(which is usually called a \define{unital $*$-homomorphism}.)
\end{point}
\end{parsec}

\begin{parsec}%
\begin{point}{Gelfand's Representation Theorem}%
Let~$\scrA$ be a commutative $C^*$-algebra.
\begin{enumerate}
\item
The \define{spectrum} of~$\scrA$, \define{$\spec(\scrA)$},
the set 
of miu-maps $\scrA\to \C$
endowed with the topology of pointwise convergence,
is a compact Hausdorff space.

\item
The \define{Gelfand transform},
the map $\gamma\colon \scrA\to C(\spec(\scrA))$
given by $\gamma(a)(\varphi)=\varphi(a)$,
is an isomorphism.
\end{enumerate}
\end{point}
\end{parsec}


\end{document}
