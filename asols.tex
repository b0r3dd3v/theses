\documentclass[b5page]{book}
\usepackage{cite}
\usepackage{xparse}
\usepackage{xcolor}
\usepackage{graphicx}
\usepackage{marginnote}
\usepackage{ifthen}
\usepackage{calc}
\usepackage{amssymb}
\usepackage{amsmath}
\usepackage{mathrsfs}
\usepackage{nicefrac}
\usepackage{sectsty}
\usepackage[all]{xypic}
\usepackage{hyperref} % <- should be last because it redefined \ref, etc..

\newcommand{\IGNORE}[1]{}

% macros concerning textstyle
\colorlet{darkblue}{blue!75!black}
\colorlet{lightblue}{blue!50!white}
\colorlet{lightgray}{gray!50!white}
\colorlet{darkgray}{gray!75!black}
\colorlet{darkred}{red!75!black}

\allsectionsfont{\sffamily\color{darkblue}}

\newcommand{\textPointHeaderI}[1]{\textcolor{darkblue}{\textbf{\textsf{#1}}}}
\newcommand{\textPointHeaderII}[1]{\textcolor{darkblue}{\textsf{#1}}}
\newcommand{\textPointHeaderIII}[1]{\textcolor{lightgray}{\textsf{#1}}}
\newcommand{\textParsecNumber}[1]{\textcolor{darkblue}{\textbf{\textsf{#1}}}}
\newcommand{\textPointNumberI}[1]{\textcolor{darkblue}{\textsf{#1}}}
\newcommand{\textPointNumberII}[1]{\textcolor{lightblue}{\textsf{#1}}}
\newcommand{\textPointNumberIII}[1]{\textcolor{lightgray}{\textsf{#1}}}
\newcommand{\textDefine}[1]{\textcolor{darkblue}{#1}}
\newcommand{\textSref}[1]{\textsf{#1}}
\newcommand{\textTodo}[1]{\textcolor{darkred}{#1}}

\newcommand{\TODO}[1]{\textTodo{\textbf{TODO}:\ #1}}
\newcommand{\Define}[1]{\textDefine{#1}}
\newcommand{\grayed}[1]{\textcolor{darkgray}{#1}}

\newcommand{\Cat}[1]{\mathbf{#1}}
\newcommand{\cW}[1]{\Cat{cW}^*_{\text{\textsc{#1}}}}
\newcommand{\W}[1]{\Cat{W}^*_{\text{\textsc{#1}}}}
\newcommand{\cCstar}[1]{\Cat{cC}^*_{\text{\textsc{#1}}}}
\newcommand{\Cstar}[1]{\Cat{C}^*_{\text{\textsc{#1}}}}
\newcommand{\CH}{\Cat{CH}}
\newcommand{\op}[1]{{#1}{^{\mathsf{op}}}}

\newcommand{\ketbra}[2]{\left|#1\right>\!\left<#2\right|}
\newcommand{\uleq}{\mathbin{\rotatebox[origin=c]{90}{$\leq$}}}


% Common symbols
\newcommand{\C}{\mathbb{C}}
\newcommand{\N}{\mathbb{N}}
\newcommand{\Z}{\mathbb{Z}}
\newcommand{\R}{\mathbb{R}}
\newcommand{\I}{\mathbb{I}}
\newcommand{\spec}{\mathrm{sp}}
\newcommand{\scrA}{\mathscr{A}}
\newcommand{\scrB}{\mathscr{B}}
\newcommand{\scrC}{\mathscr{C}}
\newcommand{\scrD}{\mathscr{D}}
\newcommand{\scrH}{\mathscr{H}}
\newcommand{\scrK}{\mathscr{K}}
\newcommand{\scrS}{\mathscr{S}}
\newcommand{\scrX}{\mathscr{X}}
\newcommand{\scrY}{\mathscr{Y}}
\newcommand{\scrZ}{\mathscr{Z}}

\newcommand{\ceil}[1]{\left\lceil#1\right\rceil}
\newcommand{\floor}[1]{\left\lfloor#1\right\rfloor}

\newcommand{\Real}[1]{#1_{\R}}
\newcommand{\Imag}[1]{#1_{\I}}
\newcommand{\sa}[1]{#1_{\R}}
\newcommand{\pos}[1]{#1_{+}}

\DeclareMathOperator{\dom}{dom}
\DeclareMathOperator{\length}{length}
\DeclareMathOperator{\interior}{in}
\DeclareMathOperator{\Stat}{Stat}
\DeclareMathOperator{\NStat}{NStat}
\DeclareMathOperator{\Cont}{C}
\DeclareMathOperator{\Colim}{Colim}
\DeclareMathOperator{\id}{id}
\DeclareMathOperator{\hZ}{hZ}
\DeclareMathOperator{\Proj}{Proj}


% typesetting theorems
\newcommand{\qed}{\hfill\textcolor{darkblue}{\ensuremath{\square}}}


\newcounter{tmp} % used for arithmetic

% The content of this thesis is grouped into numbered paragraphs,
% which are called "parsecs" (for paragraph--section),
% and these parsecs contain several points.
\newcounter{parsec} % keeps track of the current parsec number

% The first argument is the label this parsec will have--use \sref 
% 	to refer to a parsec.
\NewDocumentEnvironment{parsec}{o}{%
	\par\penalty-200\vskip1em\noindent%
	\refstepcounter{parsec}%
	%%\setcounter{point}{0}% - point is set to 0 by \numberwithin
	% In the footer of every odd page we list the parsecs present on 
	% the spread.  We pass this information to the footer via the 
	% \markboth,\leftmark,\rightmark-mechanism, which is normally
	% used to display the section and subsection names and numbers
	% in the header.
	% 	Recall that \leftmark will return the LAST value passed
	% to the first argument of \markboth on this page. (The difficulty
	% of implementing \leftmark is that a \markboth that will belong
	% to the next page can be called before the current page is shipped,
	% because this \markboth may be part of the text that overflows the
	% current page.)
	%	\rightmark will return the FIRST value passed to the second
	% argument of \markboth on this page.
	%	Since we would not only like to know if this parsec with number
	% say  N  is present on this spread, but also whether it spills over 
	% to the next spread (or has spilled over from the previous spread),
	% we keep track of whether the parsec started on this spread, 
	% encoded by  2N,  or whether the parsec ended on this spread,
	% encoded by  2N+1.
	\setcounter{tmp}{2*\value{parsec}}%
	\markboth{\the\value{tmp}}{\the\value{tmp}}%
	\label{#1}%
	% Display the parsec number in the margin.
	\marginnote{\makebox[3em][c]{\textParsecNumber{\the\value{parsec}}}}%
}{%
	\setcounter{tmp}{2*\value{parsec}+1}%
	\markboth{\the\value{tmp}}{\the\value{tmp}}%
}

\newcounter{point} % keeps track of the current point
\numberwithin{point}{parsec}
\renewcommand{\thepoint}{\theparsec\,\Roman{point}}
\newcounter{pointdepth}  % keeps track of the depth of the current point
			 % --- points may be nested.

\NewDocumentEnvironment{point}{o g}{%
	\setcounter{pointdepth}{\value{pointdepth}+1}%
	\refstepcounter{point}%
	\IfValueT{#1}{\label{#1}}%
	\ifthenelse{\equal{\value{point}}{1}}{}{%
		\ifthenelse{\equal{\value{pointdepth}}{1}}{%
			\par\penalty-100\vskip.5em\noindent%
		}{%
			\ifthenelse{\equal{\value{pointdepth}}{2}}{%
				\par\penalty-50\vskip.3em\noindent%
			}{%
				\par\penalty-25\vskip.2em\noindent%
			}%
		}%
		\marginnote{\makebox[2em][c]{%
			\ifthenelse{\equal{\value{pointdepth}}{1}}{%
				\textPointNumberI{\Roman{point}}%
			}{%
				\ifthenelse{\equal{\value{pointdepth}}{2}}{%
					\textPointNumberII{\Roman{point}}%
				}{%
					\textPointNumberIII{\Roman{point}}%
				}%
		}}}%
	}%
	\IfValueT{#2}{%
		\ifthenelse{\equal{\value{pointdepth}}{1}}{%
			\textPointHeaderI{#2}%
		}{%
			\ifthenelse{\equal{\value{pointdepth}}{2}}{%
				\textPointHeaderII{#2}%
			}{%
				\textPointHeaderIII{(#2)}%
			}}%
	\ \ }%	
}{%
\setcounter{pointdepth}{\value{pointdepth}-1}%
}

% Refer to a parsec.
\NewDocumentCommand{\sref}{m}{\textSref{\ref{#1}}}

% Adjust footer and header:
\usepackage{fancyhdr}
\pagestyle{fancy}
\renewcommand{\headrulewidth}{0pt} % we want no header line

% Since we use \markboth,\leftmark,\rightmark to keep track of the parsecs
% on a given spread, we should neutralize its old user:
\renewcommand{\chaptermark}[1]{}  
\renewcommand{\sectionmark}[1]{}

\fancyhead{}

% These counters are used for computation
\newcounter{firstParsec}
\newcounter{lastParsec}
\newcounter{firstParsecF}
\newcounter{lastParsecF}

% parsecToBeContinued is 1 if the previous spread spilled a parsec,
% and 0 otherwise.
\newcounter{parsecToBeContinued}  
\setcounter{parsecToBeContinued}{0}

% Set the footer.  It contains the parsecs on this page.
% We use "\rightmark+0", because \rightmark might be empty.
\fancyfoot[CE]{\setcounter{firstParsecF}{\rightmark+0}}
\fancyfoot[CO]{%
% firstParsecF is already set by the even page that came before
\setcounter{firstParsec}{\value{firstParsecF}/2}%
\setcounter{lastParsecF}{\leftmark+0}%
\ifthenelse{\equal{\value{lastParsecF}}{0}}%
{}% do nothing
{%
\setcounter{lastParsec}{\value{lastParsecF}/2}%
\textPointNumberI{%
\ifthenelse{\equal{\value{parsecToBeContinued}}{1}}{..}{}%
\the\value{firstParsec}%
\ifthenelse{\equal{\value{firstParsec}}{\value{lastParsec}}}%
	{}{%  no footer without parsecs
\setcounter{tmp}{\value{firstParsec}+1}%
\ifthenelse{\equal{\value{tmp}}{\value{lastParsec}}}{, }{--}%
\the\value{lastParsec}}%
\setcounter{tmp}{\value{lastParsec}*2}%
\ifthenelse{\equal{\value{tmp}}{\value{lastParsecF}}}%
{..\setcounter{parsecToBeContinued}{1}}%
{\setcounter{parsecToBeContinued}{0}}%
}% \textPointNumberI{
}% \ifthenelse{
}% \fancyfoot[CO]{


\externaldocument{a}
\externaldocument{b}

\begin{document}
\appendix
\chapter{Solutions, Addenda, Errata}
\begin{erratum}{parsec-40.160}%
    It should be assumed that $T$ is bounded.
\end{erratum}
\begin{erratum}{parsec-50.60}%
Don't follow the hints.
\end{erratum}
\begin{erratum}{parsec-60.30}%
    ``$\|x_\infty-x\|\leq \varepsilon$'' should read  ``$\|x_\infty-x_n\|\leq\varepsilon$''.
\end{erratum}
\begin{erratum}{parsec-110.60}
``$\|a\| < \|b\|$'' in point~2
    should read ``$\|a\|<  \|b^{-1}\|^{-1}$''.
\end{erratum}
\begin{erratum}{parsec-110.150}
In the hint for point 3,
``$a^n+1 = \prod_{k=1}^n a+\zeta^{2k+1}$''
should read
``$a^n+1 = \prod_{k=1}^n a-\zeta^{2k+1}$''.
\end{erratum}



\begin{solution}{parsec-40.30}%
Let~$\scrX$, $\scrY$ and~$\scrZ$ be normed (complex) vector spaces.
\begin{enumerate}
\item
To show that the operator norm does indeed
give a norm on~$\scrB(\scrX,\scrY)$,
the following
three observations
regarding $S,T\in\scrB(\scrX,\scrY)$
suffice.
\begin{enumerate}
\item
We have $\|S+T\|\leq \|S\|+\|T\|$.

To see this,
note that
given~$x\in \scrX$,
we have $\|Sx\|\leq \|S\|\|x\|$
and $\|Tx\|\leq \|T\|\|x\|$---because $\|S\|$ and $\|T\|$ are bounds
for~$S$ and~$T$, respectively,---and so 
\begin{equation*}
    \|(S+T)x\|
\ \leq\  \|Sx\|+\|Tx\|
\ \leq\  (\|S\|+\|T\|)\|x\|.
\end{equation*}
Thus~$\|S\|+\|T\|$ is a bound for~$S+T$.
Since~$\|S+T\|$ is the least bound for~$S+T$,
we get~$\|S+T\|\leq \|S\|+\|T\|$.

\item
We have
$\|\lambda S\|=\left|\lambda \right| \|S\|$
for any $\lambda \in \C$.

Surely this statement is correct when~$\lambda= 0$.
To see why it's correct for~$\lambda \neq 0$,
note that
$r\mapsto \left|\lambda\right| r$
sends bounds of~$S$ to bounds of~$\lambda S$.
Indeed,
if~$r\in[0,\infty)$ is a bound of~$S$,
then $\|Sx\|\leq r\|x\|$
for all~$x\in \scrX$,
and so~$\|\lambda Sx\| \equiv 
\left|\lambda \right|\|Sx\|\leq \left|\lambda\right| r \|x\|$
for all~$x\in\scrX$,
that is, $\left|\lambda\right|r$
is a bound for~$\lambda S$.
Similarly~$r\mapsto \left|\lambda\right|^{-1} S$
sends bounds of $\lambda S$
to bounds of~$S$.
Since the two aforementioned maps are each other's inverse,
and both are order preserving,
$r\mapsto \left|\lambda \right|r$
gives an order isomorphism from 
the bounds of~$S$ to the bounds of~$\lambda S$,
and, in particular,
    sends the least bound of~$S$ (being $\|S\|$)
to the least bound of~$\lambda S$
        (being~$\|\lambda S\|$),
that is,
$\left|\lambda\right|\|S\|=\|\lambda S\|$.
\item
We have $\|S\|=0$ iff~$S=0$.

Indeed, the following are equivalent:
$\|S\|=0$; the number~$0$ is a bound for~$S$;
$\|Sx\|\leq 0$ for all~$x\in\scrX$;
$Sx=0$ for all~$x\in\scrX$;
$S=0$.
\end{enumerate}
\item
Since $\|STx\|\leq \|S\| \|Tx\|
\leq \|S\|\|T\|\|x\|$
for all~$x\in \scrX$
the operator $ST$ is bounded by
$\|S\|\|T\|$.
\item
Indeed, $ \|\id x \|\equiv \|x\|\leq  1\cdot\|x\|$
for all~$x\in\scrX$.
\end{enumerate}
\end{solution}
\begin{solution}{parsec-40.40}%
Recall that we must prove that
    \begin{equation*}
r\|T\|\ = \ \sup_{x\in (\scrX)_r} \|Tx\|.
    \end{equation*}
Since $(\scrX)_r \equiv \{x\in\scrX\colon \|x\|\leq r\}
    = \{rx\colon x\in (\scrX)_1\}$
    the problem becomes
\begin{equation*}
r\|T\|\ = \ \sup_{x\in (\scrX)_1} \|Trx\|
    \,\equiv\, r\sup_{x\in (\scrX)_1} \|Tx\|.
\end{equation*}
So we've reduced the 
problem to the case~$r=1$, that is, to showing that
\begin{equation*}
    \|T\|\ = \ \sup_{x\in (\scrX)_1} \|Tx\|.
\end{equation*}
Since for~$x\in (\scrX)_1$
we have $\|x\|\leq 1$,
and so~$\|Tx\|\leq \|T\|\,\|x\|\leq \|T\|$,
we see that~$\sup_{x\in (\scrX)_1} \|Tx\|\leq \|T\|$.
For the other direction,
    $\|T\|\leq \sup_{x\in (\scrX)_1} \|Tx\|$,
    it suffices to show that
 $r:=\sup_{x\in(\scrX)_1} \|Tx\|$
is a bound for~$T$,
that is, we must show given~$x\in\scrX$
that $\|Tx\|\leq r\|x\|$.
When~$x=0$ this is obvious, so we'll assume that~$x\neq 0$.
Then~$\| \,x\|x\|^{-1}\,\|\leq 1$,
    so $\|x\|^{-1} \|Tx\|\equiv \|\,T x \|x\|^{-1}\, \| \leq r$,
    and thus $\|Tx\|\leq r\|x\|$,
    making $r$ is a bound for~$T$.
\end{solution}
\begin{solution}{parsec-40.100}%
Taking~$y:=x-x'$ we see that
$\|y\|^2 = \left<y,y\right>
= \left<y,x\right> - \left<y,x'\right> = 0$,
so~$\|y\|=0$,
so~$y\equiv x-x'=0$,
and thus~$x=x'$.

For the second part,
let~$T_1,T_2\colon \scrH\to\scrK$
be operators between pre-Hilbert spaces~$\scrH$
and~$\scrK$
that are both adjoint to an operator~$S\colon \scrK\to\scrH$.
We must show that~$T_1=T_2$.
Given $x\in\scrH$ we have
\begin{equation*}
\left<T_1x,y\right>
\ = \ 
\left<x,Sy\right>
\ = \ 
\left<T_2x,y\right>
\qquad 
    \text{for all~$y\in\scrK$,}
\end{equation*}
and so~$T_1x=T_2x$ by the previous part.
Whence~$T_1=T_2$.
\end{solution}
\begin{solution}{parsec-40.120}%
We prove slightly more than was requested.
Let~$\scrH$, $\scrK$ and~$\scrL$ be pre-Hilbert spaces.
\begin{enumerate}
\item
Let~$T\colon \scrH\to\scrK$ be an adjointable operator,
        so we know that~$T$ is adjoint to some (by~\sref{parsec-40.100}
        unique) operator~$T^*\colon \scrK\to\scrH$.
We must show that $T^*$ is adjoint to~$T$ too.

Let~$x\in\scrH$ and~$y\in\scrK$ be given.
Since~$T$ is adjoint to~$T^*$,
we know that~$\left<Tx,y\right>=\left<x,T^*y\right>$,
and so taking the complex conjugate, we get
        \begin{equation*}
\left<y,Tx\right>\  =\  
    \overline{\left<Tx,y\right>}
    \ =\ 
    \overline{\left<x,T^*y\right>}
    \ =\ \left<T^*y,x\right>,
\end{equation*}
and see that~$T^*$ is adjoint to~$T$.
        Thus~$T^{**}=T$ by definition, \sref{parsec-40.80}.

\item
Given adjointable operators
$S,T\colon \scrH\to\scrK$ 
and~$x\in \scrH$ and $y\in \scrK$,
\begin{equation*}
\left<(S+T)x,y\right>
\ =\  \left<Sx,y\right>\,+\,\left<Tx,y\right>
\ =\ \left<x,S^*y\right>\,+\,\left<x,T^*y\right>
    \ =\ \left<x,(S^*+T^*)y\right>,
\end{equation*}
so~$S+T$ is adjoint to~$S^*+T^*$,
and thus~$(S+T)^*=S^*+T^*$.

Given an adjointable operator~$S\colon \scrH\to\scrK$,
$\lambda\in\C$, $x\in\scrH$, and~$y\in\scrK$,
\begin{equation*}
\left<\lambda Sx,y\right>
\ = \ 
    \overline{\lambda} \left<Sx, y\right>
\ = \ 
    \overline{\lambda} \left<x, S^*y\right>
\ = \ 
     \left<x, \overline{\lambda}S^*y\right>,
\end{equation*}
so~$\lambda S$ is adjoint to $\overline{\lambda}S^*$,
and thus $(\lambda S)^*=\overline{\lambda}S^*$.
\item
Given adjointable operators
$S\colon \scrK\to\scrL$
and $T\colon \scrH\to\scrK$,
and
$x\in\scrH$  and $y\in\scrL$,
we have
\begin{equation*}
\left<STx,y\right>
\ = \ \left<Tx,S^*y\right>
\ = \ \left<x,T^*S^*y\right>,
\end{equation*}
so~$ST$ is adjoint to~$T^*S^*$,
        and thus~$(ST)^*=T^*S^*$.
\end{enumerate}
\end{solution}
\begin{solution}{parsec-40.150}%
We'll check the first two requirements
    for $\|x\|=\smash{\sqrt{\left<x,x\right>}}$
    to give a seminorm first.
Given $x\in V$,
we have $\left<x,x\right>\geq 0$,
    so~$\|x\|\equiv \smash{\sqrt{\left<x,x\right>}}\geq 0$.
Given $x\in V$ and $\lambda\in \C$,
we have $\left\|\lambda x\right\|^2
    = \left<\lambda x,\lambda x\right> = \overline{\lambda}\left<x,x\right>
    \lambda = \left|\lambda\right|^2\|x\|^2\geq 0$,
    so
    we get~$\|\lambda x\| = \left|\lambda\right|\|x\|$
    by taking the square root.

Verifying that the triangle inequality holds takes some 
preparations.
As the hint suggests, we prove the Cauchy--Schwarz inequality first.
For this, we must given $x,y\in V$  prove
    that $\left|\left<x,y\right>\right|^2 \leq \left<x,x\right>
    \left<y,y\right>$.
This inequality follows
by applying~\sref{positive-2x2matrix} 
to the matrix
$\smash{\bigl(\begin{smallmatrix}
\smash{\left<x,x\right>} & \smash{\left<x,y\right>} \\
\smash{\left<y,x\right>} & \smash{\left<y,y\right>}
\end{smallmatrix}\bigr)}$,
which is positive,
as for all~$u,v\in\C$,
    \begin{alignat*}{3}
        (
        \begin{smallmatrix}
            \bar{u} & \bar{v}
        \end{smallmatrix})
        \ 
        \bigl(
        \begin{smallmatrix}
\smash{\left<x,x\right>} & \smash{\left<x,y\right>} \\
\smash{\left<y,x\right>} & \smash{\left<y,y\right>}
\end{smallmatrix}\bigr)
        \ \bigl(
        \begin{smallmatrix}
            u \\ v
        \end{smallmatrix}
        \bigr)
        \ &=\  
        \bar{u}u\left<x,x\right>\,+\,
        \bar{u}v\left<x,y\right> \,+\,
        u\bar{v}\left<y,x\right> \,+\,
        \bar{v}v\left<y,y\right> 
        \\
        &=\ \left<\,ux+vy,\,ux+vy\,\right>\,\geq \,0.
    \end{alignat*}
Upon taking the square root
the Cauchy--Schwarz inequality
takes the forms
    \begin{equation*}
    \left|\left<x,y\right>\right|
    \ \leq\  \|x\|\,\|y\|\qquad\text{for all }x,y\in V.
    \end{equation*}
Next, note that~$\bar{z}+z \leq 2\left|z\right|$ for any~$z\in\C$,
    because writing $z\equiv a+ib$,
    we have $\left|\bar{z}+z\right|^2=(\bar{z}+z)^2 = 4a^2 \leq 4(a^2 + b^2) = 4\left|z\right|^2$,
    and so $\bar{z}+z \leq \left|\bar{z}+z\right|\leq 2\left|z\right|$.
In particular, we have,
for all~$x,y\in V$,
\begin{equation*}
\left<x,y\right>+\left<y,x\right>
\ \leq\  2\left|\left<x,y\right>\right|\ \leq\ 
2 \|x\|\,\|y\|.
\end{equation*}
Given $x,y\in V$,
the triangle inequality,
$\|x+y\|\leq \|x\|+\|y\|$,
holds,
because
\begin{alignat*}{3}
    \|x+y\|^2\ &=\ 
    \left<x+y,x+y\right> 
    \ =\ \left<x,x\right>\,+\,
    \left<x,y\right>\,+\,
    \left<y,x\right>\,+\,
    \left<y,y\right> \\
    &\leq \ \left<x,x\right>\,+\,
    2\|x\|\,\|y\|
    \,+\,
    \left<y,y\right>  
    \quad\equiv\ 
    \|x\|^2 \,+\, 2\|x\|\,\|y\| \,+\, \|y\|^2\\
    \ &=\   (\|x\|+\|y\|)^2.
\end{alignat*}

Whence~$\|x\|=\smash{\sqrt{\left<x,x\right>}}$ gives
a seminorm on~$V$.
Recall that for~$\|\,\cdot\,\|$ to be a \emph{norm},
the additional condition
$\|x\|=0\implies x=0$
for all~$x\in V$
must be met.
This is the case when
$\left<\,\cdot\,,\,\cdot\,\right>$
is definite,
because 
$\|x\|=0$
entails $0=\|x\|^2=\left<x,x\right>$,
which, by definiteness of~$\left<\,\cdot\,,\,\cdot\,\right>$,
entails $x=0$.

Let~$x,y\in V$ be given.
To complete this exercise we must establish three
identities.
\emph{Pythagoras' theorem}
holds since $\left<x,y\right>=0$ implies
\begin{alignat*}{3}
\|x+y\|^2\ \equiv  \ \left<x+y,x+y\right>
    \ &=\ 
\left<x,x\right>
\,+\, \left<x,y\right>
\,+\, \left<y,x\right>
\,+\, \left<y,y\right>\\
    \ &=\  
\left<x,x\right>\,+\,\left<y,y\right>
\ = \ \|x\|^2 + \|y\|^2.
\end{alignat*}
Similar to the first line of the display above, we have,
(for any~$x,y\in V$,)
\begin{equation*}
\|x-y\|^2\ =\ 
\left<x,x\right>
\,-\, \left<x,y\right>
\,-\, \left<y,x\right>
\,+\, \left<y,y\right>.
\end{equation*}
Taking the average of these two equations gives the \emph{parallelogram law}:
\begin{equation*}
    \textstyle \frac{1}{2}(\ \|x+y\|^2\,+\,\|x-y\|^2\ )
    \ =\ 
\left<x,x\right>
\,+\, \left<y,y\right>\ \equiv\ 
\|x\|^2 + \|y\|^2.
\end{equation*}
Concerning the \emph{polarisation identity},
note that for any~$n$, we have
\begin{equation*}
i^n\left<i^nx+y,i^nx+y\right>
\ = \ i^n\left<x,x\right>
\,+\, \left<x,y\right>
    \,+\, (-1)^n\left<y,x\right>
    \,+\, i^n\left<y,y\right>.
\end{equation*}
Since~$\sum_{n=0}^3 i^n = 0$
and $\sum_{n=0}^3 (-1)^n=0$,
we get 
\begin{equation*}
    \textstyle
    \sum_{n=0}^3i^n\left\|i^nx+y\right\|^2
\ = \ 4\left<x,y\right>.
\end{equation*}
\end{solution}
\begin{solution}{parsec-40.180}%
Let~$\scrS \subseteq \scrB(\scrH)$
denote the set of bounded adjointable operators $\scrH\to\scrH$.
Since by Exercise~\sref{parsec-40.120},
a linear combination
of adjointable operators is again adjointable,
$\scrS$ is a linear subspace of~$\scrB(\scrH)$.

To show that~$\scrS$ is a closed subset of~$\scrB(\scrH)$,
we must prove that 
the limit~$T$ in $\scrB(\scrH)$
of
a
sequence  $T_1,T_2,\dotsc$
in~$\scrS$
is adjointable.

Note that the sequence $T_1^*,\,T_2^*,\,\dotsc$
is Cauchy, because,
by~\sref{parsec-40.160}, for all~$n,m$,
    \begin{equation*}
    \|T_n^*-T_m^*\|\ =\ \|(T_n-T_m)^*\|\ =\ \|T_n-T_m\|.
    \end{equation*}
Since~$\scrB(\scrH)$ is complete
by~\sref{parsec-40.50},
we may define $S:=\lim_n T_n^*$.
We claim that~$T$ is adjoint to~$S$.
Let~$x,y\in \scrH$ be given.
To prove our claim,
we must show that
that~$\left<Tx,y\right> = \left<x,Sy\right>$.

Note that $T_n^*y$ converges to~$Sy$ as~$n\to\infty$,
because~$\|T_n^*y-Sy\|\leq \|T_n^*-S\|\,\|x\|$
and $\|T_n^*-S\|$ vanishes as~$n$ increases
    by definition of~$S$.
As a result,
$\lim_n \left<x,T_n^*y\right> = \left<x,Sy\right>$.
Note that we use here
that the map $\left<x,\,\cdot\,\right>$,
being bounded by Cauchy--Schwarz
(\sref{parsec-40.150}),
is continuous.
On the other hand,
by a similar reasoning,
$\left<x,T_n^*y\right>
\equiv \left<T_nx,y\right>$
converges to~$\left<Tx,y\right>$ too
as~$n$ increases,
and so~$\left<x,Sy\right>=\left<Tx,y\right>$.
Whence~$T$ is adjoint to~$S$,
and thus~$\scrS$ is closed.
\end{solution}
\begin{solution}{parsec-40.190}%
That~$\ketbra{x}{y}\equiv \left<y,\,\cdot\,\right>x\colon \scrH\to\scrH$ 
is linear is pretty obvious.
    
    Since $\|\,\ketbra{x}{y}z\,\|
    = \|\left<y,z\right>x\|
    =\left|\left<y,z\right>\right|\,\|x\|
    \leq \|y\|\,\|z\|\,\|x\|$
    for all~$z\in\scrH$,
    we see that~$\ketbra{x}{y}$
    is bounded by~$\|y\|\,\|x\|$,
    and so~$\|\ketbra{x}{y}\|\leq \|x\|\,\|y\|$.
    For the other direction,
    note that $\|\ketbra{x}{y}\| \,\|y\|\geq 
    \|\ketbra{x}{y}y\|=\|\left<y,y\right>x\|
    = \|y\|^2\,\|x\|$,
    and so~$\|\ketbra{x}{y}\|\geq \|y\|\,\|x\|$.
    (Even when~$\|y\|=0$.)
    Whence $\|\ketbra{x}{y}\|=\|x\|\,\|y\|$.

Finally,  $\ketbra{x}{y}$
    is adjoint~$\ketbra{y}{x}$,
    because, for all~$w,z\in\scrH$,
    \begin{equation*}
        \left<\,\ketbra{x}{y}z,\, w\,\right>
        \ = \ 
        \left<z,y\right>\,\left<x,w\right>
        \ = \ 
        \left<\, z,\,(\ketbra{y}{x}w)\,\right>.
    \end{equation*}
\end{solution}
\begin{solution}{parsec-50.30}%
Let~$x$ be an element of~$\ell^2\backslash c_{00}$
    (so $x_n$ is non-zero for infinitely many~$n$s.)
We must show that there are no element of~$c_{00}$ with minimal
distance to~$x$.
So let~$y$ be an element of~$c_{00}$,
and suppose (towards a contradiction)
that~$\|x-y\|\leq \|x-y'\|$
for all~$y'\in c_{00}$.

Since~$y$ is in $c_{00}$,
there's~$N$ such that~$y_n=0$ for all~$n> N$.
We claim that~$x_n=y_n$ for all~$n\leq N$.
    Indeed, (if not) define $y'\in c_{00}$
    by $y_n' = x_n$ for all~$n\leq 0$, and~$y_n'=0$ for all~$n >N$.
    Then
\begin{alignat*}{3}
\textstyle
    \|x-y'\|^2\ &=\ 
\textstyle
    \sum_{n=N+1}^\infty \left|x_n\right|^2\\
    \ &\leq \ 
\textstyle
    \sum_{n=0}^N \left|x_n-y_n\right|^2 
    \ +\ \sum_{n=N+1}^\infty \left|x_n\right|^2
    \ = \ \|x-y\|^2.
\end{alignat*}
Thus~$\|x-y'\|\leq \|x-y\|$.
But since~$y$ is assumed to have minimal
distance to~$x$ among the elements of~$c_{00}$,
we already had $\|x-y\|\leq \|x-y'\|$,
and so~$\|x-y\|=\|x-y'\|$.
    Whence~$\sum_{n=0}^N \left|x_n-y_n\right|^2
    = 0$,
    and so~$x_n=y_n$ for all~$n\leq N$.
    In particular, $y=y'$.

It's now easy to find a better approximation
    of~$x$ in~$c_{00}$ than~$y$.
    Indeed, since~$x_n$ is non-zero for infinitely many~$n$s,
    there's an~$M\geq 0$ with 
    \begin{equation*}
        \textstyle
        \sum_{n=M+1}^\infty \left|x_n\right|^2
    \ <\  \sum_{n=N+1}^\infty \left|x_n\right|^2.
    \end{equation*}
    Writing $y''$ for the element of~$c_{00}$
    with $y_n''=x_n$ for all~$n\leq M$ and~$y_n''=0$
    for all~$n > M$,
    we have 
    $\|x-y''\|^2=\sum_{n=M+1}^\infty \left|x_n\right|^2
    < \sum_{n=N+1}^\infty \left|x_n\right|^2 = \|x-y'\|^2 = \|x-y\|^2$,
    which contradicts $\|x-y\|\leq \|x-y''\|$.
    Hence no such~$y$ exists.
\end{solution}
\begin{solution}{parsec-50.60}%
Surely, if $y$ is closest to~$x$
among all elements of~$C$,
it is among all elements of~$y\C\subseteq C$.
Whence~$y$ is a projection of~$x$ on~$y\C$.

We claim that~$\|y\|^2 = \left<x,y\right>$.  When~$y=0$
    this is obvious, so we may assume that~$y\neq 0$,
    so that we can define $e:= y\|y\|^{-1}$.
Since~$y$ is the projection of~$x$ on~$y\C\equiv e\C$,
we know by~\sref{parsec-50.40} that
$y=\left<e,x\right>e$,
and so $\|y\|^2 y = \left<y,x\right>y$.
Whence~$\|y\|^2 = \left<y,x\right>$.
Note that the claim implies that $\left<x-y,y\right>=0$:
\begin{equation*}
0\ =\ \|y\|^2-\left<x,y\right> \ =\  \left<y,y\right>-\left<x,y\right>
\ =\  \left<y,x-y\right>.
\end{equation*}
Let~$c\in C$ be given.
Note that when~$y_1$ is a projection of
    $x$ on~$C$, then~$y_1+c$ is a projection
of~$x+c$ on~$C$, because
    $\|(y_1+c)-(x+c)\|=\|y_1-x\|\leq \|(y'-c)-x\|\equiv \|y'-(x+c)\|$
    for all~$y'\in C$.

Let~$y_1$ and~$y_2$ be projections of~$x$ on~$C$;
we will show that~$y_1=y_2$.
By the previous paragraph,
$0\equiv y_1-y_1$ and~$y_2-y_1$ are projections of~$x-y_1$ on~$C$,
and thus on~$y_1\C$.
But since there's at most one projection of~$x-y_1$ on~$y_1\C$
    by~\sref{parsec-50.40} (even when~$y=0$),
    we get~$0=y_2-y_1$, and so~$y_1=y_2$.

Let~$y'\in C$ be given.
Recall that~$y$ is a (and thus the unique) projection of~$x$
on~$C$. It remains to be shown that
$\left<y',x-y\right>=0$.
Since~$y' \equiv y+y'-y$ is a projection
of $x':=x+y'-y$ on~$C$,
    we get 
\begin{equation*}
0\ =\ \left<y',x'-y'\right>\ \equiv \ 
    \left<y',(x+y'-y)-y'\right>\ \equiv\  \left<y',x-y\right>.
\end{equation*}
\end{solution}
\begin{solution}{parsec-50.110}%
Let~$T\colon \scrH\to\scrK$
be a bounded linear map between Hilbert spaces
$\scrH$ and~$\scrK$.
We'll show that~$T$ is adjointable.

Let~$y \in \scrK$ be given. 
    Then~$\left<y,T(\,\cdot\,)\right>\colon \scrH\to\C$
    is a linear map,
    bounded by $\|y\|\|T\|$,
    because $\left|\left<y,Tx\right>\right|\leq
    \|y\|\, \|Tx\|\leq \|y\|\,\|T\|\,\|x\|$
    for all~$x\in \scrH$.
Thus $\left<y,T(\,\cdot\,)\right>\equiv \left<Sy,\,\cdot\,\right>$
    for some unique $Sy\in \scrH$ by~\sref{parsec-50.90}.

The resulting map~$S\colon \scrK\to\scrH$
is linear, because
    \begin{alignat*}{3}
\left<S(y_1+\lambda y_2),\,\cdot\,\right>
        &=\  \left< y_1+\lambda y_2, T(\,\cdot\,)\right>
    \ =\  \left<y_1,T(\,\cdot\,)\right> + 
    \bar\lambda\left<y_2,T(\,\cdot\,)\right>\\
        \ &=\  \left<Sy_1,\,\cdot\,\right> + 
    \bar\lambda\left<Sy_2,\,\cdot\,\right>
    \ =\  \left<(Sy_1+\lambda Sy_2),\,\cdot\,\right>,
\end{alignat*}
for all $y_1,y_2\in\scrK$ and~$\lambda\in\C$.
Since $\left<Sy,x\right>=\left<y,Tx\right>$
for all~$x\in\scrH$ and~$y\in\scrK$,
 $S$ is adjoint to~$T$ (and~$T$ is adjoint to~$S$).

Finally, note that~$S$ is bounded by~$\|T\|$,
because
$\|Sy\|^2 = \left<Sy,Sy\right>
= \left<y,TSy\right>
\leq \|y\|\,\|T\|\,\|Sy\|$,
and so~$\|Sy\|\leq \|y\|\|T\|$,
for all~$y\in\scrK$.
\end{solution}
\begin{solution}{parsec-70.30}
Let~$a$ be an element of a $C^*$-algebra~$\scrA$.
\begin{enumerate}
    \item[1.]
$\Real{a}$ is self-adjoint, because
$(\Real{a})^*
        = \frac{1}{2}(a^{*}+a^{**}) = \frac{1}{2}(a^*+a) = \Real{a}$.
To see that~$\Imag{a}$ is self-adjoint, recall
        that $\bar{i}=-i$,
so
        $(\Imag{a})^*
        = -\frac{1}{2i}(a^*-a) 
        = \Imag{a}$.

The identity $a=\Real{a}+i\Imag{a}$
follows from
        $2(\Real{a}+i\Imag{a})
        = a+a^* \,+\, (a-a^*)
        = 2a$.
    \item[2.]
    Since $a^*=b^*+\bar{i}c^* =b-ic$,
        we have $2\Real{a}=a+a^* = b+ic\,+\,b-ic = 2b$,
        and $2i\Imag{a} = a-a^* = b+ic \,-\, (b-ic) = 2ic$,
        so~$\Real{a}=b$, and~$\Imag{a}=c$.

\item[3.]
Since $a^* = (\Real{a}+i\Imag{a})^*
        = \Real{a} - i\Imag{a}$ by point~1,
        we get $\Real{(a^*)}=\Real{a}$
        and $\Imag{(a^*)} = -\Imag{a}$ by point~2.

\item[4.]
Indeed,
$a$ is self-adjoint iff~$a=a^*$
        iff $a+a^*=2a$
        iff $\Real{a} \equiv \frac{1}{2}(a+a^*) = a$
        iff $a-a^* = 0$
        iff $\Imag{a}\equiv \frac{1}{2i}(a-a^*) = 0$.

\item[5.]
    $\Real{(\,\cdot\,)}$
        and $\Imag{(\,\cdot\,)}$
        are $\R$-linear,
        because~$(\,\cdot\,)^*$
        is $\R$-linear.

\item[6.]
    Apply point~2 to
    $ia = i(\Real{a}+i\Imag{a}) = \Imag{a}-i\Real{a}$.

\item[7.]
The element
$a^*a$ is self-adjoint,
        because $(a^*a)^*=a^*a^{**}=a^*a$.

Further,
        $a^*a = (\Real{a}-i\Imag{a})(\Real{a}+i\Imag{a})
        = \Real{a}^2 + \Imag{a}^2 + i(\Real{a}\Imag{a}-\Imag{a}\Real{a})$.

\item[8.]
It suffices to find self-adjoint elements~$b$ and~$c$
of some~$C^*$-algebra~$\scrA$ with~$bc\neq cb$,
because then~$a:= b+ic$ will do the job.

Given any linearly independent vectors
$x$ and~$y$
from some Hilbert space~$\scrH$
with $\left<x,y\right>\neq 0$.
define~$b:=\ketbra{x}{x}$ and $c:=\ketbra{y}{y}$.
Then~$bc = \ketbra{x}{y}\,\left<x,y\right> $
and~$cb=\ketbra{y}{x}\,\left<y,x\right>$.
So if~$bc=cb$,
then~$\left|\left<x,y\right>\right|^2x = bcx
= cbx = \left<y,x\right>\|x\|^2y$,
contradicting the linear independence of~$x$ and~$y$.

\item[9.]
Combine point~3 and point~7.

\item[10.]
Indeed, $bc$ is self-adjoint
iff~$bc=(bc)^*\equiv c^*b^*\equiv cb$.

So~$x$ and~$y$ are non-orthogonal linearly independent vectors
of a Hilbert space~$\scrH$ as in point~8,
then~$\ketbra{x}{x}\ketbra{y}{y}\equiv \left<x,y\right>\,\ketbra{x}{y}$
is not self-adjoint.

\item[11.]
Surely, if~$a=0$,
then~$a^*=0$, and so~$\|a\|=0=\|a^*\|$.
So we may assume that~$a\neq 0$
(and so~$a^*\neq 0$).
Then, since~$\|a\|^2=\|a^*a\|\leq \|a^*\|\,\|a\|$,
we have $\|a\|\leq \|a^*\|$.
Since similarly $\|a^*\|\leq \|a\|$,
we get~$\|a\|=\|a^*\|$.

\item[12.]
Note that
$\|\Real{a}\|\leq
\frac{1}{2}\|a\|+\frac{1}{2}\|a^*\|
= \|a\|$
and $\|\Imag{a}\|\leq
\frac{1}{2}\|a\| + \frac{1}{2}\|a^*\|
= \|a\|$
by the triangle inequality and $\|a^*\|=\|a\|$.

\item[13.]
When~$a$ is self-adjoint,
$\|a^2\| = \|a^*a\| = \|a\|^2$.

Let~$x$ and~$y$ be non-zero orthogonal vectors
from some Hilbert space~$\scrH$.
Then~$\ketbra{x}{y}^2 = 0$, and so~$\|\ketbra{x}{y}^2\|=0$,
while $\|\ketbra{x}{y}\|^2 = \|x\|^2\|y\|^2$
(see~\sref{parsec-40.190}) is non-zero.
\end{enumerate}
\end{solution}
\begin{solution}{parsec-80.20}
For the sake of clarity,
we'll denote the zero and unit of a $C^*$-algebra~$\scrA$
in this exercise
by~$0_\scrA$ and~$1_\scrA$, respectively.
\begin{enumerate}
\item
Since~$0_\scrA=1_\scrA$ when~$\scrA=\{0_\scrA\}$,
then also $0_\C=\|0_\scrA\|=\|1_\scrA\|$.
\item
We claim that~$\|1_\scrA\|\leq 1_\C$.
Indeed, since $\|1_\scrA\|=\|1_\scrA^*1_\scrA\|=\|1_\scrA\|^2$,
we either have $\|1_\scrA\|=1_\C$
        or $\|1_\scrA\|=0_\C$.
        In either case, $\|1_\scrA\|\leq 1_\C$.

Thus $\|\lambda 1_\scrA\| = \left|\lambda\right|\|1_\scrA\|\leq
        \left|\lambda\right|$, in~$\C$.
\item
Since $\|\lambda 1_\scrA\|1_\scrA
= \left|\lambda\right| \,\|1_\scrA\|\, 1_\scrA$,
it suffices to show that $\|1_\scrA\|\, 1_\scrA = 1_\scrA$,
        that is, that $\left| 1_\C-\|1_\scrA\|\right|\,\|1_\scrA\| =0$.
But as we already saw that~$\|1_\scrA\|$
is equal to either~$1_\C$ or~$0_\C$,
this is indeed the case.
\end{enumerate}
\end{solution}
\begin{solution}{parsec-90.20}
Showing that the first three points are equivalent is not too difficult.
$\text{1.}\implies\text{2.}$:
    If~$f(x)\geq 0$ for all~$x\in X$,
    then~$g\colon X\to\C$ given by~$g(x)=\sqrt{f(x)}$
    for all~$x\in X$ is continuous,
    and~$g^2=f$.
$\text{2.}\implies\text{3.}$:
is obvious.
$\text{3.}\implies\text{1.}$:
If~$f\equiv g^*g$ for some~$g\in C(X)$,
    then~$f(x)=\smash{\overline{g(x)}}g(x)=\left|g(x)\right|^2\geq 0$
    for all~$x\in X$.
Of course,
$\text{5.}\implies \text{4.}$ is obvious.
For the remainder,
note that given~$t\in \R$ we have
    $\|f-t\|\equiv \sup\{\left|f(x)-t\right|\colon x\in X\}\leq t$
    iff $\left|f(x)-t\right|\leq t$ for all~$x\in X$
    iff $-t\leq f(x)-t \leq t $ for all~$x\in X$
    iff $0\leq f(x)\leq 2t$ for all~$x\in X$
    iff $0\leq f\leq 2t$
    iff $0\leq f$ and~$\frac{1}{2}\|f\|\leq t$.
Hence $\text{4.}\implies\text{1.}\implies\text{5.}$.
\end{solution}
\begin{solution}{parsec-90.30}
Since only~$0$ is not invertible in~$\C$,
    we see that an element $f$ of $C(X)$
    is invertible (with inverse given by~$f^{-1}(x)=f(x)^{-1}$ 
    for all~$x\in X$) precisely when~$f(x)\neq 0$ for all~$x\in X$,
    that is, when $0\notin f(X)$.
In particular,
    $f-\lambda $ is \emph{not} invertible
    iff $0\in (f-\lambda)(X)$
    iff $f(x)=\lambda$ for some~$x\in X$
    iff $\lambda \in f(X)$.
\end{solution}
\begin{solution}{parsec-90.90}
Given an element~$a$ of a $C^*$-algebra,
we have that
    $a$ is an effect iff $0\leq a\leq 1$
    iff both $a$ and~$a^\perp \equiv 1-a$ 
    are positive (using the definition of~$\leq$ here)
    iff both $a^{\perp\perp}\equiv a$ and~$a^\perp$ are positive
    iff $0\leq a^\perp\leq 1$
    iff $a^\perp$ is an effect.
\end{solution}
\begin{solution}{parsec-90.100}
\begin{enumerate}
\item[1.]
$0$ is positive, because~$0^*=0$ and~$\|0-0\|\leq 0$.

That~$a+b\in\scrA_+$ when~$a,b\in\scrA_+$
was proven in~\sref{parsec-90.70}.

To show that~$\lambda a$ is positive
for~$a\in\scrA_+$ and~$\lambda\in[0,\infty)$,
pick~$t\in \R$ with~$\|a-t\|\leq t $.
Then~$\lambda a$ is self-adjoint,
        and~$\|\lambda a -\lambda t\| = \lambda \|a-t\|\leq  \lambda t$,
so~$\lambda a$ is positive.

Since~$0$ is positive,
we have $a\leq a$ for all~$a\in\scrA$.
Further, when~$a\leq b\leq c$ for some~$a,b,c\in\scrA$,
then~$b-a$ and~$c-b$ are positive,
so~$c-a\equiv (c-b)+(b-a)$
    is positive,
        that is~$a\leq c$.
Hence~$\leq$ is a preorder, on~$\scrA$.

\item[2.]
The unit, $1$, is self-adjoint since
$1^*=1^*1 = (1^*1)^* = (1^*)^*=1$,
and positive, because $\|1-1\|\leq 1$.

Given self-adjoint $a$ in~$\scrA$,
$\|a\|+a$ and~$\|a\|-a$ are self-adjoint,
and positive,
because~$\|\,(a+\|a\|)\,-\,\|a\|\,\|\leq\|a\|$
        and $\|\,(\|a\|-a)\,-\,\|a\|\,\|\leq \|a\|$.
    Hence~$-\|a\|\leq a \leq \|a\|$
        for self-adjoint~$a\in\scrA$.

\item[3.]
Let~$x$ and~$y$ be non-orthogonal linearly independent
vectors of a Hilbert space~$\scrH$.
        Then~$\ketbra{x}{x}$ and~$\ketbra{y}{y}$ are self-adjoint.

To see that~$\ketbra{x}{x}$ is positive,
note that~$\ketbra{x}{x}^2 = \ketbra{x}{x}\|x\|^2$,
and   
\begin{alignat*}{3}
    \|\,\ketbra{x}{x}\,-\,\|x\|^2\,\|^2
        &=\  \|\,\ketbra{x}{x}^2 
        \,-\, 2\|x\|^2\ketbra{x}{x}\,+\,\|x\|^4\,\|\\
        \ &=\  \|x\|^2 \ \|\,\ketbra{x}{x}-\|x\|^2\,\|,
\end{alignat*}
        so $\|\,\ketbra{x}{x}-\|x\|^2\,\| \leq \|x\|^2$,
        and hence~$\ketbra{x}{x}$ is positive.

Similarly, $\ketbra{y}{y}$ is positive,
but their product
$\ketbra{x}{x}\,\ketbra{y}{y} 
\equiv \ketbra{x}{y}\,\left<x,y \right>$
isn't even self-adjoint,
        as we saw in 
        the solution to~\sref{parsec-70.30}(10).
\item[4.]
Note that~$\|a\|_o\leq \|a\|$, because
we saw that $-\|a\|\leq a \leq \|a\|$ in point~2.
In particular, $\|a\|_o < \infty$.
Also, clearly, $\|a\|_o\geq 0$ for all~$a\in\Real\scrA$.

Note that~$\|a\|_o = \|-a\|_o$
for all~$a\in\Real\scrA$,
because $-\lambda \leq a\leq \lambda$
iff $\lambda\geq -a \geq -\lambda$
for all~$\lambda\in[0,\infty)$.

Let~$\mu\in\R$ and~$a\in\scrA$ be given;
we want to show that $\|\mu a \|_o = \left|\mu\right|\|a\|_o$.
Since this identity holds when~$\mu=0$,
we may assume that~$\mu\neq 0$.
Suppose for now that~$\mu>0$;
we'll deal with the case that~$\mu<0$ in a moment.
Note that for~$\lambda\in [0,\infty)$,
with $-\lambda \leq \mu a \leq \lambda$
we have~$-\smash{\frac{\lambda}{\mu}}\leq a \leq \smash{\frac{\lambda}{\mu}} $,
so~$\|a\|_o\leq \smash{\frac{\lambda}{\mu}}$,
that is~$\mu \|a\|_o \leq \lambda$.
Taking the infimum, we get~$\mu \|a\|_o\leq \|\mu a\|_o$.
For the other direction,
replace~$\mu$ and $a$ by $\mu^{-1}$ and~$\mu a$,
to get $\mu^{-1} \|\mu a\|_o \leq \|\mu^{-1} \mu a\|_o$,
and so~$\|\mu a\|_o \leq \mu \|a\|_o$.
Thus~$\|\mu a\|_o = \mu\|a\|_o$,
for~$\mu>0$.
In the other case, $\mu<0$,
we have~$-\mu=\left|\mu\right| >0$, 
and so $\|\mu a\|_o = \|-(-\mu)a\|_o
= -\mu \|a\|_o = \left|\mu\right|\|a\|_o$.

Let~$a$ and~$b$ be self-adjoint elements of~$\scrA$.
To show that~$\|\,\cdot\,\|_o$ is a seminorm,
it remains to be shown that
$\|a+b\|_o\leq \|a\|_o+\|b\|_o$.
Given~$\lambda,\mu\in [0,\infty)$
with $-\lambda \leq a\leq \lambda$
and $-\mu \leq b\leq \mu$,
we have $-(\lambda+\mu)\leq a+b\leq \lambda+\mu$,
and so~$\|a+b\|_o\leq \lambda+\mu$.
Taking the infimum over all such~$\lambda$ and~$\mu$,
we get $\|a+b\|_o\leq \|a\|_o + \|b\|_o$.

\item[5.]
Please contact me if you've found a short and elementary proof for any of these
five problems.
\end{enumerate}
\end{solution}
\begin{solution}{parsec-110.60}
\begin{enumerate}
\item[1.]
Since~$\|a\|<\left|\lambda\right|$,
we have $\| \, a\lambda^{-1}\,\|
= \left|\lambda\right|^{-1}\|a\| <1$.
Thus by~\sref{parsec-110.20},
$1-a\lambda^{-1}$ is invertible.
But then~$\lambda-a \equiv \lambda (1-a\lambda^{-1})$
is invertible too.
\item[2.]
(There is an erratum to the printed
        version of this exercise.)

Since~$a-b = b(1-ab^{-1})$
it suffices to show
        that~$1-ab^{-1}$ is invertible.
    Indeed it is, by~\sref{parsec-110.20},
        because $\|ab^{-1}\|\leq \|a\|\,\|b^{-1}\| < 
        \|b^{-1}\|^{-1}\,\|b^{-1}\| = 1$.

\item[3.]
Let~$b\in U$ be given.
To show that~$U$ is open,
we must find~$\varepsilon>0$
with $a\in U$ for all~$a\in\scrA$ with $\|a-b\|\leq \varepsilon$.

Take~$\varepsilon:= \|b^{-1}\|^{-1}$. 
By the previous point,
        $a\equiv (a-b)-(-b)$
        is invertible
        provided that~$\|a-b\|\leq \varepsilon \equiv \|(-b)^{-1}\|^{-1}$,
        because~$-b$ is invertible.
\end{enumerate}
\end{solution}

\begin{solution}{parsec-110.150}
\begin{enumerate}
\item
Since~$\lambda \in \C\backslash \R$,
        we have~$\Imag{\lambda}\neq 0$.
        Then~$\smash{\frac{a-\lambda}{\Imag{\lambda}}}
        = \smash{\frac{a-\Real{\lambda}}{\Imag{\lambda}}}-i$
        is invertible by~\sref{parsec-110.130},
        since $\smash{\frac{a-\Real{\lambda}}{\Imag{\lambda}}}$
        is self-adjoint.
        Upon multiplication with~$\Imag{\lambda}$
        we
        see that~$a-\lambda$ is invertible too.

    \item
We already know that~$a^2-\lambda$ is invertible
for all~$\lambda \in\C\backslash\R$,
so the only thing that remains to be shown is that
$a^2+\lambda$ is invertible
        for $\lambda \in (0,\infty)$.
For this,
        note that since~$a-\sqrt{\lambda}i$ and~$a+\sqrt{\lambda}i$
        are invertible by point~2,
        so is their product $(a-\sqrt{\lambda}i)(a+\sqrt{\lambda}i)
        = a^2+\lambda$.
\item
We already know that~$a^n-\lambda$
is invertible for all~$\lambda\in \C\backslash \R$,
by point~1, so we only need to show
that~$a^n+\lambda$ is invertible for all~$\lambda>0$
iff $a+\lambda$ is invertible for all~$\lambda>0$.
In fact, we'll show that for any~$\lambda>0$ 
the element~$a+\lambda$ is invertible
        iff~$a^n+\lambda^n$ is invertible.
    Since~$a+\lambda = \lambda(\lambda^{-1}a+1)$
    and $a^n+\lambda^n = \lambda^n((\lambda^{-1}a)^n + 1)$,
    it suffices to show that
    $b+1$ is invertible iff $b^n+1$ is invertible, where~$b:=\lambda^{-1}a$.

    Since~$\zeta^2, \zeta^4,\zeta^6,\dotsc,\zeta^{2n}$
are the $n$ roots
of the polynomial~$x^n-1$,
        we have $x^n-1 = \prod_{k=1}^n (x-\zeta^{2k})$.
Substituting~$\zeta^{-1} b$ for~$x$,
and
multiplication by~$-1\equiv \zeta^n$ gives
\begin{equation}
\label{eq:11XV3}
b^n+1 \ =\  \zeta^n(\zeta^{-1}b)^n - \zeta^n
        \ =\  \prod_{k=1}^n b-\zeta^{2k+1}.
\end{equation}
Note that~$\zeta^{2k+1}\notin \R$ when~$k\neq \frac{1}{2}(n-1)$,
and so~$b-\zeta^{2k+1}$
is invertible for such~$k$, by point~1
of this exercise.

Thus in light of~\eqref{eq:11XV3},
$b^n+1$ is invertible
iff $b-\zeta^n\equiv b+1$ is invertible.
\end{enumerate}
\end{solution}
\begin{solution}{parsec-110.180}
Let~$a$ be an element of a $C^*$-sublalgebra~$\scrA$
of a $C^*$-algebra~$\scrB$,
    and assume~$a$ has an inverse~$a^{-1}$ in~$\scrB$.
    We must show that~$a^{-1}\in\scrA$.

Note that~$a^*a$ is invertible in~$\scrB$
    with inverse~$a^{-1} (a^{-1})^*$.
This inverse
$(a^*a)^{-1}$ is in the subalgebra~$\scrA$
by~\sref{parsec-110.160},
because~$a^*a$ is self-adjoint.
    Thus~$(a^*a)^{-1}a^*\equiv a^{-1} (a^{-1})^* a^*
    \equiv a^{-1} (aa^{-1})^* = a^{-1}$
    is in the subalgebra~$\scrA$ as well.
\end{solution}
\begin{solution}{parsec-110.200}
\begin{enumerate}
    \item
Given~$\lambda\in\C$, we have
        $\lambda \in \spec(f)$
        iff $f-\lambda$ is not invertible
        iff~$\lambda \in f(X)$,
using~\sref{parsec-90.30} in the last step.
        Thus~$\spec(f)=f(X)$.
    \item
        A square matrix is invertible iff its kernel is not~$\{0\}$.
In particilar, $A-\lambda$
        is not invertible iff $(A-\lambda)v=0$
        for some non-zero vector~$v\in \C^n$.
In that case, we have $Av=\lambda v$, and so~$\lambda$
is an eigenvalue for~$A$.  Conversely,
for any eigenvalue~$\lambda$ of~$A$
the kernel of~$A-\lambda$ will consist
of the associated eigenvectors, and so~$A-\lambda$ will not be invertible.
Hence the spectrum of~$A$ 
is the set of eigenvalues of~$A$.
\end{enumerate}
\end{solution}
\begin{solution}{parsec-110.210}
\begin{enumerate}
\item[1.]
That~$\spec(a)\subseteq \R$ for self-adjoint~$a$
follows from \sref{parsec-110.150}(1).

        To see that  $ \spec(\,\smash{\bigl(\begin{smallmatrix} 0 & 2 \\ 0 & 0
        \end{smallmatrix}\bigr)}\,)=\{0\}$
we must show that~$0$ is the only eigenvalue of
        $ \smash{\bigl(\begin{smallmatrix} 0 & 2 \\ 0 & 0
        \end{smallmatrix}\bigr)}$,
that is,
that~$0$ is the only root of the characteristic polynomial
        $\det(
        \smash{\bigl(\begin{smallmatrix} 0 & 2 \\ 0 & 0
        \end{smallmatrix}\bigr)}
        -\lambda ) = \lambda^2$,
        which is so.
\item[2.]
This follows immediately from \sref{parsec-110.150}(2).
\item[3.]
This follows directly from~\sref{parsec-110.60}(1).
\item[4.]
Let~$\lambda_1,\lambda_2,\dotsc$
        be a sequence in~$\spec(a)$
        that converges to some~$\lambda\in \C$;
        we must show that~$\lambda\in\spec(a)$.
Note that since the set of invertible elements of~$\scrA$
        is open by~\sref{parsec-110.60}(3),
        the set of non-invertible elements of~$\scrA$
        is closed.
Thus, as the $a-\lambda_1,\,a-\lambda_2,\,\dotsc$
are all non-invertible,
and converge to~$a-\lambda$,
this element~$a-\lambda$ is non-invertible as well.
        Hence~$\lambda\in\spec(a)$.

Thus, $\spec(a)$, being a closed and bounded subset of~$\C$,
is compact.
\item[5.]
Given $\lambda\in\C$
        we have $\lambda\in \spec(a)$
        iff $a-\lambda\equiv a+z-(\lambda+z)$ is not invertible
        iff $\lambda+z\in \spec(a+z)$.
        Thus $\spec(a)+z = \spec(a+z)$.
\item[6.]
Let~$\lambda\in \C$ be given.
    Since~$\lambda-a = \lambda a (a^{-1}-\lambda^{-1})$,
    we see that~$\lambda-a$ is invertible
    iff $a^{-1} - \lambda^{-1}$ is invertible.
        Hence~$\spec(a^{-1})=\spec(a)^{-1}$.
\end{enumerate}
\end{solution}



\end{document}
