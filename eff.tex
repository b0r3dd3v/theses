\documentclass[b]{subfiles}
\begin{document}

\chapter{$\dagger$ and~$\diamond$ in an effectus}

\begin{parsec}%
\begin{point}%
In the previous chapter and in \TODO{cite Bram}
    we have studied the categorical properties of von Neumann algebras.
In this chapter we change pace:
    we study categories which share part of the structure
    of the category of von Neumann algebras.
The goal of this line of study, is to identify
    axioms which uniquely pick out the category of von Neumann algebras.
We do not reach this ambitious goal ---
    instead we will build up to a characterization of
    the existence of a unique $\dagger$-structure
    on the pure maps of a von Neumann-like category.
\end{point}
\begin{point}%
As basic axiom we will require the categories we consider to be
    effectuses.
For us the definition of an effectus will play a similar role
    to that of a topological space for a geometer.
General topological spaces on their own are usually of little interest:
    the axioms are so weak that there are many pathological examples.
    These weak axioms, however, are very expressive in the sense
    that they allow for the definition of many useful notions.
    Herein lies the strength of topological spaces ---
    as a stepping stone:
    many important classes of mathematical spaces
    are just plain topological spaces with a few additional axioms.
Similarly, there will be many pathalogical effectuses.
Their use for us lies in their expressiveness.
After we have studied plain effectuses,
    we will study effectuses with various additional bits of structure.
\TODO{More intro}
\end{point}
\end{parsec}
\section{Effectus}
An effectus comes in two guises:
    axiomatizing either a category of total maps
    or a category of partial maps.
We will start of with the total form as it has the simplest axioms.
Later we will prefer to work with the partial form.
\begin{parsec}%
\begin{point}{Definition}%
A category $C$ is said to be an \Define{effectus in total form}
    if
\begin{enumerate}
\item $C$ has finite coproducts and a final object~$1$;
\item all diagrams of the following form are pullbacks
\begin{equation*}
\xymatrix{
    X+Y \ar[r]^{\id+!} \ar[d]_{!+\id} & X+1\ar[d]^{!+\id} \\
    1+Y\ar[r]_{\id+!} & 1+1
    }
    \qquad
\xymatrix{
    X \ar[r]^{!} \ar[d]_{\kappa_1} & 1 \ar[d]^{\kappa_1} \\
    X+Y\ar[r]_{!+!} & 1+1
    }
\end{equation*}
\item\label{eff-joint-monicity} and the following two arrows are jointly monic
    \begin{equation*}
        \xymatrix@C+2pc  {
            1+1+1  \ar@/^/[r]^{[\kappa_1,\kappa_2,\kappa_2]}
                    \ar@/_/[r]_{[\kappa_2,\kappa_1,\kappa_2]} & 1+1.
        }
    \end{equation*}
\end{enumerate}
An arrow~$X \to Y+1$ is called a \Define{partial map} $X \to Y$.
\begin{point}%
Our main example of an effectus in total form
    is the category~$\op{\W{NCPU}}$
    of von Neumann algebras with completely positive normal unital
    maps in the opposite direction.
The partial maps correspond to contractive maps.
We list more examples later.
Let~$C$ be an effectus in total form.
Given two arrows~$f\colon X \to Y+1$
and~$g \colon Y \to Z+1$ (i.e.~partial maps~$X \to Y$ and~$Y \to Z$)
    their composition as partial maps
    is defined as~$g \hafter f \equiv  [g, \kappa_2] \after f$.
Write~$\Define{\Par C}$ for the \Define{category of partial maps},
    which has the same objects
    as~$C$, but as arrows~$X \to Y$ in~$\Par C$
    we take arrows of the form~$X \to Y+1$ in~$C$,
    which we compose using~$\hafter$.
The identity on~$X$ in~$\Par C$ is given by~$\kappa_1 \colon X \to X+1$.
The category~$\Par C$ is not an effectus in total form --- instead it
    is an effectus \emph{in partial form}.
\end{point}
\end{point}
\begin{point}{Definition}%
A category~$C$ is called an \Define{effectus in partial form} if
\begin{enumerate}
\item
    $C$ is a finPAC \TODO{cite} --- that is: if
    \begin{enumerate}
        \item $C$ is PCM-enriched --- that is: if
            \begin{enumerate}
            \item
            every homset $C(X,Y)$ has a partial binary operation~$\ovee$
            (we write~$\Define{f \perp g}$ if~$f \ovee g$ exists)
                    and distinguished~$0 \in C(X,Y)$
                    that turns~$C(X,Y)$ into a partial commutative monoid
                    (PCM) and
            \item
            if $f \perp g$ then
                $(h \after f) \perp (h \after g)$,
                $(f \after k) \perp (g \after k)$ \emph{and}
            \begin{equation*}
                 h \after (f \ovee g) = 
                (h \after f) \ovee (h \ovee g)\qquad
                (f \ovee g) \after k = 
                (f \after k) \ovee (g \after k)
            \end{equation*}
                for any~$f,g \in X \to Y$,
                $h\colon Y \to Y'$
                and~$k \colon X'\to X$;
            \end{enumerate}
        \item 
            $C$ has finite coproducts $(+,0)$;
        \item
            (compatible sum)
            for any~$b\colon X \to Y + Y$ we have
            $\pproj_1 \after b \perp \pproj_2 \after b$,
            where~$\Define{\pproj_i}\colon Y + Y \to Y$
            are \Define{partial projectors} defined
            by~$\pproj_1 \equiv [\id, 0]$
            and~$\pproj_2 \equiv [0, \id]$ \emph{and}
        \item
            (untying) if~$f\perp g$,
            then~$\kappa_1\after f \perp \kappa_2 \after g$ \emph{and}
    \end{enumerate}
    \item it has effects --- that is: there is a distinguished object~$I$
            such that
    \begin{enumerate}
        \item for each object~$X$ the PCM~$C(X,I)$
                is an effect algebra --- that is:
        \begin{enumerate}
            \item
                for each~$f\colon X \to I$
                    there is a unique element~$f^\perp\colon X \to I$
                    such that~$f \ovee f^\perp = 1$,
                    where~$1 \equiv 0^\perp$ and
            \item
                if~$f\colon X \to I$ with~$f \perp 1$,
                    then~$f = 0$;
        \end{enumerate}
        \item if~$1 \after f \perp 1 \after g$,
            then~$f \perp g$ \emph{and}
        \item if~$1 \after f = 0$ then~$f = 0$.
    \end{enumerate}
\end{enumerate}
A map~$f\colon X \to Y$ is called \Define{total} if~$1 \after f = 1$.
\begin{point}%
At first glance an effectus in partial form seems
    to have a much richer structure than an effectus in total form.
This is not the case ---
    effectuses in total and partial form are two views
    on the same thing:
\end{point}
\end{point}
\begin{point}{Theorem}%
Let~$C$ be an effectus in total form
and~$D$ be an effectus in partial form.\TODO{cite kenta}
\begin{enumerate}
\item
The category~$\Par C$ is an effectus in partial form
        with~$I = 1$.
\item
The total maps of~$D$ form an effectus in total form~$\Tot D$.
\item
    Nothing is lost:
    $\Par \Tot D \cong D$ and~$\Tot \Par C \cong C$.
\end{enumerate}
\begin{point}{Proof}%
    \TODO{...}
\end{point}
\end{point}

\subsection{Predicates, states and scalars}
\begin{point}%
For the remainder of this text,
    we will work with effectuses in partial form.
We will write~$1$ instead of~$I$.
\end{point}
\begin{point}{Definition}%
Let~$C$ be an effectus (in partial form).
\begin{enumerate}
\item
A \Define{predicate} on~$X$ is a map~$X \to 1$.
The set of predicates $\Define{\Pred X}$ on~$X$ form an effect algebra.
\item
A \Define{scalar} is a predicate on~$1$, that is: a map~$1 \to 1$.
Write~$\Define{M} \equiv \Pred 1$ for the set of scalars.
We multiply two scalars~$\lambda,\mu \colon 1 \to 1$
    simply by composing~$\lambda \odot \mu \equiv \lambda \after \mu$.
As~$C$ is PCM-enriched, this turns~$M$ into an effect monoid.\TODO{ref}
\item
For a scalar~$\lambda\colon 1\to 1$ and a predicate~$p\colon X \to 1$
    we write~$\lambda \cdot p \equiv \lambda \after p$.
With this scalar multiplication,
    $\Pred X$ turns into a~$M$-effect module.
\item
A \Define{substate} on~$X$ is a map~$\omega\colon 1 \to X$.
A \Define{state} is a total substate.
We denote the set of states on~$X$ by~$\Stat X$.
\end{enumerate}
\end{point}
\end{parsec}

\section{Quotient}
\begin{parsec}%
\begin{point}{Definition}%
Let~$C$ be an effectus.
We say~$C$ is an \Define{effectus with quotients}
    if for each predicate~$p \colon X \to 1$,
    there exists a map~$\xi_p \colon X \to X/_p$
    with~$1 \after \xi_p \leq p^\perp$
    satisfying the following universal property:
\begin{quote}
    for any (other) map~$f\colon X \to Y$
        with~$1 \after f \leq p^\perp$,
        there is a unique map~$f' \colon X/_p \to Y$
        such that~$f' \after \xi_p = f$.
\end{quote}
Any  map with this universal property
    is called a \Define{quotient} for~$p$.
\end{point}
\begin{point}{Example}%
In~$\op{\W{NCPsU}}$ quotients are exactly the same thing
    as contractive filters, see \TODO{...}.
An example of a quotient-map for~$1-a \in \scrA$ is given
    by~$\xi\colon \ceil{a}\scrA \ceil{a} \to \scrA$
    with~$\xi(b) = \sqrt{a} b \sqrt{a}$.
\end{point}
\begin{point}[quotient-basics]{Exercise}%
Show the following basic properties of quotients.
\begin{enumerate}
    \item If~$\xi\colon X \to Y$ is a quotient for~$p$
                and~$\vartheta\colon Y \to Z$ is an isomorphism,
                then~$\vartheta \after \xi$ is a quotient for~$p$
                as well.
    \item Conversely, if~$\xi_1$ and~$\xi_2$
            are both quotients for~$p$,
            then there is unique a isomorphism~$\vartheta$
            with~$\xi_1 = \vartheta \after \xi_2$.
    \item Isomorphisms are quotients (for 0).
    \item Zero-maps are quotients (for 1).
    \item If~$\xi$ is a quotient for~$p$, then~$1\after \xi = p^\perp$
                (Hint: apply the universal property to $p^\perp$).
    \item Quotients are epi.
\end{enumerate}
\begin{point}%
The following proposition is easy to prove,
    but shows an important property of quotients:
    any map~$f$ factors as a total map after a quotient
    for~$(1\after f)^\perp$.
\end{point}
\end{point}
\begin{point}[quotient-total]{Proposition}%
Assume~$\xi_{p^\perp} \colon X \to X/_{p^\perp}$ is a quotient for~$p^\perp$.
For any~$f\colon X \to Z$
    with~$1 \after f = p$,
    there is a unique \emph{total} $g\colon X/_{p^\perp} \to Z$
    with~$f = g \after \xi_{p^\perp}$.
\begin{point}{Proof}%
By definition of quotient, there is a unique~$g\colon X/_{p^\perp} \to Z$
    with~$g \after \xi_{p^\perp} = f$.
Note~$1 \after g \after \xi_{p^\perp} = 1 \after f = p = 1 \after \xi_{p^\perp}$.
Thus, as~$\xi_{p^\perp}$ is epi (\sref{quotient-basics}),
    we conclude~$1 \after g = 1$.
That is: $g$ is total. \qed
\end{point}
\end{point}
\begin{point}{Proposition}%
In an effectus with quotients,
    quotients are closed under composition.
\begin{point}{Proof}%
Assume~$\xi_1\colon X \to Y$ is a quotient for~$p^\perp$
    and~$\xi_2\colon Y \to Z$ is a quotient for~$q^\perp$.
We will prove~$\xi_2 \after \xi_1$
    is a quotient for~$(q \after \xi_1)^\perp$.
As our effectus has quotients,
    we can pick a quotient~$\xi \colon X \to X/_{(q \after \xi_1)^\perp}$
        of~$(q \after \xi_1)^\perp$.
First, some preparation.
Note~$1 \after \xi = q \after \xi_1 \leq 1 \after \xi_1 = p$.
Thus by the universal property of~$\xi_1$,
there is a unique map~$h_1\colon Y \to X/_{(q \after \xi_1)^\perp}$
        with~$h_1 \after \xi_1 = \xi$.
As~$1 \after h_1 \after \xi_1 = 1 \after \xi = q \after \xi_1$
    and~$\xi_1$ is epi, we see~$1 \after h_1 = q$.
Thus by \sref{quotient-total},
there is a unique total map~$h_2 \colon Z \to X/_{(q \after \xi_1)^\perp}$
    with~$h_2 \after \xi_2 = h_1$.
Let~$g\colon X/_{(q \after \xi_1)^\perp} \to Z$
    be the unique map such that~$g \after \xi = \xi_2 \after \xi_1$.
We are in the following situation.
\begin{equation*}
    \xymatrix@C+2pc{
        X  \ar[r]^{\xi_1} \ar@/_/[rrd]_{\xi}
        & Y \ar[r]^{\xi_2} \ar@{.>}[rd]_{h_1}
        & Z \ar@{.>}@/^/[d]^{h_2} \\
        && X/_{(q \after \xi_1)^\perp} \ar@{.>}@/^/[u]^{g}
    }
\end{equation*}
We claim~$g$ is an isomorphism with inverse~$h_2$.
Indeed: from $g \after h_2 \after \xi_2 \after \xi_1= g \after \xi = \xi_2 \after \xi_1$
    we get~$g \after h_2 = \id$
    and from~$h_2 \after g \after \xi = h_2 \after \xi_2 \after \xi_1 = \xi$
    we find~$h_2 \after g = \id$.
Thus~$\xi_2 \after \xi_1 = g \after \xi$ for an isomorphism~$g$,
which shows~$\xi_2 \after \xi_1$ is a quotient, see \sref{quotient-basics}. \qed
\end{point}
\end{point}
\begin{point}{Exercise}
By \sref{quotient-total} we know that each
    map~$f$ factors as~$t \after \xi$
    for some total map~$t$ and quotient~$\xi$.
Show that this forms an orthogonal factorization system ---
    that is: prove that if~$t' \after \xi' = t \after \xi$
    for some (other) quotient~$\xi'$ and total map~$t'$,
    then there is a unique isomorphism~$\vartheta$
    with~$\xi' = \vartheta \after \xi$
    and~$t = t' \after \vartheta$.
\end{point}
\end{parsec}

\section{Comprehension}
\begin{parsec}%
\begin{point}{Definition}%
Let~$C$ be an effectus.
We say~$C$ has comprehension
    if for each predicate~$p \colon X \to 1$,
    there exists a map~$\pi_p \colon \cmpr{X}{p} \to X$
    with~$p \after \pi_p = 1 \after \pi_p$
    satisfying the following universal property:
\begin{quote}
    for any (other) map~$g\colon Z \to X$
        with~$p \after g = 1 \after g$,
        there is a unique map~$g' \colon Z \to \cmpr{X}{p}$
        such that~$\pi_p \after g' = g$.
\end{quote}
Any  map with this universal property
    is called a \Define{comprehension} for~$p$.
\end{point}
\begin{point}{Example}%
In~$\op{\W{NCPsU}}$ comprehensions are exactly the same thing
as corners, see \TODO{...}.
An example of a comprehension for~$a \in \scrA$
    is given by~$\pi\colon \scrA \to \floor{a}\scrA\floor{a}$
    with~$\pi(b) = \floor{a}b\floor{a}$.
\end{point}
\begin{point}[compr-basics]{Exercise}%
Show the following basic properties of comprehensions.
\begin{enumerate}
    \item If~$\pi\colon X \to Y$ is a comprehension for~$p$
                and~$\vartheta\colon Z \to X$ is an isomorphism,
                then~$\pi \after \vartheta$ is a comprehension for~$p$
                as well.
    \item Conversely, if~$\pi_1$ and~$\pi_2$
            are both comprehensions for~$p$,
            then there is unique a isomorphism~$\vartheta$
            with~$\pi_1 = \pi_2 \after \vartheta$.
    \item Isomorphisms are comprehensions (for 1).
    \item Zero-maps are comprehensions (for 0).
    \item Comprehensions are mono.
\end{enumerate}
\begin{point}%
There does not seem to be a lot more to say about comprehensions in
    an effectus without further assumptions.

\end{point}
\end{point}
\end{parsec}

\end{document}

% vim: ft=tex.latex
