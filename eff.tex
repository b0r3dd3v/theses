\documentclass[b]{subfiles}
\begin{document}

\chapter{Diamond, andthen, dagger}

\begin{parsec}%
\begin{point}%
In the previous chapter and in \TODO{cite Bram}
    we have studied the categorical properties of von Neumann algebras.
In this chapter we change pace:
    we study categories which share part of the structure
    of the category of von Neumann algebras.
The goal of this line of study, is to identify
    axioms which uniquely pick out the category of von Neumann algebras.
We do not reach this ambitious goal ---
    instead we will build up to a characterization of
    the existence of a unique $\dagger$-structure
    on the pure maps of a von Neumann-like category.
\end{point}
\begin{point}%
As basic axiom we will require the categories we consider to be
    effectuses.
For us the definition of an effectus will play a similar role
    to that of a topological space for a geometer.
General topological spaces on their own are usually of little interest:
    the axioms are so weak that there are many pathological examples.
    These weak axioms, however, are very expressive in the sense
    that they allow for the definition of many useful notions.
    Herein lies the strength of topological spaces ---
    as a stepping stone:
    many important classes of mathematical spaces
    are just plain topological spaces with a few additional axioms.
Similarly, there will be many pathalogical effectuses.
Their use for us lies in their expressiveness.
After we have studied plain effectuses,
    we will study effectuses with various additional bits of structure.
\TODO{More intro}
\end{point}
\end{parsec}
\section{Effect algebras et al}
\begin{parsec}%
\begin{point}%
Before we turn to effectuses,
    it is convenient to introcuce some lesser-known algebraic structures
    of which the effect algebra is the most important.
It will turn out that the set of predicates associated to an object in
    an effectus can be arranged into an effect algebra.
In this way, effect algebras play the same role
    for effectuses as heyting algebras for topoi.
    First, we will recall the definition of partial commutative monoid (PCM)
--- later we will see \TODO{...} that the partial maps between
    two objects in an effectus can be arranged into a PCM.
\end{point}
\begin{point}[dfn-pcm]{Definition}%
    A \Define{partial commutative monoid} (\Define{PCM})~$M$
        is a set~$M$ together with distinguished element~$0 \in M$
        and a partial binary operation~$\ovee$ such that
        for all~$a,b,c \in M$ 
        --- writing~$a \perp b$ whenever~$a \ovee b$ is defined
        --- we have
\begin{enumerate}
    \item \emph{(partial commutativity)}
        if $a \perp b$, then~$b \perp a$ and~$a \ovee b = b \ovee a$;
    \item \emph{(partial associativity)}
        if $a \perp b$ and~$a \ovee b \perp c$,
        then~$b \perp c$, ~$a \perp b \ovee c$
            and~$(a \ovee b) \ovee c = a \ovee (b \ovee c)$ \emph{and}
    \item \emph{(zero)}
        $0 \perp a$ and~$0 \ovee a = a$.
\end{enumerate}
We say~$a\mathrel{\Define{\leq}} b$
    whenever~$a \ovee c = b$ for some~$c \in M$.
\end{point}
\begin{point}[pcm-preorder]{Exercise}%
Show a PCM is preordered by~$\leq$.
% \begin{point}{Proof}%
% Reflexivity is easy:~$a \ovee 0 = 0 \ovee a = a$
%     by the partial commutativity and zero axiom, so~$a \leq a$.
% For transitivity, assume~$a \leq b$ and~$b \leq c$.
% Pick~$d,e \in M$
%     with~$a \ovee d = b$ and~$b \ovee e = c$.
% Then by partial
% associativity~$c = b \ovee e = (a \ovee d) \ovee e = a \ovee (d \ovee e)$
% so~$a \leq c$.
% \end{point}
\end{point}
\end{parsec}

\begin{parsec}%
\begin{point}[dfn-ea]{Definition}%
A PCM~$E$ together with distinguished element~$1 \in E$
is an \Define{effect algebra} (\Define{EA}) \cite{ea}
    provided
\begin{enumerate}
\item
    \emph{(orthocomplement)}
    for every~$a$
    there is a unique~$a^\perp$
   with~$a \ovee a^\perp = 1$ \emph{and}
\item
    \emph{(zero--one)}
    if~$a \perp 1$, then~$a = 0$.
\end{enumerate}
A map~$f\colon E \to F$
between effect algebras~$E$ and~$F$
is called an \Define{effect algebra homomorphism}
if
\begin{enumerate}
    \item \emph{(additivity)}
    $a \perp b$ implies $f(a) \perp f(b)$ and~$f(a)\ovee f(b) = f(a\ovee b)$
        \emph{and}
    \item \emph{(unital)}
    $f(1) = 1$.
\end{enumerate}
Write~$\Define{\textsf{EA}}$ for the category
    of effect algebras these homomorphisms.
\end{point}
\begin{point}{Examples}%
There are many examples of effect algebras
    --- we only give a few.
\begin{enumerate}
\item
The unit interval
$[0,1]$ with partial addition is an effect algebra ---
i.e.:~$x \perp y$
        whenever~$x +y \leq 1$, $x \ovee y = x + y$
        and~$a^\perp = 1-a$.
\item
Generalizing the previous:
if~$G$ is an ordered group
with distinguished element~$1$,
then the order interval~$[0,1]_G \equiv \{x;\ x\in G;\ 0 \leq x\leq 1\}$
is an effect algebra
with~$x \perp y$ whenever~$x +y \leq 1$;
$x \ovee y = x + y$ and~$x^\perp = 1-x$.
\item
In particular,
    if~$\scrA$ is any von Neumann algebra,
    then the set of \Define{effects}~$[0,1]_\scrA$
    forms an effect algebra
    with~$a \perp b$ whenever~$a +b \leq 1$;
    $a \ovee b = a + b$ and~$a^\perp = 1-a$.
The `effect' in effect algebra originates from this example.
\item
Any orthomodular lattice~$L$
    is an effect algebra
    with the same orthocomplement,
    $x \perp y$ whenever~$x \leq y^\perp$
    and~$x \ovee y = x \vee y$.  (See e.g.~\cite[Prop.~27]{basmsc}.)
\item
In particular,
    any boolean algebra~$L$
    is an effect algebra
    with complement as orthocomplement,
    ~$x \perp y$ whenever~$x \wedge y = 0$ and
    $x \ovee y = x \vee y $.
\item
The one-element boolean algebra~$1 \equiv \{0=1\}$
    is the final object in \textsf{EA}
    and the two-element boolean algebra~$2 \equiv \{0,1\}$
    is the initial object in \textsf{EA}.
\end{enumerate}
\end{point}

\begin{point}{Exercise}%
Assume~$E$ and~$F$ are effect algebras.
First show that the cartesian products~$E \times F$
    is an effect algebra with componentwise operations.
Show that this is in fact the categorical product of~$E$ and~$F$
    in \textsf{EA}.
To finish, show that~\textsf{EA} has initial
    object~$\{0,1\}$
    and final object~$\{0=1\}$.
(The category \textsf{EA} is in fact complete and cocomplete;
    for this and more categorical properties,
        see \cite{corefl}.)
\end{point}

\begin{point}{Exercise}%
There is a small redundancy in our definition of effect algebra:
show that the zero axiom ($x \ovee 0 = x$)
follows from the remaining axioms (partial commutativity,
partial associativity, orthocomplement and zero--one.)
\end{point}

\begin{point}{Proposition}%
In any effect algebra~$E$, we have
\begin{enumerate}
    \item \emph{(involution)}
        $a^{\perp\perp} = a$;
    \item
        $1^\perp= 0$ and~$0^\perp = 1$;
    \item \emph{(positivity)}
        if~$a\ovee b = 0$, then~$a = b= 0$;
    \item \emph{(cancellation)}
        if~$a\ovee c = b\ovee c$, then~$a = b$;
    \item $\leq$ partially orders~$E$;
    \item $a \leq b$ if and only if $b^\perp \leq a^\perp$;
    \item if~$a \leq b$ and~$b \perp c$, then~$a \perp c$
        and~$a \ovee c \leq b \ovee c$ \emph{and}
    \item $a \perp b$ if and only if~$a \leq b^\perp$.
\end{enumerate}
\begin{point}{Proof}%
By partial commutativity and definition
of orthocomplement both~$a^\perp \ovee a = 1$
and~$a^\perp \ovee  a^{\perp\perp} = 1$.
So by uniqueness of orthocomplement,
    we must have~$a= a^{\perp\perp}$, which is point 1.
Clearly~$0 \ovee 1 = 1$,
    so~$1^\perp = 0$ and~$0^\perp = 1$,
    which is point 2.
For point 3, assume~$a \ovee b = 0$.
Then~$a \ovee b \perp 1$
    and so by partial associativity~$b \perp 1$.
    By zero--one, we get~$b = 0$.
    Similarly~$a=0$, which shows point 3.
For point 4, assume~$ a \ovee c = b \ovee c$.
From partial associativity, we get
\begin{equation*}
    ((a \ovee b)^\perp\ovee a) \ovee c \ =\  
    ((a \ovee b)^\perp\ovee a) \ovee b \ = \ 1
\end{equation*}
and so by uniqueness of
orthocomplement~$c = ((a\ovee b)^\perp\ovee a)^\perp = b$,
    which is point 4.
By \sref{pcm-preorder},
    we only need to show that~$\leq$ is antisymmetric
    for point 5.
So assume~$a \leq b$ and~$b \leq a$.
Pick~$c,d \in E$ with~$a \ovee c = b$ and~$b \ovee d = a$.
Then~$a = (a \ovee c) \ovee d = a \ovee (c \ovee d)$.
By cancellation~$c \ovee d = 0$.
So by positivity~$c = d= 0$.
Hence~$a = b$.
For point 6, assume~$a\leq b$.
Pick~$c \in E$ with~$a \ovee c = b$.
Clearly~$a \ovee a^\perp = 1 = b \ovee b^\perp = a \ovee c \ovee b^\perp$,
so by cancellation~$a^\perp = c \ovee b^\perp$,
    which is to say~$b^\perp \leq a^\perp$.
For point 7, assume~$a \leq b$ and~$b \perp c$.
Pick~$d$ with~$a \ovee d = b$.
By partial associativity and commutativity
    we have~$b \ovee c = (a \ovee d) \ovee c = (a \ovee c) \ovee d$,
    so~$a \ovee c$ and~$a \ovee c \leq b \ovee c$.
For point 8, first assume~$a \perp b$.
Then~$a \ovee b \ovee (a \ovee b)^\perp = 1 = b \ovee b^\perp$.
So by cancellation~$b^\perp = a \ovee (a \ovee b)^\perp$,
hence~$a \leq b^\perp$.
For the converse, assume~$a \leq b^\perp$.
Then~$a \ovee c = b^\perp$ for some~$c$.
Hence~$a \ovee c \perp b$
    and so  by partial associativity and commutativity,
        we get~$a \perp b$, as desired.
    \qed
\end{point}
\end{point}
\end{parsec}%

\begin{parsec}%
\begin{point}{Definition}%
Suppose~$E$ is an effect algebra.
Write~$\Define{b \ominus a}$
for the (by cancellation) unique element (if it exists)
such that~$a \ovee (b \ominus a) = b$.
\end{point}
\begin{point}{Exercise}%
Show that for any effect algebra~$E$, we have
\begin{enumerate}
    \item[(D1)]~$a \ominus b$ is defined if and only if~$b \leq a$;
    \item[(D2)]~$a \ominus b \leq a$ (if defined);
    \item[(D3)]~$a \ominus (a \ominus b) = b$ (if defined) \emph{and}
    \item[(D4)]~if $a \leq b \leq c $,
                then~$c \ominus b \leq c \ominus a$
                and~$(c \ominus a) \ominus (c \ominus b) = b \ominus a$.
\end{enumerate}
\begin{point}%
Assume~$E$ is a poset with maximum element~$1$
    and partial binary operation~$\ominus$
    satisfying~(D1)--(D4).
Define~$a \ovee b = c \Leftrightarrow c \ominus b = a$
    and~$a^\perp = 1 \ominus a$.
Show that this turns~$E$ into an effect algebra
    with compatible order and~$\ominus$.
\end{point}
\begin{point}{Remark}%
Such a structure~$(E,\ominus,1)$ is called a difference-poset
    (D-poset) and is, as we have just seen,
    an alternative way to axiomatize effect algebras.
\end{point}
\end{point}

\begin{point}{Exercise}%
Show that for an effect algebra homomorphism~$f\colon E \to F$,
    we have
    \begin{enumerate}
        \item~\emph{(preserves zero)} $f(0) = 0$;
        \item~\emph{(order-preserving)} if $a \leq b$, then~$f(a) \leq f(b)$
                        \emph{and}
        \item~if~$a\ominus b$ is defined,
            then $f(a \ominus b) = f(a) \ominus f(b)$.
    \end{enumerate}
\end{point}
\end{parsec}

\subsection{Effect monoids}
\begin{parsec}%
\begin{point}%
In \TODO{...} we will see the set of scalars of an effectus
    can be arranged into an effect monoid.
\end{point}
\begin{point}{Definition}%
An \Define{effect monoid}~$M$ \cite{probdistrconv}
    is an effect algebra
    together with a binary operation~$\odot$
    such that for all~$a,b,c,d \in M$, we have
\begin{enumerate}
    \item \emph{(unit)}
    $1 \odot a = a \odot 1 = a$;
\item \emph{(associativity)}
    $(a \odot b) \odot c 
    =a \odot (b \odot c)$ \emph{and}
\item \emph{(distributivity)}
    $(a \ovee b) \odot (c \ovee d)
            = (a \odot c) \ovee (b \odot c) \ovee
            (a \odot d) \ovee (b \odot d)$
            whenever~$a \perp b$ and~$c \perp d$.
\end{enumerate}
(Phrased categorically:
    an effect monoid is a monoid in \textsf{EA}
    with the obvious tensor product, see \cite{corefl,probdistrconv}.)
An effect monoid~$M$ is said to be \Define{commutative}
    if we have~$a\odot b = b\odot a$ for all~$a,b \in M$.
\end{point}
\begin{point}{Examples}%
Effect monoids are less abundant.
\begin{enumerate}
\item The effect algebra~$[0,1]$
        is a commutative effect monoid with the usual product.
        (This is the only way to turn~$[0,1]$ into an effect monoid
                \cite[Prop.~41]{basmsc}.)
\item We saw earlier that every boolean algebra is an effect algebra
        with~$x \ovee y = x \vee y$ defined iff~$x \wedge y = 0$.
    The boolean algebra is turned into an effect monoid
        with~$x \odot y \equiv x \wedge y$.
    Every finite (\emph{a priori} non-commutative) effect monoid
    is of this form \cite[Prop.~40]{basmsc}.
\item
In particular: the two-element boolean algebra~$2$
    is an effect monoid.
\item There is a non-commutative effect monoid
        based on the lexicographically ordered vector space~$\R^5$,
        see ~\cite[Cor.~51]{basmsc}.
\end{enumerate}
\end{point}
\end{parsec}

\subsection{Effect modules}
\begin{parsec}%
\begin{point}%
An effect algebra of predicates in an effectus
    will have an action of the effect monoid of the scalars,
    turning them into a effect module:\TODO{ref}
\end{point}
\begin{point}{Definition}%
Suppose~$M$ is an effect monoid.
An \Define{effect module} $E$ over~$M$ \cite{corefl}
    is an effect algebra together with an operation
            $M \times E \to E$
            denoted by~$(\lambda, a) \mapsto \lambda \cdot a$
    such that for all~$a,b \in E$ and~$\lambda,\mu \in M$, we have
\begin{enumerate}
\item
    $(\lambda \odot \mu) \cdot a = \lambda \cdot (\mu \cdot a)$;
\item
    if~$a \perp b$,
     then~$\lambda \odot a \perp \lambda \odot b$
     and~$(\lambda \odot a) \ovee (\lambda\odot b) = \lambda \odot(a \ovee b)$;
\item
    if~$\lambda \perp \mu$,
     then~$\lambda \odot a \perp \mu \odot a$
     and~$(\lambda \odot a) \ovee (\mu \odot a) = (\lambda \ovee \mu) \odot a$
            \emph{and}
\item
    $1 \odot a = a$.
\end{enumerate}
(Categorically: an effect module over~$M$
    is an~$M$-action.)
An effect algebra homomorphism~$f\colon E \to F$
    between effect modules over~$M$
    is an $M$-\Define{effect module homomorphism}
    provided~$\lambda \cdot f(a) = f(\lambda \cdot a)$
    for all~$\lambda \in M$ and~$a \in E$.
Write~$\Define{\mathsf{EMod}_M}$
    for the category of effect modules over~$M$
    with effect module homomorphisms between them.
\end{point}
\begin{point}{Examples}%
There are many effect modules.
\begin{enumerate}
\item
Every effect algebra is an effect module over
    the two-element effect monoid $2$.
    (In fact $\mathsf{EA} \cong \mathsf{EMod}_{2}$.)
The only effect module over the one-element effect monoid~$1$,
    is the one-element effect algebra~$1$ itself.
\item
If~$V$ is an ordered real vector space with positive order unit~$u$,
    then~$[0,u]$ is an effect module over~$[0,1]$.
In fact, every effect module over~$[0,1]$
    is of this form \cite{gudder1998representation}.
See also \cite[Thm.~3]{jacobs2016expectation}
    for the stronger categorical equivalence.
\end{enumerate}
\end{point}
\end{parsec}

\section{Effectus}
An effectus comes in two guises:
    axiomatizing either a category of total maps
    or a category of partial maps.
We will start of with the total form as it has the simplest axioms.
Later we will prefer to work with the partial form.
\begin{parsec}%
\begin{point}{Definition}%
A category $C$ is said to be an \Define{effectus in total form}
    \cite{effintro,newdirections,statesofconvexsets}
    if
\begin{enumerate}
\item $C$ has finite coproducts and a final object~$1$;
\item all diagrams of the following form are pullbacks
\begin{equation}\label{pullbacks}
\xymatrix{
    X+Y \ar[r]^{\id+!} \ar[d]_{!+\id} & X+1\ar[d]^{!+\id} \\
    1+Y\ar[r]_{\id+!} & 1+1
    }
    \qquad
\xymatrix{
    X \ar[r]^{!} \ar[d]_{\kappa_1} & 1 \ar[d]^{\kappa_1} \\
    X+Y\ar[r]_{!+!} & 1+1
    }
\end{equation}
\item\label{eff-joint-monicity} and the following two arrows are jointly monic
    \begin{equation*}
        \xymatrix@C+2pc  {
            1+1+1  \ar@/^/[r]^{[\kappa_1,\kappa_2,\kappa_2]}
                    \ar@/_/[r]_{[\kappa_2,\kappa_1,\kappa_2]} & 1+1.
        }
    \end{equation*}
\end{enumerate}
An arrow~$X \to Y+1$ is called a \Define{partial map}
$X \mathrel{\Define{\pto}} Y$.
\begin{point}%
One interested in physics, should think of the objects
    of an effectus as physical systems and its arrows as
    the physical operations between them.
    The final object~$1$ is the physical system with a single state.
The coproduct~$X+Y$ is the system that can be prepared as~$X$ or as~$Y$.

Studying programming languages, one would do better thinking of
    the objects of an effectus as data types and
    its arrows as the allowed operations between them (semantics of programs).
The final object~$1$ is the unit data type.
The coproduct~$X + Y$ is the union data type of~$X$ and~$Y$.
\end{point}
\begin{point}%
Our main example of an effectus in total form
    is the category~$\op{\W{NCPU}}$
    of von Neumann algebras with completely positive normal unital
    maps in the opposite direction.
The partial maps correspond to contractive maps.
We list more examples later on.
Let~$C$ be an effectus in total form.
Given two arrows~$f\colon X \to Y+1$
and~$g \colon Y \to Z+1$ (i.e.~partial maps~$X \pto Y$ and~$Y \pto Z$)
    their composition as partial maps
    is defined as~$g \hafter f \equiv  [g, \kappa_2] \after f$.
Write~$\Define{\Par C}$ for the \Define{category of partial maps},
    which has the same objects
    as~$C$, but as arrows~$X \to Y$ in~$\Par C$
    we take arrows of the form~$X \to Y+1$ in~$C$,
    which we compose using~$\hafter$
    and with identity on~$X$ in~$\Par C$
    given by~$\kappa_1 \colon X \to X+1$.\footnote{In categorical
            parlance: $\Par C$ is the Kleisli category of
            the monad~$(\ )+1\colon C \to C$.}
The category~$\Par C$ is not an effectus in total form --- instead it
    is an effectus \emph{in partial form}.
\end{point}
\end{point}
\begin{point}{Definition}%
A category~$C$ is called an \Define{effectus in partial form}
    \cite{effintro,kentapartial} if
\begin{enumerate}
\item
    $C$ is a finPAC \cite{kentapartial} (cf.~\cite{arbib}) --- that is:
    \begin{enumerate}
        \item 
            $C$ has finite coproducts $(+,0)$;
        \item $C$ is PCM-enriched --- that is:
            \begin{enumerate}
            \item
            every homset $C(X,Y)$ has a partial binary operation~$\ovee$
                    and distinguished~$0 \in C(X,Y)$
                    that turns~$C(X,Y)$ into a partial commutative monoid,
                    see \sref{dfn-pcm}, and
            \item
            if $f \perp g$ then both
                $(h \after f) \perp (h \after g)$ and
                $(f \after k) \perp (g \after k)$ \emph{and}
            \begin{equation*}
                 h \after (f \ovee g) = 
                (h \after f) \ovee (h \ovee g)\qquad
                (f \ovee g) \after k = 
                (f \after k) \ovee (g \after k)
            \end{equation*}
                for any~$f,g \in X \to Y$,
                $h\colon Y \to Y'$
                and~$k \colon X'\to X$;
            \end{enumerate}
        \item
            (compatible sum)
            for any~$b\colon X \to Y + Y$ we have
            $\pproj_1 \after b \perp \pproj_2 \after b$,
            where~$\Define{\pproj_i}\colon Y + Y \to Y$
            are \Define{partial projectors} defined
            by~$\pproj_1 \equiv [\id, 0]$
            and~$\pproj_2 \equiv [0, \id]$ \emph{and}
        \item
            (untying) if~$f\perp g$,
            then~$\kappa_1\after f \perp \kappa_2 \after g$ \emph{and}
    \end{enumerate}
    \item it has effects --- that is: there is a distinguished object~$I$
            such that
    \begin{enumerate}
        \item for each object~$X$, the PCM~$C(X,I) \equiv \Define{\Pred X}$
            is an effect algebra, see \sref{dfn-ea}
            --- in particular~$C(X,I)$
                has a maximum element~$1$;
        \item if~$1 \after f \perp 1 \after g$,
            then~$f \perp g$ \emph{and}
        \item if~$1 \after f = 0$ then~$f = 0$.
    \end{enumerate}
\end{enumerate}
A map~$f\colon X \to Y$ is called \Define{total} if~$1 \after f = 1$.
\begin{point}{Remark}%
The untying axiom is redundant: it follows from the others.
We include it, as it is part
    of the definition of a finPAC.
\end{point}
\begin{point}%
At first glance an effectus in partial form seems
    to have a much richer structure than an effectus in total form.
This is not the case ---
    effectuses in total and partial form are two views
    on the same thing: \cite{kentapartial,introeff}
\end{point}
\end{point}
\begin{point}{Theorem (Cho)}%
Let~$C$ be an effectus in total form
and~$D$ be an effectus in partial form.
\begin{enumerate}
\item
The category~$\Par C$ is an effectus in partial form
        with~$I = 1$.
\item
The total maps of~$D$ form an effectus in total form~$\Tot D$.
\item
    Nothing is lost:
    $\Par \Tot D \cong D$ and~$\Tot \Par C \cong C$.
\end{enumerate}
\begin{point}%
We postpone the proof until \TODO{}.
\end{point}
\end{point}
\end{parsec}
\subsection{From partial to total}
\begin{parsec}%
\begin{point}%
We will first show that the subcategory of
    total maps of an effectus in partial form
    is an effectus in total form.
This proof and especially the demonstration
    that the squares in \eqref{pullbacks} are pullbacks,
    will elucidate the axioms of an effectus in total form
    and will make the proof in the opposite direction more palatable.
\end{point}
\begin{point}[coproj-total]{Lemma}%
In an effectus in partial form,
    coprojections are total.
\begin{point}{Proof}%
Note
$1 = 1 \after \id = (1 \after [\id, 0]) \after \kappa_1
                    \leq 1 \after \kappa_1 \leq 1$.
Indeed~$1 \after \kappa_1 = 1$. \qed
\end{point}
\end{point}
\begin{point}[cotupl-pcm]{Proposition}%
In an effectus in partial form,
the cotupling bijection~$(f,g) \mapsto [f,g]$
    is a PCM-isomorphism ---
    that is:
\begin{enumerate}
\item
    $[f,g] \perp [f',g']$ if and only if
        $f \perp f'$ and~$g \perp g'$;
\item
    if~$[f,g] \perp [f', g']$,
    then~$[f,g] \ovee [f',g'] = [f \ovee f', g\ovee g']$ \emph{and}
\item
    $[0,0] = 0$.
\end{enumerate}
Furthermore~$[1,1]=1$ for maps into~$I$,
so the cotupling map is an effect algebra
isomorphism~$\Pred (X) \times \Pred (Y) \cong \Pred (X+Y) $.
\begin{point}{Proof}%
First we show~$[h,l] = [h,0] \ovee [0,l]$.
By the compatible sum axiom
\begin{equation*}
    [h, 0] \ = \ 
    \pproj_1 \after (h + l)
    \ \perp\  \pproj_2 \after (h + l)
    \ = \ [0, l].
\end{equation*}
By PCM-enrichtment
$([h, 0] \ovee [0, l]) \after \kappa_1
         =  ([h, 0] \after \kappa_1) \ovee
        ([0, l] \after \kappa_1)  = h$.
Similarly~$([h, 0] \ovee [0, l]) \after \kappa_2 = l$.
Thus indeed~$[h,l] = [h,0] \ovee [0,l]$.

Assume~$[f,g] \perp [f',g']$.
    By PCM-enrichment
    we have~$f = [f,g] \after \kappa_1 \perp [f',g'] \after \kappa_1 = f'$.
Similarly~$g \perp g'$.
Conversely, assume~$f \perp f'$ and~$g \perp g'$.
Again, by PCM-enrichment~$
[f,0] = f  \after \pproj_1 \perp f' \after \pproj_1 = [f', 0]$
    and~$[f\ovee f', 0 ] = [f,0] \ovee [f', 0]$.
Similarly~$[0,g \ovee g'] = [0,g] \ovee [0,g']$.
Putting it all together
\begin{align*}
[f, g] \ovee [f', g']
&\ = \ 
 [f, 0] \ovee [0,g] \ovee [f', 0] \ovee [0, g'] \\
 &\ = \ 
 [f\ovee f', 0]\ovee [0, g\ovee g'] \\
 &\ = \ 
 [f\ovee f', g \ovee g'].
\end{align*}
To show cotupling is a PCM-isomorphism,
    it only remains to be shown~$[0,0]=0$.
As~$0 \after \kappa_1 = 0$
    and~$0 \after \kappa_2 = 0$,
    we indeed have~$0 = [0,0]$.
Similarly by \sref{coproj-total} we have~$1 \after \kappa_1 = 1$
    and~$1 \after \kappa_2 = 1$,
    so~$1 = [1,1]$. \qed
\end{point}
\end{point}

\begin{point}%
The coproduct in an effectus in partial form
is almost a (bi)product:
\end{point}

\begin{point}[coprod-prod]{Proposition}%
In an effectus in partial form, we have a correspondence
\begin{prooftree}
\AxiomC{$h\colon Z \to X+Y$}
\doubleLine
\UnaryInfC{$f\colon Z \to X \quad g\colon Z \to Y \quad 1 \after f \perp 1 \after g$}
\end{prooftree}
as follows:
for every~$f\colon Z \to X$
    and~$g\colon Z \to Y$
    with~$1 \after f \perp 1 \after g$,
    there is a unique map~$\Define{\langle f, g\rangle} \colon Z \to X +Y$
    such that~$\pproj_1 \after \langle f, g \rangle = f$
    and~$\pproj_2 \after \langle f, g \rangle = g$,
    where~$\pproj_1 = [\id,0]$ and~$\pproj_2=[0,\id]$.
In fact~$\langle f, g\rangle = (\kappa_1 \after f) \ovee (\kappa_2 \after g)$.
Conversely~$h = \langle \pproj_1 \after h, \pproj_2 \after h \rangle$
    for any~$h \colon Z \to X+Y$.
\begin{point}{Proof}%
Assume~$f\colon Z \to X$ and~$g\colon Z \to Y$ with~$1\after f \perp 1\after g$.
By \sref{coproj-total},
we have~$1 \after \kappa_1 \after f = 1 \after f \perp 1 \after g =
        1 \after \kappa_2 \after g$.
Thus~$\kappa_1 \after f \perp \kappa_2 \after g$.
Define~$\langle f, g\rangle = (\kappa_1 \after f) \ovee (\kappa_2 \after g)$.
By the PCM-enrichedness, we have
\begin{align*}
\pproj_1 \after \langle f, g\rangle 
&\  =\  [\id, 0] \after ((\kappa_1 \after f) \ovee (\kappa_2 \after g)) \\
 &\  =\  ([\id, 0] \after \kappa_1 \after f) \ovee 
    ([\id, 0] \after \kappa_2 \after g) \\
    & \ =\  f \ovee (0 \after g) \ = \ f.
\end{align*}
Similarly~$\pproj_2 \after \langle f, g \rangle = g$.
To show uniqueness, assume~$f = \pproj_1 \after h$
    and~$g = \pproj_2 \after h$ for some (other)
    $h\colon Z \to X+Y$.
Note~$\kappa_1 \after \pproj_1 = [\kappa_1, 0]$
and~$\kappa_2 \after \pproj_2 = [0, \kappa_2]$,
so by~\sref{cotupl-pcm} we have~$(\kappa_1 \after \pproj_1)
    \ovee (\kappa_2 \after \pproj_2) = [\kappa_1, \kappa_2] = \id$,
    hence
\begin{align*}
    \langle f, g\rangle & \ = \ 
    \langle \pproj_1 \after h, \pproj_2 \after h \rangle \\
    & \ = \ (\kappa_1 \after\pproj_1 \after h) 
    \ovee (\kappa_2 \after\pproj_2 \after h)  \\
    & \ = \ ((\kappa_1 \after\pproj_1 )
    \ovee (\kappa_2 \after\pproj_2 ))  \after h \\
    & \ = \  h,
\end{align*}
which demonstrates uniqueness.

Finally, let~$h\colon Z\to X+Y$ be any map.
Note~$1 \after \pproj_1 = [1,0]$
    and~$1 \after \pproj_2 = [0,1]$.
So by PCM-enrichment~$1 \after \pproj_1 \after h \perp 1 \after \pproj_2 \after h$ and so~$\pproj_1 \after h \perp \pproj_2 \after h$.
By the previous
$\langle \pproj_1 \after h, \pproj_2 \after h\rangle
= (\kappa_1 \after \pproj_1 \after h) \ovee
 (\kappa_2 \after \pproj_2 \after h) 
 = h $ as desired.\qed
\end{point}
\end{point}

\begin{point}[eff-prod-rules]{Exercise}%
Show that in an effectus in partial form, we have
\begin{enumerate}
    \item~$1 \after \langle f, g\rangle = (1 \after f) \ovee(1 \after g)$;
    \item~$(k+l) \after \langle f, g\rangle
        = \langle k \after f, l \after g \rangle$ \emph{and}
    \item~$\langle f, g\rangle \after k = \langle f \after k,
                                g \after k \rangle$.
\end{enumerate}
\end{point}

\begin{point}[eff-partial-to-total]{Theorem}%
Let~$C$ be an effectus in partial form.
The total maps of~$C$ are an effectus in total form.
\begin{point}{Proof}%
The total maps form a subcategory: $1 \after \id = 1$
    and $1 \after f \after g = 1 \after g = 1$ for composable total~$f,g$.
In \sref{coproj-total} we saw coprojections are total.
By \sref{cotupl-pcm}
    we have~$1 \after [f,g] = [1 \after f, 1 \after g] = [1,1] = 1$
    for total~$f\colon X \to Z$ and~$g\colon Y \to Z$,
    so~$\Tot C$ has binary coproducts.
The unique map~$!\colon 0 \to X$ must be total
    as~$1 \after ! = 1$ is the unique map~$0 \to I$,
    so~$\Tot C$ has initial object~$0$, hence all finite coproducts.
\end{point}
\begin{point}%
To show~$I$ is the final object of~$\Tot C$,
    we need~$\id_I = 1$.
As~$C(I,I)$ is an effect algebra~$1 = \id \ovee \id^\perp$
    for some~$\id^\perp$.
So by PCM-enrichtment~$1 = 1 \after 1 = (1 \after \id) \ovee (1 \after\id^\perp) = 
1 \ovee (1 \after \id^\perp)$.
By the zero--one axiom~$1 \after \id^\perp = 0$.
So~$\id^\perp = 0$ and indeed~$\id = 1$.
To show~$I$ is final in~$\Tot C$, pick any object~$X$ in~$\Tot C$.
We claim~$1 \colon X \to I$ is the unique total map.
Indeed, by the previous $1 = \id_I \after 1 = 1 \after 1$, so~$1$ is total
and if~$h\colon X \to I$ is total, then~$1=1 \after h = \id_I \after h = h$.
\end{point}
\begin{point}%
To show the square on the left of \eqref{pullbacks} is a pullback
    in~$\Tot C$,
assume~$f\colon Z \to X+I$ 
and~$g\colon Z \to I+Y$
are total maps with~$(\id+1) \after g = (1+\id) \after f$.
By~\sref{coprod-prod}, $f = \langle \alpha, a \rangle$
    and~$g = \langle b, \beta \rangle$
    for some maps~$\alpha\colon Z \to X$, $\beta\colon Z \to Y$
        and~$a,b\colon Z \to I$ in~$C$.
\begin{equation*}
\xymatrix@C+1pc@R-1pc{ 
Z \ar@/^1pc/[rrd]^{\langle\alpha,a \rangle}
    \ar@/_1pc/[rdd]_{\langle b, \beta\rangle}
    \ar@{.>}[rd]|{\langle \alpha, \beta\rangle}
    \\
    &  X+Y \ar[r]^{\id+1} \ar[d]_{1+\id} & X+I\ar[d]^{1+\id} \\
    &I+Y\ar[r]_{\id+1} & I+I
    }
\end{equation*}
By~\sref{eff-prod-rules},
    we have~$1 = 1 \after f = 1 \after \langle \alpha, a\rangle
                = (1 \after \alpha) \ovee a$
        and so~$a^\perp = 1\after \alpha$.
Similarly~$b^\perp = 1 \after \beta$.
Again, using~\sref{eff-prod-rules},
    we see
\begin{equation*}
    \langle b, 1 \after \beta \rangle \ =\ 
    (\id + 1) \after \langle b, \beta \rangle \ =\ 
    (\id + 1) \after g \ =\ 
    (1 + \id) \after f
               \  = \ \langle 1 \after \alpha, a\rangle,
\end{equation*}
so~$1 \after \beta = a = (1 \after \alpha)^\perp$,
hence~$1 \after \beta \perp 1 \after \alpha$,
so~$\langle \alpha,\beta\rangle \colon Z \to X+Y$ exists
    and is total as~$1 \after \langle \alpha,\beta\rangle = 
        (1 \after \alpha) \ovee (1 \after \beta) = 1$.
We compute~$(\id + 1) \after \langle\alpha, \beta\rangle = \langle \alpha,
    1 \after \beta\rangle = \langle \alpha, a\rangle = f$.
    Similarly~$(1 + \id) \after \langle \alpha, \beta\rangle = g$.
    Assume~$h\colon Z \to X+Y$ is any (other) map with~$(1 +\id) \after h
        = g$ and~$(\id+1) \after h = f$.
Say~$h = \langle h_1, h_2 \rangle$.
Then~$\langle \alpha, a\rangle = f= (\id + 1)\after h = \langle h_1, 1 \after h_2 \rangle$, so~$\alpha = h_1$. Similarly~$\beta=h_2$.
Thus~$h = \langle \alpha, \beta\rangle$,
    which shows our square is indeed a pullback.
\end{point}
\begin{point}%
To show the square on the right of \eqref{pullbacks} is a pullback
    in~$\Tot C$,
assume (using \sref{coprod-prod})
$\langle \alpha, \beta \rangle\colon Z \to X+Y$
is some total map
with~$(1+1) \after \langle\alpha,\beta\rangle = \kappa_1 \after 1$.
    \begin{equation*}
\xymatrix@C+1pc@R-1pc{ 
    Z \ar@{.>}@/^1pc/[rrd]^{1}
    \ar@/_1pc/[rdd]_{\langle \alpha,\beta\rangle}
    \ar@{.>}[rd]|{\alpha}\\
    &X \ar[r]^{1} \ar[d]_{\kappa_1} & I \ar[d]^{\kappa_1} \\
    &X+Y\ar[r]_{1+1} & I+I
    }
    \end{equation*}
With \sref{eff-prod-rules},
we see~$\langle 1 \after \alpha, 1 \after \beta \rangle
        = (1 + 1) \after \langle \alpha, \beta \rangle
        = \kappa_1 \after 1
        = \langle 1, 0 \rangle$.
So~$\alpha$ is total and~$\beta=0$.
Hence~$\langle \alpha, \beta\rangle = \langle \alpha, 0\rangle
        = \kappa_1 \after \alpha$ as desired.
\end{point}
\begin{point}%
    Finally, to show~$m_1\equiv [\kappa_1,\kappa_2,\kappa_2], m_2 \equiv [\kappa_1,\kappa_2,\kappa_2]\colon
    I+I+I \to I+I$ are jointly monic,
    let~$f_1 \equiv \langle a_1,b_1,c_1\rangle,f_2\equiv \langle a_2,b_2,c_2\rangle\colon X\to I+I+I$ be any total maps
    with~$m_1 \after f_1 = m_1 \after f_2$
    and~$m_2 \after f_1 = m_2 \after f_2$.
Then
\begin{equation*}
    a_1 \ = \ [\id, 0, 0] \after f_1
        \ = \ \pproj_1 \after m_1 \after f_1
        \ = \ \pproj_1 \after m_1 \after f_2
        \ = \ a_2.
\end{equation*}
and similarly from the equality involving~$m_2$, we get~$b_1 = b_2$.
As~$f_1$ is total, we have~$1 = 1 \after f_1=  a_1 \ovee b_1 \ovee c_1$,
    so~$c_1 = (a_1 \ovee b_1)^\perp$.
With the same reasoning~$c_2 = (a_2 \ovee b_2)^\perp$.
Thus~$c_1 = c_2$ and so~$f_1 = f_2$, as desired. \qed
\end{point}
\end{point}
\end{parsec}

\subsection{From total to partial}
\begin{parsec}%
\begin{point}%
Let~$C$ be an effectus in total form.
In this section we will show that~$\Par C$ is an effectus in partial form.
Before we get to work, it is helpful to discuss the axioms
    of an effectus in total form, now we have some experience
    with an effectus in partial form.
\begin{enumerate}
\item
In \sref{coprod-prod} we saw that the coproduct in an effectus
    (in partial form) is almost a biproduct.
This structure is hidden (for the most part)
    in the left pullback square of
    \eqref{pullbacks}, which
    allows the formation of~$\langle \alpha, \beta \rangle$ given
    partial maps~$\alpha,\beta$.
\item
The right pullback square of \eqref{pullbacks}
    is used to extract a total map in~$C$
    from a partial map~$f$ in~$\Par C$ that is total (i.e.~$1 \hafter f = 1$).
\item
The joint-monicity
    of~$[\kappa_1,\kappa_2,\kappa_2]$
    and~$[\kappa_2,\kappa_1,\kappa_2]$
    will imply the joint monicity of~$\pproj_1$ and $\pproj_2$,
    which is required for uniqueness of the partial sum of partial maps.
\end{enumerate}
To prove the theorem, we need to study pullbacks:
    first in any category,
    then in an effectus~$C$ in total form
    and finally in~$\Par C$.
\end{point}
\end{parsec}

\begin{parsec}%
\begin{point}%
    We start with two classic facts about pullbacks.
\end{point}
\begin{point}{Exercise}%
Show that if we have any pullback square --- say
\begin{equation*}
    \vcenter{\vbox{\xymatrix@R-1pc{
        {P\ } \pullback \ar[r]^{m_1} \ar[d]_{m_2}
        & B \ar[d]^f
                \\ {A\ } \ar[r]_{g}
    & X}}} \text{,}
\end{equation*}
then~$m_1$ and~$m_2$ are jointly-monic.
\end{point}

\begin{point}[pullback-lemma]{Exercise}%
Show the \Define{pullback lemma} --- that is:
    if we have a commuting diagram
\begin{equation*}
    \vcenter{\vbox{\xymatrix@R-1pc{
                A \ar[r]^f \ar[d]_k
                & B \ar[r]^g \ar[d]_l
                & C \ar[d]^m
                \\ X \ar[r]_{f'}
                & Y \ar[r]_{g'}
                & Z
    }}} \text{,}
\end{equation*}
\end{point}
then we have the following two implications.
\begin{enumerate}
\item
If the left and right inner squares are pullbacks,
    then so is the outer square.
\item
If the outer square is a pullback
    and~$l$ and~$g$ are jointly monic,
    then the left inner square is a pullback.
\end{enumerate}
\end{parsec}

\begin{parsec}%
\begin{point}%
It is well-known that monos are stable under pullbacks ---
that is:
\begin{equation*}
    \text{if} \quad
    \vcenter{\vbox{\xymatrix@R-1pc{
                {P\ } \pullback \ar[r]^n \ar[d]_g
        & X \ar[d]^f
                \\ {A\ } \ar@{>->}[r]_{m}
    & B}}} ,
    \quad \text{then} \quad
    \vcenter{\vbox{\xymatrix@R-1pc{
                {P\ } \pullback \ar@{>->}[r]^n \ar[d]_g
        & X \ar[d]^f
                \\ {A\ } \ar@{>->}[r]_{m}
    & B}}}.
\end{equation*}
We will need an analogous result for jointly monic maps.
\end{point}
\begin{point}[joint-monicity-stable]{Lemma}%
If the pairs~$(m_1,m_2)$, $(n_1,g_1)$, $(n_2,g_2)$ and~$(h_1,h_2)$
in the following commuting diagram
are jointly-monic (for instance: if they span pullback squares),
then~$(n_1 \after h_1, n_2 \after h_2)$ is jointly-monic as well.
\begin{equation*}
    \xymatrix@C+1pc{
        P \ar[r]^{h_1} \ar[d]_{h_2}
        &P_1\ar[r]^{n_1} \ar[d]^{g_1}
        &X_1 \ar[d]^{f_1}
        \\ P_2\ar[r]_{g_2} \ar[d]_{n_2}
        & A \ar[r]_{m_1} \ar[d]^{m_2}
        & B_1 
        \\ X_2 \ar[r]_{f_2}
        & B_2
    }
\end{equation*}
\begin{point}{Proof}%
To show joint monicity of~$n_1 \after h_1$ and~$n_2 \after h_2$,
assume~$\alpha_1,\alpha_2  \colon Z \to P$
    are maps with~$
        n_1 \after h_1 \after \alpha_1
        = n_1 \after h_1 \after \alpha_2$
    and~$n_2 \after h_2 \after \alpha_1
        = n_2 \after h_2 \after \alpha_2$.
To start
\begin{equation*}
    m_1 \after g_1 \after h_1 \after \alpha_2
    \ =\  f_1 \after n_1 \after h_1 \after \alpha_2 
    \ =\  f_1 \after n_1 \after h_1 \after \alpha_1 
            \ =\  m_1 \after g_1 \after h_1 \after \alpha_1.
\end{equation*}
Reasoning on the other side of the diagram, we find
\begin{align*}
    m_2 \after g_1 \after h_1 \after \alpha_2
    &\ =\  m_2 \after g_2 \after h_2 \after \alpha_2 \\ 
    &\ =\  f_2 \after n_2 \after h_2 \after \alpha_2 \\ 
    &\ =\  f_2 \after n_2 \after h_2 \after \alpha_1 \\ 
    &\ =\  m_2 \after g_2 \after h_2 \after \alpha_1 \\ 
    &\ =\  m_2 \after g_1 \after h_1 \after \alpha_1.
\end{align*}
So by joint monicity of~$m_1$ and~$m_2$,
    we conclude~$g_1 \after h_1 \after \alpha_2
                = g_1 \after h_1 \after \alpha_1$.
So by the joint monicity of~$n_1$ and~$g_1$
    (and~$n_1 \after h_1 \after \alpha_2
    = n_1 \after h_1 \after \alpha_1$ by assumption),
    we conclude~$h_1 \after \alpha_1 = h_1 \after \alpha_2$.
Reasoning in the same way mirrored over the diagonal,
    we find~$h_2 \after \alpha_1 = h_2 \after \alpha_2$.
Thus, by joint monicity of~$h_1$ and~$h_2$, we conclude~$\alpha_1 = \alpha_2$, as desired. \qed
\end{point}
\end{point}
\end{parsec}

\begin{parsec}%
\begin{point}[tot-pullbacks]{Proposition}%
The following squares are pullbacks in an effectus in total form.
\begin{equation*}
    \xymatrix{
        X+A \pullback \ar[r]^{\id+f} \ar[d]_{g + \id}
        & X+B \ar[d]^{g+\id}
        \\ Y+A \ar[r]_{\id+f}
        & Y+B
    } \qquad
    \xymatrix{
        X \pullback \ar[d]_{\kappa_1} \ar[r]^{f}
        & Y \ar[d]^{\kappa_1}
        \\
        X+A \ar[r]_{f+g}
        & Y+B
    }
\end{equation*}
\begin{point}{Proof}%
To start with the left square, consider the following commuting diagram.
\begin{equation*}
    \xymatrix{
        X+A \ar@{}[rd]|{(1)} \ar[r]^{\id+f} \ar[d]_{g + \id}
        & X+B \ar@{}[rd]|{(2)} \ar[d]^{g+\id} \ar[r]^{\id+!}
        & X+1 \ar[d]^{g+\id}
        \\ Y+A \ar@{}[rd]|{(3)} \ar[r]_{\id+f} \ar[d]_{!+\id}
        & Y+B \ar@{}[rd]|{(4)} \ar[r]_{\id+!} \ar[d]^{!+\id}
        & Y+1 \ar[d]^{!+\id}
        \\ 1+A \ar[r]_{\id+f}
        & 1+B \ar[r]_{\id+!}
        & 1+1
    }
\end{equation*}
We want to show (1) is a pullback.
The inner square~(4) and right rectangle~(2,4) are pullbacks by axiom.
Thus the inner square~(2) is also a pullback
    by the pullback lemma, see \sref{pullback-lemma}.
With the same reasoning, we see square (3) is a pullback.
Thus by the pullback lemma, it is sufficient to show that
    left rectangle (1,3) is a pullback.
The left rectangle is indeed a pullback
as both the outer square (1,2,3,4)
    and the right rectangle (2,4) are pullbacks.

For the right square, we consider the following diagram.
\begin{equation*}
    \xymatrix{
         X \ar[r]^f \ar[d]_{\kappa_1}
            %\ar@{}[rd]|{(1)}
        & Y \ar[r]^{!} \ar[d]^{\kappa_1}
            %\ar@{}[rd]|{(2)}
        &1 \ar[d]^{\kappa_1}
        \\ X+A \ar[r]_{f+g}
        & Y+B \ar[r]_{!+!}
        & 1+1
    }
\end{equation*}
The inner right and outer squares are pullbacks by axiom.
So the inner left square is also a pullback by the pullback lemma, as desired.
    \qed
\end{point}
\end{point}
\end{parsec}

\begin{parsec}%
\begin{point}{Definition}%
Assume~$C$ is an effectus in total form.
For a map~$f\colon X \to Y$,
    write~$\hat{f} = \kappa_1 \after f \colon X \to Y+1$.
    (This is the Kleisli embedding~$C \to \Par C$.)
\end{point}
\begin{point}[par-c-coprod]{Exercise}%
Assume~$C$ is an effectus in total form.
Show that if
\begin{equation*}
\kappa_1 \colon X \to X+Y \leftarrow Y \colon \kappa_2
\end{equation*}
    is a coproduct in~$C$, then~
\begin{equation*}
    \hat\kappa_1 \colon X \pto X+Y \pfrom Y \colon \hat\kappa_2
\end{equation*}
    is a coproduct in~$\Par C$.
Show~$0$ is also the initial object of~$\Par C$.
\begin{point}{Beware}%
Cotupling in~$C$ and~$\Par C$ coincide:
$[f,g]_{C} = [f,g]_{\Par C}$
for any (properly typed) partial maps~$f,g$.
Also, we have the following rule for the sum of maps:
    $\widehat{f+_C g} = \hat{f} +_{\Par C} \hat{g}$.
    However, in general
    \begin{equation*}
    [\kappa_1 \after f, \kappa_2 \after g] \ =\  f+_C g \ \neq\  f +_{\Par C} g
        \ =\  [[\kappa_1,\kappa_3] \after f,
            [\kappa_2, \kappa_3] \after g].
    \end{equation*}
\end{point}
\end{point}
\begin{point}[par-pullbacks]{Proposition}%
Assume~$C$ is an effectus in total form.
The following squares are pullbacks in~$\Par C$.
\begin{equation*}
    \xymatrix{
        X+A \pullback \ar@^{>}[r]^{\id+\hat{f}} \ar@^{>}[d]_{\hat{g} + \id}
        & X+B \ar@^{>}[d]^{\hat{g}+\id}
        \\ Y+A \ar@^{>}[r]_{\id+\hat{f}}
        & Y+B
    } \qquad
    \xymatrix{
        A+X \pullback \ar@^{>}[r]^{\hat{f}+\id} \ar@^{>}[d]_{\pproj_1}
        & B+X \ar@^{>}[d]^{\pproj_1}
        \\ A \ar@^{>}[r]_{\hat{f}}
        & B
    }
\end{equation*}
\begin{point}{Proof}%
It is sufficient to show that the following squares
are pullbacks in~$C$.
\begin{equation*}
    \xymatrix{
        X+(A+1) \ar[r]^{\id + (f+\id)} \ar[d]_{g+ \id}
        & X+(B+1) \ar[d]^{g+\id}
        \\ Y+(A+1) \ar[r]_{\id+(f +\id)}
        & Y+(B+1)
    } \qquad
    \xymatrix{
        A+(X+1) \ar[r]^{f+\id} \ar[d]_{\id+!}
        & B+(X+1)  \ar[d]^{\id+!}
        \\ A+1 \ar[r]_{f+\id}
        & B+1
    }
\end{equation*}
These are indeed pullbacks by~\sref{tot-pullbacks}. \qed
\end{point}
\end{point}
\begin{point}[zero-and-one-parc]{Definition}%
Assume~$C$ is an effectus in total form.
For arbitrary objects~$X,Y$ in~$C$,
    write~$0 \equiv \kappa_2 \after !\colon X \to Y+1$
    and~$1 \equiv \kappa_1 \after ! \equiv \hat! \colon X \to 1+1$.
\end{point}
\begin{point}{Exercise}%
    Show~$0$ is a zero object in~$\Par C$
        with unique zero map
        as in \sref{zero-and-one-parc}.
\end{point}
\begin{point}[pardp]{Proposition}%
Assume~$C$ is an effectus in total form
    and~$f$ is a map in~$\Par C$.
\begin{enumerate}
\item
If~$1 \hafter f = 1$,
    then~$f = \hat{g}$ for some~$g$ in~$C$.
\item
If~$0 \hafter f = 0$, then~$f = 0$.
\end{enumerate}
\begin{point}{Proof}%
Both points follow from the right pullback square of~\eqref{pullbacks}
 as follows.
\begin{equation*}
    \xymatrix@R-.8pc{
    X \ar@{.>}[rd]|g
    \ar@/^1pc/[rrd]^{!}
        \ar@/_1pc/[rdd]_f
        \\& Y \pullback
        \ar[r]_{!}
        \ar[d]^{\kappa_1}
    & 1
        \ar[d]^{\kappa_1}
    \\& Y+1
        \ar[r]_{!+!}
&1+1
}
\qquad
    \xymatrix@R-.8pc{
        X \ar@{.>}[rd]|{!}
    \ar@/^1pc/[rrd]^{!}
        \ar@/_1pc/[rdd]_f
        \\& 1 \pullback
        \ar[r]_{!}
        \ar[d]^{\kappa_2}
    & 1
        \ar[d]^{\kappa_2}
    \\& Y+1
        \ar[r]_{!+!}
&1+1
}
\end{equation*}
If~$1 \hafter f = 1$, then~$(!+!)\after f = \kappa_1 \after !$
    and so there is a unique~$g$ with~$f = \kappa_1 \after g$
    as shown above on the left.
For the other point, if~$1 \hafter f = 0$,
    then~$(!+!)\after f = \kappa_2 \after !$
    and so~$f = \kappa_2 \after !$ as shown above on the right. \qed
\end{point}
\end{point}
\begin{point}[pproj-joint-monicity]{Proposition}%
    Assume~$C$ is an effectus in total form.
    The partial projectors~$\pproj_1 \equiv [\id,0]$
        and~$\pproj_2 \equiv [0, \id]$
        are jointly monic in~$\Par C$.
\begin{point}{Proof}%
Consider the following commuting diagram in~$\Par C$.
\begin{equation*}
\xymatrix {
    X+Y \pullback
    \ar@/^1.5pc/[rr]^{\pproj_1}
    \ar@/_3pc/[dd]_{\pproj_2}
        \ar@^{>}[r]^{\id+\hat!}
        \ar@^{>}[d]_{\hat! + \id}
& X+1 \pullback
        \ar@^{>}[r]^{\pproj_1}
        \ar@^{>}[d]^{\hat! + \id}
& X
        \ar@^{>}[d]^{\hat!}
\\ 1+Y \pullback
        \ar@^{>}[r]_{\id+\hat!}
        \ar@^{>}[d]_{\pproj_2}
& 1+1
        \ar@^{>}[r]_{\pproj_1}
        \ar@^{>}[d]^{\pproj_2}
& 1
\\ Y
        \ar@^{>}[r]_{\hat!}
& 1
}
\end{equation*}
Each of the inner squares is a pullback by~\sref{par-pullbacks}.
The maps~$\pproj_1,\pproj_2\colon 1 + 1 \pto 1$
    are jointly monic,
    which is a reformulation
    of the joint monicity axiom of an effectus in total form.
Thus by \sref{joint-monicity-stable}
    the outer~$\pproj_1$ and~$\pproj_2$ are jointly monic. \qed
\end{point}
\end{point}
\end{parsec}

\begin{parsec}%
\begin{point}{Theorem}%
If~$C$ is an effectus in total form,
then~$\Par C$ is an effectus in partial form.
\begin{point}{Proof}%
In \sref{par-c-coprod}, we saw that~$\Par C$ has finite coproducts.
\begin{point}{PCM-enrichment, I}%
Assume~$f,g \colon X \pto Y$.
We say~$f \perp g$
whenever there is a \Define{bound}~$b\colon X \pto Y+Y$
    with~$\pproj_1 \hafter b = f$
    and~$\pproj_2 \hafter b = g$.
    (By \sref{pproj-joint-monicity}
    this~$b$ is unique if it exists.)
In this case, define~$f \ovee g = \nabla \hafter b$,
where $\Define{\nabla} \equiv [\id,\id]$.
We will show~$\ovee$ with~$0$
    as defined in \sref{zero-and-one-parc}
    forms a PCM.
Partial commutativity is obvious.
The map~$\kappa_1 \hafter f\colon X \pto Y+Y$
    is a bound for~$f \perp 0$ and
    so~$f \ovee 0 = \nabla \hafter \kappa_1 \hafter f = f$,
    which shows~$0$
    is indeed a zero for~$\ovee$.
Only partial associativity remains.
Assume~$f \perp g$ via bound~$b$
and~$f \ovee g \perp h$ via bound~$c$.
Note~$\nabla \hafter b = f \ovee g = \pproj_1 \hafter c$
and thus by the right pullback square of \sref{par-pullbacks} (see diagram below),
    there is a unique~$d\colon X \pto (Y+Y)+Y$
    with~$\pproj_1 \hafter d = b$ and~$(\nabla +\id)\hafter d = c$.
\begin{equation*}
    \xymatrix@R-.8pc{
    X \ar@^{.>}[rd]|d
        \ar@/^1pc/[rrd]^c
        \ar@/_1pc/[rdd]_b
        \\& (Y+Y)+Y \pullback
        \ar@^{>}[r]_{\nabla+\id}
        \ar@^{>}[d]^{\pproj_1}
    & Y+Y
        \ar@^{>}[d]^{\pproj_1}
    \\& Y+Y
        \ar@^{>}[r]_{\nabla}
&Y
}
\end{equation*}
The map $(\pproj_2 + \id) \hafter d$
    is a bound for~$g \perp h$,
    indeed:
\begin{align*}
    \pproj_1 \hafter (\pproj_2 + \id) \hafter d
    & \ =\  \pproj_2 \hafter \pproj_1 \hafter d 
    & \pproj_2 \hafter (\pproj_2 + \id) \hafter d
    & \ =\  \pproj_2 \hafter d
    \\
    &\ =\  \pproj_2 \hafter b
    && \ =\  \pproj_2 \hafter (\nabla + \id) \hafter d
    \\
    & \ =\  g
    && \ = \ \pproj_2 \hafter c\\
    &&& \ = \ h.
\end{align*}
We need one more ---
    $[\id,\kappa_2]\hafter d$ is a bound for~$f \perp g \ovee h$:
\begin{align*}
    \pproj_1 \hafter [\id,\kappa_2] \hafter d
        & \ = \ \pproj_1 \hafter \pproj_1 \hafter d
    & \pproj_2 \hafter [\id,\kappa_2]\hafter d & \ =\ \nabla \hafter (\pproj_2 + \id) \hafter d \\
        & \ = \ \pproj_1 \hafter b
        && \ =\  g \ovee h \\
        & \ = \ f.
\end{align*}
Finally,
we compute
\begin{equation*}
f \ovee (g \ovee h)
\ = \ \nabla \hafter [\id, \kappa_2] \hafter d
\ = \ \nabla \hafter (\nabla+ \id) \hafter d
\ = \ \nabla \hafter c
\ = \ (f \ovee g) \ovee h,
\end{equation*}
which shows the partial associativity.
Homsets of~$\Par C$ are indeed PCMs.
\end{point}
\begin{point}{PCM-enrichment, II}%
Assume~$f \perp g$ with bound~$b$.
It is easy to see~$b \hafter k$
    is a bound for~$f \hafter k \perp g \hafter k$
    and consequently~$(f \hafter k) \ovee (g \hafter k)
                = \nabla \hafter b \hafter k = 
                (f \ovee g) \hafter k$.
For the other side:
$(h + h) \hafter b$ is a bound for~$h \hafter f \perp h \hafter g$,
    indeed
\begin{align*}
    \pproj_1 \hafter (h+h) \hafter b
    &\ = \ 
    h\hafter \pproj_1 \hafter b \ =\  h \hafter f \\
    \pproj_2 \hafter (h+h) \hafter b
    &\ = \ 
    h\hafter \pproj_2 \hafter b\  =\  h \hafter g.
\end{align*}
And so~$(h \hafter f) \ovee (h \hafter g)
            = \nabla \hafter (h+h) \hafter b
            = h \hafter \nabla \hafter b = h \hafter (f \ovee g)$.
\end{point}
\begin{point}{FinPAC}%
We just saw~$\Par C$ is PCM-enriched.
We already know~$\Par C$ has coproducts.
The compatible sum axiom holds by definition of~$\perp$.
To show~$\Par C$ is a finPAC, only the untying axiom remains to be proven.
If~$f \perp g$ with bound~$b$,
then~$(\kappa_1 + \kappa_2) \hafter b$ is a bound for~$\kappa_1 \hafter f
    \perp \kappa_2 \hafter g$,
    which proves the untying axiom.
\end{point}
\begin{point}{Effect algebra of predicates}%
Pick any predicate~$p \colon X \to 1+1$.
We define~$p^\perp = [\kappa_2,\kappa_1]\after p$.
($p^\perp$ is~$p$ with swapped outcomes.)
We compute
\begin{align*}
    \pproj_1 \hafter \hat{p}
    &\ = \ [\id,\kappa_2]\after\kappa_1 \after p
        \  = \ p \\
        \pproj_2 \hafter \hat{p}
    &\ = \ [[\kappa_2, \kappa_1],\kappa_2]\after\kappa_1 \after p
        \  = \ [\kappa_2,\kappa_1]\after p \\
    \nabla \hafter \hat{p}
    &\ = \ [[\kappa_1,\kappa_1],\kappa_2]\after\kappa_1 \after p
        \  = \ \kappa_1 \after ! \ = \ 1.
\end{align*}
So~$\hat{p}$ is a bound for~$p \perp p^\perp$
and~$p \ovee p^\perp = 1$.
To show~$p^\perp$ is the unique orthocomplement,
    assume~$p \ovee q = 1$ for some (other)~$q$ via a bound~$b$.
Note~$1 \hafter b =\nabla \hafter b = 1$
    and so by \sref{pardp}
    we know~$b = \hat{c}$ for some~$c\colon X \to 1+1$.
    And so~$p = \pproj_1 \hafter b = [\id, \kappa_2]\after \kappa_1 \after c
                    = c$
                    hence
\begin{equation*}
q \ =\  \pproj_2 \hafter b \ =\  [[\kappa_2, \kappa_1],\kappa_2] \after
    \kappa_1\after c \ =\  [\kappa_2,\kappa_1]\after p \ =\  p^\perp.
\end{equation*}
\end{point}
\end{point}
\end{point}
\end{parsec}

\subsection{Predicates, states and scalars}
\begin{parsec}%
\begin{point}%
For the remainder of this text,
    we will work with effectuses in partial form.
We will write~$1$ instead of~$I$.
\end{point}
\begin{point}{Definition}%
Let~$C$ be an effectus (in partial form).
\begin{enumerate}
\item
A \Define{predicate} on~$X$ is a map~$X \to 1$.
The set of predicates $\Define{\Pred X}$ on~$X$ form an effect algebra.
\item
A \Define{scalar} is a predicate on~$1$, that is: a map~$1 \to 1$.
Write~$\Define{M} \equiv \Pred 1$ for the set of scalars.
We multiply two scalars~$\lambda,\mu \colon 1 \to 1$
    simply by composing~$\lambda \odot \mu \equiv \lambda \after \mu$.
As~$C$ is PCM-enriched, this turns~$M$ into an effect monoid.\TODO{ref}
\item
For a scalar~$\lambda\colon 1\to 1$ and a predicate~$p\colon X \to 1$
    we write~$\lambda \cdot p \equiv \lambda \after p$.
With this scalar multiplication,
    $\Pred X$ turns into a~$M$-effect module.
\item
A \Define{substate} on~$X$ is a map~$\omega\colon 1 \to X$.
A \Define{state} is a total substate.
We denote the set of states on~$X$ by~$\Stat X$.
\end{enumerate}
\end{point}
\end{parsec}

\section{Quotient}
\begin{parsec}%
\begin{point}{Definition}%
Let~$C$ be an effectus.
We say~$C$ is an \Define{effectus with quotients}
    if for each predicate~$p \colon X \to 1$,
    there exists a map~$\xi_p \colon X \to X/_p$
    with~$1 \after \xi_p \leq p^\perp$
    satisfying the following universal property:
\begin{quote}
    for any (other) map~$f\colon X \to Y$
        with~$1 \after f \leq p^\perp$,
        there is a unique map~$f' \colon X/_p \to Y$
        such that~$f' \after \xi_p = f$.
\end{quote}
Any  map with this universal property
    is called a \Define{quotient} for~$p$.
\end{point}
\begin{point}{Example}%
In~$\op{\W{NCPsU}}$ quotients are exactly the same thing
    as contractive filters, see \TODO{...}.
An example of a quotient-map for~$1-a \in \scrA$ is given
    by~$\xi\colon \ceil{a}\scrA \ceil{a} \to \scrA$
    with~$\xi(b) = \sqrt{a} b \sqrt{a}$.
\end{point}
\begin{point}[quotient-basics]{Exercise}%
Show the following basic properties of quotients.
\begin{enumerate}
    \item If~$\xi\colon X \to Y$ is a quotient for~$p$
                and~$\vartheta\colon Y \to Z$ is an isomorphism,
                then~$\vartheta \after \xi$ is a quotient for~$p$
                as well.
    \item Conversely, if~$\xi_1$ and~$\xi_2$
            are both quotients for~$p$,
            then there is unique a isomorphism~$\vartheta$
            with~$\xi_1 = \vartheta \after \xi_2$.
    \item Isomorphisms are quotients (for 0).
    \item Zero-maps are quotients (for 1).
    \item If~$\xi$ is a quotient for~$p$, then~$1\after \xi = p^\perp$
                (Hint: apply the universal property to $p^\perp$).
    \item Quotients are epi.
\end{enumerate}
\begin{point}%
The following proposition is easy to prove,
    but shows an important property of quotients:
    any map~$f$ factors as a total map after a quotient
    for~$(1\after f)^\perp$.
\end{point}
\end{point}
\begin{point}[quotient-total]{Proposition}%
Assume~$\xi_{p^\perp} \colon X \to X/_{p^\perp}$ is a quotient for~$p^\perp$.
For any~$f\colon X \to Z$
    with~$1 \after f = p$,
    there is a unique \emph{total} $g\colon X/_{p^\perp} \to Z$
    with~$f = g \after \xi_{p^\perp}$.
\begin{point}{Proof}%
By definition of quotient, there is a unique~$g\colon X/_{p^\perp} \to Z$
    with~$g \after \xi_{p^\perp} = f$.
Note~$1 \after g \after \xi_{p^\perp} = 1 \after f = p = 1 \after \xi_{p^\perp}$.
Thus, as~$\xi_{p^\perp}$ is epi (\sref{quotient-basics}),
    we conclude~$1 \after g = 1$.
That is: $g$ is total. \qed
\end{point}
\end{point}
\begin{point}[quotients-composition]{Proposition}%
In an effectus with quotients,
    quotients are closed under composition.
\begin{point}{Proof}%
Assume~$\xi_1\colon X \to Y$ is a quotient for~$p^\perp$
    and~$\xi_2\colon Y \to Z$ is a quotient for~$q^\perp$.
We will prove~$\xi_2 \after \xi_1$
    is a quotient for~$(q \after \xi_1)^\perp$.
As our effectus has quotients,
    we can pick a quotient~$\xi \colon X \to X/_{(q \after \xi_1)^\perp}$
        of~$(q \after \xi_1)^\perp$.
First, some preparation.
Note~$1 \after \xi = q \after \xi_1 \leq 1 \after \xi_1 = p$.
Thus by the universal property of~$\xi_1$,
there is a unique map~$h_1\colon Y \to X/_{(q \after \xi_1)^\perp}$
        with~$h_1 \after \xi_1 = \xi$.
As~$1 \after h_1 \after \xi_1 = 1 \after \xi = q \after \xi_1$
    and~$\xi_1$ is epi, we see~$1 \after h_1 = q$.
Thus by \sref{quotient-total},
there is a unique total map~$h_2 \colon Z \to X/_{(q \after \xi_1)^\perp}$
    with~$h_2 \after \xi_2 = h_1$.
Let~$g\colon X/_{(q \after \xi_1)^\perp} \to Z$
    be the unique map such that~$g \after \xi = \xi_2 \after \xi_1$.
We are in the following situation.
\begin{equation*}
    \xymatrix@C+2pc{
        X  \ar[r]^{\xi_1} \ar@/_/[rrd]_{\xi}
        & Y \ar[r]^{\xi_2} \ar@{.>}[rd]_{h_1}
        & Z \ar@{.>}@/^/[d]^{h_2} \\
        && X/_{(q \after \xi_1)^\perp} \ar@{.>}@/^/[u]^{g}
    }
\end{equation*}
We claim~$g$ is an isomorphism with inverse~$h_2$.
Indeed: from $g \after h_2 \after \xi_2 \after \xi_1= g \after \xi = \xi_2 \after \xi_1$
    we get~$g \after h_2 = \id$
    and from~$h_2 \after g \after \xi = h_2 \after \xi_2 \after \xi_1 = \xi$
    we find~$h_2 \after g = \id$.
Thus~$\xi_2 \after \xi_1 = g \after \xi$ for an isomorphism~$g$,
which shows~$\xi_2 \after \xi_1$ is a quotient, see \sref{quotient-basics}. \qed
\end{point}
\end{point}
\begin{point}{Exercise}
By \sref{quotient-total} we know that each
    map~$f$ factors as~$t \after \xi$
    for some total map~$t$ and quotient~$\xi$.
Show that this forms an orthogonal factorization system ---
    that is: prove that if~$t' \after \xi' = t \after \xi$
    for some (other) quotient~$\xi'$ and total map~$t'$,
    then there is a unique isomorphism~$\vartheta$
    with~$\xi' = \vartheta \after \xi$
    and~$t = t' \after \vartheta$.
\end{point}
\end{parsec}

\section{Comprehension and images}
\begin{parsec}%
\begin{point}{Definition}%
Let~$C$ be an effectus.
We say~$C$ has comprehension
    if for each predicate~$p \colon X \to 1$,
    there exists a map~$\pi_p \colon \cmpr{X}{p} \to X$
    with~$p \after \pi_p = 1 \after \pi_p$
    satisfying the following universal property:
\begin{quote}
    for any (other) map~$g\colon Z \to X$
        with~$p \after g = 1 \after g$,
        there is a unique map~$g' \colon Z \to \cmpr{X}{p}$
        such that~$\pi_p \after g' = g$.
\end{quote}
Any  map with this universal property
    is called a \Define{comprehension} for~$p$.
\begin{point}{Beware}%
    In this text we do not assume comprehensions are total
        (in contrast to \TODO{}).
\end{point}
\begin{point}[compr-not]{Notation}%
Unless otherwise specified~$\pi_p$ will be some
    comprehension of~$p$.
\end{point}
\end{point}
\begin{point}{Example}%
In~$\op{\W{NCPsU}}$ comprehensions are exactly the same thing
as corners, see \TODO{...}.
An example of a comprehension for~$a \in \scrA$
    is given by~$\pi\colon \scrA \to \floor{a}\scrA\floor{a}$
    with~$\pi(b) = \floor{a}b\floor{a}$.
\end{point}
\begin{point}[compr-basics]{Exercise}%
Show the following basic properties of comprehensions.
\begin{enumerate}
    \item If~$\pi\colon X \to Y$ is a comprehension for~$p$
                and~$\vartheta\colon Z \to X$ is an isomorphism,
                then~$\pi \after \vartheta$ is a comprehension for~$p$
                as well.
    \item Conversely, if~$\pi_1$ and~$\pi_2$
            are both comprehensions for~$p$,
            then there is unique a isomorphism~$\vartheta$
            with~$\pi_1 = \pi_2 \after \vartheta$.
    \item Isomorphisms are comprehensions (for 1).
    \item Zero-maps are comprehensions (for 0).
    \item Comprehensions are mono.
    \item $p^\perp \after \pi = 0$ if~$\pi$ is a comprehension of~$p$.
\end{enumerate}
\end{point}
\end{parsec}

\begin{parsec}%
\begin{point}%
We are ready to define purity in effectuses.
\end{point}
\begin{point}{Definition}%
In an effectus a map~$f$ is called \Define{pure}
    if~$f = \pi \after \xi$
    for some comprehension~$\pi$
    and quotient~$\xi$.
\end{point}
\begin{point}{Example}%
In $\op{\W{NCPsU}}$
the pure maps~$\scrB(\scrH) \to \scrB(\scrK)$
    are exactly the maps of the form~$\ad_T$
    where~$T$ is a contractive map~$\scrK \to \scrH$. \TODO{ref}
\end{point}
\begin{point}%
We do not have a good handle on the structure of pure maps in
    an arbitrary effectus.  We will need a few additional assumptions.
\end{point}
\end{parsec}

\begin{parsec}%
\begin{point}{Definition}%
Let~$C$ be an effectus.
\begin{enumerate}
\item
We say~$C$ has \Define{images}
    if for each map~$f\colon X \to Y$,
    there is a least predicate~$\Define{\IM f}$ on~$Y$
        with the property~$(\IM f) \after f = 1 \after f$ ---
        that is, 
    for every (other) predicate~$p$ on~$Y$
    with~$p \after f = 1 \after f$,
    we must have~$\IM f \leq p$.
For brevity, write~$\Define{\IMperp f} \equiv (\IM f)^\perp$. 
Note~$\IMperp f$ is the greatest predicate
with the property~$(\IMperp f) \after f = 0$.
\item
We say a map~$f\colon X \to Y$ is \Define{faithful}
    if~$\IM f = 1$.
That is: $f$ is faithful if and only if
    $p \after f = 0$ implies~$f = 0$
    for every predicate~$p$.
\end{enumerate}
\begin{point}{Notation}%
The expression~$\IM f \after g$
    is read as~$\IM (f \after g)$.
\end{point}
\begin{point}%
The predicate~$\IM f$ will play for comprehension the same role
    as~$1 \after f$ plays for quotients.
Similarly faithful is the analogue of total.
\end{point}
\end{point}
\begin{point}{Example}%
In $\op{\W{NCPsU}}$ the image of a map~$f\colon \scrA \to \scrB$
    is given by the least projection~$e \in \scrA$
    with the property~$f(1-e) = 0$.
This~$e$ is the complement of the projection onto the kernel of~$f$.
The map~$f$ is faithful if and only
    if~$f(a^*a)=0$ implies~$a^*a = 0$
    for all~$a\in \scrA$.
\end{point}
\begin{point}[im-ineq]{Exercise}%
Show~$\IM f \after g \leq \IM f$.
Conclude~$\IM f \after \alpha = \IM f$
    for any iso~$\alpha$.
\end{point}
\begin{point}%
In contrast to quotients,
    it is not clear at all whether (without additional assumptions)
    comprehensions are total;
    they are part of a factorization system
    or are closed under composition.
\end{point}
\begin{point}[compr-total]{Lemma}%
In an effectus with quotients, comprehensions are total.
\begin{point}{Proof}%
Assume~$\pi$ is some comprehension for~$p$.
Let~$\xi$ be a quotient for~$(1 \after \pi)^\perp$.
There is a total~$\pi_t$ with~$\pi = \pi_t \after \xi$.
As~$ 1 \after \pi_t \after \xi
        = 1 \after \pi
        = p \after \pi
        = p \after \pi_t \after \xi$
        and~$\xi$ is epi,
        we see~$1 \after \pi_t = p \after \pi_t$.
Thus, as~$\pi$ is a comprehension for~$p$,
    there exists an~$f$ with~$\pi_t = \pi \after f$.
Now~$\pi = \pi_t \after \xi = \pi \after f \after \xi$
    and so~$\id = f \after \xi$, since~$\pi$ is mono.
Hence~$1 = 1 \after \id =1 \after f \after \xi \leq 1 \after \xi = 1 \after \pi$
    and so~$\pi$ is total. \qed
\end{point}
\end{point}
\begin{point}{Proposition}%
Assume~$\pi_s\colon \cmpr{X}{s} \to X$ is a comprehension for~$s$.
For any~$f\colon Y \to X$ with~$\IM f = s$,
    there is a unique \emph{faithful}~$g\colon Y \to \cmpr{X}{s}$
    with~$\pi \after g = f$.
\begin{point}{Proof}
There is some~$g \colon Y \to \cmpr{X}{s}$ with~$\pi_s \after g = f$. \qed
\TODO{}
\end{point}
\end{point}
\end{parsec}
\subsection{Sharp predicates}
\begin{parsec}%
\begin{point}{Definition}%
Let~$C$ be an effectus with comprehension and images.
\begin{enumerate}
\item
    We say a predicate~$p$ is (image) \Define{sharp}
        if~$p = \IM f$ for some map~$f$.
    Write~$\SPred X$ for the set of all sharp predicates on~$X$.
\item
We define~$\Define{\floor{p}} = \IM \pi_p$,
    where~$\pi_p$ is some comprehension for~$p$.
(Comprehensions for the same predicate have the same image
    by \sref{im-ineq}.)
Also write~$\Define{\ceil{p}} = \floor{p^\perp}^\perp$.
\end{enumerate}
\begin{point}{Beware}%
In \TODO{...} $p$ is called sharp whenever~$p \wedge p^\perp=0$.
In general this is weaker than image-sharpness, which we use in this text.
In~\sref{floor-basics}
    we will see~$\floor{p}$ is sharp.
It is unclear whether~$\ceil{p}$ is sharp (without additional
    assumptions).
\end{point}
\end{point}
\begin{point}{Example}%
In $\op{\W{NCPsU}}$ the sharp predicates are the projections.
For a predicate~$a \in \scrA$,
    the sharp predicate~$\ceil{a}$
    is the least projection above~$a$.
\end{point}
\begin{point}[floor-basics]{Lemma}%
In an effectus with comprehension and images, we have
\begin{multicols}{2}
\begin{enumerate}
\item
    $\floor{p} \leq p$
\item
    $\pi_p = \pi_{\floor{p}} \after \alpha$
        for some iso~$\alpha$;
\item
    $\floor{\floor{p}} = \floor{p}$;
\item
    $p \leq q$ $\implies$ $\floor{p} \leq \floor{q}$;
\item
    $\ceil{p} \after f \leq \ceil{p \after f}$ \emph{and}
\item
    $\ceil{p} \after f =0$ iff~$p \after f = 0$,
\end{enumerate}
\end{multicols}
\noindent for any map~$f\colon X \to Y$,
    predicates~$p,q$ on~$X$ and
$\pi_p$ and~$\pi_{\floor{p}}$
    comprehensions for~$p$ and~$\floor{p}$ respectively.
\begin{point}{Proof}%
We will prove the statements in listed order.
\begin{point}{Ad 1}%
    Let~$\pi$ be a comprehension for~$p$.
    By definition~$p \after \pi = 1 \after \pi$.
    Thus~$\floor{p} = \IM \pi \leq p$, as desired.
\end{point}
\begin{point}{Ad 2}%
It is sufficient to show~$\pi_p$ is a comprehension for~$\floor{p}$.
First, note~$\floor{p}\after \pi_p = (\IM \pi_p) \after \pi_p = 1 \after \pi_p$.
To show the universal property,
    assume~$g\colon Z \to X$
    is some map with~$\floor{p} \after g = 1 \after g$.
Then~$1 \after g = \floor{p} \after g \leq p \after g \leq 1 \after g$
    and so~$1 \after g = p \after g$.
As~$\pi_p$ is a comprehension for~$p$,
    there is a unique~$g'$ with~$\pi_p \after  g' = g$
    and so~$\pi_p$ is indeed a comprehension for~$\floor{p}$ as well.
\end{point}
\begin{point}{Ad 3}%
Follows from the previous and \sref{im-ineq}.
\end{point}
\begin{point}{Ad 4}%
Pick a comprehension~$\pi_p$ for~$p$ and~$\pi_q$ for~$q$.
Note~$1 \after \pi_p = p \after \pi_p
                \leq q \after \pi_p \leq 1 \after \pi_p $
so~$q \after \pi_p = 1 \after \pi_p$
and thus~$\pi_p = \pi_q \after f$ for some~$f$.
By~\sref{im-ineq}
    we see~$\floor{p} = \IM \pi_p = \IM \pi_q \after f \leq \IM \pi_q = \floor{q}$.
\end{point}
\begin{point}{Ad 5}%
Clearly~$p \after f \after \pi_{(p \after f)^\perp} = 0$.
(Recall our convention \sref{compr-basics} for $\pi$'s.)
Thus there is some~$h$
with~$f \after \pi_{(p \after f)^\perp} = \pi_{p^\perp}\after h$.
By point 2 there is some (isomorphism)~$\alpha$
    with~$\pi_{p^\perp} =\pi_{\floor{p^\perp}} \after \alpha$.
    We compute
\begin{equation*}
    \ceil{p} \after f \after \pi_{(p \after f)^\perp}
        = \ceil{p} \after \pi_{p^\perp}\after h
        = \ceil{p} \after \pi_{\floor{p^\perp}}\after \alpha\after h
    = \ceil{p} \after \pi_{\ceil{p}^\perp}\after \alpha\after h
    = 0.
\end{equation*}
Thus~$\ceil{p} \after f \leq \IMperp \pi_{(p \after f)^\perp}
= \lfloor(p \after f)^\perp\rfloor^\perp = \ceil{p \after f}$, as promised.
\end{point}
\begin{point}{Ad 6}%
Assume~$\ceil{p}\after f = 0$.
From~$p \leq \ceil{p}$
it follows~$p \after f \leq \ceil{p} \after f = 0$.
Thus~$p \after f = 0$.
For the converse, assume~$p \after f = 0$.
Then~$\ceil{p \after f} = \ceil{0} = (\IM \id)^\perp = 0$
    as~$\id$ is a comprehension for~$1$.
    Thus~$\ceil{p} \after f \leq \ceil{p \after f} = 0$. \qed
\end{point}
\end{point}
\end{point}
\begin{point}[img-of-compr]{Exercise}%
Let~$C$ be an effectus with comprehension and images.
Show that~$p$ is sharp if and only if~$\floor{p}= p$.
Conclude~$\IM \pi_s = s$ for sharp~$s$.
\end{point}
\begin{point}[ceiling-within-ceiling]{Exercise}%
Show that in an effectus with comprehension and images
    we have the equality
    $\ceil{\ceil{p}\after f} = \ceil{p \after f}$
    for any map~$f\colon X \to Y$ and predicate~$p$ on~$X$.
\end{point}
\begin{point}[compr-is-full]{Lemma}%
In an effectus with comprehension and images
    we have
\begin{equation*}
    s \leq t
    \quad \iff \quad
    \pi_s = \pi_t \after h
    \quad \text{for some $h$},
\end{equation*}
for all sharp predicates~$s,t$ on the same object.
\begin{point}{Proof}%
Assume~$\pi_s = \pi_t \after h$.
Then
\begin{equation*}
    s \ \overset{\sref{img-of-compr}}{=}\  \floor{s} \ =\  \IM \pi_s \  = \ \IM \pi_t \after h
    \ \overset{\sref{im-ineq}}{\leq} \ \IM \pi_t = \floor{t} 
            \ \overset{\sref{img-of-compr}}{=} \ t.
\end{equation*}
as desired.
Conversely assume~$s \leq t$.
Then~$t^\perp \after \pi_s \leq s^\perp \after \pi_s = 0$
    and so~$\pi_s = \pi_t \after h$
    for some~$h$ by the universal property of~$\pi_t$. \qed
\end{point}
\end{point}
\end{parsec}

\section{$\diamond$-effectus}
\begin{parsec}%
\begin{point}%
In quantum physics it is not uncommon to restrict one's attention
    to sharp predicates (i.e.~projections).
Does this restriction hurt the expressivity?
A partial answer is given by Gleason's famous theorem:
    it states that every (normal) state on~$\scrB(\scrH)$
    is determined by its values on projections
    (if~$\dim \scrH \geq 3$).
We take the idea of restricting oneself to projections
    one step further: we also want to restrict to projections
    for our outcomes.
It is rare for quantum processes to send projections to projections
    (see \TODO{mult}),
    so instead we consider the least projection above the outcome.
The simple idea of taking the restriction to sharp predicates
    seriously, leads to a host of interesting new notions.
We will give these right of the bat and study their relevance later on.
\end{point}
\begin{point}{Definition}%
A~\Define{$\diamond$-effectus}
    is an effectus with quotient, comprehension and images
    such that~$s^\perp$ is sharp for every sharp predicate~$s$.
In a~$\diamond$-effectus
    we define for any map~$f\colon X \to Y$
    the following restrictions to sharp predicates
    \begin{equation*}
        \xymatrix{
            \SPred X  \ar@/^/[r]^{f_\diamond}
            & \SPred Y \ar@/^/[l]^{f^\diamond}}
            \quad \text{by} \quad
            \Define{f^\diamond(s)} = \ceil{s \after f}
            \quad
            \text{and}
            \quad
            \Define{f_\diamond(s)} = \IM f \after \pi_s.
    \end{equation*}
    \begin{enumerate}
        \item
    We say maps~$f \colon X \leftrightarrows Y\colon g$
        are~\Define{$\diamond$-adjoint}
        if~$f^\diamond = g_\diamond$.
    \item
    An endomap~$f\colon X \to X$ is \Define{$\diamond$-self-adjoint}
        if~$f$ is $\diamond$-adjoint to itself.
    \item
    Two maps~$f,g\colon X \to Y$
        are~\Define{$\diamond$-equivalent}
        if~$f^\diamond = g^\diamond$.
    \item
        A pure endomap~$f$ is~\Define{$\diamond$-positive}
            if $f = g\after g$ for some~$\diamond$-self-adjoint~$g$.
    \end{enumerate}
For brevity, write $\Define{f^\BOX(s)} = f^\diamond(s^\perp)^\perp$.
\end{point}
\end{parsec}
\begin{parsec}%
\begin{point}%
Let's investigate the basic properties of~$(\ )^\diamond$, $(\ )_\diamond$
    and $(\ )^\BOX$.
\end{point}
\begin{point}{Exercise}%
Show that in a~$\diamond$-effectus
    both~$f^\diamond$ and~$f^\BOX$ are order preserving maps.
\end{point}
\begin{point}[diamond-adjunction]{Proposition}%
For~$f\colon X \to Y$ in a~$\diamond$-effectus we have
\begin{equation}
    f^\diamond(s) \ \leq\  t^\perp
    \quad \iff
    \quad f_\diamond(t) \ \leq\  s^\perp \label{diamond-main-lemma}
\end{equation}
for all sharp~$s,t$.
In other words: $f_\diamond$ is the left order-adjoint of~$f^\BOX$.
\begin{point}{Proof}%
To start, let's prove the order-adjunction reformulation
\begin{equation*}
    f_\diamond(s) \leq t
    \ \overset{\eqref{diamond-main-lemma}}{\iff} \ 
    f^\diamond(t^\perp) \equiv f^\BOX(t)^\perp \leq s^\perp
            \ \iff \
    s \leq f^\BOX(t).
\end{equation*}
To prove \eqref{diamond-main-lemma},
first assume~$f^\diamond(s) \leq t^\perp$.
Then~$s \after f \leq \ceil{s \after f} = f^\diamond(s) \leq t^\perp
= \IMperp \pi_t$, where the last equality is
due to \sref{img-of-compr}.
Thus~$s \after f \after \pi_t = 0$
    which is to say~$s \leq \IMperp f \after \pi_t$,
    so~$f_\diamond(t) = \IM f \after \pi_t \leq s^\perp$.

For the converse, assume
    $f_\diamond(t) \leq s^\perp$.
Then as before (but in the other direction)
    we find~$s \after f \after \pi_t = 0$
    and so~$s \after f \leq t^\perp$.
    Hence~$f^\diamond(s) =  \ceil{s \after f} \leq \lceil t^\perp\rceil
                = \floor{t}^\perp = t^\perp$, as desired. \qed
\end{point}
\end{point}
\begin{point}[order-adj-basics]{Exercise}%
Use the fact that there is an
order adjunction between~$f_\diamond$ and $f^\BOX$
to show that in a~$\diamond$-effectus
\begin{multicols}{2}
\begin{enumerate}
    \item $f_\diamond$ is order-preserving;
    \item $f_\diamond$ preserves suprema;
    \item $f^\BOX$ preserves infima;
    \item $f^\diamond$ preserves suprema;
    \item $f_\diamond \after f^\BOX \after f_\diamond = f_\diamond$ \emph{and}
    \item $f^\BOX \after f_\diamond \after f^\BOX = f^\BOX$.
\end{enumerate}
\end{multicols}
\end{point}
\begin{point}[diamond-functor]{Lemma}%
In a~$\diamond$-effectus,
we have
\begin{multicols}{3}
\begin{enumerate}
\item
$(\id)^\diamond = \id$,
\item
$(f\after g)^\diamond
            =   g^\diamond \after f^\diamond$,
\item
$(\id)^\BOX = \id$,
\item
$(f\after g)^\BOX
            =   g^\BOX \after f^\BOX$,
\item
$(\id)_\diamond= \id$ and
\item
            $(f \after g)_\diamond = f_\diamond \after g_\diamond$.
\end{enumerate}
\end{multicols}
\begin{point}{Proof}%
We get~$(\id)^\diamond = \id$
directly from~\sref{img-of-compr}.
For 2 we only need a single line:
\begin{equation*}
(f\after g)^\diamond(s)
    \ =\  \ceil{s \after f \after g}
    \ \overset{\sref{ceiling-within-ceiling}}{=}\ 
    \ceil{\ceil{s \after f} \after g}
    \ =\  g^\diamond(f^\diamond(s)).
\end{equation*}
Note that we used that ceilings are sharp.
Point 3 and 4 follow easily from~1 and 2 respectively.
The identity~$(\id)_\diamond = \id$ is again~\sref{img-of-compr}.
We claim~$f_\diamond \after g_\diamond$
    is left order-adjoint to~$(f\after g)^\BOX$,
     indeed
\begin{equation*}
    f_\diamond (g_\diamond(s)) \leq t
        \ \iff\  g_\diamond(s) \leq f^\BOX (t)
        \ \iff\  s \leq g^\BOX (f^\BOX (t)) = (f \after g)^\BOX(t).
\end{equation*}
Thus by uniqueness of order adjoints, we find
$(f\after g)_\diamond = f_\diamond \after g_\diamond$. \qed
\end{point}
\end{point}
\end{parsec}
\begin{parsec}%
\begin{point}{Theorem}%
\TODO{attribution for this thm}
In a~$\diamond$-effectus,
the poset~$\SPred X$ of sharp predicates on~$X$,
    is an orthomodular lattice.
\begin{point}{Proof}%
Let~$C$ be a~$\diamond$-effectus with some object~$X$.
\begin{point}[spred-infimum]{Infima}%
Let~$s,t$ be sharp predicates on~$X$.    
We will show~$(\pi_s)_\diamond (\pi_s^\BOX (t))$
is the infimum of~$s$ and~$t$.
It is easy to see
$(\pi_s)_\diamond (\pi_s^\BOX (t))$ is a lower bound:
$(\pi_s)_\diamond (\pi_s^\BOX (t)) \leq t$
and~$(\pi_s)_\diamond (\pi_s^\BOX (t)) \leq (\pi_s)_\diamond(1)
    = \IM \pi_s = s$.
We have to show $(\pi_s)_\diamond (\pi_s^\BOX (t))$ is the greatest lower bound.
Let~$r$ be any sharp predicate with~$r \leq s$ and~$r \leq t$.
By~\sref{compr-is-full}
    we have~$\pi_r = \pi_s \after h$ for some~$h$.
Thus
\begin{equation*}
    (\pi_r)_\diamond
        \ = \  (\pi_s)_\diamond \after h_\diamond
        \ =\  (\pi_s)_\diamond \after \pi_s^\BOX  
            \after (\pi_s)_\diamond \after h_\diamond
        \ =\  (\pi_s)_\diamond \after \pi_s^\BOX \after (\pi_r)_\diamond.
\end{equation*}
Hence~$r = (\pi_r)_\diamond(1)  
=((\pi_s)_\diamond \after \pi_s^\BOX \after (\pi_r)_\diamond )(1)
=  (\pi_s)_\diamond ( \pi_s^\BOX (r))
\leq  (\pi_s)_\diamond ( \pi_s^\BOX (t))$.
\end{point}
\begin{point}[spred-ortholattice]{Ortholattice}%
Thanks to the order anti-isomorphism~$(\ )^\perp$
suprema (satisfying De Morgan's laws) come for free:
    $s \vee t = (s^\perp \wedge t^\perp)^\perp$.
Now note
\begin{equation*}
\pi_s^\BOX(s^\perp) \ =
    \  (s \after \pi_s)^\perp 
    \ =\  (1 \after \pi_s)^\perp 
    \ \overset{\sref{compr-total}}{=}\ 1^\perp \ =\   0
\end{equation*}
and so~$s\wedge s^\perp = 0$.
We have shown~$\SPred X$ with~$(\ )^\perp$ is an ortholattice.
\end{point}
\begin{point}{Orthomodular law}%
Assume~$s \leq  t$.
We have to show~$s \vee (s^\perp \wedge t) = t$.
Indeed
\begin{alignat*}{2}
s \vee (s^\perp \wedge t) 
&\  =\  (s \wedge t) \vee (s^\perp \wedge t) &\qquad& \text{as $s \leq t$}\\
&\  =\ 
        ((\pi_t)_\diamond \after \pi_t^\BOX)
        (s) \vee
        ((\pi_t)_\diamond \after \pi_t^\BOX)
        (s^\perp)
        &\qquad& \text{by \sref{spred-infimum}}\\
&\  =\ 
        (\pi_t)_\diamond ( \pi_t^\BOX(s) \vee 
                            \pi_t^\BOX(s^\perp))
                            &\qquad& \text{by \sref{order-adj-basics}} \\
&\  =\ 
        (\pi_t)_\diamond ( \pi_t^\diamond(s) \vee 
                            \pi_t^\diamond(s)^\perp)
                            &\qquad& \text{as $\pi_t$ total,~\sref{compr-total}}\\
&\  =\ 
                            (\pi_t)_\diamond (1)
                            &\qquad& \text{by \sref{spred-ortholattice}} \\
& \ =\  \IM \pi_t \ = \ t && \text{by \sref{img-of-compr}}.
\end{alignat*}
We have shown~$\SPred X$ is an orthomodular lattice. \qed
\end{point}
\end{point}
\begin{point}{Corollary}%
In a $\diamond$-effectus~$C$
the assignment~$X\mapsto \SPred X$, ~$f \mapsto (f_\diamond,f^\BOX)$
    yields a functor from~$C$
    to~$\mathsf{OMLatGal}$,
    the category of orthomodular lattices
    with galois connection between them,
    as defined in~\cite{jacobs2009orthomodular}.
\end{point}
\end{point}
\begin{point}[image-sharp-is-order-sharp]{Lemma}%
In a $\diamond$-effectus,
sharp predicates are \Define{order sharp} ---
    that is:
    for any predicate~$p$
    and sharp predicate~$s$
    with~$p \leq s$ and~$p \leq s^\perp$,
    we must have~$p = 0$.
\begin{point}{Proof}%
Note~$\ceil{p} \leq \ceil{s} = s$
    and~$\ceil{p} \leq \lceil s^\perp\rceil = s^\perp$.
So by~\sref{compr-is-full},
there is an~$h$ with~$\pi_{\ceil{p}} = \pi_s \after h$.
We compute
\begin{equation*}
    1 \after \pi_{\ceil{p}} \ =\  \ceil{p} \after \pi_{\ceil{p}}  \ = \
    \ceil{p} \after \pi_s \after h \ \leq \ 
        s^\perp \after \pi_s \after h \ = \ 0.
\end{equation*}
Thus~$\pi_{\ceil{p}} = 0$
and so~$p \leq \ceil{p} = \IM\pi_{\ceil{p}} = 0$, as desired.\qed
\end{point}
\end{point}
\end{parsec}

\begin{parsec}%
\begin{point}%
We turn to~$\diamond$-adjointness.
\end{point}
\begin{point}{Exercise}%
Show the following basic properties of~$\diamond$-adjointness
\begin{enumerate}
    \item
$f^\diamond = g_\diamond$
    ($f$ is $\diamond$-adjoint to~$g$)
    if and only if~$f_\diamond = g^\diamond$.
    \item
If~$f$ and~$g$ are~$\diamond$-adjoint,
    then~$\IM f = \ceil{1 \after g}$.
\end{enumerate}
\end{point}
\begin{point}[diamond-squares]{Exercise}%
Show in order:
\begin{enumerate}
\item
If~$f$ is~$\diamond$-self-adjoint,
    then~$f \after f$ is~$\diamond$-self-adjoint.
\item
If~$f$ is~$\diamond$-positive,
    then~$f$ is~$\diamond$-self-adjoint.
\item
If~$f$ is~$\diamond$-positive,
    then~$f \after f$ is~$\diamond$-positive.
\end{enumerate}
\end{point}
\begin{point}[iso-diamond-adjoint]{Lemma}%
Let~$\alpha$ be an isomorphism in a~$\diamond$-effectus.
Then
\begin{enumerate}
\item
    $s\after\alpha$ is sharp for sharp predicates~$s$ \emph{and}
\item
    $\alpha^\diamond(s) = s \after \alpha$
    and~$\alpha_\diamond(s) = s\after \alpha^{-1}$
    (so~$\alpha$ and~$\alpha^{-1}$ are~$\diamond$-adjoint).
\end{enumerate}
\begin{point}{Proof}%
Let~$s$ be a sharp predicate. Then~$s = \IM \pi_s$.
Note~$\IM \alpha^{-1}\after \pi_s = s \after \alpha$
    --- indeed, $s \after \alpha \after \alpha^{-1} \after \pi_s=1$
    and when~$p \after \alpha^{-1} \after \pi_s=1$,
    we must have~$p \after \alpha^{-1} \geq s$,
    which gives~$p \geq s \after \alpha$ as desired.
    So~$s \after \alpha$ is indeed sharp.

So~$\alpha^\diamond(s) = \ceil{s \after \alpha} = s\after \alpha$
    and~$\alpha_\diamond(s) = \IM \alpha \after \pi_s = s \after \alpha^{-1}$
    as promised. \qed
\end{point}
\end{point}
\end{parsec}

\begin{parsec}%
\begin{point}[sharp-map]{Definition}%
A map~$f$ in a~$\diamond$-effectus is a~\Define{sharp map}
    provided~$s \after f$ is sharp for all sharp predicates~$s$.
\end{point}
\begin{point}[sharp-ceil]{Exercise}%
Show that the following are equivalent.
\begin{enumerate}
    \item $f$ is a sharp map.
    \item $\ceil{p \after f} = \ceil{p}\after f$
            for every predicate~$p$.
\end{enumerate}
\end{point}

\begin{point}{Example}%
In~$\op{\W{NCPsU}}$ the sharp maps are exactly the~MNIU-maps
(i.e.~the normal $*$-homomorphisms). \TODO{ref}
\end{point}
    
\end{parsec}

\section{$\&$-effectus}
\begin{parsec}%
\begin{point}%
In a~$\diamond$-effectus
    quotient and comprehension are not tied together by its axioms.
    With two additional axioms, we will see quotient and comprehension
    become tightly interwoven. \TODO{contrast old attempts}
\end{point}
\begin{point}{Definition}%
A~\Define{$\&$-effectus}
is a $\diamond$-effectus
such that
\begin{enumerate}
\item
    for each object~$X$
    and each predicate on~$p$,
    there is a unique $\diamond$-positive map~$\Define{\asrt_p}\colon X \to X$
    with~$1 \after \asrt_p = p$ \emph{and}
\item
    for every quotient~$\xi \colon X \to Y$
    and comprehension~$\pi \colon Y \to Z$
    the composite~$\pi \after \xi$ is pure.
\end{enumerate}
In an $\&$-effectus,
we define~$\Define{\andthen{p}{q}} = q \after \asrt_p$
    and~$\Define{p^2} = \andthen{p}{p}$.
\end{point}
\begin{point}{Example}%
The category$\op{\W{NCPsU}}$ is a~$\&$-effectus
    with~$\andthen{a}{b} = \sqrt{a}b\sqrt{a}$. \TODO{ref}
\end{point}
\begin{point}[upm-basics]{Proposition}%
In an $\&$-effectus
    the following holds.
\begin{enumerate}
\item
For a predicate~$p$,
the following are equivalent
\begin{inparaenum}
\item $p$ is sharp,
\item $\andthen{p}{p}=p$ and
\item $\asrt_p \after \asrt_p = \asrt_p$.
\end{inparaenum}
\item
Let~$s$ be any sharp predicate.
There exist comprehension~$\pi_s$ of~$s$
        and quotient~$\zeta_s$ of~$s^\perp$
        such that~$\zeta_s \after \pi_s = \id$
        and~$\pi_s \after \zeta_s = \asrt_s$.

In fact, for every comprehension~$\pi$ of~$s$,
    we can find a quotient~$\xi$ of~$s^\perp$
    with~$\xi \after \pi = \id$ and~$\pi \after \xi = \asrt_s$
    \emph{and} conversely for every quotient~$\xi$ of~$s^\perp$
    there exists a comprehension~$\pi$ of~$s$
    with~$\xi \after \pi = \id$ and~$\pi \after \xi = \asrt_s$.
\item Comprehensions are closed under composition.
\item Pure maps are closed under composition.
\end{enumerate}
\begin{point}{Proof}%
We prove the points in order.
\begin{point}{Ad 1}%
First we prove that~$p$ is sharp if and only if~$\andthen{p}{p}=p.$
So, assume~$p$ is sharp.
As $\diamond$-positive maps are~$\diamond$-self-adjoint
    we have~$\IM \asrt_p = \ceil{1 \after \asrt_p} = \ceil{p} = p$.
Thus~$\andthen{p}{p} = p \after \asrt_p = 1 \after \asrt_p = p$.
For the converse, assume~$\andthen{p}{p}=p$.
From~$\IM \asrt_p = \ceil{1 \after \asrt_p} = \ceil{p}$,
we get~$\ceil{p} \after \asrt_p = \andthen{p}{\ceil{p}} = 1 \after \asrt_p = p$.
By assumption~$\andthen{p}{p}=p$.
So~$\andthen{p}{(\ceil{p} \ominus p)} = 0$.
Hence~$\ceil{p}\ominus p \leq \IMperp \asrt_p = \ceil{p}^\perp$.
However $\ceil{p}\ominus p \leq \ceil{p}$.
Thus by~\sref{image-sharp-is-order-sharp}
    get~$\ceil{p} \ominus p = 0$. So~$p$ is indeed sharp.

Clearly, if~$\asrt_p \after \asrt_p = \asrt_p$,
then~$p = 1 \after \asrt_p = 1\after\asrt_p\after\asrt_p = \andthen{p}{p}$.
It only remains to be shown~$\asrt_p \after \asrt_p = \asrt_p$
    whenever~$p$ is sharp.
So assume~$p$ is sharp.
By definition~$\asrt_p$ is pure: $\asrt_p = \pi \after \xi$
    for some quotient~$\xi$ and comprehension~$\pi$.
By assumption~$\xi \after \pi$ is pure as well,
    so there is a quotient~$\xi'$ and comprehension~$\pi'$
    with~$\xi \after \pi = \pi' \after \xi'$.
Note~$1 \after \xi = 1 \after \asrt_p = p$
and~$1 \after \asrt_p \after \asrt_p = \andthen{p}{p} = p = 1 \after \xi$,
hence
\begin{equation*}
 1\after \xi  
    \ =\  1 \after \asrt_p \after \asrt_p 
    \ =\  1 \after \pi \after \xi \after \pi \after \xi 
    \ =\  1 \after \pi \after \pi' \after \xi' \after \xi 
    \ =\  1 \after \xi' \after \xi,
\end{equation*}
so~$1 \after \xi' = 1$. Thus~$\xi'$ is an iso.
Also~$
    (\IM \pi') \after \xi \after \pi
    = (\IM \pi') \after \pi' \after \xi'
    = 1 \after \xi' = 1 $
    from which it follows~$p = \IM \pi \leq  (\IM \pi') \after \xi 
            \leq 1 \after \xi = p$.
Thus~$(\IM \pi') \after \xi = p = 1\after \xi$.
Hence~$\IM \pi' = 1$ and so~$\pi'$ is an isomorphism.
Now we know~$\pi \after \pi'$
    is a comprehension and~$\xi' \after \xi$ is a quotient,
    we see~$\asrt_p\after\asrt_p$ is pure.
As~$\asrt_p$ is $\diamond$-self-adjoint,
    we see~$\asrt_p\after\asrt_p$ is $\diamond$-positive.
By uniqueness of positive maps~$\asrt_p \after \asrt_p = \asrt_p$,
    as desired.
\end{point}
\begin{point}{Ad 2}%
Pick quotient~$\xi$ and comprehension~$\pi$
    with~$\pi \after \xi = \asrt_s$.
By point 1:
\begin{equation*}
   \pi \after \xi \ =\  \asrt_s \ =\  \asrt_s\after\asrt_s \ =\  \pi \after \xi \after \pi \after \xi.
\end{equation*}
Thus~$\xi \after \pi = \id$.
We compute~$s = 1 \after \asrt_s = 1 \after \xi$
and so
\begin{alignat*}{2}
    \IM \pi &\ = \ 
    \IM \pi \after \xi \after \pi &\qquad& \text{as $\xi \after \pi=\id$}\\
                  &\ = \ \IM \asrt_s \after \pi &&\text{by dfn.~$\xi$ and~$\pi$}\\
                  &\ = \ (\asrt_s)_{\diamond}(\IM \pi)&& \text{by dfn.~$(\ )_\diamond$} \\
                  & \ = \ (\asrt_s)^{\diamond}(\IM \pi)&&
        \text{by $\diamond$-s.a.~$\asrt_s$}\\
        &\ = \ \ceil{(\IM \pi) \after \pi \after \xi} && \text{by dfn.~$(\ )^\diamond$} \\
&\ = \ \ceil{1 \after \xi} \\
    & \ = \ s.
\end{alignat*}
Thus~$\pi$ is a comprehension of~$s$
and~$\xi$ is a quotient of~$s^\perp$.
We have proven the first part.

For the second part,
let~$\pi'$ be any (other) comprehension of~$s$.
Then~$\pi' = \pi \after \alpha$ for some iso~$\alpha$.
Define~$\xi' = \alpha^{-1} \after \xi$.
It is easy to see~$\pi' \after \xi' = \asrt_s$
and~$\xi' \after \pi' = \id$.
The other statement is proven in a similar way.
\end{point}
\begin{point}{Ad 3}%
Assume~$\pi_1 \colon X \to Y$
    and~$\pi_2 \colon Y \to Z$
    are comprehensions with~$s = \IM \pi_1$
    and~$t = \IM \pi_2$.
We will show~$\pi_2 \after \pi_1$
    is a comprehension for~$\IM \pi_2 \after \pi_1$.
To this end, let~$f\colon V\colon Z$ be any map
with~$(\IM \pi_2 \after \pi_1)^\perp \after f = 0$.
As~$\IM \pi_2 \after \pi_1 \leq \IM \pi_2 = t$
    we get~$t^\perp \after f = 0$,
    so~$f = \pi_2 \after g_2$ for a unique~$g_2\colon V \to Y$.
Let~$\zeta_2$ be a quotient for~$t^\perp$
such that~$\zeta_2 \after \pi_2$,
which exists by the previous point.
Then~$s \after \zeta_2 \after \pi_2 \after \pi_1
            = s \after \pi_1 = 1$
            so~$s \after \zeta_2 \geq \IM \pi_2 \after \pi_1$.
Consequently
\begin{equation*}
    s \after g_2 \ =\  s \after \zeta_2 \after \pi_2 \after g_2
        \ =\  s \after \zeta_2 \after f
        \ \geq\   (\IM \pi_2 \after \pi_1) \after f \ =\  1 \after f
                \ =\  1 \after g_2.
\end{equation*}
Hence there exists a unique~$g_1\colon V \to X$
    with~$\pi_1 \after g_1 = g_2$
    and so~$\pi_2 \after \pi_1 \after g_1 = f$.
By monicity of~$\pi_2 \after \pi_1$,
    this~$g_1$ is unique and so~$\pi_2 \after \pi_1$
    is indeed a comprehension.
\end{point}
\begin{point}{Ad 4}%
Assume~$g\colon X \to Y$ and~$f\colon Y \to Z$ are pure maps.
Say~$f = \pi_1 \after \xi_1$
    and~$g = \pi_2 \after \xi_2$
    for some comprehensions~$\pi_1$, $\pi_2$
        and quotients~$\xi_1$, $\xi_2$.
By definition of $\&$-effectus
    there is a comprehension~$\pi'$ and quotient~$\xi'$
    such that~$\xi_1 \after \pi_2 = \pi' \after \xi'$.
By the previous point~$\pi_1 \after \pi'$
    is a comprehension
    and~$\xi' \after \xi_2$ is a quotient by \sref{quotients-composition}.
Thus
$f \after g 
=\pi_1 \after \pi' \after \xi' \after \xi_2$
is pure. \qed
\end{point}
\end{point}
\begin{point}[zeta-s-convention]{Notation}%
In a~$\&$-effectus
    together with chosen comprehension~$\pi_s$ for~$s$,
    we will write~$\zeta_s$
    for the unique quotient for~$s^\perp$
    satisfying~$\zeta_s \after \pi_s = \id$
    and~$\pi_s \after \zeta_s = \asrt_s$.
We call this~$\zeta_s$
    the \Define{corresponding quotient} of~$\pi_s$
    and vice versa~$\pi_s$ the \Define{corresponding comprehension}
    of~$\zeta_s$.
\end{point}
\end{point}
\begin{point}[andthen-square-rule]{Exercise}%
Show that in a~$\&$-effectus,
    we have~$\asrt_p \after \asrt_p = \asrt_{\andthen{p}{p}}$.
\end{point}
\begin{point}[asrt-absorp-rule]{Exercise}%
    Show that in a~$\&$-effectus,
    we have
    \begin{align*}
        \IM f \leq s \quad&\iff\quad  \asrt_s \after f = f \\
        1 \after f \leq t \quad&\iff\quad  f \after \asrt_t = f
    \end{align*}
for any sharp predicates~$s$ and~$t$.
\end{point}
\begin{point}{Definition}%
Let~$C$ be a~$\&$-effectus.
In \sref{upm-basics} we saw pure maps are closed under composition
Write~$\Define{\Pure C}$ for the subcategory
    of pure map.
\end{point}
\end{parsec}

\begin{parsec}%
\begin{point}[standard-form-map]{Corollary}%
Every map~$f$
in a~$\&$-effectus
factors as
    \begin{equation*}
        f \ =\ \pi_{\IM f} \after g \after \zeta_{\ceil{1 \after f}} \after \asrt_{1 \after f}
    \end{equation*}
    for some total and faithful map~$g$.
If~$f$ is pure, then~$g$ is an isomorphism.
\begin{point}{Proof}%
    \TODO{...}
\end{point}
\end{point}
\end{parsec}

\section{$\dagger$-effectus}
\begin{parsec}%
\begin{point}{Definition}%
A \Define{$\dagger$-category} \TODO{refs}
is category~$C$ together with an involutive identity-on-objects
    functor~$(\ )^\dagger \colon C \to \op{C}$;
    that is, for all objects~$X$ and maps~$f,g$ in~$C$
\begin{multicols}{2}
\begin{enumerate}
    \item $(f \after g)^\dagger = g^\dagger \after f^\dagger$
    \item $\id^\dagger = \id$
    \item $f^{\dagger\dagger} = f$ \emph{and}
    \item $X^\dagger = X$.
\end{enumerate}
\end{multicols}
\noindent
In aby~$\dagger$-category we may define the following.
\begin{enumerate}
\item
    An endomap~$f$ is called \Define{$\dagger$-self-adjoint}
    if~$f^\dagger = f$.
\item
    An endomap~$f$ is \Define{$\dagger$-positive}
    if~$f = g^\dagger \after g$ for some other map~$g$.
\item
    An isomorphism~$\alpha$ is called~\Define{unitary}
        whenever~$\alpha^{-1} = \alpha^\dagger$.
\end{enumerate}
\end{point}
\begin{point}{Example}%
The category~$\mathsf{Hilb}$
    of Hilbert spaces with bouned linear maps
    is a~$\dagger$-category
    with the familiar adjoint as~$\dagger$.
\end{point}
\end{parsec}
\begin{parsec}%
\begin{point}{Definition}%
We call a~$\&$-effectus~$C$,
    a~\Define{$\dagger$-effectus} provided
\begin{enumerate}
\item
    $\Pure C$ is a  $\dagger$-category
     satisfying~$\asrt_p^\dagger = \asrt_p$
        and~$f$ is $\diamond$-adjoint to~$f^\dagger$;
\item
    for every~$\dagger$-positive~$f$,
        there is a unique~$\dagger$-positive~$g$
        with~$g \after g = f$ \emph{and}
\item
    $\diamond$-positive maps are $\dagger$-positive.
\end{enumerate}
\end{point}
\begin{point}[dagger-theorem]{Theorem}%
    A $\&$-effectus
        is a~$\dagger$-effectus if and only if
\begin{enumerate}
\item
for every predicate~$p$, there is a unique predicate~$q$
    with~$\andthen{q}{q} = p$;
\item
    $\asrt^2_{\andthen{p}{q}}
        = \asrt_p \after \asrt^2_q \after \asrt_p$
     for all predicates~$p,q$ \emph{and}
\item
    a quotient for a sharp predicate (e.g.~$\zeta_s$)
    is a sharp map, see \sref{sharp-map}.
\end{enumerate}
\begin{point}{Proof}%
Necessity will be proven in
    \sref{dagger-thm-necessity}
        and sufficiency in \sref{dagger-thm-sufficiency}. \qed
\end{point}
\begin{point}%
Especially the sufficiency requires quite some preparation.
For convenience, call~$C$ a \Define{$\dagger'$-effectus}
    if~$C$ is a~$\&$-effectus
    satisfying axioms 1, 2 and 3 from the Theorem above.
\end{point}
\begin{point}{Note}%
    In a~$\&$-effectus
    with seperating predicates,
    the second axiom is equivalent
    to~$\andthen{(\andthen{p}{q})^2}{r}
    = \andthen{p}{(\andthen{q^2}{(\andthen{p}{r})})}$
    for all predicates~$p,q,r$.
\end{point}
\end{point}
\end{parsec}


\begin{parsec}%
\begin{point}[diamond-is-dagger-positive]{Lemma}%
In a~$\dagger$-effectus,
    a map is~$\dagger$-positive if and only if it is~$\diamond$-positive.
\begin{point}{Proof}%
Assume~$f$ is~$\dagger$-positive.
By assumption 2, there is a (unique) $\dagger$-positive
    $g$ with~$f = g \after g$.
By~$\dagger$-positivity of~$g$,
    there is an~$h$ with~$g = h^\dagger \after h$.
Using the fact~$h$ is~$\diamond$-adjoint
    to~$h^\dagger$, we see~$g$ is~$\diamond$-self-adjoint:
\begin{equation*}
    g^\diamond
    \ =\  (h^\dagger \after h)^\diamond
    \ =\  h^\diamond \after (h^\dagger)^\diamond 
    \ =\  (h^\dagger)_\diamond \after h_\diamond 
    \ =\  (h^\dagger\after h)_\diamond
    \ =\  g_\diamond.
\end{equation*}
Thus~$f$ is the square of the~$\diamond$-self-adjoint map $g$,
    hence~$f$ is~$\diamond$-positive. \qed
\end{point}
\end{point}

\begin{point}[dagger-eff-square-root]{Lemma}%
Predicates in a~$\dagger$-effectus have a unique square root:
    for every predicate~$p$,
    there is a unique predicate~$q$
    with~$\andthen{q}{q}=p$.
\begin{point}{Proof}%
Let~$p\colon X \to 1$ be any predicate.
By assumption 3,
    the~$\diamond$-positive map~$\asrt_p$
        is also~$\dagger$-positive.
So by assumption 2,
    there is (a unique)~$\dagger$-positive map~$f$
    with~$f \after f = \asrt_p$.
Define~$q = 1 \after f$.
By \sref{diamond-is-dagger-positive}
    $f$ is~$\diamond$-positive
    and so by uniqueness of~$\diamond$-positive maps,
    we get~$f = \asrt_{1\after f} \equiv \asrt_q$.
We compute
\begin{equation*}
    \andthen{q}{q}
        \ =\    
        q \after \asrt_q \ \equiv \ 
        1 \after f \after \asrt_{1 \after f} \ = \ 
        1 \after f \after f \ = \ 1 \after \asrt_p \ = \ p,
\end{equation*}
which shows~$p$ has as square root~$q$.

To show uniqueness, assume~$p = \andthen{r}{r}$ for some (other)
    predicate~$r$.
Note
\begin{equation*}
    \asrt_p
        \ = \ \asrt_{\andthen{r}{r}}
        \ \overset{\smash{\sref{andthen-square-rule}}}{=} \ \asrt_r \after \asrt_r.
\end{equation*}
As~$\asrt_r$ is~$\diamond$-positive,
    it is also~$\dagger$-positive by the third axiom.
    So by the second axiom~$\asrt_r = \asrt_q$.
Thus~$r = q$, which shows uniqueness of the square root.
\qed
\end{point}
\end{point}
\begin{point}[asrt-iso]{Proposition}%
In a~$\&$-effectus with square roots
    (e.g.~$\dagger$- or~$\dagger'$-effectus) we have
\begin{equation*}
    \asrt_p \after \alpha
        \ =\  \alpha \after \asrt_{p\after \alpha}
\end{equation*}
for every isomorphism~$\alpha$.
\begin{point}{Proof}%
There is some~$q$ with~$\andthen{q}{q}=p$.
The map~$\alpha^{-1} \after \asrt_q \after \alpha$
    is~$\diamond$-self-adjoint:
\begin{alignat*}{2}
    (\alpha^{-1} \after \asrt_q \after \alpha)_\diamond
    &\ = \ \alpha^{-1}_\diamond \after (\asrt_q)_\diamond \after \alpha_\diamond \\
    &\ \overset{\mathclap{\sref{iso-diamond-adjoint}}}{=} \ \alpha^\diamond \after (\asrt_q)^\diamond \after (\alpha^{-1})^\diamond\\
    &\ = \ (\alpha^{-1} \after \asrt_q \after \alpha)^\diamond.
\end{alignat*}
Thus~$
\alpha^{-1} \after \asrt_q \after \alpha \after
\alpha^{-1} \after \asrt_q \after \alpha
= \alpha^{-1} \after \asrt_p \after \alpha
$ is~$\diamond$-positive.  By uniqueness of~$\diamond$-positive maps,
we get~$\alpha^{-1} \after \asrt_p \after \alpha
    = \asrt_{1 \after \alpha^{-1}\after\asrt_p \after \alpha}
    = \asrt_{p \after \alpha} $.
Postcomposing~$\alpha$, we find~$\asrt_p\after\alpha = 
        \alpha \after \asrt_{p \after \alpha}$, as desired. \qed
\end{point}
\end{point}
\begin{point}[dagger-of-zeta]{Proposition}%
In a $\dagger$-effectus:~$\zeta_s^\dagger = \pi_s$
(recall convention \sref{zeta-s-convention} for~$\pi_s$ and~$\zeta_s$).
\begin{point}{Proof}%
As~$\pi_s$ is~$\diamond$-adjoint to~$\pi_s^\dagger$,
we have
\begin{equation*}
\IM \pi_s^\dagger
 \ = \  (\pi^\dagger_s)_\diamond(1) \ =\ 
 (\pi_s)^\diamond(1) \ = \  \ceil{1 \after \pi_s}
 \ =\  1
\end{equation*}
and similarly~$\ceil{1 \after \pi_s^\dagger} = \IM \pi_s = s$.
As~$1 \after \pi_s^\dagger \leq \ceil{1 \after \pi_s^\dagger} = s$,
    there is some pure~$h$ with~$\pi_s^\dagger = h \after \zeta_s$.
There is also some pure and faithful \TODO{...}~$g$
    with~$\zeta_s^\dagger = \pi_s \after g$.
Using~$\zeta_s \after \pi_s = \id$ twice we find
\begin{equation}\label{eq-dagger-zeta-pi-conn}
    \id \ =\  \id^\dagger
    \ = \ \pi_s^\dagger \after \zeta_s^\dagger
    \ = \ h \after \zeta_s \after \pi_s \after g 
    \ = \ h \after g.
\end{equation}
Now~$1= 1 \after h \after g \leq 1 \after g$ and so~$g$ is total.
Clearly~$\zeta_s^\dagger\after\zeta_s$ is~$\dagger$-positive,
    hence~$\diamond$-positive and so by uniqueness of~$\diamond$-positive maps:
\begin{equation*}
\zeta_s^\dagger \after \zeta_s
    \ = \ \asrt_{1 \after \zeta_s^\dagger \after \zeta_s}
    \ =\  \asrt_{1 \after \pi_s \after g\after \zeta_s}
    \ =\  \asrt_s
    \ =\  \pi_s \after \zeta_s.
\end{equation*}
So by epicity of~$\zeta_s$, we find~$\zeta_s^\dagger = \pi_s$, as desired.\qed
\end{point}
\end{point}

\begin{point}[dagger-of-iso]{Corollary}%
    In a~$\dagger$-effectus,
        $\pi_s$ is~$\diamond$-adjoint to~$\zeta_s$.
    Also~$\alpha^\dagger = \alpha^{-1}$
        for any iso~$\alpha$.
\end{point}

\begin{point}[zeta-through-asrt]{Exercise}%
Show that in a~$\&$-effectus
        where every predicate has a square root
        and where~$\pi_s$ is~$\diamond$-adjoint to~$\zeta_s$
        (e.g.~a~$\dagger$-effectus)
        we have
    $\asrt_p \after \zeta_s \ = \ \zeta_s \after \asrt_{p \after \zeta_s}$.
    (Hint: mimic the proof of \sref{asrt-iso}.)
\end{point}

\begin{point}[dagger-thm-necessity]{Theorem}%
    A~$\dagger$-effectus is a~$\dagger'$-effectus.
\begin{point}{Proof}%
Assume~$C$ is a~$\dagger$-effectus.
Axiom 1 is already proven in \sref{dagger-eff-square-root}.
\begin{point}[pqqp-from-dagger]{Ax.~2}%
Let~$p,q$ be predicates.
To start, note~$1 \after \asrt_q\after  \asrt_p = \andthen{p}{q}$ and
\begin{equation*}
    \IM \asrt_q \after \asrt_p \ =\ 
        (\asrt_q)_\diamond(p) \ = \ 
        (\asrt_q)^\diamond(p) \ = \ 
        \ceil{\andthen{q}{p}}.
\end{equation*}
So using \sref{standard-form-map}
    we know
\begin{equation*}
    \asrt_q \after \asrt_p \ =\ 
    \pi_{\ceil{\andthen{q}{p}}}
    \after \alpha \after \zeta_{\ceil{\andthen{p}{q}}} \after \asrt_{\andthen{p}{q}}
\end{equation*}
for some iso~$\alpha$.
Applying the dagger to both sides
we get, using \sref{dagger-of-zeta} and \sref{dagger-of-iso}:
\begin{align*}
    \asrt_p \after \asrt_q &\ =\ 
    (\asrt_q \after \asrt_p)^\dagger \\
    & \ =\  \asrt_{\andthen{p}{q}}
    \after \zeta^\dagger_{\ceil{\andthen{p}{q}}}
    \after \alpha^\dagger \after
    \pi^\dagger_{\ceil{\andthen{q}{p}}} \\
    & \ = \ 
    \asrt_{\andthen{p}{q}}
    \after \pi_{\ceil{\andthen{p}{q}}}
    \after \alpha^{-1} \after
    \zeta_{\ceil{\andthen{q}{p}}}.
\end{align*}
Combining both:
\begin{align*}
    & \asrt_p \after \asrt_q^2 \after \asrt_p \\
    &\qquad = \ \asrt_{\andthen{p}{q}}
    \after \pi_{\ceil{\andthen{p}{q}}}
    \after \alpha^{-1} \after
    \zeta_{\ceil{\andthen{q}{p}}}
    \after \pi_{\ceil{\andthen{q}{p}}}
    \after \alpha \after \zeta_{\ceil{\andthen{p}{q}}} \after \asrt_{\andthen{p}{q}} \\
    &\qquad = \ \asrt_{\andthen{p}{q}}
    \after \pi_{\ceil{\andthen{p}{q}}}
    \after \zeta_{\ceil{\andthen{p}{q}}} \after \asrt_{\andthen{p}{q}} \\
    &\qquad = \ \asrt_{\andthen{p}{q}}
    \after \asrt_{\ceil{\andthen{p}{q}}}
    \after \asrt_{\andthen{p}{q}} \\
    &\qquad \overset{\smash{\mathclap{\sref{asrt-absorp-rule}}}}{=} \ 
    \asrt_{\andthen{p}{q}}^2,
\end{align*}
as desired.
\end{point}
\begin{point}{Ax.~3}%
Pick any sharp predicates~$s,t$.
We want to show~$t \after \zeta_s$ is sharp.
To this end, we will show~$t \after \zeta_s$
    is the image of~$\pi_s \after \pi_t$.
Clearly~$t \after \zeta_s \after \pi_s \after \pi_t = 1$.
Let~$p$ be any (other) sharp predicate with~$p \after \pi_s \after \pi_t = 1$.
Then~$p \after \pi_s \geq \IM \pi_t = t$.
So~$  p^\perp \after \pi_s \leq t^\perp$
and hence~$\ceil{p^\perp \after \pi_s} \leq t^\perp$.
As~$\pi_s$ is~$\diamond$-adjoint to~$\zeta_s$ by \sref{dagger-of-iso},
we get~$t \after \zeta_s \leq \ceil{t \after \zeta_s} \leq p$,
    which shows~$t \after \zeta_s$
    is the image of~$\pi_s \after \pi_t$
    and consequently sharp. \qed
\end{point}
\end{point}
\end{point}

    
\end{parsec}

\begin{parsec}%
\begin{point}%
Let~$f$ be a pure map in a~$\dagger'$-effectus.
We will work towards the definition of~$f^\dagger$.
By \sref{standard-form-map} we
know there is an iso~$\alpha$ with
\begin{equation*}
f  \ =\   \pi_{\IM f} \after \alpha \after \zeta_{\ceil{1\after f}}
            \after \asrt_{1 \after f}.
\end{equation*}
In a~$\dagger$-effectus
    we have~$\asrt_p^\dagger = \asrt_p$,
    $\zeta_s^\dagger = \pi_s$,
    $\alpha^\dagger = \alpha^{-1}$
    and~$\pi_s^\dagger = \zeta_s$ 
    (for corresponding~$\zeta_s$ and~$\pi_s$)
    ---
    so we are forced to define~
\begin{equation}\label{dagger-definition}
    f^\dagger
        \ =\  \asrt_{1 \after f} \after
    \pi_{\ceil{1\after f}} \after
    \alpha^{-1} \after
    \zeta_{\IM f},
\end{equation}
    where~$\zeta_{\IM f}$
    is the unique corresponding
        quotient of~$\pi_{\IM f}$
        and~$\pi_{\ceil{1 \after f}}$
        the unique corresponding comprehension of
        and~$\zeta_{\ceil{1 \after f}}$,
        see \sref{zeta-s-convention}.
Before we declare~\eqref{dagger-definition} a definition,
 we have to check whether it is independant
    of choice of~$\pi$ (and corresponding $\zeta$).
    So suppose~$ f  =  \pi' \after \alpha' \after \zeta'
        \after \asrt_{1 \after f}$
        for some (other) iso~$\alpha'$, comprehension~$\pi'$ of~$\IM f$
        and quotient~$\zeta'$ of~$\smash{\ceil{1 \after f}^\perp}$.
There are isos~$\beta$ and~$\gamma$
    such that~$\pi' = \pi_{\IM f} \after \beta$
    and~$\zeta' = \gamma \after \zeta_{\ceil{1 \after f}}$.
We will take a moment to relate~$\alpha$ and~$\alpha'$:
as~$\pi_{\IM f}$ is mono
    and~$\zeta_{1 \after f}\after\asrt_{1 \after f}$
    is epic,
    we have~$\beta\after\alpha'\after\gamma = \alpha$
    and so~$(\alpha')^{-1}= \gamma \after\alpha^{-1}\after\beta$.
To continue,
it is easy to see~$\beta^{-1} \after \zeta_{\IM f}$
    is the unique corresponding quotient to~$\pi'$
    and~$\pi_{\ceil{1 \after f}}\after \gamma^{-1}$
    is the unique corresponding comprehension to~$\zeta'$.
So with this choice of quotient and comprehension,
    we are forced to define
\begin{align*}
    f^\dagger 
    &\ = \ \asrt_{1 \after f}\after \pi_{\ceil{1 \after f}}
    \after \gamma^{-1} \after {\alpha'}^{-1}
                \after \beta^{-1}\after \zeta_{\IM f} \\
    &\ = \ \asrt_{1 \after f}\after \pi_{\ceil{1 \after f}}
                \after \gamma^{-1} \after \gamma \after \alpha^{-1} \after \beta 
                \after \beta^{-1}\after \zeta_{\IM f} \\
    &\ = \ \asrt_{1 \after f}\after \pi_{\ceil{1 \after f}}
                \after \alpha^{-1} \after \zeta_{\IM f},
\end{align*}
which is indeed consistent with~\eqref{dagger-definition}.
So we are justified to declare:
\end{point}
\begin{point}[dagger-definition2]{Definition}%
In a~$\dagger'$-effectus, for a pure map~$f$, define
\begin{equation*}
    \Define{f^\dagger}
        \ =\  \asrt_{1 \after f} \after
    \pi_{\ceil{1\after f}} \after
    \alpha^{-1} \after
    \zeta_{\IM f},
\end{equation*}
where~$\alpha$ is the unique iso such that
$f  =  \pi_{\IM f} \after \alpha \after \zeta_{\ceil{1\after f}}
            \after \asrt_{1 \after f}$.
\end{point}
\begin{point}[dagger-prime-basics]{Exercise}%
Show that in a~$\dagger'$-effectus,
    we have
\begin{align*}
    \asrt_p^\dagger &= \asrt_p &
    \pi_s^\dagger &= \zeta_s &
    \zeta_s^\dagger &= \pi_s &
    \alpha^\dagger &= \alpha^{-1}
\end{align*}
for a quotient~$\zeta_s$ corresponding to~$\pi_s$
and iso~$\alpha$.
\end{point}
\end{parsec}

\begin{parsec}%
\begin{point}%
To compute~$f^{\dagger\dagger}$,
    we need to put~$f^\dagger$
    in the standard form of~\sref{standard-form-map}.
To do this, we need to pull~$\asrt_p$ from one side to the other.
In this section we will work towards a general result for this.
\end{point}
\begin{point}[quotcompr-diamond-adjoint]{Lemma}%
In a~$\dagger'$-effectus,
$\pi_s$ is~$\diamond$-adjoint to~$\zeta_s$.
\begin{point}{Proof}%
    Pick any sharp~$t,u$.
We have to
    show~$t \after \zeta_s \leq u^\perp$
    if and only if~$u \after \pi_s \leq t^\perp$.
So assume~$t \after \zeta_s \leq u^\perp$.
Then~$t =  t\after \zeta_s \after \pi_s  \leq  u^\perp \after \pi_s
                 =  (u \after \pi_s)^\perp$
    so~$u \after \pi_s \leq t^\perp$.
For the converse, assume~$u \after \pi_s \leq t^\perp$.
Then~$u \after \asrt_s = u \after \pi_s \after \zeta_s \leq t^\perp \after \zeta_s$.
From this,
    the
    $\diamond$-self-adjointness of~$\asrt_s$
    and the fact that~$t^\perp \after \zeta_s$ is sharp,
    we find~$(t^\perp \after \zeta_s)^\perp \after \pi_s \after \zeta_s \leq
        u^\perp$
    and so
\begin{alignat*}{2}
    t \after \zeta_s
    &\ =\  (t^\perp \after \id)^\perp \after \zeta_s \\
    &\ =\  (t^\perp \after \zeta_s \after \pi_s)^\perp \after \zeta_s \\
    &\ =\  (t^\perp \after \zeta_s)^\perp \after \pi_s \after \zeta_s \\
    &\ \leq \ u^\perp,
\end{alignat*}
as desired. \qed
\end{point}
\end{point}
\begin{point}{Definition}%
In a~$\&$-effectus, a map~$f$ is \Define{pristine}
        if it is pure and~$1\after f$ is sharp.
\end{point}
\begin{point}[standard-form-pristine]{Exercise}%
Show that in a~$\&$-effectus, every pristine map~$h$ is of the form
        \begin{equation*}
            h \ =\ \pi_{\IM h} \after \alpha \after \zeta_{1 \after h}
        \end{equation*}
        for some iso~$\alpha$.
\end{point}
\begin{point}[pristine-asrt]{Proposition}%
In a~$\dagger'$-effectus,
    with pristine map~$h$, we have
\begin{equation*}
    \asrt_p \after h
        \ =\  h \after \asrt_{p \after h}
\end{equation*}
for any predicate~$p$ with~$p \leq \IM h$.
\begin{point}{Proof}%
For brevity,
write~$t = 1 \after h$
    and~$s = \IM h$.
By~\sref{standard-form-pristine}
    there is some iso~$\alpha$
    such that~$h = \pi_s \after \alpha \after \zeta_t$.
    Note~$\IM \asrt_p = \ceil{p } \leq \ceil{s} = s$
    and so~$\asrt_p = \asrt_{s} \after \asrt_p$
    by the first rule of \sref{asrt-absorp-rule}.
By the second rule of \sref{asrt-absorp-rule}
    and~$1 \after p = p \leq s$,
    we see~$p = p \after \asrt_{s}$.
Thus
\begin{alignat*}{2}
    \asrt_p \after \pi_{s}
    &\ =\  \asrt_{s} \after \asrt_{p \after \asrt_{s}} \after \pi_{s} \\
    &\ =\  \pi_{s} \after \zeta_{s} \after
    \asrt_{p \after \pi_{s} \after \zeta_{s}} \after \pi_{s} \\
    &\ =\  \pi_{s}  \after
    \asrt_{p \after \pi_{s}} \after \zeta_{s} \after \pi_{s} 
    &\qquad& \text{by \sref{zeta-through-asrt}
                    and \sref{quotcompr-diamond-adjoint} }\\
    &\ =\  \pi_{s}  \after
    \asrt_{p \after \pi_{s}}.
\end{alignat*}
Putting everything together
\begin{alignat*}{2}
   \asrt_p \after h
   & \ =\  \asrt_p \after \pi_s \after \alpha \after \zeta_t \\
   & \ =\  \pi_s \after \asrt_{p\after\pi_s} \after \alpha \after \zeta_t  \\
   & \ =\  \pi_s \after \alpha \after \asrt_{p\after\pi_s \after \alpha} \after \zeta_t 
   &\qquad&\text{by \sref{asrt-iso}}\\
   & \ =\  \pi_s \after \alpha \after \zeta_t \after \asrt_{p\after\pi_s \after \alpha \after \zeta_t} 
    &\qquad& \text{by \sref{zeta-through-asrt}
                    and \sref{quotcompr-diamond-adjoint} }\\
   & \ = \ h \after \asrt_{p \after h},
\end{alignat*}
as desired.\qed
\end{point}
\end{point}
\begin{point}[asrt-pristine-reverse]{Exercise}%
Working in a~$\dagger'$-effectus, show in order
\begin{enumerate}
    \item if~$h \equiv \pi_{\IM h} \after \alpha \after \zeta_{1 \after h}$
            is some pristine map,
            then~$h^\dagger = \pi_{1 \after h} \after \alpha^{-1} \after \zeta_{\IM h}$;
    \item $h^{\dagger\dagger} = h$ for any pristine~$h$;
    \item $h^\dagger \after h = \asrt_{1 \after h}$ for any pristine map~$h$;
    \item $p \after h^\dagger \leq \IM h$ for any predicate~$p$ \emph{and}
    \item
        if~$p \leq 1 \after h$,
        then~$\asrt_{p\after h^\dagger} \after h = h \after \asrt_p$.
\end{enumerate}
\end{point}
\begin{point}[prist-asrt-decomp]{Proposition}%
In a~$\dagger'$-effectus,
        for every pure map~$f$,
    there exists a unique pristine map~$h$
    with~$1 \after h = \ceil{1 \after f}$
    and~$f = h \after \asrt_{1 \after f}$.
    Furthermore~$f^\dagger = \asrt_{1 \after f}\after h^\dagger$.
\begin{point}{Proof}%
        \TODO{}
\end{point}
\end{point}
\begin{point}[dagger-idempotent]{Proposition}%
    In a~$\dagger'$-effectus, we have~$f^{\dagger\dagger}=f$
        for any pure map~$f$.
\begin{point}{Proof}%
By~\sref{prist-asrt-decomp}
    we have~$f = h \after \asrt_{1\after f}$
    and~$f^\dagger = \asrt_{1 \after f} \after h^\dagger$
    for some pristine~$h$
    with~$1 \after h = \ceil{1 \after f}$.
Clearly~$1 \after f \leq \ceil{1 \after f} = 1 \after h = \IM h^\dagger$,
    so by \sref{pristine-asrt}
    we get~$f^\dagger = \asrt_{1 \after f} \after h^\dagger
              = h^\dagger \after \asrt_{1 \after f \after h^\dagger}$.
Consequently
\begin{equation*}
    f^{\dagger\dagger}
    \ =\  \asrt_{1 \after f\after h^\dagger} \after h^{\dagger\dagger}
    \ \overset{\sref{asrt-pristine-reverse}}{=}\  \asrt_{1 \after f\after h^\dagger} \after h
    \ \overset{\sref{asrt-pristine-reverse}}{=}\  h\after \asrt_{1 \after f}
    \ = \ f,
\end{equation*}
as desired. \qed
\end{point}
\end{point}
\end{parsec}

\begin{parsec}%
\begin{point}%
Now we tackle the most tedious part
    of \sref{dagger-theorem}:
    we will show~$(f \after g)^\dagger = g^\dagger \after f^\dagger$
    in a~$\dagger'$-effectus.
To avoid too much repetition, let us fix the setting.
\end{point}%
\begin{point}[dagger-setting]{Setting}%
Let~$f,g$ be two composable pure maps in a~$\dagger'$-effectus.
For brevity, write~$p = 1 \after f$, $q = 1 \after g$,
    $s = \IM f$ and $t = \IM g$.
Let~$\varphi$ and~$\psi$
be the unique isomorphisms (see \sref{standard-form-map}) such that
\begin{equation*}
    f \ =\  \pi_s \after \varphi \after \zeta_{\ceil{p}} \after \asrt_p 
    \qquad \text{and} \qquad
    g \ =\  \pi_t \after \psi \after \zeta_{\ceil{q}} \after \asrt_q.
\end{equation*}
Define~$
    h = \pi_s \after \varphi \after \zeta_{\ceil{p}}$
    and~$k = \pi_t \after \psi \after \zeta_{\ceil{q}}$.
To compute~$(f\after g)^\dagger$,
    we have to put~$f\after g$
    in the standard form
    of~\sref{standard-form-map}.
We will do this step-by-step,
    first we put~$\zeta_{\ceil{p}} \after \asrt_p \after \pi_t$
        in standard form.
\begin{align*}
1 \after \zeta_{\ceil{p}} \after \asrt_p \after \pi_t
&\ =\  \ceil{p} \after \asrt_p \after \pi_t
    \ \overset{\sref{asrt-absorp-rule}}{=}\  p \after \pi_t\\
    \IM \zeta_{\ceil{p}}  \after \asrt_p \after \pi_t &\ = \ 
(\zeta_{\ceil{p}}  \after \asrt_p)_\diamond(t)
\ \overset{\sref{quotcompr-diamond-adjoint}}{=} \ \ceil{t \after \asrt_p \after \pi_{\ceil{p}}}.
\end{align*}
Thus there is a unique isomorphism~$\chi$ with
\begin{equation}\label{dagger-iso-chi}
    \zeta_{\ceil{p}} \after \asrt_p \after \pi_t
    \ = \ \pi_{\lceil t \after \asrt_p \after \pi_{\ceil{p}}\rceil} \after \chi \after \zeta_{\ceil{p \after \pi_t}} \after \asrt_{p \after \pi_t}.
\end{equation}
Next, we consider~$\asrt_{p \after k}
                        \after \asrt_q$, clearly
\begin{equation*}
    1 \after \asrt_{p \after k}
\after \asrt_q 
\ =\ p \after k
\after \asrt_q \ = \ p \after g.
\end{equation*}
Concerning the image, first
    note~$p \after k = p \after \pi_t \after \psi \after \zeta_{\ceil{q}}
    \leq 1 \after \zeta_{\ceil{q}} = \ceil{q}$
        and so we must have~$\IM \asrt_{p \after k} \leq \ceil{q}$, which implies
\begin{equation*}
\IM \asrt_{p \after k}
\after \asrt_q
\ = \ (\asrt_{p \after k})_\diamond(q)
\ =\  \ceil{\ceil{q} \after \asrt_{p \after k}}
\ =\  \ceil{p \after k}.
\end{equation*}
So there is a unique isomorphism~$\omega$ such that
\begin{equation}\label{dagger-iso-omega}
\asrt_{p \after k} \after \asrt_q
    \ =\   \pi_{\lceil p \after k \rceil}
    \after \omega \after \zeta_{\ceil{p \after g}} \after \asrt_{p \after g}.
\end{equation}
Next, we consider~$
\zeta_{\ceil{p \after \pi_t}} \after \psi \after
                \zeta_{\ceil{q}}$.
Note~$\psi \after \zeta_{\ceil{q}}$
    is a quotient for a sharp predicate,
    hence sharp and so~$\ceil{p \after \pi_t} \after \psi \after \zeta_{\ceil{q}}
                =\lceil p \after \pi_t \after \psi
                \after \zeta_{\ceil{q}}\rceil = \lceil p \after k \rceil$
                by \sref{sharp-ceil}.
As quotients are closed under composition
(\sref{quotients-composition}),
    there is an iso~$\beta$
    with
    \begin{equation}\label{dagger-iso-beta}
    \zeta_{\ceil{p \after \pi_t}} \after \psi \after \zeta_{\ceil{q}}
    \ = \ \beta \after \zeta_{\ceil{p \after k}}.
\end{equation}
Finally, we deal with~$
\pi_s \after \varphi \after 
\pi_{\lceil t \after \asrt_p \after \pi_{\ceil{p}}\rceil}
$.
By \sref{upm-basics} this is again a comprehension.
We compute
\begin{align*}
    \IM
\pi_s \after \varphi \after 
\pi_{\lceil t \after \asrt_p \after \pi_{\ceil{p}}\rceil}
& \ = \ 
(\pi_s \after \varphi \after 
\pi_{\lceil t \after \asrt_p \after \pi_{\ceil{p}}\rceil})_\diamond(1)
\\
& \ \overset{\smash{\mathclap{\sref{quotcompr-diamond-adjoint}}}}{=}\ 
\zeta_s^\diamond ( \varphi_\diamond (
\lceil t \after \asrt_p \after \pi_{\ceil{p}}\rceil)) \\
& \ =\ 
\lceil t \after \asrt_p \after \pi_{\ceil{p}} \rceil
\after \varphi^{-1} \after \zeta_s \\
& \ \overset{\smash{\mathclap{\sref{sharp-ceil}}}}{=}\ 
\lceil t \after \asrt_p \after \pi_{\ceil{p}}
\after \varphi^{-1} \after \zeta_s \rceil \\
& \ =\ 
\lceil t \after f^\dagger\rceil
\end{align*}
and so there must be a unique iso~$\alpha$ with
\begin{equation}\label{dagger-iso-alpha}
    \pi_s \after \varphi \after \pi_{\lceil t \after \asrt_p \after
    \pi_{\ceil{p}} \rceil} = \pi_{\ceil{t \after f^\dagger}} \after \alpha.
\end{equation}
\end{point}
\begin{point}{Lemma}%
In setting \sref{dagger-setting}, we have
    $f \after g  =  \pi_{\ceil{t \after f^\dagger}}
        \after \alpha \after \chi \after \beta \after \omega
        \after \zeta_{\ceil{p \after g}} \after
        \asrt_{p \after g}$.
\begin{point}{Proof}%
It's a long, but easy verification,
    either with a diagram
\begin{equation*}
    \xymatrix@C+3pc {
        \bullet \ar[rr]^g
        \ar[rd]|{\asrt_q}
        \ar[ddd]_{\rotatebox{90}{$\scriptstyle\omega\after\zeta_{\ceil{p \after g}} \after \asrt_{p \after g}$}}
        && \bullet \ar[rr]^f
            \ar[rd]|{\zeta_{\ceil{p}}\after \asrt_p}
        && \bullet
            \\ \ar@{}[rd]|{\text{\eqref{dagger-iso-omega}}}
            & \bullet
                \ar[r]^{\psi \after \zeta_{\ceil{q}}}
                \ar[d]|{\asrt_{p \after k}}
                \ar@{}[rd]|{\text{\sref{pristine-asrt}}}
 & \bullet
            \ar[u]^{\pi_t}
            \ar[d]|{\asrt_{p \after \pi_t}}
            & \bullet \ar[ru]_{\pi_s\after\varphi}
            \ar@{}[rd]|{\text{\eqref{dagger-iso-alpha}}}
            \\& \bullet \ar[r]^{\psi \after \zeta_{\ceil{q}}}
                        \ar[d]|{\zeta_{\ceil{p \after k}}}
                \ar@{}[rd]|{\text{\eqref{dagger-iso-beta}}}
            &\bullet \ar[d]|{\zeta_{\ceil{p \after \pi_t}}}
                        \ar@{}[ru]|{\text{\eqref{dagger-iso-chi}}}
                        &&
            \\ \bullet \ar@{=}[r]
            \ar[ru]^{\pi_{\ceil{p \after k}}}
            &\bullet \ar[r]_\beta
            &\bullet \ar[rr]_\chi
            &&\bullet \ar[uuu]_{\rotatebox{90}{$\scriptstyle\pi_{\ceil{t \after f^\dagger}} \after \alpha$}}
                        \ar[luu]|{\pi_{\lceil t \after \asrt_p \after \pi_{\ceil{p}} \rceil}}
        }
\end{equation*}
or with equational reasoning
\begin{align*}
   f \after g
    & \ = \ \pi_s \after \varphi \after \zeta_{\ceil{p}}
        \after \asrt_p \after \pi_t \after \psi \after \zeta_{\ceil{q}}
        \after \asrt_q \\
    & \ \overset{\smash{\mathclap{\eqref{dagger-iso-chi}}}}{=} \ 
        \pi_s \after \varphi \after
        \pi_{\lceil t \after \asrt_p \after \pi_{\ceil{p}} \rceil}
        \after \chi \after \zeta_{\ceil{p \after \pi_t}}
        \after \asrt_{p \after \pi_t}
    \after \psi \after 
        \zeta_{\ceil{q}}
        \after \asrt_q \\
    & \ \overset{\smash{\mathclap{\sref{pristine-asrt}}}}{=} \ 
        \pi_s \after \varphi \after
        \pi_{\lceil t \after \asrt_p \after \pi_{\ceil{p}} \rceil}
        \after \chi \after \zeta_{\ceil{p \after \pi_t}}
        \after \psi \after \zeta_{\ceil{q}}
        \after \asrt_{p \after k}
        \after \asrt_q \\
    & \ \overset{\smash{\mathclap{\eqref{dagger-iso-beta}}}}{=} \ 
        \pi_s \after \varphi \after
        \pi_{\lceil t \after \asrt_p \after \pi_{\ceil{p}} \rceil}
        \after \chi
        \after \beta \after \zeta_{\ceil{p \after k}}
        \after \asrt_{p \after k}
        \after \asrt_q \\
    & \ \overset{\smash{\mathclap{\eqref{dagger-iso-alpha}}}}{=} \ 
        \pi_{\ceil{t \after f^\dagger}}\after\alpha
        \after \chi
        \after \beta \after \zeta_{\ceil{p \after k}}
        \after \asrt_{p \after k}
        \after \asrt_q \\
    & \ \overset{\smash{\mathclap{\eqref{dagger-iso-omega}}}}{=} \ 
        \pi_{\ceil{t \after f^\dagger}}\after\alpha
        \after \chi
        \after \beta \after \zeta_{\ceil{p \after k}}
        \after \pi_{\ceil{p \after k}} \after \omega
            \after \zeta_{\ceil{p \after g}}
            \after \asrt_{p \after g} \\
    & \ = \ 
        \pi_{\ceil{t \after f^\dagger}}\after\alpha
        \after \chi
        \after \beta \after
        \omega
            \after \zeta_{\ceil{p \after g}}
            \after \asrt_{p \after g},
\end{align*}
whichever the Reader might prefer. \qed
\end{point}
\end{point}
\begin{point}[dagger-of-fg]{Corollary}%
    $(f \after g)^\dagger = \asrt_{p \after g}
                \after \pi_{\ceil{p \after g}}
                \after \omega^{-1}
                \after \beta^{-1}
                \after \chi^{-1}
                \after \alpha^{-1}
                \after \zeta_{\ceil{t \after f^\dagger}} $ .
\begin{point}%
To show~$(f \after g)^\dagger = g^\dagger \after f^\dagger$,
    it is sufficient to proof that
    the `daggered' version
    of each of the subdiagrams
        \eqref{dagger-iso-alpha},
        \eqref{dagger-iso-beta},
        \eqref{dagger-iso-chi},
        \eqref{dagger-iso-omega} and
        \sref{pristine-asrt} of the above diagram holds.
We start with the simple ones.
\end{point}
\end{point}
\begin{point}[dagger-iso-beta2]{Lemma}%
In setting \sref{dagger-setting},
    the daggered version of
        \eqref{dagger-iso-beta} holds --- that is:
    \begin{equation*}
        \pi_{\ceil{p \after k}} \after \beta^{-1}
            \ =\  \pi_{\ceil{q}} \after \psi^{-1} \after \pi_{\ceil{p \after \pi_t}}.
    \end{equation*}
\begin{point}{Proof}%
The map~$
\pi_{\ceil{q}} \after \psi^{-1} \after \pi_{\ceil{p \after \pi_t}} \after \beta$
is a comprehension for~$\ceil{p \after k}$ --- indeed
\begin{align*}
    (\pi_{\ceil{q}} \after \psi^{-1} \after \pi_{\ceil{p \after \pi_t}}\after\beta)_\diamond(1)
    &\ = \ 
    (\zeta^\diamond_{\ceil{q}} \after \psi^\diamond) (\ceil{p \after \pi_t})\\
    &\ = \ \ceil{p \after \pi_t} \after \zeta_{\ceil{q}} \after \psi \\
    &\ = \ \ceil{p \after \pi_t \after \zeta_{\ceil{q}} \after \psi} \\
    &\ = \ \ceil{p \after k}.
\end{align*}
Furthermore
\begin{align*}
    &\zeta_{\ceil{p \after k}} \after
    \pi_{\ceil{q}} \after \psi^{-1} \after \pi_{\ceil{p \after \pi_t}}
    \after \beta \\
    & \qquad \ \overset{\mathclap{\eqref{dagger-iso-beta}}}{=}\  
    \beta^{-1}\after \zeta_{\ceil{p \after \pi_t}}
    \after \psi \after \zeta_{\ceil{q}}
    \after \pi_{\ceil{q}} \after \psi^{-1} \after \pi_{\ceil{p \after \pi_t}}
    \after \beta \\
    & \qquad \ = \ \id.
\end{align*}
So~$
\pi_{\ceil{q}} \after \psi^{-1} \after \pi_{\ceil{p \after \pi_t}}
\after \beta
$ is the unique comprehension corresponding to~$\zeta_{\ceil{p \after k}}$ --- that is:~$
\pi_{\ceil{q}} \after \psi^{-1} \after \pi_{\ceil{p \after \pi_t}}
\after \beta \ = \ \pi_{\ceil{p \after k}} $,
as desired. \qed
\end{point}
\end{point}
\begin{point}[dagger-iso-alpha2]{Exercise}%
Show that in the setting \sref{dagger-setting},
    the daggered version of
        \eqref{dagger-iso-alpha} holds --- i.e.
    \begin{equation*}
        \alpha^{-1} \after \zeta_{\ceil{t \after f^\dagger}}
        \ =\   \zeta_{\lceil t \after \asrt_p \after \pi_{\ceil{p}} \rceil}
        \after \varphi^{-1} \after \zeta_s.
    \end{equation*}
(Hint: mimic the proof of~\sref{dagger-iso-beta2}.)
\end{point}
\begin{point}[dagger-iso-zeta2]{Exercise}%
Show that in the setting \sref{dagger-setting},
we have
\begin{equation*}
    \pi_{\ceil{q}} \after \psi^{-1} \after \asrt_{p \after \pi_t}
    \ = \ \asrt_{p \after k} \after \pi_{\ceil{q}} \after \psi^{-1}.
\end{equation*}
(This is the daggered version of the subdiagram
marked~\sref{pristine-asrt}.)
\end{point}
\begin{point}[dagger-iso-mu]{Proposition}%
If in a~$\dagger'$-effectus,
$\nu$ is the unique iso (cf.~\sref{pqqp-from-dagger}) such that
\begin{equation*}
    \asrt_a \after \asrt_b\  =\  \pi_{\ceil{\andthen{a}{b}}} 
    \after \nu \after \zeta_{\ceil{\andthen{b}{a}}} \after \asrt_{\andthen{b}{a}},
\end{equation*}
then~$\asrt_b \after \asrt_a\  =\  \asrt_{\andthen{b}{a}}
        \after \pi_{\ceil{\andthen{b}{a}}} 
        \after \nu^{-1} \after \zeta_{\ceil{\andthen{a}{b}}}$.
\begin{point}{Proof}%
Let~$\mu$ be the unique iso with~$
\asrt_b \after \asrt_a\  =\  
        \pi_{\ceil{\andthen{b}{a}}} 
        \after \mu \after \zeta_{\ceil{\andthen{a}{b}}}
                \after \asrt_{{\andthen{a}{b}}}$.
We will see~$\mu = \nu^{-1}$.
For brevity, write
\begin{align*}
    \overline{\andthen{a}{b}} &\ = \ (\andthen{a}{b}) \after \pi_{\ceil{\andthen{a}{b}}} &
\overline{\andthen{b}{a}} &\ = \ (\andthen{b}{a}) \after \pi_{\ceil{\andthen{b}{a}}}
\end{align*}
By \sref{asrt-pristine-reverse} and~\sref{asrt-absorp-rule}, we have
\begin{equation*}
    \pi_{\ceil{\andthen{a}{b}}} \after
\asrt_{\overline{\andthen{a}{b}}} \after
 \zeta_{\ceil{\andthen{a}{b}}}
    \ = \ 
    \pi_{\ceil{\andthen{a}{b}}} \after
 \zeta_{\ceil{\andthen{a}{b}}} \after
\asrt_{\andthen{a}{b}}
    \ = \ 
\asrt_{\andthen{a}{b}}.
\end{equation*}
Now the second axiom of a~$\dagger'$-effectus comes into play
\begin{align*}
    &\pi_{\ceil{\andthen{a}{b}}} \after
    \asrt^2_{\overline{\andthen{a}{b}}} \after \zeta_{\ceil{\andthen{a}{b}}} \\
    &\qquad \ = \ 
    \pi_{\ceil{\andthen{a}{b}}} \after
\asrt_{\overline{\andthen{a}{b}}} \after
 \zeta_{\ceil{\andthen{a}{b}}} \after
\pi_{\ceil{\andthen{a}{b}}} \after
\asrt_{\overline{\andthen{a}{b}}}
\after \zeta_{\ceil{\andthen{a}{b}}} \\
    &\qquad \ = \ 
\asrt_{\andthen{a}{b}}^2
\\
    &\qquad \ = \ 
\asrt_{a} \after
\asrt^2_{b}\after
\asrt_{a}
\\
    &\qquad \ = \ 
\pi_{\ceil{\andthen{a}{b}}} \after
    \nu \after
    \asrt_{\overline{\andthen{b}{a}}} \after
    \zeta_{\ceil{\andthen{b}{a}}} \after
\pi_{\ceil{\andthen{b}{a}}} \after
    \mu \after
    \asrt_{\overline{\andthen{a}{b}}} \after
    \zeta_{\ceil{\andthen{a}{b}}} 
\\
    &\qquad \ = \ 
\pi_{\ceil{\andthen{a}{b}}} \after
    \nu \after
    \asrt_{\overline{\andthen{b}{a}}} \after
    \mu \after
    \asrt_{\overline{\andthen{a}{b}}} \after
    \zeta_{\ceil{\andthen{a}{b}}}.
\end{align*}
Thus as quotients are epis and comprehensions are monos:
\begin{equation} \label{dagger-second-axiom-intermediate}
\asrt^2_{\overline{\andthen{a}{b}}}
         \ = \ \nu \after \asrt_{\overline{\andthen{b}{a}}}
            \after \mu \after \asrt_{\overline{\andthen{a}{b}}}.
\end{equation}
We want to show~$\asrt_{\overline{\andthen{a}{b}}}$ is an epi.
First note
\begin{equation*}
    \ceil{\overline{\andthen{a}{b}}}\after\zeta_{\ceil{\andthen{a}{b}}}
    \ \overset{\sref{sharp-ceil}}{=} \ 
    \ceil{\overline{\andthen{a}{b}}\after\zeta_{\ceil{\andthen{a}{b}}}}
    \ \overset{\sref{asrt-absorp-rule}}{=} \ 
    \ceil{\andthen{a}{b}} \ = \ 1 \after \zeta_{\ceil{\andthen{a}{b}}}
\end{equation*}
and so~$\IM \asrt_{\overline{\andthen{a}{b}}}
= \ceil{\overline{\andthen{a}{b}}} = 1 $,
    which tells us~$\asrt_{\overline{\andthen{a}{b}}}$
    is a quotient and therefore an epi.
    So, from \eqref{dagger-second-axiom-intermediate} we get
    $ \asrt_{\overline{\andthen{a}{b}}}
     =  \nu \after \asrt_{\overline{\andthen{b}{a}}} \after \mu$ and so
\begin{equation}\label{dagger-seqprod-inversion}
    \overline{\andthen{a}{b}}
    \ = \ 1 \after \asrt_{\overline{\andthen{a}{b}}}
    \ = \ 1 \after \nu \after \asrt_{\overline{\andthen{b}{a}}} \after \mu
    \ = \ \overline{\andthen{b}{a}} \after \mu.
\end{equation}
Using this equation again in the previous, we find
\begin{equation*}
    \nu \after \mu \after \asrt_{\overline{\andthen{a}{b}}}
            \ = \ 
    \nu \after \mu \after \asrt_{\overline{\andthen{b}{a}}\after \mu}
            \ = \ 
    \nu \after \asrt_{\overline{\andthen{b}{a}}} \after \mu
            \ = \ \asrt_{\overline{\andthen{a}{b}}}.
\end{equation*}
Thus~$\nu \after \mu = \id$ and so~$\mu = \nu^{-1}$.
Write~$l = \pi_{\ceil{\andthen{b}{a}}} \after \nu^{-1}
                        \after \zeta_{\ceil{\andthen{a}{b}}}$. Then
\begin{alignat*}{2}
    (\andthen{a}{b}) \after l^\dagger & \ = \ 
    (\andthen{a}{b})
        \after \pi_{\ceil{\andthen{a}{b}}} 
        \after \nu
        \after \zeta_{\ceil{\andthen{b}{a}}}  
        &\qquad& \text{by \sref{asrt-pristine-reverse}}\\
        & \ = \ 
        \overline{\andthen{a}{b}}
        \after \nu
        \after \zeta_{\ceil{\andthen{b}{a}}}  \\
        & \ = \ 
        \overline{\andthen{b}{a}}
        \after \zeta_{\ceil{\andthen{b}{a}}}  
        &&\text{by \eqref{dagger-seqprod-inversion} and~$\mu = \nu^{-1}$}\\
        & \ = \ 
        \andthen{b}{a}.
\end{alignat*}
And so, keeping in mind~$\andthen{a}{b} \leq \ceil{\andthen{a}{b}}
    =  1 \after l$,
we have
\begin{alignat*}{2}
    \asrt_b \after \asrt_a &\ = \ 
        l \after \asrt_{\andthen{a}{b}} \\
        & \ = \ 
        \asrt_{(\andthen{a}{b}) \after l^\dagger} \after l 
    &\qquad&\text{by \sref{asrt-pristine-reverse}} \\
        &\ = \ 
        \asrt_{\andthen{b}{a}} \after l \\
        &\ = \ 
        \asrt_{\andthen{b}{a}}
        \after \pi_{\ceil{\andthen{b}{a}}} 
        \after \nu^{-1} \after \zeta_{\ceil{\andthen{a}{b}}},
\end{alignat*}
as promised. \qed
\end{point}
\end{point}
\begin{point}[dagger-iso-omega2]{Corollary}
In setting \sref{dagger-setting},
    the daggered version of
        \eqref{dagger-iso-omega} holds --- that is:
    \begin{equation*}
\asrt_q \after
\asrt_{p \after k}
    \ =\ 
    \asrt_{p \after g} \after
    \pi_{\ceil{p \after g}} \after
    \omega^{-1} \after
    \zeta_{\lceil p \after k \rceil}.
    \end{equation*}
\end{point}
\begin{point}[dagger-iso-chi2]{Lemma}%
In setting \sref{dagger-setting},
    the daggered version of
        \eqref{dagger-iso-chi} holds --- that is:
    \begin{equation*}
    \zeta_t \after
    \asrt_p \after
    \pi_{\ceil{p}}
    \ = \ 
    \asrt_{p \after \pi_t} \after
    \pi_{\ceil{p \after \pi_t}} \after
    \chi^{-1} \after
    \zeta_{\lceil t \after \asrt_p \after \pi_{\ceil{p}}\rceil}.
    \end{equation*}
\begin{point}{Proof}%
The heavy lifting has been done in \sref{dagger-iso-mu} already. To start, note
    \begin{align*}
\asrt_p \after \asrt_t
& \ \overset{\mathclap{\smash{\sref{asrt-absorp-rule}}}}{=}
    \ \asrt_{\ceil{p}} \after \asrt_p  \after \asrt_t \\
    & \ = \ \pi_{\ceil{p}} \after \zeta_{\ceil{p}} \after \asrt_p \after
                    \pi_t \after \zeta_t \\
                    & \ \overset{\mathclap{\smash{\eqref{dagger-iso-chi}}}}{=} \ 
        \pi_{\ceil{p}} \after 
        \pi_{\lceil t \after
        \asrt_p \after \pi_{\ceil{p}}\rceil} \after
        \chi \after \zeta_{\ceil{p \after \pi_t}} \after
        \asrt_{p \after \pi_t} \after
        \zeta_t \\
                    & \ \overset{\mathclap{\smash{\sref{pristine-asrt}}}}{=} \ 
        \pi_{\ceil{p}} \after 
        \pi_{\lceil t \after
        \asrt_p \after \pi_{\ceil{p}}\rceil} \after
        \chi \after \zeta_{\ceil{p \after \pi_t}} \after
        \zeta_t \after
        \asrt_{p \after \pi_t\after \zeta_t} \\
                    & \  = \ 
        \pi_{\ceil{p}} \after 
        \pi_{\lceil t \after
        \asrt_p \after \pi_{\ceil{p}}\rceil} \after
        \chi \after
        \zeta_{\ceil{p \after \pi_t}} \after
        \zeta_t \after
        \asrt_{\andthen{t}{p}} \\
                    & \  = \ 
        \pi_{\ceil{\andthen{p}{t}}} \after \alpha_2 \after
        \chi \after \beta_2 \after
        \zeta_{\ceil{\andthen{t}{p}}} \after
        \asrt_{\andthen{t}{p}},
    \end{align*}
where~$\alpha_2$ and~$\beta_2$ are the unique isomorphisms such that
\begin{align*}
        \pi_{\ceil{p}} \after 
        \pi_{\lceil t \after
        \asrt_p \after \pi_{\ceil{p}}\rceil} &\ = \ 
            \pi_{\ceil{\andthen{p}{t}}} \after \alpha_2 &
        \zeta_{\ceil{p \after \pi_t}} \after
        \zeta_t
        &\ = \ 
        \beta_2 \after \zeta_{\ceil{\andthen{t}{p}}}.
\end{align*}
With the same reasoning as in \sref{dagger-iso-alpha2}
        and~\sref{dagger-iso-beta2}, we see
\begin{align*}
        \zeta_{\lceil t \after
        \asrt_p \after \pi_{\ceil{p}}\rceil} \after
        \zeta_{\ceil{p}}
        &\ = \ 
        \alpha_2^{-1} \after
            \zeta_{\ceil{\andthen{p}{t}}}
            &
        \pi_t \after
        \pi_{\ceil{p \after \pi_t}}
        &\ = \ 
        \pi_{\ceil{\andthen{t}{p}}}
        \after \beta_2^{-1}.
\end{align*}
Now we can apply \sref{dagger-iso-mu}:
\begin{align*}
    \zeta_t \after \asrt_p \after \pi_{\ceil{p}} 
    &\ \overset{\smash{\mathclap{\sref{asrt-absorp-rule}}}}{=} \ 
    \zeta_t \after \asrt_t \after
    \asrt_p \after
    \pi_{\ceil{p}}
    \\
    &\ \overset{\smash{\mathclap{\sref{dagger-iso-mu}}}}{=} \ 
    \zeta_t \after \asrt_{\andthen{t}{p}} \after
    \pi_{\ceil{\andthen{t}{p}}} \after
    \beta_2^{-1} \after
    \chi^{-1} \after
    \alpha_2^{-1} \after
    \zeta_{\ceil{\andthen{p}{t}}} \after
    \pi_{\ceil{p}}
    \\
    &\ = \ 
    \zeta_t \after \asrt_{\andthen{t}{p}} \after
        \pi_t \after
        \pi_{\ceil{p \after \pi_t}} \after
    \chi^{-1} \after
        \zeta_{\lceil t \after
        \asrt_p \after \pi_{\ceil{p}}\rceil} \after
        \zeta_{\ceil{p}} \after
    \pi_{\ceil{p}}
    \\
    &\ = \ 
    \zeta_t \after \asrt_{p \after \pi_t \after \zeta_t} \after
        \pi_t \after
        \pi_{\ceil{p \after \pi_t}} \after
    \chi^{-1} \after
        \zeta_{\lceil t \after
        \asrt_p \after \pi_{\ceil{p}}\rceil}
    \\
    &\ = \ 
    \asrt_{p \after \pi_t} \after 
        \pi_{\ceil{p \after \pi_t}} \after
    \chi^{-1} \after
        \zeta_{\lceil t \after
        \asrt_p \after \pi_{\ceil{p}}\rceil},
\end{align*}
as desired. \qed
\end{point}
\end{point}
\begin{point}[dagger-is-functor]{Proposition}%
In a~$\dagger'$-effectus~$(f \after g)^\dagger = g^\dagger \after f^\dagger$
    holds.
\begin{point}{Proof}%
We work in setting~\sref{dagger-setting}.
The equality~$(f\after g)^\dagger = g^\dagger \after f^\dagger$
follows from~\sref{dagger-of-fg} and the commutativity
of the following diagram
\begin{equation*}
    \xymatrix@C+3pc {
        \bullet \ar@{<-}[rr]^{g^\dagger}
        \ar@{<-}[rd]|{\asrt_q}
        \ar@{<-}[ddd]_{\rotatebox{90}{$\scriptstyle
            \asrt_{p \after g}
            \pi_{\ceil{p \after g}} \after
            \omega^{-1}
        $}}
        && \bullet \ar@{<-}[rr]^{f^\dagger}
            \ar@{<-}[rd]|{ \asrt_p \after \zeta_{\ceil{p}} }
        && \bullet
            \\ \ar@{}[rd]|{\text{\sref{dagger-iso-omega2}}}
            & \bullet
            \ar@{<-}[r]^{ \pi_{\ceil{q}} \after \psi^{-1} }
            \ar@{<-}[d]|{\asrt_{p \after k}}
                \ar@{}[rd]|{\text{\sref{dagger-iso-zeta2}}}
 & \bullet
                \ar@{<-}[u]^{\zeta_t}
                \ar@{<-}[d]|{\asrt_{p \after \pi_t}}
                & \bullet \ar@{<-}[ru]_{\varphi^{-1} \after \zeta_s}
            \ar@{}[rd]|{\text{\sref{dagger-iso-alpha2}}}
            \\& \bullet \ar@{<-}[r]^{\pi_{\ceil{q}} \after \psi^{-1}}
            \ar@{<-}[d]|{\pi_{\ceil{p \after k}}}
                \ar@{}[rd]|{\text{\sref{dagger-iso-beta2}}}
                &\bullet \ar@{<-}[d]|{\pi_{\ceil{p \after \pi_t}}}
                        \ar@{}[ru]|{\text{\sref{dagger-iso-chi2}}}
                        &&
            \\ \bullet \ar@{=}[r]
            \ar@{<-}[ru]^{\zeta_{\ceil{p \after k}}}
            &\bullet \ar@{<-}[r]_{\beta^{-1}}
            &\bullet \ar@{<-}[rr]_{\chi^{-1}}
            &&\bullet \ar@{<-}[uuu]_{\rotatebox{90}{$\scriptstyle
    \alpha^{-1}\after \zeta_{\ceil{t \after f^\dagger}}$}}
    \ar@{<-}[luu]|{ \zeta_{\lceil t \after \asrt_p \after \pi_{\ceil{p}} \rceil}} }
\end{equation*}
or alternatively by
\begin{align*}
    g^\dagger \after f^\dagger
        & \ = \ 
            \asrt_q \after
            \pi_{\ceil{q}} \after
            \psi^{-1} \after
            \zeta_t \after
            \asrt_p \after
            \pi_{\ceil{p}} \after
            \varphi^{-1} \after
            \zeta_s
        \\
        & \ \overset{\smash{\mathclap{\sref{dagger-iso-chi2}}}}{=} \ 
            \asrt_q \after
            \pi_{\ceil{q}} \after
            \psi^{-1} \after
    \asrt_{p \after \pi_t} \after
    \pi_{\ceil{p \after \pi_t}} \after
    \chi^{-1} \after
    \zeta_{\lceil t \after \asrt_p \after \pi_{\ceil{p}}\rceil} \after
            \varphi^{-1} \after
            \zeta_s
        \\
        & \ \overset{\smash{\mathclap{\sref{dagger-iso-zeta2}}}}{=} \ 
            \asrt_q \after
    \asrt_{p \after k} \after
    \pi_{\ceil{q}} \after
    \psi^{-1} \after
    \pi_{\ceil{p \after \pi_t}} \after
    \chi^{-1} \after
    \zeta_{\lceil t \after \asrt_p \after \pi_{\ceil{p}}\rceil} \after
            \varphi^{-1} \after
            \zeta_s
        \\
        & \ \overset{\smash{\mathclap{\sref{dagger-iso-beta2}}}}{=} \ 
            \asrt_q \after
    \asrt_{p \after k} \after
        \pi_{\ceil{p \after k}} \after 
        \beta^{-1} \after
    \chi^{-1} \after
    \zeta_{\lceil t \after \asrt_p \after \pi_{\ceil{p}}\rceil} \after
            \varphi^{-1} \after
            \zeta_s
        \\
        & \ \overset{\smash{\mathclap{\sref{dagger-iso-alpha2}}}}{=} \ 
            \asrt_q \after
    \asrt_{p \after k} \after
        \pi_{\ceil{p \after k}} \after 
        \beta^{-1} \after
    \chi^{-1} \after
        \alpha^{-1} \after
        \zeta_{\ceil{t \after f^\dagger}}
        \\
        & \ \overset{\smash{\mathclap{\sref{dagger-iso-omega2}}}}{=} \ 
    \asrt_{p \after g} \after
    \pi_{\ceil{p \after g}} \after
    \omega^{-1} \after
    \zeta_{\lceil p \after k \rceil} \after
        \pi_{\ceil{p \after k}} \after 
        \beta^{-1} \after
    \chi^{-1} \after
        \alpha^{-1} \after
        \zeta_{\ceil{t \after f^\dagger}}
        \\
        & \ = \ 
    \asrt_{p \after g} \after
    \pi_{\ceil{p \after g}} \after
    \omega^{-1} \after
        \beta^{-1} \after
    \chi^{-1} \after
        \alpha^{-1} \after
        \zeta_{\ceil{t \after f^\dagger}}
        \\
        & \ \overset{\smash{\mathclap{\sref{dagger-of-fg}}}}{=} \ 
        (f \after g)^\dagger,
\end{align*}
whichever the Reader might prefer. \qed
\end{point}
\end{point}
\end{parsec}

\begin{parsec}%
\begin{point}%
We are ready to finish the proof of \sref{dagger-theorem}.
\end{point}
\begin{point}[dagger-thm-sufficiency]{Theorem}%
    A~$\dagger'$-effectus is a~$\dagger$-effectus
        with~$\dagger$
            as defined in \sref{dagger-definition2}.
\begin{point}{Proof}%
Let~$C$ be a~$\dagger'$-effectus.
\begin{point}{Ax.~1}%
By~\sref{dagger-is-functor}, \sref{dagger-idempotent}
    and~\sref{dagger-prime-basics}
    the~$\dagger$ defined in~\sref{dagger-definition2}
    turns~$\Pure C$ into a~$\dagger$-category.
Also by~\sref{dagger-prime-basics},
    we have~$\asrt_p^\dagger = \asrt_p$
    for any predicate~$p$.
Pick any pure~$f$.
We have to show~$f$ is~$\diamond$-adjoint to~$f^\dagger$.
By~\sref{standard-form-map}
    we have~$f =
    \pi_{\IM f} \after \varphi \after \zeta_{\ceil{1 \after f}}
        \after \asrt_{1 \after f}$
        for some iso~$\alpha$.
We compute
\begin{align*}
   f_\diamond 
   & \ = \ 
   (\pi_{\IM f})_\diamond \after \varphi_\diamond \after (\zeta_{\ceil{1 \after f}})_\diamond
   \after (\asrt_{1 \after f})^\diamond \\
   & \ \overset{\mathclap{\smash{\sref{quotcompr-diamond-adjoint}}}}{=} \ 
   (\zeta_{\IM f})^\diamond \after \varphi_\diamond \after (\pi_{\ceil{1 \after f}})^\diamond
   \after (\asrt_{1 \after f})^\diamond \\
   & \ \overset{\mathclap{\smash{\sref{iso-diamond-adjoint}}}}{=} \ 
   (\zeta_{\IM f})^\diamond \after (\varphi^{-1})^\diamond \after (\pi_{\ceil{1 \after f}})^\diamond
   \after (\asrt_{1 \after f})^\diamond \\
   & \ = \ 
   (\asrt_{1 \after f} \after
   \pi_{\ceil{1 \after f}} \after
   {\varphi^{-1}} \after
   \zeta_{\IM f})^\diamond
   \\
   & \ = \ 
   (f^\dagger)^\diamond,
\end{align*}
so~$f$ is indeed~$\diamond$-adjoint to~$f^\dagger$.
\end{point}
\begin{point}{Ax.~2}%
Let~$f$ be a~$\dagger$-positive map.
That is: $f = h^\dagger \after h$ for some pure map~$h$.
By~\sref{standard-form-map}
    we have~$h =
    \pi_{\IM h} \after \alpha \after \zeta_{\ceil{1 \after h}}
                    \after \asrt_{1 \after h}
                    $
                    for some iso~$\alpha$.
            We compute
\begin{align*}
    f &\ = \ h^\dagger \after h \\
    &\ = \ 
    \asrt_{1 \after h} \after \pi_{\ceil{1 \after  h}} \after \alpha^{-1} \after \zeta_{\IM h} \after
    \pi_{\IM h} \after \alpha \after \zeta_{\ceil{1 \after h}}
                    \after \asrt_{1 \after h}
                    \\
    &\ = \ 
    \asrt_{1 \after h} \after \asrt_{\ceil{1 \after  h}}
                    \after \asrt_{1 \after h}
                    \\
                    &\ \overset{\smash{\mathclap{\sref{asrt-absorp-rule}}}}{=} \ 
    \asrt_{1 \after h} \after \asrt_{1 \after h}.
\end{align*}
Form this it follows that $\dagger$-positive maps are~$\diamond$-positive.
Let~$q$ be the predicate such that~$\andthen{q}{q} = 1 \after h$.
Then
\begin{equation*}
\asrt_{q}^\dagger \after \asrt_{q}  
    \ =\  \asrt_q\after \asrt_q
    \ \overset{\sref{andthen-square-rule}}{=}\  
    \asrt_{\andthen{q}{q}}
    \  =\  \asrt_{1 \after h}.
\end{equation*}
Thus~$\asrt_{1 \after h}$ is a~$\dagger$-positive
    with~$\asrt_{1 \after h} \after \asrt_{1 \after h} = f$.
We have to show~$\asrt_{1 \after h}$
    is the unique map with this property.
Let~$g$ be any (other)~$\dagger$-positive map with~$g \after g = f$.
Recall both~$g$ and~$f$ are~$\diamond$-positive, hence
\begin{equation*}
    \asrt_{1 \after f} \ =\  f\ =\ g \after g \ = \ \asrt_{1 \after g}
            \after \asrt_{1 \after g}
            \ \overset{\smash{\sref{andthen-square-rule}}}{=}\  
            \asrt_{\andthen{(1 \after g)}{(1 \after g)}}.
\end{equation*}
So~$\andthen{(1 \after g)}{(1\after g)} = 1\after f =
            \andthen{(1 \after h)}{(1 \after h)}$.
Hence~$1\after g = 1 \after h$ by uniqueness of the square root.
So~$g = \asrt_{1 \after g} = \asrt_{1 \after h} = h$, as desired.
\end{point}
\begin{point}{Ax.~3}%
Let~$\asrt_p$ be any~$\diamond$-positive map.
Write~$q$ for the unique predicate with~$\andthen{q}{q}=p$.
Then
\begin{equation*}
    \asrt_p
     \ = \ \asrt_{\andthen{q}{q}}
            \ \overset{\smash{\sref{andthen-square-rule}}}{=}\ 
     \asrt_q \after \asrt_q  \ =\ 
     \asrt_q^\dagger \after \asrt_q ,
\end{equation*}
so~$\asrt_p$ is~$\dagger$-positive, as desired. \qed
\end{point}
\end{point}
\end{point}
\end{parsec}
\end{document}

% vim: ft=tex.latex
