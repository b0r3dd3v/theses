\documentclass[main]{subfiles}
\begin{document}
\chapter{Von Neumann algebras}
%
% von Neumann algebras
%
\begin{parsec}[vna]%
\begin{point}{Definition}%
A $C^*$-algebra~$\scrA$
is a \Define{von Neumann algebra}
when
\begin{enumerate}
\item
every bounded directed subset~$D$
of self-adjoint elements of~$\scrA$ (so $D\subseteq \sa{\scrA}$) 
has a supremum $\bigvee D$ in $\sa{\scrA}$, and
\item
if $a$ is a positive element of~$\scrA$
with $\omega(a)=0$ for every \emph{normal} (see below) positive 
linear map $\omega\colon \scrA\to \C$,
then~$a=0$.
\end{enumerate}
\begin{point}%
A positive linear map $\omega\colon \scrA\to \C$
is called \Define{normal}
if $\omega(\bigvee D) = \bigvee_{d\in D} \omega(d)$
for every bounded directed subset of self-adjoint elements of~$D$
which has a supremum $\bigvee D$ in $\sa{\scrA}$.
\end{point}%
\begin{point}%
The \Define{ultraweak topology} on $\scrA$
is the least topology on~$\scrA$
that makes all normal positive linear maps $\omega\colon \scrA\to \C$
continuous.
\end{point}
\begin{point}%
The \Define{ultrastrong topology} on~$\scrA$
is the least topology on~$\scrA$
that makes $a\mapsto \omega(a^*a)$ continuous
for every np-map~$\omega\colon \scrA\to \C$.
\end{point}
\end{point}
\end{parsec}
%
% multiplication turns suprema into ultraweak limits
%
\begin{parsec}%
\begin{point}%
It is clear that translation and scaling
on a von Neumann algebra
are ultraweakly (and ultrastrongly) continuous.
We will see in this paragraph that multiplication
is also ultraweakly (and ultrastrongly) continuous
in each coordinate.
Quite surprisingly,
the problem reduces to the ultraweak continuity
of $b\mapsto a^*ba$ by the following identity
(cf.~\TODO{refer to polarization identity}).
\end{point}
\begin{point}{Exercise}%
Show that for elements~$a,b,c$ of a $C^*$-algebra,
\begin{equation*}
\textstyle
a^*\,c\,b\ =\ \frac{1}{4}\,\sum_{k=0}^3\ i^k\  (i^ka+b)^*\,c\,(i^ka+b).
\end{equation*}
\end{point}
\begin{point}[vanishing-effects]{Lemma}%
Let~$(x_\alpha)_\alpha$ be 
a net of effects of a von Neumann algebra~$\scrA$
which converges ultraweakly to~$0$.
Then $(x_\alpha a)_\alpha$ converges ultraweakly
to $0$ for every~$a\in\scrA$.
\begin{point}{Proof}%
Let~$\omega\colon \scrA\to \C$ be a normal positive linear map.
We have, for each~$\alpha$,
\begin{alignat*}{3}
\left|\,\omega(x_\alpha a)\,\right|^2
\ &=\ 
\left|\, \omega(\,\sqrt{x_\alpha}\,\sqrt{x_\alpha}\,a\,)\, \right|^2
\qquad&&\text{since $x_\alpha\geq 0$}\\
\ &\leq\ 
\omega(x_\alpha)\  \omega(\,a^* x_\alpha a\,) 
\qquad&&\text{by \sref{cstar-cs}}\\
\ &\leq\ 
\omega(x_\alpha)\ \omega(a^*a)
\qquad&&\text{since $x_\alpha\leq 1$}.
\end{alignat*}
Thus,
since $(\omega(x_\alpha))_\alpha$
converges to~$0$,
we see that $(\omega(x_\alpha a))_\alpha$
converges to~$0$,
and so $(x_\alpha a)_\alpha$ converges ultraweakly to~$0$.\qed
\end{point}
\end{point}
\begin{point}{Exercise}
Let~$D$ be a bounded directed set of self-adjoint
elements of a von Neumann algebra~$\scrA$,
and let~$a\in \scrA$.
\begin{point}[vna-supremum-uwlimit]%
Show that the net~$(d)_{d\in D}$ converges ultraweakly to~$\bigvee D$.
\end{point}
\begin{point}[vna-supremum-mult]%
Use~\sref{vanishing-effects}
to show that $(da)_d$ converges ultraweakly to~$(\bigvee D)a$,
and that~$(a^*d)_d$ converges ultraweakly to~$a^* (\bigvee D)$.
\end{point}
\begin{point}[vna-supremum-commutes]%
Show that if~$ad=da$ for all~$d\in D$,
then $a(\bigvee D) = (\bigvee D)a $.
\end{point}
\end{point}
%
%  ad is normal
%
\begin{point}[ad-normal]{Proposition}%
Let~$a$ be an element of a von Neumann algebra~$\scrA$.
Then~$\bigvee_{d\in D} a^*\,d\,a = a^*\,(\bigvee D)\, a$
for every bounded directed subset~$D$ of self-adjoint
elements of~$\scrA$.
\begin{point}[ad-normal-1]{Proof}%
If~$a$ is invertible,
then the (by~\sref{ad-monotone}) order preserving map $b\mapsto a^*ba$
has an order preserving inverse (namely $b\mapsto (a^{-1})^* b a^{-1}$),
and therefor preserves all suprema.
\begin{point}%
The general case reduces to the case that~$a$ 
is invertible
in the following way.
There is (by~\sref{spectrum-bounded})
 $\lambda>0$ such that $\lambda+a$ is invertible.
Then as $d$ increases 
\begin{equation*}
a^*\,d\,a \ \equiv\  (\lambda+a)^*\,d\,(\lambda+a) \,-\,
 \lambda^2 \,-\, \lambda a^*d \,-\, \lambda da
\end{equation*}
converges ultraweakly
to~$a^* \,(\bigvee D)\,a$,
because $(\ (\lambda+a)^*\,d\,(\lambda+a)\ )_d$
converges ultraweakly to $(\lambda+a)^*\,(\bigvee D)\,(\lambda+a)$
by~\sref{ad-normal-1} and~\sref{vna-supremum-uwlimit},
and $(a^*d+da)_d$ converges ultraweakly to $a^*(\bigvee D)+(\bigvee D)a$
by~\sref{vna-supremum-mult}.
Since~$(a^*da)_d$ converges to~$\bigvee_{d\in D} a^*d a$ too,
we conclude that~$\bigvee_{d\in D} a^* \,d\, a = a^*\,(\bigvee D)\,a$.\qed
\end{point}
\end{point}
\end{point}
\begin{point}{Exercise}%
Show that for a positive linear map $f\colon \scrA\to\scrB$
between von Neumann algebras,
the following are equivalent.
\begin{enumerate}
\item
$f$ is ultraweakly continuous;
\item 
$\omega\circ f\colon \scrA\to\C$ is normal 
for each np-map $f\colon \scrB\to\C$;
\item
$f(\bigvee D)=\bigvee_{d\in D}f(d)$ for each bounded 
directed~$D\subseteq\sa{\scrA}$.
\end{enumerate}
Conclude that $b\mapsto a^*ba,\,\scrA\to\scrA$
is ultraweakly continuous for every element~$a$ of a von Neumann 
algebra~$\scrA$.
\end{point}
\begin{point}{Proposition}%
The map $b\mapsto a^*ba,\,\scrA\to\scrA$
is ultrastrongly continuous for every
 element~$a$ of a von Neumann algebra~$\scrA$.
\begin{point}{Proof}%
To show that the linear map~$b\mapsto a^*ba$
is ultrastrongly continuous,
it suffices to show that it is ultrastrongly continuous at~$0$.
So let~$(b_\alpha)_\alpha$ be a net in~$\scrA$
which converges ultrastrongly to~$0$;
we must show that $(a^*b_\alpha a)_\alpha$
converges ultrastrongly to~$0$, c.q.~that
$(a^*b^*_\alpha aa^*b_\alpha a)_\alpha$ converges ultraweakly to~$0$.
Since
$a^*b^*_\alpha a a^* b_\alpha a \leq \|a\|^2 a^* b_\alpha^* b_\alpha a$
it suffices to show that~$(a^*b_\alpha^*b_\alpha a)_\alpha$
converges ultraweakly to~$0$,
but this follows from the
fact that~$c\mapsto a^*ca$ is ultraweakly continuous,
and~$(b_\alpha^*b_\alpha)_\alpha$
converges ultraweakly to~$0$.
\end{point}
\end{point}
\begin{point}{Exercise}%
Show that the maps $b\mapsto ab,\,ba\colon\ \scrA\to\scrA$
are ultraweakly and ultrastrongly continuous
for every element~$a$ of a von Neumann algebra~$\scrA$.
\end{point}
\end{parsec}



\subsection{Projections}
Let us now turn to the projections in a von Neumann algebra.

\TODO{Supremum/infimum of projections}

\TODO{Relation between  $\floor{\cdot}$, $\ceil{\cdot}$ and NCPsU-maps}

\TODO{NCPsU-maps with sharp $f(1)$ as restriction category}

\TODO{Central carier and von Neumann--Murray equivalence (for Paschke.)}

\TODO{$\ceil{pqp}=p\cdot q$ (Sakai product) and counter-example non-projections}

\TODO{$\floor{aba}=\floor{a}\cap \floor{b}$}



\begin{parsec}[ad-contraposed]%
\begin{point}{Lemma}%
Let~$a$ be an element of a $C^*$-algebra~$\scrA$
with $\|a\|\leq 1$,
and let~$p$ and~$q$ be projections on~$\scrA$.
Then 
$a^* p a \leq q^\perp$
iff $paq=0$
iff  $aqa^*\leq p^\perp$.
\end{point}
\begin{point}{Proof}%
Suppose that~$a^*pa\leq q^\perp$.
Then we have $q a^*pa q \leq qq^\perp q = 0$
(see \sref{ad-monotone})
and so $paq=0$,
because $\|paq\|^2=\|(paq)^*paq\|=0$
by the $C^*$-identity.
Applying $(\,\cdot\,)^*$,
we get $qa^*p=0$, and so both $qa^* = qa^*p^\perp$
and $aq = p^\perp aq$, giving
us $aqa^* = p^\perp a q a^* p^\perp 
\leq p^\perp$,
where we used that $aqa^*\leq aa^*\leq \|aa^*\|=\|a\|^2\leq 1$.
By a similar reasoning,
we get $aqa^*\leq p^\perp \implies paq=0\implies a^*pa\leq q^\perp$.
\end{point}
\end{parsec}
\begin{parsec}%
\begin{point}{Exercise}%
Let~$a$ be an effect of a $C^*$-algebra~$\scrA$,
and~$p$ be a projection from~$\scrA$.
\begin{point}[projection-above-effect]%
Show that $a\leq p$
iff $p\sqrt{a} = \sqrt{a}$
iff $\sqrt{a}p = \sqrt{a}$
iff $p^\perp\sqrt{a} = 0$
iff $\sqrt{a}p^\perp = 0$
iff $a^2\leq p$
iff $p a  = a$
iff $ a p = a $
iff $p^\perp a  = 0$
iff $ap^\perp = 0$
iff $\sqrt{a}\leq p$.
\end{point}
\begin{point}[projection-below-effect]%
Show that $p\leq a$
iff $p \sqrt{a} = p$
iff $\sqrt{a} p = p$
iff $ p\sqrt{a}^\perp = 0$
iff $\sqrt{a}^\perp p = 0$
iff $p\leq a^2$
iff $ap=p$
iff $pa = p$
iff $pa^\perp =0$
iff $a^\perp p =0$
iff $p\leq \sqrt{a}$.
\end{point}
\end{point}
\begin{point}[projection-below-projection]{Exercise}%
Let~$p$ and~$q$ be projections from a $C^*$-algebra
with~$p\leq q$.\\
Show that~$q-p$ is a projection.
\end{point}
\end{parsec}

%
% Projections
%
\begin{parsec}[vna-ceil]%
\begin{point}{Proposition}%
Above every effect~$b$ of a von Neumann algebra~$\scrA$,
there is a smallest projection, \Define{$\ceil{b}$},
called the \Define{ceiling} of~$b$,
 given by $\ceil{b}=\bigvee_{n=0}^\infty b^{\nicefrac{1}{2^n}}$.
\begin{point}[vna-ceil-commutes]%
Moreover, if $a\in \scrA$ commutes with $b$,
then~$a$ commutes with~$\ceil{b}$.
\end{point}
\end{point}
\begin{point}{Proof}
Let~$p$ denote the supremum of~$0\leq b\leq b^{\nicefrac{1}{2}}\leq
b^{\nicefrac{1}{4}}\leq\dotsb\leq 1$.
\begin{point}[vna-ceil-point-1]%
To begin,
note that if~$a\in \scrA$
commutes with~$b$,
then~$a$ commutes with~$p$.
Indeed, for such~$a$ we have~$a\sqrt{b}=\sqrt{b}a$
by~\sref{cstar-square-commutes},
and so $a b^{\nicefrac{1}{2^n}} = b^{\nicefrac{1}{2^n}} a$
for each~$n$
by induction.
Thus~$ap=pa$ by~\sref{vna-supremum-commutes}.
\end{point}
\begin{point}%
Let us prove that~$p$ is a projection, c.q.~$p^2=p$. 
Since~$p\leq 1$, we already have $p^2\equiv \sqrt{p}p\sqrt{p}\leq p$
by~\sref{ad-monotone},
and so we only need to show that $p\leq p^2$. We have:
\begin{alignat*}{3}
 p^2 \ &=\  \textstyle \bigvee_m \sqrt{p} \,b^{\nicefrac{1}{2^m}} \,\sqrt{p}
\qquad&&\text{by \sref{ad-normal}} \\
&=\ \textstyle\bigvee_m b^{\nicefrac{1}{2^{m+1}}}\, p\,
b^{\nicefrac{1}{2^{m+1}}} 
\qquad&&\text{by \sref{vna-ceil-point-1} and \sref{cstar-square-commutes}} \\
&=\ \textstyle \bigvee_m \bigvee_n \, 
b^{\nicefrac{1}{2^{m+1}}}\, b^{\nicefrac{1}{2^n}}\,
b^{\nicefrac{1}{2^{m+1}}} \qquad && \text{by \sref{ad-normal}}
\end{alignat*}
Thus $p^2 \geq b^{\nicefrac{1}{2^k}}$
for each~$k$ (taking $n=m=k+1$,)
and so~$p^2 \geq p$.
\end{point}
\begin{point}%
It remains to be shown that~$p$ is the \emph{least} projection
above~$b$.
Let~$q$ be a projection in~$\scrA$ with $b\leq q$;
we must show that~$q\leq p$.
We have $b^{\nicefrac{1}{2}}\leq q$
by~\sref{projection-above-effect},
and so $b^{\nicefrac{1}{2^n}}\leq q$ for each~$n$ by induction.
Hence $p\leq q$.
\end{point}
\end{point}
\end{parsec}

%
%	floor
%
\begin{parsec}[vna-floor]%
\begin{point}{Proposition}%
Below every effect~$b$ of a von Neumann algebra~$\scrA$,
there is greatest projection, $\floor{b}$,
called the \Define{floor} of~$b$,
given by~$\floor{b} = \bigwedge_{n=0}^\infty b^{2^{n}}$.
\begin{point}%
Moreover, if~$a\in \scrA$ commutes with~$b$,
then~$b$ commutes with~$\floor{b}$.
\end{point}
\end{point}
\begin{point}{Proof}%
Let~$p$ denote the infimum of $1\geq b\geq b^2 \geq b^4 \geq  \dotsb \geq 0$.
\begin{point}[vna-floor-point-1]%
If~$a\in \scrA$ commutes with~$b$,
then~$a$ commutes with~$p$.
Indeed, such~$a$ commutes with~$b^2$ (because
$ab^2 = bab = b^2a$,)
and so~$a$ commutes with~$b^{2^n}$ for each~$n$ by induction.
Thus~$a$ commutes with~$p\equiv\bigwedge_n b^{2^n}$ 
(by a variation on~\sref{vna-supremum-commutes}.)
\end{point}
\begin{point}%
To see that~$p$ is a projection, c.q.~$p^2=p$,
we only need to show that~$p\leq p^2$,
because we get $p^2\equiv \sqrt{p}\,p\,\sqrt{p}\leq p$
from $p\leq 1$ (using~\sref{ad-monotone}.)
Now, since
\begin{alignat*}{3}
p^2 \ &=\ \textstyle \bigwedge_m\  \sqrt{p}\, b^{2^m} \sqrt{p}\qquad
&&\text{by a variation on~\sref{ad-normal}}\\
&=\ \textstyle \bigwedge_m \ b^{2^{m-1}} p\, b^{2^{m-1}}\qquad
&&\text{by~\sref{vna-floor-point-1} and~\sref{cstar-square-commutes}}\\
&=\ \textstyle \bigwedge_m \bigwedge_n \ 
b^{2^{m-1}}\, b^{2^n}\, b^{2^{m-1}}\qquad
&&\text{by~\sref{ad-normal},}
\end{alignat*}
and $p\leq b^{2^{m-1}}\, b^{2^n}\,b^{2^{m-1}}$
for all~$n,m$, we get~$p\leq p^2$.
\end{point}
\begin{point}%
It remains to be shown that~$p$ is the greatest projection above~$b$.
Let~$q$ be a projection in~$\scrA$ with~$q\leq b$
we must show that~$q\leq p$.
Since~$q\leq b$,
we have~$q\leq b^2$ (by~\sref{projection-below-effect}),
and so~$q\leq b^{2^n}$ for each~$n$ by induction.
Thus~$q\leq p\equiv\bigwedge_n b^{2^n}$.
\end{point}
\end{point}
\end{parsec}
%
%
%
\begin{parsec}%
\begin{point}{Exercise}%
Let~$a,b$ be effects of a von Neumann algebra~$\scrA$,
and let~$\lambda\in [0,1]$.
\begin{point}%
Show that $\ceil{a}^\perp = \floor{a^\perp}$
and $\floor{a}^\perp = \ceil{a^\perp}$.
\end{point}
\begin{point}[vna-binary-supremum-projections]%
Show that~$\ceil{\lambda a} = \ceil{a}$
when~$\lambda\neq 0$.
Use this to prove that~$\ceil{\lambda a+\lambda^\perp b}$
is the supremum of~$\ceil{a}$ and~$\ceil{b}$
in the poset of projections of~$\scrA$
when~$\lambda\neq 0$ and~$\lambda\neq 1$.
\end{point}
\begin{point}[vna-floor-square]%
Show that $\floor{a}=\floor{a^2}$.
\end{point}
\end{point}
\end{parsec}

%
%	directed supremum of projections
%
\begin{parsec}%
\begin{point}[vna-directed-supremum-projections]{Lemma}%
The supremum of a directed set~$D$ of projections
from a von Neumann algebra~$\scrA$ is a projection.
\end{point}
\begin{point}{Proof}%
Writing $p=\bigvee D$,
we must show that $p^2=p$.
Note that $dp=d$ for all~$d\in D$
(by~\sref{projection-below-effect} because~$d\leq p$.)
Now, on the one hand, $(d)_{d\in D}$
converges ultraweakly to~$p$.
On the other hand,
$(dp)_{d\in D}$
converges ultraweakly to~$p^2$ by~\sref{vna-supremum-mult}.
Hence~$p=p^2$ by uniqueness of ultraweak limits.
\end{point}
\begin{point}{Exerise}%
Deduce from this result
 that every set~$A$ of projections from~$\scrA$
has a supremum $\Define{\bigcup A}$
and an infimum $\Define{\bigcap A}$
\emph{in the poset of projections from~$\scrA$}.\\
(Hint: use~\sref{vna-binary-supremum-projections},
and the fact that $p\mapsto p^\perp$ 
is an order isomorphism on the poset of projections on~$\scrA$.)
\end{point}
\end{parsec}
%
%
%
\begin{parsec}[floor-sequential-product]%
\begin{point}{Lemma}%
Let~$a,b$ be effects of a von Neumann algebra~$\scrA$.
Then~$\floor{\sqrt{a}b\sqrt{a}}$ is the greatest projection
below~$a$ and~$b$, that is, in symbols, 
$\floor{\sqrt{a}b\sqrt{a}}=\floor{a}\cap \floor{b}$.
\end{point}
\begin{point}{Proof}%
Surely, $\floor{\sqrt{a}b\sqrt{a}}\leq \sqrt{a}b\sqrt{a} \leq a$.
Let us prove that~$\floor{\sqrt{a}b\sqrt{a}}\leq b$.
To this end,
recall
that (by~\sref{projection-below-effect})
a projection~$e$ is below an effect~$c$
iff $ec=e$ iff $e\sqrt{c}=e$.
In particular,
since~$\floor{\sqrt{a}b\sqrt{a}}\leq \sqrt{a}b\sqrt{a}$ and 
$\floor{\sqrt{a}b\sqrt{a}}\leq a$,
we get
\begin{equation*}
\floor{\sqrt{a}b\sqrt{a}}
\ =\ \floor{\sqrt{a}b\sqrt{a}}\sqrt{a}b\sqrt{a}\floor{\sqrt{a}b\sqrt{a}} \ =\ 
\floor{\sqrt{a}b\sqrt{a}}b\floor{\sqrt{a}b\sqrt{a}},
\end{equation*}
and so $\floor{\sqrt{a}b\sqrt{a}}b^\perp\floor{\sqrt{a}b\sqrt{a}}=0$,
which implies that
$\floor{\sqrt{a}b\sqrt{a}}\leq b$ by~\sref{ad-contraposed}.
\begin{point}%
Now,
let~$e$ be a projection below~$a$ and~$b$,
that is, $e\sqrt{a}=e$ and~$eb=e$.
We must show that~$e\leq \floor{\sqrt{a}b\sqrt{a}}$,
or equivalently, $e\leq \sqrt{a}b\sqrt{a}$,
or put yet differently, $e\sqrt{a}b\sqrt{a}=e$.
But this is obvious: $e=e\sqrt{a}=eb\sqrt{a}=e\sqrt{a}b\sqrt{a}$.
\end{point}
\end{point}
\end{parsec}

\begin{parsec}%
\begin{point}%
Having seen that~$\floor{\sqrt{a}b\sqrt{a}} = \floor{a}\cap\floor{b}$
in~\sref{floor-sequential-product}
one might wonder whether
there is a similar expression for $\ceil{\sqrt{a}b\sqrt{a}}$.
If~$a$ and~$b$ are projection,
$\ceil{aba}$ turns out to coincide with the \Define{Sasaki product},
$a\cap (a^\perp \cup b)$,
as we will show below.
\TODO{Reference for the Sasaki product.}
\TODO{Thank Kenta}
\end{point}
\begin{point}[floor-difference]{Lemma}%
Let~$p$ be a projection,
and let~$a$ be an effect of a von Neumann algebra
with $a\leq p$.
We have $p-\ceil{a}=\floor{p-a}$.
\end{point}
\begin{point}{Proof}%
We must show that $p-\ceil{a}$ is the greatest projection below $p-a$.
To begin, $p-\ceil{a}\leq p-a$,
because $a\leq \ceil{a}$.
Further, since~$a\leq p$, we have $\ceil{a}\leq p$,
and so~$p-\ceil{a}$ is a projection
(by~\sref{projection-below-projection}).
\begin{point}%
Let~$q$ be a projection below~$p-a$.
We must show that~$q\leq p-\ceil{a}$.
The trick is to note that~$a\leq p-q$.
Since~$p-q$ is a projection (by~\sref{projection-below-projection}
because $q\leq p-a\leq p$),
we have $\ceil{a}\leq p-q$,
and so $q\leq p-\ceil{a}$.
\begin{point}[ceil-sequential-product]{Lemma}%
For projections $p,q$ from a von Neumann algebra,
$\ceil{pqp}=p\cap (p^\perp \cup q)$.
\end{point}
\end{point}
\end{point}
\begin{point}[ceil-sequential-product-1]{Proof}%
Observe that $(\ p\cap (p^\perp \cup q)\ )^\perp 
= p^\perp \cup(p\cap q^\perp)$.
Since~$p^\perp$ and $p\cap q^\perp$ are disjoint,
we have $p^\perp \cup (p\cap q^\perp) = p^\perp + p\cap q^\perp$,
and so $p\cap (p^\perp \cup q) = p-p\cap q^\perp$.
\begin{point}%
By point~\sref{ceil-sequential-product-1}, 
it suffices to show that~$\ceil{pqp}=p- p\cap q^\perp$,
that is, $p-\ceil{pqp}=p\cap q^\perp$.
Since $p-\ceil{pqp} = \floor{p-pqp}$
by~\sref{floor-difference} and $\floor{pq^\perp p}=p\cap q^\perp$
by~\sref{floor-sequential-product} we are done.
\end{point}
\end{point}
\end{parsec}

%
% About maps between von Neumann algebras
%

\begin{parsec}%
\begin{point}{Proposition}%
For a cp-map~$f\colon \scrA\to\scrB$
 between $C^*$-algebras, and $a,b\in\scrA$, 
\begin{equation*}
f(b^*a)\,f(a^*b)\ \leq\ \|f(a^*a)\|\,f(b^*b).
\end{equation*}
\end{point}%
\begin{point}{Proof}%
\TODO{add}
\end{point}
\end{parsec}
%
%
%
\begin{parsec}%
\begin{point}{Exercise}%
Let~$f\colon \scrA\to\scrB$ be a ncpsu-map
between von Neumann algebras
with~$f(1)\leq 1$,
and let~$a\in \scrA$ be positive.
\begin{point}[cp-kadisons-ineq]%
Show that~$f(a)^2 \leq f(a^2)$.
\end{point}
\begin{point}%
Use this, and~\sref{cstar-sqrt-monotone},
to show that~$f(a^{\nicefrac{1}{2}}) \leq f(a)^{\nicefrac{1}{2}}$.
\end{point}
\begin{point}[cpsu-2nthroot]%
Go on, and show that $f(a^{\nicefrac{1}{2^n}})
\leq f(a)^{\nicefrac{1}{2^n}}$ for each~$n$.
\end{point}
\end{point}
\end{parsec}

%
%
%
\begin{parsec}%
\begin{point}{Proposition}%
Let $f\colon \scrA\to\scrB$ be a ncpsu-map
between von Neumann algebras.
Then $\ceil{f(a)}=\ceil{f(\ceil{a})}$
for every effect~$a$ from~$\scrA$.

\TODO{Use Kadison's inequality instead of complete positivity?}
\end{point}
\begin{point}{Proof}%
Since~$a\leq \ceil{a}$
we have $f(a)\leq f(\ceil{a})$,
and so~$\ceil{f(a)}\leq \ceil{f(\ceil{a})}$.
\begin{point}%
It remains to be shown that $\ceil{f(\ceil{a})}\leq \ceil{f(a)}$,
that is, $f(\ceil{a})\leq \ceil{f(a)}$.
Since  $\ceil{a}=\bigvee_n a^{\nicefrac{1}{2^n}}$ 
(by~\sref{vna-ceil})
and~$f$ is normal,
we have $f(\ceil{a})=\bigvee_n f(a^{\nicefrac{1}{2^n}})$.
Now, since $f(a^{\nicefrac{1}{2^n}})\leq f(a)^{\nicefrac{1}{2^n}}
\leq \ceil{f(a)}$
by~\sref{cpsu-2nthroot} for each~$n$,
we have~$f(\ceil{a})\leq \ceil{f(a)}$.
\end{point}
\end{point}
\end{parsec}
%
%
%
\begin{parsec}%
\begin{point}{Proposition}%
Let~$f\colon \scrA\to\scrB$ be a ncpsu-map
between von Neumann algebras.
Then~$\floor{f(a)}=\floor{f(\floor{a})}$
for every effect~$a$ from~$\scrA$.
\end{point}
\begin{point}{Proof}%
Since~$\floor{a}\leq a$,
we have~$\floor{f(\floor{a})}\leq \floor{f(a)}$.
Thus we only need to show that~$\floor{f(a)}\leq \floor{f(\floor{a})}$,
or equivalently, $\floor{f(a)}\leq f(\floor{a})$.
We have
\begin{equation*}
\floor{f(a)}
\ \stackrel{\sref{vna-floor-square}}{=}\ 
\floor{f(a)^2}
\ \stackrel{\sref{cp-kadisons-ineq}}{\leq}\  
\floor{f(a^2)} \ \leq\ \floor{f(a)},
\end{equation*}
and so~$\floor{f(a)}=\floor{f(a^2)}$.
By induction,
and similar reasoning,
we get~$\floor{f(a)}=\floor{f(a^{2^n})}\leq f(a^{2^n})$
for every~$n$,
and so
$\floor{f(a)}\leq \bigwedge_n f(a^{2^n})
= f(\bigwedge_n a^{2^n})=f(\floor{a})$,
where we used that~$f$ is normal,
and~$\floor{a}=\bigwedge_n a^{2^n}$ (see~\sref{vna-floor}).
\end{point}
\end{parsec}

\begin{parsec}%
\begin{point}%
Parsec about commutative $C^*$-algebras:
\end{point}
\begin{point}[ccstar-proj]{Theorem}%
The projections~$\Proj(\scrA)$ of a commutative $C^*$-algebra~$\scrA$
form a Boolean algebra.
If~$\scrA$ is monotone complete,
then~$\Proj(\scrA)$ is a complete Boolean algebra.
\end{point}
\end{parsec}

\begin{parsec}%
\begin{point}%
Let us now turn to quotients of von Neumann algebras.
Recall that given a norm closed two-sided ideal~$\scrD$
of a $C^*$-algebra $\scrA$
we can form the quotient $\scrA/\scrD$,
(which is again a $C^*$-algebra),
and the quotient map $q\colon \scrA\to \scrA/\scrD$
which is a \textsc{miu}-map.
If~$\scrA$ is a von Neumann algebra,
then~$\scrA/\scrD$ might not be a von Neumann algebra,
either because~$\scrA/\scrD$ is not be monotone complete 
(see~\sref{vn-quotient-not-monotone-complete}),
or because~$\scrA/\scrD$ does not have a separating set of normal states
(see~\TODO{}).
However, if~$\scrD$ is ultraweakly closed,
then~$\scrA/\scrD$ is a von Neumann algebra,
as one might have suspected.
To prove this,
we need the somewhat surprising
fact that every ultraweakly closed
ideal~$\scrD$ is of the form $\scrD\equiv c\scrA$,
where~$c$ is a central projection~$c$ of~$\scrA$.
(The quotient~$\scrA/c\scrA$ is then simply~$c^\perp\scrA$.)
\TODO{connection between uw-closed ideals and central projections}
\TODO{universal property for \textsc{n(c)p(s)(u)}-maps}
\end{point}
\begin{point}[vn-quotient-not-monotone-complete]{Example}%
The quotient of a von Neumann algebra
by an closed ideal
need not be a von Neumann algebra:
we will show that $\ell^\infty/c_0$
is not monotone complete,
where~$\ell^\infty$ is the von Neumann algebra
of bounded sequences,
and~$c_0\subseteq \ell^\infty$
is the norm closed ideal of
sequences which converge to~$0$.
By~\sref{ccstar-proj}
it suffices to show that the Boolean algebra~$\Proj(\ell^\infty/c_0)$
of projections of~$\ell^\infty/c_0$ is not complete.
\begin{point}%
We claim that $\Proj(\ell^\infty/c_0)$
is isomorphic to the Boolean algebra
 $\wp(\N)/\wp_{fin}(\N)$
of subsets of~$\N$ modulo 
the filter~$\wp_{fin}(\N)$ of finite subsets.

It is easy to see that the assignment $A\mapsto [\mathbf{1}_A]_{c_0}$
(where $\mathbf{1}_A$ is the indicator function of~$A$)
gives a Boolean algebra homomorphism 
$f\colon \wp(\N)\to \Proj(\ell^\infty/c_0)$,
that the kernel of~$f$
is $\wp_{fin}(\N)$
(because for $A\subseteq \N$
we have $\mathbf{1}_A\in c_0$ iff $A$ is finite).
It remains to be shown that~$f$ is surjective,
which requires some fiddling.

Let~$\alpha\in \ell^\infty$
be
such that $[\alpha]_{c_0}$ is a projection in $\ell^\infty/c_0$.
We claim that
$\mathbf{1}_A - \alpha \in c_0$ (and so $\alpha=f(A)$)
for some~$A\subseteq \N$.
Note that since $[\alpha]_{c_0}$ is self-adjoint,
we have $[\alpha]_{c_0}=(\,[\alpha]_{c_0}\,)_\R = [\,\alpha_\R\,]_{c_0}$,
and so we may assume without loss of generality
that~$\alpha$ is self-adjoint
(replacing $\alpha$ by $\alpha_\R$ is necessary).
Define $A=\{n\in\N\colon \left|1-\alpha(n)\right|<\frac{1}{2}\}$.

Let~$\varepsilon>0$ with $\varepsilon<\nicefrac{1}{2}$ be given.
We must find~$N\in\N$ with 
$\left| \mathbf{1}_A(n)-\alpha(n)\right|\leq \varepsilon$
for all~$n\geq N$.
Since~$[\alpha]_{c_0}$ is a projection,
$\alpha-\alpha^2\in c_0$,
and so
there is~$N\in \N$
with $\left|\alpha(n)\right| \,\left| 1-\alpha(n)\right|
\,\equiv\, \left| \alpha(n) - \alpha(n)^2 \right| \leq 
\nicefrac{1}{2}\,\varepsilon < \nicefrac{1}{4}$
for all~$n\geq N$.

Let~$n\geq N$ be given.
Note that either 
$\left|1-\alpha(n)\right|\geq \nicefrac{1}{2}$
or 
$\left|\alpha(n)\right|\geq \nicefrac{1}{2}$.
If~$\left|1-\alpha(n)\right|\geq\nicefrac{1}{2}$,
then $\mathbf{1}_A(n)=0$
(by definition of~$A$),
and so $\left|\mathbf{1}_A(n)-\alpha(n)\right|\equiv \left|\alpha(n)\right| 
\leq 2 \,\left|\alpha(n)\right|\,\left| 1-\alpha(n)\right|
\leq \varepsilon$.
On the other hand,
if  $\left|\alpha(n)\right| \geq \nicefrac{1}{2}$,
then $\left|1-\alpha(n)\right| < \nicefrac{1}{2}$
(because otherwise $\left|\alpha(n)\right|\left|1-\alpha(n)\right|
\geq \nicefrac{1}{4}$),
so $\mathbf{1}_A(n)=1$,
and thus
$\left|\mathbf{1}_A(n)-\alpha(n)\right|
\equiv \left|1-\alpha(n)\right|
\leq 2 \left|\alpha(n)\right|\left|1-\alpha(n)\right|
\leq \varepsilon$.

Hence $\mathbf{1}_A - \alpha \in c_0$,
and so~$f$ is surjective.
It follows that $\Proj(\ell^\infty/c_0)$
is isomorphic to $\wp(\N)/\wp_{fin}(\N)$.
\end{point}
\begin{point}%
We claim that the Boolean algebra~$\wp(\N)/\wp_{fin}(\N)$
is not complete.
To see this,
find a partition of~$\N$ into infinite subsets
$A_1,A_2,\dotsc$.
We claim that~$A_1,A_2,\dotsc$ has no supremum in~$\wp(\N)/\wp_{fin}(\N)$
because there is no minimal upper bound.
Indeed,
let~$A$ be an upper bound of~$A_1,A_2,\dotsc$ in~$\wp(\N)/\wp_{fin}(\N)$,
that is, $A_n\backslash A$ is finite for every~$n$;
we will define an upper bound~$A'$ of $A_1,A_2,\dotsc$
which is strictly below~$A$ in~$\wp(\N)/\wp_{fin}(\N)$.
Since~$A_n$ is infinite,
and~$A_n\backslash A$ is finite,
we can pick $a_n\in A_n\cap A$ for each~$n$.
Then~$A':=A \backslash \{a_1,a_2,a_3,\dotsc\}$
is an upper bound for~$A_1,A_2,\dotsc$,
because $A_n\backslash A' = (A_n\backslash A)\cup\{a_n\}$ is finite
for each~$n$,
and~$A'$ is strictly below~$A$ in~$\wp(\N)/\wp_{fin}(\N)$,
because $A'\backslash A$ is infinite.
Hence~$\wp(\N)/\wp_{fin}(\N)$
is not complete.
\end{point}
\begin{point}%
\TODO{connection with $\beta\N\backslash\N$
being non-Stonean.}
\end{point}
\end{point}
\begin{point}{Proposition}%
\label{prop:weakly-closed-ideal}
For every ultraweakly closed two-sided
ideal~$\mathscr{D}$ of a von Neumann algebra~$\scrA$
there is a unique central projection~$c\in\scrA$
such that $\mathscr{D} = c\scrA$.

Moreover, $c$ is the greatest projection in~$\mathscr{D}$.
\begin{point}{Proof}%
\emph{(if $a\in \mathscr{D}\cap [0,1]_{\scrA}$, 
then $\ceil{a}\in\mathscr{D}$)}\ 
Let~$a\in\mathscr{D}$ with $0\leq a \leq 1$
be given.
Recall that $\ceil{a}$
is the ultraweak limit of $a,\ a^{\frac{1}{2}},\ a^{\frac{1}{4}},\ \dotsc$.
Since~$\mathscr{D}$ is a $C^*$-subalgebra of~$\scrA$,
we have $a^{\frac{1}{2}}\in \mathscr{D}$.
Then $(a^{\frac{1}{2}})^{\frac{1}{2}} \equiv a^{\frac{1}{4}}\in \mathscr{D}$
by the same token,
and so on.
Thus, as~$\mathscr{D}$ is ultraweakly closed, $\ceil{a}\in\mathscr{D}$.

\emph{($\mathscr{D}\cap [0,1]_\scrA$ is directed)}\ 
Let~$a,b\in \mathscr{D}\cap[0,1]_\scrA$ be given.
We're looking for an element~$u\in \mathscr{D}\cap[0,1]_\scrA$
with $a\leq u$ and $b\leq u$.
Note that $\frac{1}{2}a + \frac{1}{2}b$
is in~$\mathscr{D}\cap[0,1]_{\scrA}$.
Define
$u:=\ceil{\frac{1}{2}a+\frac{1}{2}b}$.
Then  $u\in \mathscr{D}\cap[0,1]_\scrA$
by the previous point.
Now, 
as $\frac{1}{2}a \leq \frac{1}{2}a+\frac{1}{2}b \leq u$,
we have $a \leq \ceil{a} =\ceil{\frac{1}{2}a} \leq 
\ceil{ \frac{1}{2}a+\frac{1}{2}b} = u$.
Similarly, $b\leq u$.


\emph{($\mathscr{D}\cap [0,1]_\scrA$
has a greatest element, $c$)}\ 
As~$\mathscr{D}\cap[0,1]_\scrA$
is a directed subset of self-adjoint elements 
of~$\scrA$ bounded above by~$1$,
it has a supremum in~$\scrA_\mathrm{sa}$,
say~$c$.
As~$c$ is the ultraweak limit of~$(c)_{c\in \mathscr{D}\cap[0,1]_\scrA}$
and~$\mathscr{D}\cap[0,1]_\scrA$ is ultraweakly closed,
we have~$c\in \mathscr{D}\cap[0,1]_\scrA$.
Thus~$c$ is the greatest element of~$\mathscr{D}\cap[0,1]_\scrA$.

\emph{($c$ is a projection)}\ 
It suffices to show that~$c=\ceil{c}$ since~$\ceil{c}$ is a projection.
Since~$c\leq \ceil{c}$, it suffices to show that~$\ceil{c}\leq c$.
As before,~$c\in \mathscr{D}\cap[0,1]_\scrA$
entails~$\ceil{c}\in\mathscr{D}\cap[0,1]_\scrA$.
Thus~$\ceil{c}\leq c$, since~$c$ is the
greatest element of~$\mathscr{D}\cap[0,1]_\scrA$.
Hence~$c=\ceil{c}$ is a projection.

\emph{(for all $a\in \scrA$
we have: $a\in \mathscr{D}$ iff $ca=a$)}\ 
Since~$c\in \mathscr{D}$,
and~$\mathscr{D}$ is a two-sided ideal,
$ca\in \mathscr{D}$, and so $ca=a$ entails~$a\in\mathscr{D}$.
Thus we only need to show that~$a\in\mathscr{D}$
entails~$ca=a$.
Let~$a\in\mathscr{D}$
be given.
We must show that~$ca=a$.

If~$a\in [0,1]_\scrA$,
then this is clear:
$a\leq c$, since~$c$ is the greatest element
of~$\mathscr{D}\cap [0,1]_\scrA$,
and so~$ca=ac=a$ by Lemma~\ref{lem:projection-order}.

If~$a\in \scrA_{\mathrm{sa}}$
with~$a\neq 0$,
then $a\cdot\|a\|^{-1} \in [0,1]_\scrA\cap \mathscr{D}$,
and so $ca \|a\|^{-1} = a \|a\|^{-1}$,
which entails $ca = a$.

(For arbitrary $a\in \scrA$,)
we have $a= a_{\mathbb{R}} + i a_{\mathbb{I}}$,
where $a_\mathbb{R}=\frac{1}{2}(a+a^*)$
and $a_\mathbb{I} = \frac{1}{2i}(a-a^*)$
are self-adjoint elements of~$\mathscr{D}$.
Then~$ca_\mathbb{R}=a_\mathbb{R}$
and $ca_\mathbb{I}=a_\mathbb{I}$
by the previous paragraph.
Thus~$ca=ca_\mathbb{R}+ica_\mathbb{I}=a_\mathbb{R}+ia_\mathbb{I}=a$.

\emph{($c$ is central)}\ 
Let~$a\in\scrA$ be given.
We must show that~$ca=ac$.
Since~$\mathscr{D}$ is a two-sided ideal,
we have~$ac\in \mathscr{D}$,
and so~$c(ac)=ac$ by the previous point.
By a similar reasoning we get~$(ca)c=ca$.
Thus~$ac=ca$.
Thus~$c$ is central.

\emph{($\mathscr{D} = c\scrA$)}\ 
Clearly $c\scrA\subseteq\mathscr{D}$
since~$c\in\mathscr{D}$ and~$\mathscr{D}$ is an ideal.
On the other hand,
if~$a\in\mathscr{D}$,
then~$ca=a$
(as we saw),
and thus~$a\in c\scrA$.
Hence~$\mathscr{D} = c\scrA$.

\emph{(uniqueness)}\ 
Let~$c_1$ and~$c_2$ be central projections 
with~$c_1\scrA = \mathscr{D}=c_2\scrA$.
We must show that~$c_1=c_2$.

As~$c_2\in\mathscr{D}=c_1 \scrA$,
there is~$a_1\in \scrA$
with~$c_2 = c_1a_1$.
Then~$c_2 = c_2c_2^* = c_1a_1a_1^*c_1^*\leq 
c_1c_1^*\|a_1a_1^*\|=c_1\|a_1\|^2$.
Thus~$c_1c_2=c_2c_1=c_2$ by Lemma~\ref{lem:projection-order}.

Similarly, $c_2c_1 = c_1c_2 = c_1$, and so $c_1=c_1c_2=c_2$.

\emph{($c$ is the greatest projection of~$\mathscr{D}$)}\ 
Let~$p$ be a projection in~$\mathscr{D}$.
We must show that~$p\leq c$.
Since~$p$ is a projection,
we have $0\leq p\leq 1$, and so $p\in \mathscr{D}\cap [0,1]_\scrA$.
Thus, $p\leq c$, by definition of~$c$.
\end{point}
\end{point}
\end{parsec}
\end{document}

