\documentclass[b5paper]{book}
\usepackage{cite}
\usepackage{xparse}
\usepackage{xcolor}
\usepackage{graphicx}
\usepackage{marginnote}
\usepackage{ifthen}
\usepackage{calc}
\usepackage{amssymb}
\usepackage{amsmath}
\usepackage{mathrsfs}
\usepackage{nicefrac}
\usepackage{sectsty}
\usepackage[all]{xypic}
\usepackage{hyperref} % <- should be last because it redefined \ref, etc..

\newcommand{\IGNORE}[1]{}

% macros concerning textstyle
\colorlet{darkblue}{blue!75!black}
\colorlet{lightblue}{blue!50!white}
\colorlet{lightgray}{gray!50!white}
\colorlet{darkgray}{gray!75!black}
\colorlet{darkred}{red!75!black}

\allsectionsfont{\sffamily\color{darkblue}}

\newcommand{\textPointHeaderI}[1]{\textcolor{darkblue}{\textbf{\textsf{#1}}}}
\newcommand{\textPointHeaderII}[1]{\textcolor{darkblue}{\textsf{#1}}}
\newcommand{\textPointHeaderIII}[1]{\textcolor{lightgray}{\textsf{#1}}}
\newcommand{\textParsecNumber}[1]{\textcolor{darkblue}{\textbf{\textsf{#1}}}}
\newcommand{\textPointNumberI}[1]{\textcolor{darkblue}{\textsf{#1}}}
\newcommand{\textPointNumberII}[1]{\textcolor{lightblue}{\textsf{#1}}}
\newcommand{\textPointNumberIII}[1]{\textcolor{lightgray}{\textsf{#1}}}
\newcommand{\textDefine}[1]{\textcolor{darkblue}{#1}}
\newcommand{\textSref}[1]{\textsf{#1}}
\newcommand{\textTodo}[1]{\textcolor{darkred}{#1}}

\newcommand{\TODO}[1]{\textTodo{\textbf{TODO}:\ #1}}
\newcommand{\Define}[1]{\textDefine{#1}}
\newcommand{\grayed}[1]{\textcolor{darkgray}{#1}}

\newcommand{\Cat}[1]{\mathbf{#1}}
\newcommand{\cW}[1]{\Cat{cW}^*_{\text{\textsc{#1}}}}
\newcommand{\W}[1]{\Cat{W}^*_{\text{\textsc{#1}}}}
\newcommand{\cCstar}[1]{\Cat{cC}^*_{\text{\textsc{#1}}}}
\newcommand{\Cstar}[1]{\Cat{C}^*_{\text{\textsc{#1}}}}
\newcommand{\CH}{\Cat{CH}}
\newcommand{\op}[1]{{#1}{^{\mathsf{op}}}}

\newcommand{\ketbra}[2]{\left|#1\right>\!\left<#2\right|}
\newcommand{\uleq}{\mathbin{\rotatebox[origin=c]{90}{$\leq$}}}


% Common symbols
\newcommand{\C}{\mathbb{C}}
\newcommand{\N}{\mathbb{N}}
\newcommand{\Z}{\mathbb{Z}}
\newcommand{\R}{\mathbb{R}}
\newcommand{\I}{\mathbb{I}}
\newcommand{\spec}{\mathrm{sp}}
\newcommand{\scrA}{\mathscr{A}}
\newcommand{\scrB}{\mathscr{B}}
\newcommand{\scrC}{\mathscr{C}}
\newcommand{\scrD}{\mathscr{D}}
\newcommand{\scrH}{\mathscr{H}}
\newcommand{\scrK}{\mathscr{K}}
\newcommand{\scrS}{\mathscr{S}}
\newcommand{\scrX}{\mathscr{X}}
\newcommand{\scrY}{\mathscr{Y}}
\newcommand{\scrZ}{\mathscr{Z}}

\newcommand{\ceil}[1]{\left\lceil#1\right\rceil}
\newcommand{\floor}[1]{\left\lfloor#1\right\rfloor}

\newcommand{\Real}[1]{#1_{\R}}
\newcommand{\Imag}[1]{#1_{\I}}
\newcommand{\sa}[1]{#1_{\R}}
\newcommand{\pos}[1]{#1_{+}}

\DeclareMathOperator{\dom}{dom}
\DeclareMathOperator{\length}{length}
\DeclareMathOperator{\interior}{in}
\DeclareMathOperator{\Stat}{Stat}
\DeclareMathOperator{\NStat}{NStat}
\DeclareMathOperator{\Cont}{C}
\DeclareMathOperator{\Colim}{Colim}
\DeclareMathOperator{\id}{id}
\DeclareMathOperator{\hZ}{hZ}
\DeclareMathOperator{\Proj}{Proj}


% typesetting theorems
\newcommand{\qed}{\hfill\textcolor{darkblue}{\ensuremath{\square}}}


\newcounter{tmp} % used for arithmetic

% The content of this thesis is grouped into numbered paragraphs,
% which are called "parsecs" (for paragraph--section),
% and these parsecs contain several points.
\newcounter{parsec} % keeps track of the current parsec number

% The first argument is the label this parsec will have--use \sref 
% 	to refer to a parsec.
\NewDocumentEnvironment{parsec}{o}{%
	\par\penalty-200\vskip1em\noindent%
	\refstepcounter{parsec}%
	%%\setcounter{point}{0}% - point is set to 0 by \numberwithin
	% In the footer of every odd page we list the parsecs present on 
	% the spread.  We pass this information to the footer via the 
	% \markboth,\leftmark,\rightmark-mechanism, which is normally
	% used to display the section and subsection names and numbers
	% in the header.
	% 	Recall that \leftmark will return the LAST value passed
	% to the first argument of \markboth on this page. (The difficulty
	% of implementing \leftmark is that a \markboth that will belong
	% to the next page can be called before the current page is shipped,
	% because this \markboth may be part of the text that overflows the
	% current page.)
	%	\rightmark will return the FIRST value passed to the second
	% argument of \markboth on this page.
	%	Since we would not only like to know if this parsec with number
	% say  N  is present on this spread, but also whether it spills over 
	% to the next spread (or has spilled over from the previous spread),
	% we keep track of whether the parsec started on this spread, 
	% encoded by  2N,  or whether the parsec ended on this spread,
	% encoded by  2N+1.
	\setcounter{tmp}{2*\value{parsec}}%
	\markboth{\the\value{tmp}}{\the\value{tmp}}%
	\label{#1}%
	% Display the parsec number in the margin.
	\marginnote{\makebox[3em][c]{\textParsecNumber{\the\value{parsec}}}}%
}{%
	\setcounter{tmp}{2*\value{parsec}+1}%
	\markboth{\the\value{tmp}}{\the\value{tmp}}%
}

\newcounter{point} % keeps track of the current point
\numberwithin{point}{parsec}
\renewcommand{\thepoint}{\theparsec\,\Roman{point}}
\newcounter{pointdepth}  % keeps track of the depth of the current point
			 % --- points may be nested.

\NewDocumentEnvironment{point}{o g}{%
	\setcounter{pointdepth}{\value{pointdepth}+1}%
	\refstepcounter{point}%
	\IfValueT{#1}{\label{#1}}%
	\ifthenelse{\equal{\value{point}}{1}}{}{%
		\ifthenelse{\equal{\value{pointdepth}}{1}}{%
			\par\penalty-100\vskip.5em\noindent%
		}{%
			\ifthenelse{\equal{\value{pointdepth}}{2}}{%
				\par\penalty-50\vskip.3em\noindent%
			}{%
				\par\penalty-25\vskip.2em\noindent%
			}%
		}%
		\marginnote{\makebox[2em][c]{%
			\ifthenelse{\equal{\value{pointdepth}}{1}}{%
				\textPointNumberI{\Roman{point}}%
			}{%
				\ifthenelse{\equal{\value{pointdepth}}{2}}{%
					\textPointNumberII{\Roman{point}}%
				}{%
					\textPointNumberIII{\Roman{point}}%
				}%
		}}}%
	}%
	\IfValueT{#2}{%
		\ifthenelse{\equal{\value{pointdepth}}{1}}{%
			\textPointHeaderI{#2}%
		}{%
			\ifthenelse{\equal{\value{pointdepth}}{2}}{%
				\textPointHeaderII{#2}%
			}{%
				\textPointHeaderIII{(#2)}%
			}}%
	\ \ }%	
}{%
\setcounter{pointdepth}{\value{pointdepth}-1}%
}

% Refer to a parsec.
\NewDocumentCommand{\sref}{m}{\textSref{\ref{#1}}}

% Adjust footer and header:
\usepackage{fancyhdr}
\pagestyle{fancy}
\renewcommand{\headrulewidth}{0pt} % we want no header line

% Since we use \markboth,\leftmark,\rightmark to keep track of the parsecs
% on a given spread, we should neutralize its old user:
\renewcommand{\chaptermark}[1]{}  
\renewcommand{\sectionmark}[1]{}

\fancyhead{}

% These counters are used for computation
\newcounter{firstParsec}
\newcounter{lastParsec}
\newcounter{firstParsecF}
\newcounter{lastParsecF}

% parsecToBeContinued is 1 if the previous spread spilled a parsec,
% and 0 otherwise.
\newcounter{parsecToBeContinued}  
\setcounter{parsecToBeContinued}{0}

% Set the footer.  It contains the parsecs on this page.
% We use "\rightmark+0", because \rightmark might be empty.
\fancyfoot[CE]{\setcounter{firstParsecF}{\rightmark+0}}
\fancyfoot[CO]{%
% firstParsecF is already set by the even page that came before
\setcounter{firstParsec}{\value{firstParsecF}/2}%
\setcounter{lastParsecF}{\leftmark+0}%
\ifthenelse{\equal{\value{lastParsecF}}{0}}%
{}% do nothing
{%
\setcounter{lastParsec}{\value{lastParsecF}/2}%
\textPointNumberI{%
\ifthenelse{\equal{\value{parsecToBeContinued}}{1}}{..}{}%
\the\value{firstParsec}%
\ifthenelse{\equal{\value{firstParsec}}{\value{lastParsec}}}%
	{}{%  no footer without parsecs
\setcounter{tmp}{\value{firstParsec}+1}%
\ifthenelse{\equal{\value{tmp}}{\value{lastParsec}}}{, }{--}%
\the\value{lastParsec}}%
\setcounter{tmp}{\value{lastParsec}*2}%
\ifthenelse{\equal{\value{tmp}}{\value{lastParsecF}}}%
{..\setcounter{parsecToBeContinued}{1}}%
{\setcounter{parsecToBeContinued}{0}}%
}% \textPointNumberI{
}% \ifthenelse{
}% \fancyfoot[CO]{


\usepackage{subfiles}
%\usepackage{layouts} %for printing the dimensions
\externaldocument{a}
\externaldocument{b}

\makeindex

\begin{document}
%\title{%
%\sffamily\color{darkblue}%
%The Category of Von Neumann Algebras}
%\author{Abraham A.~Westerbaan}
%\maketitle

\newcommand{\titelpagina}[8]{%
\vspace*{4em}
\begin{center}
{\huge\sffamily\color{darkblue}The Category of Von Neumann Algebras}
\end{center}
\vspace{2em}
\begin{center}
{\par\noindent\sffamily\color{darkblue}\large\textbf{#1}}
\end{center}
\begin{center}
#2
\end{center}
\vspace{1em}
\begin{center}
#3
\end{center}
\begin{center}
#4
\end{center}
\begin{center}
#5
\end{center}
\begin{center}
#6
\end{center}
\vspace{10em}
\begin{center}%
#7
\end{center}%
\begin{center}%
\large\color{darkblue}Abraham Anton \textsc{Westerbaan}
\end{center}%
\begin{center}%
#8
\end{center}
\newpage
}

\newcommand{\achterkanttitelpagina}[3]{
\vspace*{8em}
\begin{center}
\textbf{\large\sffamily\color{darkblue}#1:}
\end{center}
\begin{center}
Prof.~dr.~B.P.F.~\textsc{Jacobs}
\end{center}
\vspace{5em}
\begin{center}
\textbf{\large\sffamily\color{darkblue}#2:}
\end{center}
\begin{center}
Prof.~dr.~J.D.M.~\textsc{Maassen}\\
\vspace{1em}
    Prof.~dr.~P.~\textsc{Panangaden} \\ {\footnotesize(McGill University, Canada)} \\
\vspace{1em}
    Prof.~dr.~P.~\textsc{Selinger} \\ {\footnotesize(Dalhousie University, Canada)} \\
\vspace{1em}
    Dr.~C.J.M.~\textsc{Heunen} \\ {\footnotesize(University of Edinburgh, #3)} \\
\vspace{1em}
Dr.~A.R.~\textsc{Kissinger}
\end{center}
\newpage
}

\newcommand{\whiteout}[1]{{\color{white}#1}}

\titelpagina{\whiteout{P}$\,$}{$\,$\whiteout{ter}\\
\whiteout{aan}$\,$\\
\whiteout{op}$\,$\\
\whiteout{volgens}$\,$\\
\whiteout{in}$\,$}{\whiteout{op}$\,$}{\whiteout{dinsdag}$\,$}{\whiteout{om}$\,$}{\whiteout{10.30}$\,$}{\whiteout{door}}{$\,$\whiteout{geboren}$\,$\\
\whiteout{te}$\,$}

\vspace*{1em}
\newpage

\titelpagina{Proefschrift}{ter verkrijging van de graad van doctor\\
aan de Radboud Universiteit Nijmegen\\
op gezag van de rector magnificus prof. dr. J.H.J.M. \textsc{van Krieken},\\
volgens besluit van het college van decanen \\
in het openbaar te verdedigen}{op}{\textbf{dinsdag 14 mei 2019}}{om}{\textbf{10.30 uur precies}}{door}{geboren op 30 augustus 1988\\
te Nijmegen}

\achterkanttitelpagina{Promotor}{Manuscriptcommissie}{Verenigd Koninkrijk}

\titelpagina{Doctoral Thesis}{to obtain the degree of doctor \\
from  Radboud University Nijmegen \\
on the authority of the Rector Magnificus prof. dr. J.H.J.M. \textsc{van Krieken},\\
according to the decision of the Council of Deans \\
to be defended in public}{on}{\textbf{Tuesday, May 14, 2019}}{at}{\textbf{10.30 hours}}{by}{born on August 30, 1988\\
in Nijmegen (the Netherlands)}

\achterkanttitelpagina{Supervisor}{Doctoral Thesis Committee}{United Kingdom}

\makeatletter\@starttoc{parsectoc}\makeatother
\chapter{Introduction}

% Code to print the dimensions of the paper:
%\newpage
%\currentpage
%\pagedesign
%\currentpage
%\pagedesign

\begin{parsec}{10}
\begin{point}{10}
What does this Ph.D.~thesis offer?
Proof, perhaps,
to the doctoral thesis committee
of passable academic work;
an advertisement, as it may be,
of my school's perspective
to colleagues;
a display, even,
of intellectual achievement
to friends and family.
But I believe such narrow and selfish goals \emph{alone}
barely serve to keep a writer's spirits 
energised---and are definitely detrimental to that of the readers.
That is why I have foolhardily
challenged
myself
not just 
to drily list contributions,
but to write this thesis 
as the introduction,
that I would have liked to read
when I started
research for this thesis
back in May~2014.

The topic is von Neumann algebras,
the category they form,
and how they may be used
to model aspects of quantum computation.
Let us just say for now that a von Neumann algebra
is a special type of complex vector
space endowed with
a multiplication operation among some other additional structure.
An important example is the complex vector space~$M_2$
of~$2\times 2$ complex matrices,
because it models (the predicates on) a qubit;
but all~$N\times N$-complex matrices form a von Neumann algebra~$M_N$ as well.
Using von Neumann algebras
(and their little cousins, $C^*$-algebras) 
to describe quantum data types 
seems to be quite a recent idea
(see e.g.~\cite{jacobs2013block,rennela2015operator,furber2013kleisli}, 
	and~\cite{cho2016semantics} for an overview)
and has two distinct features.
Firstly, classical data types
are neatly incorporated:
$\C^2 \equiv \C\oplus \C$
models a bit,
and the direct sum $M_2\oplus M_3$
models the union type of a qubit and a qutrit.
Secondly,
von Neumann algebras
allow for infinite data types as well
	such as~$\scrB(\ell^2(\Z))$,
which represents a ``quantum integer.''\footnote{Though
	other methods of modelling infinite dimensional
	quantum computing have been proposed as well
	e.g.~using non-standard analysis\cite{Gogioso2017},
	pre-sheaves\cite{malherbe2013categorical},
	the geometry of interaction\cite{hasuo2017semantics},
	and quantitive semantics\cite{pagani2014applying}.}
It should be said that this last feature
is both a boon and a bane:
it brings with it all the inherent
intricacies of dealing with infinite dimensions;
and it is no wonder that
most authors choose 
to restrict themselves
to finite dimensions,
especially since
this seems to be enough to describe quantum algorithms,
see e.g.~\cite{nielsen2002quantum}.
\end{point}
\begin{point}{20}%
In this thesis, however,
we do face infinite dimensions,
because the two main results demand it:
\begin{enumerate}
\item
For the first result,
that von Neumann algebras
form  a model of Selinger and Valiron's quantum lambda calculus,
		as Cho and I explained in~\cite{model}
and for which I'll provide the foundation here,
we need to interpret function types,
some of which are essentially infinite dimensional.
\item
The second result,
an axiomatisation
of the map $a\mapsto \sqrt{p}a\sqrt{p}\colon \scrA\to\scrA$
representing measurement
of an element $p\in[0,1]_\scrA$
of a von Neumann algebra~$\scrA$
was tailored by B.E.~Westerbaan (my twin brother) and myself to work for
both finite and infinite dimensional~$\scrA$.
\end{enumerate}
These results
are part of a line of research that
tries to find patterns
in the category of von Neumann algebras,
that may also be cut from other categories
modelling computation---ideally in order to arrive at categorical axioms
for (probabilistic) computation in general.
When I joined the fray
the notion of \emph{effectus}\cite{newdirections} had already
been established by Jacobs,
and the two results
above offer potential additional axioms.
The work in this area has largely been a collaborative effort,
primarily between Jacobs, Cho, my twin brother, and myself,
and many of their insights have ended up in this thesis.

Of this I'd say no more
than that my work appears conversely, and proportionally,
in their writings too, except that the close cooperation
with my brother begs further explanation.
Our efforts on certain topics have been like interleaving 
of the pages of two phone books:
separating them  would be nigh impossible,
especially the work on the axiomatisation
of~$a\mapsto \sqrt{p}a\sqrt{p}$ and Paschke dilations.
So that's why we decided to write our theses
as two volumes of the same work;
preliminaries on von Neumann algebras,
and the axiomatisation of $a\mapsto \sqrt{p}a\sqrt{p}$
appear in this thesis,
while the work on dilations,
and effectus theory appear in my brother's thesis,
\cite{bas}.
\end{point}
\begin{point}{30}
The two results mentioned above only
make up 
about a third of this thesis;
the rest of it is devoted to 
the introduction to the theory of von Neumann algebras
needed to understand these results.
My aim is that anyone 
with, say, a bachelor's degree in mathematics
(more specifically, basic knowledge of linear algebra,
analysis\cite{rudin1964principles}, 
topology\cite{willard}
and set theory\cite{devlin2012joy})
should at least be able to follow the lines of reasoning
with only minimal recourse to external sources.
But I hope that they will gain some deeper understanding
of the material as well.
To this end, and because I wanted to gain some of this insight
for myself too,
I've not just mixed
and matched
results from the literature,
but I tailored a thorough treatise
of everything that's needed,
including proofs.
Whenever possible,
I've taken shortcuts
(e.g.~avoiding for example
the theory of Banach algebras
and locally convex spaces entirely)
to  prevent the mental tax
the added concepts
(and pages) would have brought.
For the same reasons
I've refrained from putting
everything in its proper abstract (and categorical\cite{maclane}) context
trusting that it'll shine through of its own accord.
I've however not been able to restrain
myself in making perhaps frivolous variations on the existing
theory whenever not strictly necessary,
taking for example Kadison's characterisation\cite{kadison1956}
of von Neumann algebras
as my definition,
and developing the elementary theory for it;
in my defence I'll just say this adds to
the original element that is expected of a thesis.
\end{point}
\begin{point}{40}{Advertisements}%
Due to space--time constraints
this thesis is based only on a selection
\cite{model,cho2015quotient,cho2016duplicable,qpakm,westerbaan2016universal}
of the works
I produced under supervision of Jacobs,
and while~\cite{wwpaschke,effintro,statesofconvexsets}
are incorporated in my brother's thesis,
this means~\cite{jacobs2015effect,jacobs2017distances} 
are unfortunately left out.
If you like this thesis,
then
you might also want to take a look
at these~\cite{rennela2017infinite,
rennela2015complete,
furber2013kleisli,
kornell2012,
heunen2015domains,
Maassen2010} recent works on von Neumann algebras,
and $C^*$-algebras.
If you're curious
about effectus theory
and related matters,
please have a look at~\cite{jacobs2017quantum,
cho2017disintegration,
jacobs2016hyper,
jacobs2017channel,
jacobs2017formal,
cho2017efprob,
jacobs2017probability,
jacobs2017recipe,
jacobs2016effectuses,
jacobs2016affine,
jacobs2016relating,
effintro,
statesofconvexsets,
cho2015quotient,
jacobs2017distances,
jacobs2015effect,
jacobs2016expectation,
jacobs2016predicate,
newdirections}.
But if you'd like more pictures instead,
I'd suggest~\cite{coecke2017picturing}.
\end{point}
\begin{point}{50}[on-writing-style]{Writing style}
I've replaced page numbers by
paragraph numbers
such as~\sref{on-writing-style}
for this paragraph.
The numbers after~\sref{final-bram} refer to paragraphs
	in my twin brother's thesis\cite{bas}.
\Define{Definitions}\index{definitions} are set like that
(i.e.~in blue),
and can be found in the index.
Proofs of certain facts
that are easily obtained on the back of an envelope,
and would clutter this manuscript,
have been left out.
Instead these facts have been phrased as exercises
as a challenge to the reader.
\end{point}
\begin{point}{51}
    The symbol ``\Define{$:=$}''\index{(((defequal@$:=$, is defined to be}
should be interpretted as ``is defined to be'',
    while ``\Define{$\equiv$}''\index{(((equiv@$\equiv$, being of the form}
should be read as ``being of the form''.
Sometimes ``$\equiv$'' is
used to define something on its right-hand side,
as in ``let $A\equiv\left(
\begin{smallmatrix} a&b\\b^* &c\end{smallmatrix}
\right)$ be a self-adjoint matrix.''
Other times~``$\equiv$''
indicates a simple rewrite step, as in
``since~$a=2$, we have $a+2=2+2\equiv 4$,''
where it's not suggested $a=2$ implies $2+2=4$.
\end{point}
\begin{point}{60}{Acknowledgements}
The work in this thesis specifically
has benefited greatly from
discussions
with John van de Wetering,
Robert Furber,
Kenta Cho,
and Bas Westerbaan,
but I've also had the pleasure
of discussing a variety
of other topics 
with 
Aleks Kissinger,
Andrew Polonsky,
Bert Lindenhovius,
Frank Roumen, 
Hans Maassen,
Henk Barendregt,
Joshua Moerman,
Martti Karvonen,
Robbert Krebbers,
Robin Adams, 
Robin Kaarsgaard,
Sam Staton, 
Sander Uijlen,
Sebastiaan Joosten,
and many others.
I'm especially honoured to have been received
in Edinburgh by Chris Heunen 
and in Oberwolfach
by Jianchao Wu.
I'm very grateful to
Arnoud van Rooij,
Bas Westerbaan
and
John van de Wetering
for proofreading large parts of
this manuscript,
without whose efforts
even more shameful errors would have remained.
I'm very grateful too for the manuscript committee's members' 
various suggestions and comments, and hope the improvements I made to this text
do them justice.
I should of course not forget to mention
the contribution
of friends (both close and distant),
family,
and colleagues---too numerous to name---of keeping me sane
these past years.

This is the second dissertation topic
I've worked on;
my first attempt
under different supervision
was unfortunately cut short after $1\sfrac{1}{2}$ years.
When Bart Jacobs graciously offered
me a second chance,
I initially had my reservations,
but accepted on account of the challenging topic.
Little did I know 
that behind the ambition and suit
	one finds a man
of singular moral fibre,
embodying
what was said
	about von Neumann himself:
	``[he] had to understand and accept much that most 
of us do not want to accept and do not even wish to understand.''%
\footnote{An excerpt from Eugene P.~Wigner's writings,
see page~130 of~\cite{wigner2013collected}.}
\end{point}
\begin{point}{70}{Funding} was received from the 
European Research Council under grant agreement \textnumero~320571.
\end{point}
\end{parsec}

%\subfile{apropos}
\subfile{cstar}
\subfile{vn}
\subfile{proc}
\begin{parsec}{1330}
\begin{point}{10}[conclusion]{Conclusion}
Here ends this thesis,
but not the entire story.
There's much more to be said
about self-dual Hilbert $\scrA$-modules,
about dilations and their relation to purity,
and about the abstract theory of corners, filters,
and $\diamond$-positivity.
You'll see all this,
and more,
in the sequel,
``Dagger and dilations in the category of von Neumann algebras''\cite{bas},
brought to you by my twin brother.
\end{point}
\end{parsec}
\begin{parsec}{1340}[final-bram]
\begin{point}{10}
(Paragraphs numbered~\sref{final-bram}
and up can be found in~\cite{bas}.)
\end{point}
\end{parsec}

\backmatter

\fancyfoot[CE]{}
\fancyfoot[CO]{}
\fancypagestyle{plain}{
    \fancyfoot[CE]{}
    \fancyfoot[CO]{}
}

\printindex

\begingroup
\renewcommand\chapter[2]{\backmattertitle{Bibliography}}
\bibliography{main}{}
\endgroup

\bibliographystyle{plain}

\end{document}
