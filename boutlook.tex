
\chapter{Outlook}
\begin{parsec}{2290}%
\begin{point}{10}%
For my closing remarks, I would like to speculate
    on possible applications
    and suggest possible
        directions for future research.

We started this thesis by asking the question
    whether the incredibly useful Stinespring dilation theorem
    extends to any ncp-map between von Neumann algebras.
Now that we have seen it does, the next question is obvious:
    do the applications that make the Stinespring dilation so useful
    also extend to the Paschke dilation?
I want to draw attention to one application in particular:
    proving security bounds on quantum protocols.
Here one uses a continuous version
    of the Stinespring dilation theorem \cite{werner2}.
Is there also such a continuous version for Paschke dilation?
\begin{point}{20}%
Next, we have seen that self-dual Hilbert~C$^*$-modules
    over von Neumann algebras
    are very well-behaved compared
    to arbitrary Hilbert C$^*$-modules.
    Frankly, I'm surprised they haven't been studied
        more extensively before:
        the way results about Hilbert spaces
        generalize elegantly to self-dual Hilbert~C$^*$-modules
        over von Neumann algebras seems hard to ignore.
Only one immediate open question remains here:
    does the normal form of~\sref{selfdual-normalish-form}
    extend in some way for to arbitrary von Neumann algebras?
\end{point}
\begin{point}{30}%
As announced, in the second part of this thesis, we did not reach
    our goal of axiomatizing the category of von Neumann algebras
    categorically.
In our attempt, we did find several new concepts
    such as~$\diamond$-adjointness, $\diamond$-positivity,
        purity defined with quotient and comprehension and
        the~$\dagger$ on these pure maps.
Building on this work,
    van de Wetering recently announced~\cite{wetering,weteringeffthe}
    a reconstruction of finite-dimensional quantum theory.
    To discuss it, we need a definition: call an effectus \emph{operational}
    \index{effectus!operational}
        if its scalars are isomorphic to the real interval;
        both the states and the predicates are order separating;
        all predicate effect modules are embeddable in
        finite-dimensional order unit spaces
        \emph{and} each state space is a closed subset
        of the base norm space of all
        unital positive functionals on the corresponding
        order unit space of predicates.
It follows from \cite{wetering}
    that the state space of any operational~$\&$-effectus
    is a spectral convex set in the sense of Alfsen and Shultz
    and that any operational~$\dagger$-effectus
    is equivalent to a subcategory of the category of Euclidean Jordan Algebras
    with positive maps between them in the opposite direction.
Are there infinite-dimensional generalizations of these results?
Are the state spaces of a real~$\&$-effectuses perhaps
    spectral convex sets?
Is any real~$\dagger$-effectus equivalent to a subcategory of, say,
    the category of JBW-algebras with positive maps between them
        in the opposite direction?
\end{point}
\end{point}
\end{parsec}


% vim: ft=tex.latex
