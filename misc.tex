\documentclass[a]{subfiles}
\begin{document}
\chapter{Miscellaneous}
\begin{parsec}%
\begin{point}{Theorem}%
For an ncp-map $\varphi\colon \scrA\to\scrB$ between von Neumann algebras
the following are equivalent.
\begin{enumerate}
\item
\label{extreme-1}
$\varphi$ is extreme among 
ncp-maps $f\colon \scrA\to\scrB$ with~$f(1)=\varphi(1)$.
\item
\label{extreme-2}
The map $T\mapsto \left< 1\otimes 1,T\,1\otimes 1\right>$
from~$\scrB^a(\scrA\otimes_\varphi\scrB)$
to~$\scrB$ is injective on~$\varrho(\scrA)^\square$.
\end{enumerate}
\begin{point}{Proof}%
Based on Lemma 1.4.6~of ``Subalgebras of $C^*$-algebras''.

Assume~\ref{extreme-2}.
Suppose that $\varphi\equiv t \psi_1+t^\perp \psi_2$
for ncp-maps $\psi_1,\psi_2\colon \scrA\to\scrB$
with~$\psi_1(1)=\varphi(1)=\psi_2(1)$
and~$t \in(0,1)$.
We must show that~$\psi_1=\psi_2$.
Since~$t \psi_1\leq \varphi$,
there is (by ``Paschke's correspondence'')
an operator $T\in \varrho(\scrA)^\square$
with $0\leq T\leq 1$
and $t\psi_1(a)=\left<1\otimes 1,T \varrho(a)\, 1\otimes 1\right>$
for all~$a\in\scrA$,
so that $\left<1\otimes 1, T \,1\otimes 1\right>
= t\psi_1(1) = t\varphi(1)= t \left<1\otimes 1,1\otimes 1\right>
= \left<1\otimes 1, t1\otimes 1\right>$,
which entails that~$t=T$,
because~$t$ and~$T$ are from~$\varrho(\scrA)^\square$
on which~$\left<1\otimes 1,(\,\cdot\,)1\otimes 1\right>$
is injective.
But then~$t\psi_1=\left<1\otimes 1,t \varrho(\,\cdot\,)\,1\otimes 1\right>
=t \varphi$,
so that $\varphi = t\psi_1+t^\perp \psi_2
= t \varphi+t^\perp\psi_2$,
implying that~$t^\perp \psi_2= t^\perp \varphi$,
and so~$\psi_1=\psi_2=\varphi$.

Assume~\ref{extreme-1}.
Let~$h\colon \varrho(\scrA)^\square\to \scrB$
be the ncp-map given by~$h(S)=\left<1\otimes 1,S1\otimes 1\right>$,
which we must prove to be injective.
It suffices to show that~$h$ is injective on self-adjoint elements.
Let~$S$ be a self-adjoint element of~$\varrho(\scrA)^\square$
with $h(S)=0$;
we must show that~$S=0$.
Define $A:=\frac{1}{4}+\frac{1}{2}\|S\|^{-1}S$
so that~$\frac{1}{4}\leq A\leq \frac{3}{4}$.
Note that~$h(A)=\frac{1}{4}$ and
so $\varphi_A(1)=h(A)=\frac{1}{4}$
and similarly~$\varphi_{A^\perp}(1)=\frac{3}{4}$.
Since clearly
$\varphi_A+\varphi_{A^\perp} = \varphi$
we have $\frac{1}{4}(4\varphi_A)\,+\,\frac{3}{4}(\frac{4}{3}\varphi_{A^\perp})
\,=\,\varphi$,
and so~$4\varphi_A=\frac{4}{3}\varphi_{A^\perp}=\varphi$
since~$\varphi$ is extreme.
Since then~$\varphi_{4A}=\varphi_{1}$, we get~$4A=1$,
and so~$1=4A=1+2\|S\|^{-1}S$
implying~$S=0$,
and thus that~$h$ is injective.
\end{point}

\end{point}
\end{parsec}

\begin{parsec}%
\begin{point}%
Parsec about commutative $C^*$-algebras:
\end{point}
\begin{point}[ccstar-proj]{Theorem}%
The projections~$\Proj(\scrA)$ of a commutative $C^*$-algebra~$\scrA$
form a Boolean algebra.
If~$\scrA$ is monotone complete,
then~$\Proj(\scrA)$ is a complete Boolean algebra.
\end{point}
\end{parsec}

\begin{parsec}%
\begin{point}%
Let us now turn to quotients of von Neumann algebras.
Recall that given a norm closed two-sided ideal~$\scrD$
of a $C^*$-algebra $\scrA$
we can form the quotient $\scrA/\scrD$,
(which is again a $C^*$-algebra),
and the quotient map $q\colon \scrA\to \scrA/\scrD$
which is a \textsc{miu}-map.
If~$\scrA$ is a von Neumann algebra,
then~$\scrA/\scrD$ might not be a von Neumann algebra,
either because~$\scrA/\scrD$ is not be monotone complete 
(see~\sref{vn-quotient-not-monotone-complete}),
or because~$\scrA/\scrD$ does not have a separating set of normal states
(see~\TODO{}).
However, if~$\scrD$ is ultraweakly closed,
then~$\scrA/\scrD$ is a von Neumann algebra,
as one might have suspected.
To prove this,
we need the somewhat surprising
fact that every ultraweakly closed
ideal~$\scrD$ is of the form $\scrD\equiv c\scrA$,
where~$c$ is a central projection~$c$ of~$\scrA$.
(The quotient~$\scrA/c\scrA$ is then simply~$c^\perp\scrA$.)
\TODO{connection between uw-closed ideals and central projections}
\TODO{universal property for \textsc{n(c)p(s)(u)}-maps}
\end{point}
\begin{point}[vn-quotient-not-monotone-complete]{Example}%
The quotient of a von Neumann algebra
by an closed ideal
need not be a von Neumann algebra:
we will show that $\ell^\infty/c_0$
is not monotone complete,
where~$\ell^\infty$ is the von Neumann algebra
of bounded sequences,
and~$c_0\subseteq \ell^\infty$
is the norm closed ideal of
sequences which converge to~$0$.
By~\sref{ccstar-proj}
it suffices to show that the Boolean algebra~$\Proj(\ell^\infty/c_0)$
of projections of~$\ell^\infty/c_0$ is not complete.
\begin{point}%
We claim that $\Proj(\ell^\infty/c_0)$
is isomorphic to the Boolean algebra
 $\wp(\N)/\wp_{fin}(\N)$
of subsets of~$\N$ modulo 
the filter~$\wp_{fin}(\N)$ of finite subsets.

It is easy to see that the assignment $A\mapsto [\mathbf{1}_A]_{c_0}$
(where $\mathbf{1}_A$ is the indicator function of~$A$)
gives a Boolean algebra homomorphism 
$f\colon \wp(\N)\to \Proj(\ell^\infty/c_0)$,
that the kernel of~$f$
is $\wp_{fin}(\N)$
(because for $A\subseteq \N$
we have $\mathbf{1}_A\in c_0$ iff $A$ is finite).
It remains to be shown that~$f$ is surjective,
which requires some fiddling.

Let~$\alpha\in \ell^\infty$
be
such that $[\alpha]_{c_0}$ is a projection in $\ell^\infty/c_0$.
We claim that
$\mathbf{1}_A - \alpha \in c_0$ (and so $\alpha=f(A)$)
for some~$A\subseteq \N$.
Note that since $[\alpha]_{c_0}$ is self-adjoint,
we have $[\alpha]_{c_0}=(\,[\alpha]_{c_0}\,)_\R = [\,\alpha_\R\,]_{c_0}$,
and so we may assume without loss of generality
that~$\alpha$ is self-adjoint
(replacing $\alpha$ by $\alpha_\R$ is necessary).
Define $A=\{n\in\N\colon \left|1-\alpha(n)\right|<\frac{1}{2}\}$.

Let~$\varepsilon>0$ with $\varepsilon<\nicefrac{1}{2}$ be given.
We must find~$N\in\N$ with 
$\left| \mathbf{1}_A(n)-\alpha(n)\right|\leq \varepsilon$
for all~$n\geq N$.
Since~$[\alpha]_{c_0}$ is a projection,
$\alpha-\alpha^2\in c_0$,
and so
there is~$N\in \N$
with $\left|\alpha(n)\right| \,\left| 1-\alpha(n)\right|
\,\equiv\, \left| \alpha(n) - \alpha(n)^2 \right| \leq 
\nicefrac{1}{2}\,\varepsilon < \nicefrac{1}{4}$
for all~$n\geq N$.

Let~$n\geq N$ be given.
Note that either 
$\left|1-\alpha(n)\right|\geq \nicefrac{1}{2}$
or 
$\left|\alpha(n)\right|\geq \nicefrac{1}{2}$.
If~$\left|1-\alpha(n)\right|\geq\nicefrac{1}{2}$,
then $\mathbf{1}_A(n)=0$
(by definition of~$A$),
and so $\left|\mathbf{1}_A(n)-\alpha(n)\right|\equiv \left|\alpha(n)\right| 
\leq 2 \,\left|\alpha(n)\right|\,\left| 1-\alpha(n)\right|
\leq \varepsilon$.
On the other hand,
if  $\left|\alpha(n)\right| \geq \nicefrac{1}{2}$,
then $\left|1-\alpha(n)\right| < \nicefrac{1}{2}$
(because otherwise $\left|\alpha(n)\right|\left|1-\alpha(n)\right|
\geq \nicefrac{1}{4}$),
so $\mathbf{1}_A(n)=1$,
and thus
$\left|\mathbf{1}_A(n)-\alpha(n)\right|
\equiv \left|1-\alpha(n)\right|
\leq 2 \left|\alpha(n)\right|\left|1-\alpha(n)\right|
\leq \varepsilon$.

Hence $\mathbf{1}_A - \alpha \in c_0$,
and so~$f$ is surjective.
It follows that $\Proj(\ell^\infty/c_0)$
is isomorphic to $\wp(\N)/\wp_{fin}(\N)$.
\end{point}
\begin{point}%
We claim that the Boolean algebra~$\wp(\N)/\wp_{fin}(\N)$
is not complete.
To see this,
find a partition of~$\N$ into infinite subsets
$A_1,A_2,\dotsc$.
We claim that~$A_1,A_2,\dotsc$ has no supremum in~$\wp(\N)/\wp_{fin}(\N)$
because there is no minimal upper bound.
Indeed,
let~$A$ be an upper bound of~$A_1,A_2,\dotsc$ in~$\wp(\N)/\wp_{fin}(\N)$,
that is, $A_n\backslash A$ is finite for every~$n$;
we will define an upper bound~$A'$ of $A_1,A_2,\dotsc$
which is strictly below~$A$ in~$\wp(\N)/\wp_{fin}(\N)$.
Since~$A_n$ is infinite,
and~$A_n\backslash A$ is finite,
we can pick $a_n\in A_n\cap A$ for each~$n$.
Then~$A':=A \backslash \{a_1,a_2,a_3,\dotsc\}$
is an upper bound for~$A_1,A_2,\dotsc$,
because $A_n\backslash A' = (A_n\backslash A)\cup\{a_n\}$ is finite
for each~$n$,
and~$A'$ is strictly below~$A$ in~$\wp(\N)/\wp_{fin}(\N)$,
because $A'\backslash A$ is infinite.
Hence~$\wp(\N)/\wp_{fin}(\N)$
is not complete.
\end{point}
\begin{point}%
\TODO{connection with $\beta\N\backslash\N$
being non-Stonean.}
\end{point}
\end{point}
\end{parsec}
\begin{parsec}%
\begin{point}%
We will construct the left adjoint
to the inclusion $U\colon \cW{nmiu}\longrightarrow \cW{npu}$
(see~\sref{cw-giry}),
which 
is perhaps to commutative von Neumann algebras,
what the Giry monad is to  measure spaces,
and the Radon monad is to compact Hausdorff spaces.

We'll first find a left adjoint to $\cW{nmiu}\longrightarrow \cW{pu}$,
and then modify it
to the left adjoint to $\cW{nmiu}\longrightarrow \cW{npu}$
using~\sref{make-pu-map-normal}.
\end{point}
\begin{point}[make-pu-map-normal]{Lemma}%
Let~$f\colon \scrA\to\scrB$
be a \textsc{pu}-map between
von Neumann algebras.
Let~$\scrD$ be the ultraweakly closed ideal of~$\scrB$
generated by
\begin{equation*}
\textstyle
\scrD' \ :=\ \{\ f(\bigvee D)-\bigvee_{d\in D}f(d)\colon \, 
D\subseteq \scrA\text{ is directed and bounded}\ \}.
\end{equation*}
Let~$q\colon \scrB\to \scrB/\scrD$ be the quotient
map (see \TODO{paragraph on quotients}).

Then~$q\circ f$ is normal.
Moreover,
for every von Neumann algebra~$\scrC$
and \textsc{nmiu}-map $g\colon \scrB\to \scrC$
such that~$g\circ f$ is normal,
there is a unique \textsc{nmiu}-map $h\colon \scrB/\scrD\to \scrC$
such that $g=h \circ q$.
\begin{point}{Proof}%
Since for every bounded and directed~$D\subseteq \scrA$
we have
$q(f(\bigvee D)) - \bigvee_{d\in D}q(f(d))
\,=\, q(\ f(\bigvee D)-\bigvee_{d\in D}f(d)\ ) \,=\, 0$
by definition of~$\scrD$,
we see that~$q\circ f$ is normal.
\end{point}
\begin{point}%
Let~$g\colon \scrB\to \scrC$ be an \textsc{nmiu}-map
such that $g\circ f$ is normal.
Since for every bounded directed subset~$D$ of~$\scrA$
we have $g(\,f(\bigvee D)-\bigvee_{d\in D}f(d)\,)=0$,
we see that~$\scrD'$ is a subset of $\ker(g)$, the kernel of~$g$,
which is an ultraweakly closed ideal of~$\scrB$.
Hence $\scrD\subseteq \ker(g)$ by definition of~$\scrD$.
Then by~\TODO{universal prop.~of quotients}
there is a unique \textsc{nmiu}-map $h\colon \scrB/\scrD\to \scrC$
such that $g=h\circ q$.\qed
\end{point}
\end{point}
\begin{point}[cw-giry]{Theorem}%
Let~$\scrA$ be a von Neumann algebra.
Let~$\eta\colon\scrA\to (\Cont\Stat\scrA)^{**}$
be universal arrow from~$\scrA$ to
the inclusion $\cW{nmiu}\longrightarrow \cW{pu}$
from~\TODO{add}.

Let~$\scrD$ and $q\colon  (\Cont\Stat\scrA)^{**}
\longrightarrow  (\Cont\Stat\scrA)^{**}/\scrD$
be as in~\sref{make-pu-map-normal}
(taking $f:=\eta$).

Then~$q\circ \eta$ is a universal arrow from~$\scrA$
to the inclusion $\cW{nmiu}\longrightarrow \cW{npu}$.
\begin{point}{Proof}%
Since~$q\circ \eta$ is normal, $q\circ \eta$ is an arrow in
$\cW{npu}$.
Let~$\scrB$ be a von Neumann algebra,
and let~$f\colon \scrA\to \scrB$
be a \textsc{npu}-map.
We must show that there is a unique \textsc{nmiu}-map
$g\colon 
(\Cont\Stat\scrA)^{**}/\scrD \longrightarrow \scrB$
such that~$g\circ q\circ \eta = f$.
\begin{point}{Existence}
Since~$g$ is also just a \textsc{pu}-map,
there is (by the universal property of~$\eta$)
a unique \textsc{nmiu}-map $g'\colon 
(\Cont\Stat\scrA)^{**} \longrightarrow \scrB$
such that $f=g'\circ \eta$.

But since~$f\equiv g'\circ \eta$ is also normal,
there is (by the universal property of~$q$) 
a unique \textsc{nmiu}-map $g\colon 
(\Cont\Stat\scrA)^{**}/\scrD \longrightarrow \scrB$
such that $g'=g\circ q$. 

Hence $f=g\circ q \circ \eta$.
\end{point}
\begin{point}{Uniqueness}%
Let~$g_1,g_2\colon 
(\Cont\Stat\scrA)^{**}/\scrD \longrightarrow \scrB$
be \textsc{nmiu}-maps
such that $g_1\circ q\circ \eta = g_2 \circ q \circ \eta$.
We must show that~$g_1=g_2$.
Since~$g_1\circ q \circ \eta = g_2 \circ q \circ \eta$,
we get  $g_1\circ q = g_2 \circ q$ by the universal property of~$\eta$.
This in turn implies that~$g_1=g_2$ by the universal property of~$q$.\qed
\end{point}
\end{point}
\end{point}
\end{parsec}
\section{Hereditarily Central}
\begin{parsec}%
\begin{point}{Definition}%
Let~$\scrA$ be a von Neumann algebra.
\begin{enumerate}
\item
$c\in \scrA$
is called \Define{hereditarily central}
if~$ca$ is central for all~$a\in\scrA$.
\item
The set of hereditarily central elements of~$\scrA$
is denoted by~$\hZ(\scrA)$.
\end{enumerate}
\end{point}
\begin{point}{Lemma}%
Let~$c$ be a central element of a von Neumann algebra.\\
Then~$c$ is hereditarily central
iff~$c\scrA$ is Abelian.
\begin{point}{Proof}%
If~$c$ is hereditarily central,
then all elements of~$c\scrA$
are central,
and so~$c\scrA$ is Abelian.
For the other direction,
suppose that~$c\scrA$ is Abelian,
and let~$a\in\scrA$ be given.
We must show that~$ca$ is central.
Let~$b\in\scrA$ be given.
We must show that~$(ca)b=b(ca)$.
Since~$c\scrA$ is Abelian
and~$c$ is central,
we have $cab=(ca)(cb)=(cb)(ca)=bca$.
Thus~$c$ is hereditarily central.
\end{point}
\end{point}
\begin{point}{Proposition}%
Let~$\scrA$ be a von Neumann algebra.
There is a greatest hereditarily central projection~$h$ in~$\scrA$,
and~$\hZ(\scrA)=h\scrA$.
\begin{point}{Proof}%
By Proposition~\ref{prop:weakly-closed-ideal} it suffices
to show that~$\hZ(\scrA)$ is an ultraweakly closed
two-sided ideal of~$\scrA$.
Using that every hereditarily central
element is also central, it is easily seen
 that~$\hZ(\scrA)$ is a two-sided ideal.

It remains to be shown that~$\hZ(\scrA)$ is ultraweakly closed.
It suffices to prove that~$\hZ(\scrA)$
is ultrastrongly closed.
Let~$(c_i)_i$ be a net in~$\hZ(\scrA)$
which converges ultrastrongly to~$c$ in~$\scrA$.
We must show that~$c$ is hereditarily central.
Let~$a\in \scrA$ be given.
We must show that~$ca$ is central.
Since the set of central elements is ultraweakly closed,
it suffices to show that~$(c_ia)_i$ converges ultraweakly to~$ca$.
Let~$\omega\colon\scrA\to\mathbb{C}$
be a normal state.
We must show that $|\omega(ca-c_ia)|$
converges to~$0$ as~$i$ increases.
By Kadison's inequality,
\begin{equation*}
|\omega(ca -c_ia)| \ =\  |\omega((c-c_i)a)|
\ \leq\  \sqrt{\omega(\,(c-c_i)^*(c-c_i)\,)}\cdot\sqrt{\omega(a^*a)}.
\end{equation*}
The number $\sqrt{\omega((c-c_i)^*(c-c_i))}$ 
vanishes as~$i$ increases
since~$(c_i)_i$ converges to~$c$ ultrastrongly.
Thus~$|\omega(ca-c_ia)|$ vanishes as~$i$ increases.
Thus~$(c_ia)_i$ converges ultrastrongly to~$ca$,
and~$ca$ is central.
Hence~$c$ is hereditarily central,
and so~$\hZ(\scrA)$
is ultrastrongly closed.
\end{point}
\end{point}
\begin{point}{Definition}%
Let~$\scrA$ be a von Neumann algebra.
\begin{enumerate}
\item
Let~$h_\scrA$ denote the greatest hereditarily central
element of~$\scrA$.

\item
Let~$\eta_\scrA\colon \scrA\to \hZ(\scrA)$
be the map given by $\eta_\scrA(a)=h_\scrA a$
for all~$a\in\scrA$.
\end{enumerate}
\end{point}
\begin{point}{Theorem}%
Let~$\scrA$
be a von Neumann algebra.
For every Abelian von Neumann algebra~$\mathscr{B}$
and normal unital $*$-homomorphism $f\colon\scrA\to\mathscr{B}$
there is a unique normal unital $*$-homomorphism 
$g\colon\hZ(\scrA)\to \mathscr{B}$
such that~$g\circ \eta_\scrA = f$.
\begin{point}{Proof}%
\emph{(uniqueness)}\ 
Let~$g_1,g_2\colon \hZ(\scrA)\to\mathscr{B}$
be normal unital $*$-homomorphisms
with $g_1\circ \eta = f = g_2 \circ \eta$.
We must show that~$g_1 = g_2$.
Let~$a\in \hZ(\scrA)$
be given.
Then~$ha=a$, and so $g_1(a) = g_1(h_\scrA a) 
= g_1(\eta_\scrA(a)) = f(a)$.
Similarly, $g_2(a)=f(a)$. Hence $g_1=g_2$.

\emph{(existence)}\ 
Define $g\colon \hZ(\scrA)\to \mathscr{B}$
by~$g(a)=f(a)$.  Then~$g$ is a normal $*$-homomorphism.
It remains to be shown that~$g$ is unital and $g\circ \eta_\scrA = f$.
Both issues are solved by proving that~$f(h_\scrA^\perp)=0$.

Let~$\mathscr{D}:=\{a\in \scrA\colon f(a)=0\}$
be the kernel of~$f$.
We must show that~$h_\scrA^\perp \in \mathscr{D}$.
Since~$f$ is a normal unital $*$-homomorphism,
$\mathscr{D}$ is an ultraweakly closed two-sided ideal of~$\scrA$.
Thus~$\mathscr{D} = c\scrA$
for some central projection~$c$ of~$\scrA$
by Proposition~\ref{prop:weakly-closed-ideal}.
Moreover, $c$ is the greatest projection of~$\mathscr{D}$.
Thus, to prove that~$h_\scrA^\perp \in\mathscr{D}$
we it suffices to show that~$h_\scrA^\perp \leq c$,
or in other words, $c^\perp \leq h_\scrA$.
For this, it suffices to show that~$c^\perp$
is hereditarily central,
or equivalently that $c^\perp\scrA$
is Abelian (see Lemma~\ref{lem:hz-abelian-ca}).

The restriction of~$f$ to~$c^\perp\scrA$
gives a multiplicative map $c^\perp\scrA \to\mathscr{B}$,
which is injective (because the kernel of~$f$ is~$c\scrA$).
Thus, since~$\mathscr{B}$ is Abelian, so is~$c^\perp \scrA$.
Hence~$f(h_\scrA^\perp)=0$.
\end{point}
\end{point}
\begin{point}{Corollary}%
The assignment $\scrA\mapsto \hZ(\scrA)$
extends to 
a functor 
\begin{equation*}
\hZ\colon \W{nmiu}\longrightarrow \cW{nmiu},
\end{equation*}
which is left adjoint
to the forgetfull functor $\cW{nmiu}\longrightarrow \W{nmiu}$.
The unit of this adjunction
is given by the maps $\eta_\scrA
\colon \scrA\to\hZ(\scrA)$.
\end{point}
\begin{point}{Lemma}%
Let~$\scrA$ and~$\mathscr{B}$
be von Neumann algebras.
Then~$h_{\scrA\otimes\mathscr{B}}=
h_\scrA\otimes h_\mathscr{B}$.
\begin{point}{Proof}%
Note that $(h_\scrA \otimes h_\mathscr{B})(
\scrA\otimes\mathscr{B})
\cong (h_\scrA\scrA)\otimes(h_\mathscr{B}\mathscr{B})$
is Abelian,
and so $h_\scrA\otimes h_\mathscr{B}$
is hereditarily central
by Lemma~\ref{lem:hz-abelian-ca}.

It remains to be shown that
$h_\scrA\otimes h_\mathscr{B}$
is the greatest hereditarily central element,
$h_{\scrA\otimes \mathscr{B}}$.
To this end, observe that
\begin{equation*}
1_\scrA\otimes 1_\mathscr{B}
\ = \ 
h_\scrA\otimes h_\mathscr{B}
\ +\ 
h_\scrA^\perp\otimes h_\mathscr{B}
\ +\ 
h_\scrA\otimes h_\mathscr{B}^\perp
\ +\ 
h_\scrA^\perp\otimes h_\mathscr{B}^\perp.
\end{equation*}
Thus,
if we prove that
$h_\scrA^\perp\otimes h_\mathscr{B}
+ 
h_\scrA\otimes h_\mathscr{B}^\perp
+
h_\scrA^\perp\otimes h_\mathscr{B}^\perp$
is an element of~$\mathscr{D}:=
 h_{\scrA\otimes \mathscr{B}}^\perp
(\scrA\otimes\mathscr{B})$,
then $h_{\scrA\otimes \mathscr{B}}
= h_\scrA\otimes h_\mathscr{B}$.
(To see this, unfold $h_{\scrA\otimes \mathscr{B}}
\cdot (1_\scrA\otimes 1_\mathscr{B})$.)

Let us first prove that 
$h_\scrA^\perp \otimes h_\mathscr{B}
= (h_\scrA^\perp \otimes 1)\cdot(1\otimes h_\mathscr{B})$
is in~$\mathscr{D}$.
It suffices to show that 
$(h_\scrA^\perp \otimes 1)$
is in~$\mathscr{D}$.
Note that~$\mathscr{D}$ is the kernel 
of~$\eta_{\scrA\otimes\mathscr{B}}$,
so we must show that~$\eta_{\scrA\otimes\mathscr{B}}(h_\scrA
\otimes 1)=0$.
We have the following commutative diagram.
\begin{equation*}
\xymatrix{
\scrA
\ar[rr]^-{a\mapsto a\otimes 1}
\ar[d]_{\eta_\scrA}
&&
\scrA\otimes\mathscr{B}
\ar[d]^{\eta_{\scrA\otimes\mathscr{B}}}
\\
\hZ(\scrA)
\ar[rr]_-{\hZ((-)\otimes 1)}
&&
\hZ(\scrA\otimes \mathscr{B})
}
\end{equation*}
Thus,
as~$\eta_\scrA(h^\perp_\scrA)=0$,
it follows that
\begin{equation*}
\eta_{\scrA\otimes \mathscr{B}}(h_\scrA^\perp
\otimes 1)
\ =\ 
\hZ((-)\otimes 1) (\, \eta_\scrA(h_\scrA^\perp)\,)
\ = \ 
\hZ((-)\otimes 1) (0)\,=\,0.
\end{equation*}
So
$h_\scrA^\perp \otimes h_\mathscr{B}$
is in~$\mathscr{D}$.
By a similar reasoning,
we see that 
$h_\scrA \otimes h^\perp_\mathscr{B}$
and
$h_\scrA^\perp \otimes h^\perp_\mathscr{B}$
are in~$\mathscr{D}$.
Hence~$h_{\scrA\otimes\mathscr{B}}
=h_\scrA\otimes h_\mathscr{B}$.
\end{point}
\end{point}
\begin{point}{Corollary}%
$\hZ(\scrA\otimes\mathscr{B})
\cong \hZ(\scrA)\otimes\hZ(\mathscr{B})$.
\end{point}
\end{parsec}
The proof of 
Lemma~\ref{lem:continuous-finite-measure-space-not-duplicable} 
is an variation
of the proof of Theorem~6.4 of~\cite{Kornell2012}.








\section{Some categorical consequences}

\begin{parsec}%
\begin{point}%
Let $\W{nmiu}$ denote the category of von Neumann algebras
and normal MIU-maps, and $\cW{nmiu}\hookrightarrow\W{nmiu}$
the full subcategory of commutative von Neumann algebras.

It will be shown that the inclusion functor
$\cW{nmiu}\hookrightarrow\W{nmiu}$ has a left adjoint.
This means that $\cW{nmiu}$ is a reflective subcategory of $\W{nmiu}$.

We have the following general categorical results.
\end{point}
\begin{point}{Proposition}%
Let $G\colon\Cat{B}\to\Cat{A}$ be a full and faithful functor
with a left adjoint $F\colon\Cat{A}\to\Cat{B}$.
Let $T=GF$ be the induced monad on $\Cat{A}$.
Then:
\begin{enumerate}
\item
The counit $\epsilon\colon FG\to\id_{\Cat{B}}$
of the adjunction is an isomorphism.
\item
The monad $T=GF$ is idempotent,
i.e.\ the multiplication $\mu\colon TT\to T$
is an isomorphism.
\item
The adjunction $F\dashv G$ is monadic,
i.e.\ the comparison functor $\Cat{B}\to\Cat{A}^{T}$
is an equivalence.
\item
The functor $G\colon\Cat{B}\to\Cat{A}$ creates limits
(not necessarily strictly).
\item
Let $D\colon\Cat{I}\to\Cat{B}$ be a diagram.
If we have a colimit $\Colim GD$ of $GD\colon\Cat{I}\to\Cat{A}$,
then $F\Colim GD$ is a colimit of $D$
(with the obvious cocone).
\end{enumerate}
\begin{point}{Proof}%
TODO: give references.
See also nLab's article ``reflective subcategory''.
\end{point}
\end{point}
\begin{point}%
The category $\W{nmiu}$ is complete and cocomplete.
Applying the above results to
the reflective subcategory $\cW{nmiu}\hookrightarrow\W{nmiu}$,
we see that $\cW{nmiu}$ is also complete and cocomplete.
The creation of limits means
that limits in $\cW{nmiu}$ are simply those in $\W{nmiu}$,
and thus $\cW{nmiu}$ is closed under limits.
On the other hand, $\cW{nmiu}$ is not closed under colimits,
since a free product of commutative von Neumann algebras
may be noncommutative (TODO: proof). This implies that
there is no right adjoint to
the inclusion $\cW{nmiu}\hookrightarrow\W{nmiu}$.
Conjecture: In $\cW{nmiu}$,
the tensor product of von Neumann algebras
is a coproduct.
\end{point}
\end{parsec}
\subsection{Functionals}
\begin{parsec}
\begin{point}{Definition}%
Given a functional $f\colon \scrA\to \C$
on a $C^*$-algerbra~$\scrA$,
we define
\begin{enumerate}
\item
$\Define{\Real{f}}\colon \scrA\to\C$,
the \Define{real part} of~$f$,
by~$\Real{f}(a)=\Real{f(\Real{a})}
+i\Real{f(\Imag{a})}$, and
\item
$\Define{\Imag{f}}\colon \scrA\to\C$,
the \Define{imaginary part} of~$f$,
by~$\Imag{f}(a)=\Imag{f(\Real{a})}
+i\Imag{f(\Imag{a})}$.
\end{enumerate}
\end{point}
\begin{point}{Exercise}%
Show that,
given a functional~$f\colon \scrA\to \C$
on a $C^*$-algebra~$\scrA$,
\begin{enumerate}
\item
	$f=\Real{f}+i\Imag{f}$, and 
	$\Real{f}$ and $\Imag{f}$
	are i-maps (linear and involution preserving);
\item
	if $f=g+ih$ for some i-maps
	$g,h\colon \scrA\to\C$,
	then $\Real{f}=g$ and $\Imag{f}=h$.
\end{enumerate}
\end{point}
\end{parsec}
\end{document}
